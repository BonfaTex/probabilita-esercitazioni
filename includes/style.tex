%%%%%%%%%%%%%%%%%%%%%%%%%%%%%%%%%%%%%%%%%
% Template Dispense
%
% Autore:
% Teo Bucci
%
%%%%%%%%%%%%%%%%%%%%%%%%%%%%%%%%%%%%%%%%%

%----------------------------------------------------------------------------------------
% FONTS AND LANGUAGE
%----------------------------------------------------------------------------------------

\usepackage[T1]{fontenc} % Use 8-bit encoding that has 256 glyphs
\usepackage[utf8]{inputenc} % Required for including letters with accents
\usepackage[italian]{babel}

%----------------------------------------------------------------------------------------
%	VARIOUS REQUIRED PACKAGES AND CONFIGURATIONS
%----------------------------------------------------------------------------------------

\usepackage{latexsym}

%\usepackage{amsmath,amsfonts,amssymb,amsthm} % For math equations, theorems, symbols, etc

%----------------------------------------------------------------------------------------
%	FIGURE MATHCHA
%----------------------------------------------------------------------------------------

\usepackage{dsfont} % per \mathds{} (funzione indicatrice)
\usepackage{amsmath,amssymb,amsthm} % amssymb carica anche amsfonts
\usepackage{tikz}
\usetikzlibrary{positioning,shapes.misc,intersections,shapes.symbols,patterns,fadings,shadows.blur}
\usepackage{mathdots}
\usepackage{cancel}
\usepackage{color}
\usepackage{siunitx}
\usepackage{array}
\usepackage{multirow}
\usepackage{makecell}
\usepackage{tabularx}
\usepackage{caption}
\captionsetup{belowskip=12pt,aboveskip=4pt}
\usepackage{subcaption}
\usepackage{placeins} % \FloatBarrier
\usepackage{flafter}  % The flafter package ensures that floats don't appear until after they appear in the code.
\usepackage[shortlabels]{enumitem}
\usepackage[italian]{varioref}


%----------------------------------------------------------------------------------------
%	INCLUSIONE FIGURE
%----------------------------------------------------------------------------------------

\usepackage{import}
\usepackage{pdfpages}
\usepackage{transparent}
\usepackage{xcolor}
\usepackage{graphicx}
\graphicspath{ {./images/} } % Path relative to the main .tex file 
\usepackage{float}

\newcommand{\fg}[3][\relax]{%
  \begin{figure}[H]%[htp]%
    \centering
    \captionsetup{width=0.7\textwidth}
      \includegraphics[width = #2\textwidth]{#3}%
      \ifx\relax#1\else\caption{#1}\fi
      \label{#3}
  \end{figure}%
  \FloatBarrier%
}

%----------------------------------------------------------------------------------------
%	PARAGRAFI, INTERLINEA E MARGINE
%----------------------------------------------------------------------------------------

\usepackage[none]{hyphenat} % Per non far andare a capo le parole con il trattino

\emergencystretch 3em % Per evitare che il testo vada oltre i margini

% \parindent 0ex % TOGLIE INTENDAMENTO PARAGRAFI, E' INCLUSO NEL PACCHETTO \parskip
% \setlength{\parindent}{4em} % VARIANTE DI QUELLO SOPRA

% \setlength{\parskip}{\baselineskip} % CAMBIA SPAZIO TRA PARAGRAFI (POSSO METTERE ANCHE 1em) INCLUSA LA TABLE OF CONTENTS, PERTANTO USO IL COMANDO CHE SEGUE

\usepackage[skip=0.2\baselineskip+2pt]{parskip}

% \renewcommand{\baselinestretch}{1.5} % CAMBIA INTERLINEA

%----------------------------------------------------------------------------------------
%	HEADERS AND FOOTERS
%----------------------------------------------------------------------------------------

\usepackage{fancyhdr}

\pagestyle{empty} % Il fancy serve a partire al primo capitolo
\fancyhead{} % Pulisci header
\fancyfoot{} % Pulisci footer

\fancyhead[RE]{\nouppercase{\leftmark}}
%\fancyhead[LO]{\emph{Trascrizioni di Teo Bucci}}
\fancyhead[LO]{\nouppercase{\rightmark}}
\fancyhead[LE,RO]{\thepage}

% Removes the header from odd empty pages at the end of chapters
\makeatletter
\renewcommand{\cleardoublepage}{
\clearpage\ifodd\c@page\else
\hbox{}
\vspace*{\fill}
\thispagestyle{empty}
\newpage
\fi}

%----------------------------------------------------------------------------------------
%	COMANDI PERSONALIZZATI
%----------------------------------------------------------------------------------------


\newcommand{\latex}{\LaTeX\xspace}
\newcommand{\tex}{\TeX\xspace}

\newcommand{\Tau}{\mathcal{T}}
\newcommand{\Ind}{\mathds{1}} % indicatrice

\newcommand{\transpose}{^{\mathrm{T}}}
\newcommand{\complementary}{^{\mathrm{C}}} % alternative ^{\mathrm{C}} ^{\mathrm{c}} ^{\mathsf{c}}
\newcommand{\degree}{^\circ\text{C}} % simbolo gradi

\newcommand{\notimplies}{\mathrel{{\ooalign{\hidewidth$\not\phantom{=}$\hidewidth\cr$\implies$}}}}
\newcommand{\questeq}{\overset{?}{=}} % è vero che?

\newcommand{\Bot}{\perp \!\!\! \perp} % indipendenza
\newcommand{\iid}{\stackrel{\mathrm{iid}}{\sim}}
\newcommand{\event}[1]{\emph{``#1''}}

\newcommand{\boxedText}[1]{\noindent\fbox{\parbox{\textwidth}{#1}}}

\renewcommand{\qed}{\tag*{$\blacksquare$}}
\renewcommand{\emptyset}{\varnothing} % simbolo insieme vuoto
\renewcommand{\tilde}{\widetilde}
\renewcommand{\hat}{\widehat}

\DeclareMathOperator{\sgn}{sgn}

\usepackage{mathtools} % Serve per i comandi dopo
\DeclarePairedDelimiter{\abs}{\lvert}{\rvert}
\DeclarePairedDelimiter{\norm}{\lVert}{\rVert}
\DeclarePairedDelimiter{\sca}{\left\langle}{\right\rangle}

% Bold
% \newcommand{\AA}{\mathbb{A}} already defined in LaTeX
\newcommand{\BB}{\mathbb{B}}
\newcommand{\CC}{\mathbb{C}}
\newcommand{\DD}{\mathbb{D}}
\newcommand{\EE}{\mathbb{E}}
\newcommand{\FF}{\mathbb{F}}
\newcommand{\GG}{\mathbb{G}}
\newcommand{\HH}{\mathbb{H}}
\newcommand{\II}{\mathbb{I}}
\newcommand{\JJ}{\mathbb{J}}
\newcommand{\KK}{\mathbb{K}}
\newcommand{\LL}{\mathbb{L}}
\newcommand{\MM}{\mathbb{M}}
\newcommand{\NN}{\mathbb{N}}
\newcommand{\OO}{\mathbb{O}}
\newcommand{\PP}{\mathbb{P}}
\newcommand{\QQ}{\mathbb{Q}}
\newcommand{\RR}{\mathbb{R}}
%\newcommand{\SS}{\mathbb{S}} already defined in LaTeX
\newcommand{\TT}{\mathbb{T}}
\newcommand{\UU}{\mathbb{U}}
\newcommand{\VV}{\mathbb{V}}
\newcommand{\WW}{\mathbb{W}}
\newcommand{\XX}{\mathbb{X}}
\newcommand{\YY}{\mathbb{Y}}
\newcommand{\ZZ}{\mathbb{Z}}

% Calligraphic
\newcommand{\Ac}{\mathcal{A}}
\newcommand{\Bc}{\mathcal{B}}
\newcommand{\Cc}{\mathcal{C}}
\newcommand{\Dc}{\mathcal{D}}
\newcommand{\Ec}{\mathcal{E}}
\newcommand{\Fc}{\mathcal{F}}
\newcommand{\Gc}{\mathcal{G}}
\newcommand{\Hc}{\mathcal{H}}
\newcommand{\Ic}{\mathcal{I}}
\newcommand{\Jc}{\mathcal{J}}
\newcommand{\Kc}{\mathcal{K}}
\newcommand{\Lc}{\mathcal{L}}
\newcommand{\Mc}{\mathcal{M}}
\newcommand{\Nc}{\mathcal{N}}
\newcommand{\Oc}{\mathcal{O}}
\newcommand{\Pc}{\mathcal{P}}
\newcommand{\Qc}{\mathcal{Q}}
\newcommand{\Rc}{\mathcal{R}}
\newcommand{\Sc}{\mathcal{S}}
\newcommand{\Tc}{\mathcal{T}}
\newcommand{\Uc}{\mathcal{U}}
\newcommand{\Vc}{\mathcal{V}}
\newcommand{\Wc}{\mathcal{W}}
\newcommand{\Xc}{\mathcal{X}}
\newcommand{\Yc}{\mathcal{Y}}
\newcommand{\Zc}{\mathcal{Z}}

% differenziale
\newcommand{\de}{\mathrm{d}}
\newcommand{\dx}{\de x}
\newcommand{\dy}{\de y}
\newcommand{\dt}{\de t}
\newcommand{\ds}{\de s}
\newcommand{\dz}{\de z}
\newcommand{\dw}{\de w}
\newcommand{\du}{\de u}
\newcommand{\dv}{\de v}
\newcommand{\dxy}{\de x\ \de y}
\newcommand{\dP}{\de\PP}

\newcommand{\SDP}{(\Omega,\Ac,\PP)} % spazio di probabilità
\newcommand{\Cz}{\Cc^0}
\newcommand{\Cu}{\Cc^1}
\newcommand{\Lu}{\mathcal{L}^1}

\newcommand{\fXY}{f_{(X,Y)}}
\newcommand{\fXYxy}{\fXY(x,y)}
\newcommand{\indep}{\mathrel{\text{\scalebox{1.07}{$\perp\mkern-10mu\perp$}}}}

%----------------------------------------------------------------------------------------
%	APPENDICE
%----------------------------------------------------------------------------------------

\usepackage[toc,page]{appendix}

%----------------------------------------------------------------------------------------
%	SIMBOLI CARTE DA GIOCO
%
% Sono i seguenti:
%   \varheartsuit
%   \vardiamondsuit
%   \clubsuit
%   \spadesuit
%----------------------------------------------------------------------------------------

\DeclareSymbolFont{extraup}{U}{zavm}{m}{n}
\DeclareMathSymbol{\varheartsuit}{\mathalpha}{extraup}{86}
\DeclareMathSymbol{\vardiamondsuit}{\mathalpha}{extraup}{87}


%----------------------------------------------------------------------------------------
%----------------------------------------------------------------------------------------
%----------------------------------------------------------------------------------------
%----------------------------------------------------------------------------------------
%----------------------------------------------------------------------------------------
%----------------------------------------------------------------------------------------
%----------------------------------------------------------------------------------------
%----------------------------------------------------------------------------------------
%----------------------------------------------------------------------------------------
%----------------------------------------------------------------------------------------
%%%% TESTING

\definecolor{grey245}{RGB}{245,245,245}

% questo fa si che le footnote siano "fuori" dall'environment dimostrazione
% \usepackage{footnote}
% \usepackage{etoolbox}
% \BeforeBeginEnvironment{dimostrazione}{\savenotes}
% \AfterEndEnvironment{dimostrazione}{\spewnotes}
% \BeforeBeginEnvironment{theorem}{\savenotes}
% \AfterEndEnvironment{theorem}{\spewnotes}

\newtheoremstyle{blacknumbox} % Theorem style name
{0pt}% Space above
{0pt}% Space below
{\normalfont}% Body font
{}% Indent amount
{\bf\scshape}% Theorem head font --- {\small\bf}
{.\;}% Punctuation after theorem head
{0.25em}% Space after theorem head
{\small\thmname{#1}\nobreakspace\thmnumber{\@ifnotempty{#1}{}\@upn{#2}}% Theorem text (e.g. Theorem 2.1)
%{\small\thmname{#1}% Theorem text (e.g. Theorem)
\thmnote{\nobreakspace\the\thm@notefont\normalfont\bfseries---\nobreakspace#3}}% Optional theorem note

\newtheoremstyle{unnumbered} % Theorem style name
{0pt}% Space above
{0pt}% Space below
{\normalfont}% Body font
{}% Indent amount
{\bf\scshape}% Theorem head font --- {\small\bf}
{.\;}% Punctuation after theorem head
{0.25em}% Space after theorem head
{\small\thmname{#1}\thmnumber{\@ifnotempty{#1}{}\@upn{#2}}% Theorem text (e.g. Theorem 2.1)
%{\small\thmname{#1}% Theorem text (e.g. Theorem)
\thmnote{\nobreakspace\the\thm@notefont\normalfont\bfseries---\nobreakspace#3}}% Optional theorem note


\newcounter{dummy} 
\numberwithin{dummy}{chapter}


\theoremstyle{blacknumbox}
\newtheorem{definitionT}[dummy]{Definizione}
\newtheorem{theoremT}[dummy]{Teorema}
\newtheorem{corollarioT}[dummy]{Corollario}
\newtheorem{lemmaT}[dummy]{Lemma}

% Per gli unnumbered tolgo il \nobreakspace subito dopo {\small\thmname{#1} perché altrimenti c'è uno spazio tra Teorema e il ".", lo spazio lo voglio solo se sono numerati per distanziare Teorema e "(2.1)"
\theoremstyle{unnumbered}
\newtheorem*{NBT}{NB}
\newtheorem*{ossT}{Osservazione}
\newtheorem*{ricalgT}{Richiamo di Algebra}
\newtheorem*{pseudocodiceT}{Pseudocodice}
\newtheorem*{dimT}{Dimostrazione}

\RequirePackage[framemethod=default]{mdframed} % Required for creating the theorem, definition, exercise and corollary boxes

\usepackage{lipsum}



% orange box
\newmdenv[skipabove=7pt,
skipbelow=7pt,
rightline=false,
leftline=true,
topline=false,
bottomline=false,
linecolor=orange,
backgroundcolor=orange!5,
innerleftmargin=5pt,
innerrightmargin=5pt,
innertopmargin=5pt,
leftmargin=0cm,
rightmargin=0cm,
linewidth=2pt,
innerbottommargin=5pt]{oBox}

% green box
\newmdenv[skipabove=7pt,
skipbelow=7pt,
rightline=false,
leftline=true,
topline=false,
bottomline=false,
linecolor=green,
backgroundcolor=green!5,
innerleftmargin=5pt,
innerrightmargin=5pt,
innertopmargin=5pt,
leftmargin=0cm,
rightmargin=0cm,
linewidth=2pt,
innerbottommargin=5pt]{gBox}

% blue box
\newmdenv[skipabove=7pt,
skipbelow=7pt,
rightline=false,
leftline=true,
topline=false,
bottomline=false,
linecolor=blue,
backgroundcolor=blue!5,
innerleftmargin=5pt,
innerrightmargin=5pt,
innertopmargin=5pt,
leftmargin=0cm,
rightmargin=0cm,
linewidth=2pt,
innerbottommargin=5pt]{bBox}

% dim box
\newmdenv[skipabove=7pt,
skipbelow=7pt,
rightline=false,
leftline=true,
topline=false,
bottomline=false,
linecolor=black,
backgroundcolor=grey245!0,
innerleftmargin=5pt,
innerrightmargin=5pt,
innertopmargin=5pt,
leftmargin=0cm,
rightmargin=0cm,
linewidth=2pt,
innerbottommargin=5pt]{dimBox}

\newenvironment{theorem}{\begin{gBox}\begin{theoremT}}{\end{theoremT}\end{gBox}}	
\newenvironment{definition}{\begin{bBox}\begin{definitionT}}{\end{definitionT}\end{bBox}}	
\newenvironment{nb}{\begin{oBox}\begin{NBT}}{\end{NBT}\end{oBox}}	
\newenvironment{oss}{\begin{oBox}\begin{ossT}}{\end{ossT}\end{oBox}}
\newenvironment{ricalg}{\begin{oBox}\begin{ricalgT}}{\end{ricalgT}\end{oBox}}
\newenvironment{pseudocode}{\begin{oBox}\begin{pseudocodiceT}}{\end{pseudocodiceT}\end{oBox}}
\newenvironment{corollario}{\begin{oBox}\begin{corollarioT}}{\end{corollarioT}\end{oBox}}
\newenvironment{lemma}{\begin{oBox}\begin{lemmaT}}{\end{lemmaT}\end{oBox}}
\newenvironment{dimostrazione}{\begin{dimBox}\begin{dimT}}{\[\qed\]\end{dimT}\end{dimBox}}

%----------------------------------------------------------------------------------------
%	INDICE INIZIALE
%----------------------------------------------------------------------------------------

\setcounter{secnumdepth}{3} % DI DEFAULT LE SUBSUBSECTION NON SONO NUMERATE, COSÌ SÌ
\setcounter{tocdepth}{2} % FISSA LA PROFONDITÀ DELLE COSE MOSTRATE NELL'INDICE

\usepackage[hidelinks]{hyperref} % Rende l'indice interattivo e hidelinks nasconde il bordo rosso dai riferimenti

\usepackage{glossaries} % certain packages that must be loaded before glossaries, if they are required: hyperref, babel, polyglossia, inputenc and fontenc
\setacronymstyle{long-short}

% Questo comando e quello dopo servono per avere il comando \tocless da mettere prima di una sezione che non voglio far apparire nell'indice
\newcommand{\nocontentsline}[3]{}
\newcommand{\tocless}[2]{\bgroup\let\addcontentsline=\nocontentsline#1{#2}\egroup}

%\usepackage{showframe}
\usepackage[textsize=tiny, textwidth=1.5cm]{todonotes} %add disable to options to not show in pdf




