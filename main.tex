\documentclass[11pt,a4paper,twoside,openright]{book}

% emph per gli eventi
% spazio dopo \exists e \forall
% rimuovere leqslant, geqslant in favore di leq, geq
% usare versioni var di varepsilon, vartheta, varrho, varphi


\usepackage{geometry}
\geometry{
	nomarginpar, % Toglie doppi margini
	margin=1in, % Imposta i margini a 1 inch
}

%%%%%%%%%%%%%%%%%%%%%%%%%%%%%%%%%%%%%%%%%
% Template Dispense
%
% Autore:
% Teo Bucci
%
%%%%%%%%%%%%%%%%%%%%%%%%%%%%%%%%%%%%%%%%%

%----------------------------------------------------------------------------------------
%	FONTS
%----------------------------------------------------------------------------------------

\usepackage[T1]{fontenc} % Use 8-bit encoding that has 256 glyphs
\usepackage[utf8]{inputenc} % Required for including letters with accents
\usepackage{dsfont} % per funzione indicatrice

%----------------------------------------------------------------------------------------
%	VARIOUS REQUIRED PACKAGES AND CONFIGURATIONS
%----------------------------------------------------------------------------------------

\usepackage[italian]{babel} % Italian language/hyphenation

\usepackage{latexsym}

%\usepackage{amsmath,amsfonts,amssymb,amsthm} % For math equations, theorems, symbols, etc

%----------------------------------------------------------------------------------------
%	FIGURE MATHCHA
%----------------------------------------------------------------------------------------

\usepackage{amsmath}
%\usepackage[fleqn]{amsmath}  % per avere l'allineamento a sinistra delle equazioni
\usepackage{amssymb} % carica anche \usepackage{amsfonts}
\usepackage{amsthm}
\usepackage{tikz}
\usepackage{mathdots}
\usepackage{cancel}
\usepackage{color}
\usepackage{siunitx}
\usepackage{array}
\usepackage{multirow}
%\usepackage{gensymb} % The gensymb package provides a number of ‘generic’ macros, which produce the same output in text and math mode: \degree \celsius \perthousand \ohm \micro
\usepackage{makecell}
\usepackage{tabularx}
\usepackage{booktabs}
\usepackage{caption}\captionsetup{belowskip=12pt,aboveskip=4pt}
\usepackage{subcaption}
\usetikzlibrary{fadings}
\usetikzlibrary{patterns}
\usetikzlibrary{shadows.blur}
\usepackage{placeins} % The placeins package gives the command \FloatBarrier, which will make sure any floats will be put in before this point
\usepackage{flafter}  % The flafter package ensures that floats don't appear until after they appear in the code.

%----------------------------------------------------------------------------------------
%	INCLUSIONE FIGURE
%----------------------------------------------------------------------------------------

\usepackage{import}
\usepackage{pdfpages}
\usepackage{transparent}
\usepackage{xcolor}
\usepackage{graphicx}
\graphicspath{ {./images/} } % Path relative to the main .tex file 
\usepackage{float}

\newcommand{\fg}[3][\relax]{%
  \begin{figure}[H]%[htp]%
    \centering
    \captionsetup{width=0.7\textwidth}
      \includegraphics[width = #2\textwidth]{#3}%
      \ifx\relax#1\else\caption{#1}\fi
      \label{#3}
  \end{figure}%
  \FloatBarrier%
}

%----------------------------------------------------------------------------------------
%	PARAGRAFI, INTERLINEA E MARGINE
%----------------------------------------------------------------------------------------

\usepackage[none]{hyphenat} % Per non far andare a capo le parole con il trattino

\emergencystretch 3em % Per evitare che il testo vada oltre i margini

% \parindent 0ex % TOGLIE INTENDAMENTO PARAGRAFI, E' INCLUSO NEL PACCHETTO \parskip
% \setlength{\parindent}{4em} % VARIANTE DI QUELLO SOPRA

% \setlength{\parskip}{\baselineskip} % CAMBIA SPAZIO TRA PARAGRAFI (POSSO METTERE ANCHE 1em) INCLUSA LA TABLE OF CONTENTS, PERTANTO USO IL COMANDO CHE SEGUE

\usepackage[skip=0.2\baselineskip+2pt]{parskip}

% \renewcommand{\baselinestretch}{1.5} % CAMBIA INTERLINEA

%----------------------------------------------------------------------------------------
%	HEADERS AND FOOTERS
%----------------------------------------------------------------------------------------

\usepackage{fancyhdr}

\pagestyle{empty} % Il fancy serve a partire al primo capitolo
\fancyhead{} % Pulisci header
\fancyfoot{} % Pulisci footer

\fancyhead[RE]{\nouppercase{\leftmark}}
%\fancyhead[LO]{\emph{Trascrizioni di Teo Bucci}}
\fancyhead[LO]{\nouppercase{\rightmark}}
\fancyhead[LE,RO]{\thepage}

% Removes the header from odd empty pages at the end of chapters
\makeatletter
\renewcommand{\cleardoublepage}{
\clearpage\ifodd\c@page\else
\hbox{}
\vspace*{\fill}
\thispagestyle{empty}
\newpage
\fi}

%----------------------------------------------------------------------------------------
%	COMANDI PERSONALIZZATI
%----------------------------------------------------------------------------------------

\newcommand{\Tau}{\mathcal{T}}
\newcommand{\myNewEmptyPage}{\newpage $ $}
\newcommand{\sgn}{\mathop{\mathrm{sgn}}} % segno di funzione
%\renewcommand{\epsilon}{\varepsilon}
%\renewcommand{\theta}{\vartheta}
%\renewcommand{\rho}{\varrho}
%\renewcommand{\phi}{\varphi}
\newcommand{\degree}{^\circ\text{C}} % SIMBOLO GRADI
\newcommand{\notimplies}{\mathrel{{\ooalign{\hidewidth$\not\phantom{=}$\hidewidth\cr$\implies$}}}}
\renewcommand{\qed}{\tag*{$\blacksquare$}}
\newcommand{\latex}{\LaTeX\xspace}
\newcommand{\tex}{\TeX\xspace}
\newcommand{\questeq}{\overset{?}{=}} % è vero che?
\renewcommand{\leqslant}{\leq}
\renewcommand{\geqslant}{\geq}
\newcommand{\complementary}{^{\mathrm{C}}}
\renewcommand{\emptyset}{\varnothing} % simbolo insieme vuoto
\newcommand{\Bot}{\perp \!\!\! \perp} % indipendenza
\newcommand{\boxedText}[1]{\noindent\fbox{\parbox{\textwidth}{#1}}}
\renewcommand{\P}{\mathbb{P}}
\newcommand{\Ind}{\mathds{1}}

\usepackage{mathtools} % Serve per i due comandi dopo
\DeclarePairedDelimiter{\abs}{\lvert}{\rvert} % CREA UN COMANDO abs()
\DeclarePairedDelimiter{\norma}{\lVert}{\rVert} % CREA UN COMANDO norma()

% Simbolo di integrale barrato

\makeatletter
\newcommand*{\mint}[1]{%
  % #1: overlay symbol
  \mint@l{#1}{}%
}
\newcommand*{\mint@l}[2]{%
  % #1: overlay symbol
  % #2: limits
  \@ifnextchar\limits{%
    \mint@l{#1}%
  }{%
    \@ifnextchar\nolimits{%
      \mint@l{#1}%
    }{%
      \@ifnextchar\displaylimits{%
        \mint@l{#1}%
      }{%
        \mint@s{#2}{#1}%
      }%
    }%
  }%
}
\newcommand*{\mint@s}[2]{%
  % #1: limits
  % #2: overlay symbol
  \@ifnextchar_{%
    \mint@sub{#1}{#2}%
  }{%
    \@ifnextchar^{%
      \mint@sup{#1}{#2}%
    }{%
      \mint@{#1}{#2}{}{}%
    }%
  }%
}
\def\mint@sub#1#2_#3{%
  \@ifnextchar^{%
    \mint@sub@sup{#1}{#2}{#3}%
  }{%
    \mint@{#1}{#2}{#3}{}%
  }%
}
\def\mint@sup#1#2^#3{%
  \@ifnextchar_{%
    \mint@sup@sub{#1}{#2}{#3}%
  }{%
    \mint@{#1}{#2}{}{#3}%
  }%
}
\def\mint@sub@sup#1#2#3^#4{%
  \mint@{#1}{#2}{#3}{#4}%
}
\def\mint@sup@sub#1#2#3_#4{%
  \mint@{#1}{#2}{#4}{#3}%
}
\newcommand*{\mint@}[4]{%
  % #1: \limits, \nolimits, \displaylimits
  % #2: overlay symbol: -, =, ...
  % #3: subscript
  % #4: superscript
  \mathop{}%
  \mkern-\thinmuskip
  \mathchoice{%
    \mint@@{#1}{#2}{#3}{#4}%
        \displaystyle\textstyle\scriptstyle
  }{%
    \mint@@{#1}{#2}{#3}{#4}%
        \textstyle\scriptstyle\scriptstyle
  }{%
    \mint@@{#1}{#2}{#3}{#4}%
        \scriptstyle\scriptscriptstyle\scriptscriptstyle
  }{%
    \mint@@{#1}{#2}{#3}{#4}%
        \scriptscriptstyle\scriptscriptstyle\scriptscriptstyle
  }%
  \mkern-\thinmuskip
  \int#1%
  \ifx\\#3\\\else_{#3}\fi
  \ifx\\#4\\\else^{#4}\fi  
}
\newcommand*{\mint@@}[7]{%
  % #1: limits
  % #2: overlay symbol
  % #3: subscript
  % #4: superscript
  % #5: math style
  % #6: math style for overlay symbol
  % #7: math style for subscript/superscript
  \begingroup
    \sbox0{$#5\int\m@th$}%
    \sbox2{$#5\int_{}\m@th$}%
    \dimen2=\wd0 %
    % => \dimen2 = width of \int
    \let\mint@limits=#1\relax
    \ifx\mint@limits\relax
      \sbox4{$#5\int_{\kern1sp}^{\kern1sp}\m@th$}%
      \ifdim\wd4>\wd2 %
        \let\mint@limits=\nolimits
      \else
        \let\mint@limits=\limits
      \fi
    \fi
    \ifx\mint@limits\displaylimits
      \ifx#5\displaystyle
        \let\mint@limits=\limits
      \fi
    \fi
    \ifx\mint@limits\limits
      \sbox0{$#7#3\m@th$}%
      \sbox2{$#7#4\m@th$}%
      \ifdim\wd0>\dimen2 %
        \dimen2=\wd0 %
      \fi
      \ifdim\wd2>\dimen2 %
        \dimen2=\wd2 %
      \fi
    \fi
    \rlap{%
      $#5%
        \vcenter{%
          \hbox to\dimen2{%
            \hss
            $#6{#2}\m@th$%
            \hss
          }%
        }%
      $%
    }%
  \endgroup
}

%----------------------------------------------------------------------------------------
%	APPENDICE
%----------------------------------------------------------------------------------------

\usepackage[toc,page]{appendix}

%----------------------------------------------------------------------------------------
%	SIMBOLI CARTE DA GIOCO
%
% Sono i seguenti:
%   \varheartsuit
%   \vardiamondsuit
%   \clubsuit
%   \spadesuit
%----------------------------------------------------------------------------------------

\DeclareSymbolFont{extraup}{U}{zavm}{m}{n}
\DeclareMathSymbol{\varheartsuit}{\mathalpha}{extraup}{86}
\DeclareMathSymbol{\vardiamondsuit}{\mathalpha}{extraup}{87}


%----------------------------------------------------------------------------------------
%----------------------------------------------------------------------------------------
%----------------------------------------------------------------------------------------
%----------------------------------------------------------------------------------------
%----------------------------------------------------------------------------------------
%----------------------------------------------------------------------------------------
%----------------------------------------------------------------------------------------
%----------------------------------------------------------------------------------------
%----------------------------------------------------------------------------------------
%----------------------------------------------------------------------------------------
%%%% TESTING

\definecolor{grey245}{RGB}{245,245,245}

% questo fa si che le footnote siano "fuori" dall'environment dimostrazione
% \usepackage{footnote}
% \usepackage{etoolbox}
% \BeforeBeginEnvironment{dimostrazione}{\savenotes}
% \AfterEndEnvironment{dimostrazione}{\spewnotes}
% \BeforeBeginEnvironment{theorem}{\savenotes}
% \AfterEndEnvironment{theorem}{\spewnotes}

\newtheoremstyle{blacknumbox} % Theorem style name
{0pt}% Space above
{0pt}% Space below
{\normalfont}% Body font
{}% Indent amount
{\bf\scshape}% Theorem head font --- {\small\bf}
{.\;}% Punctuation after theorem head
{0.25em}% Space after theorem head
{\small\thmname{#1}\nobreakspace\thmnumber{\@ifnotempty{#1}{}\@upn{#2}}% Theorem text (e.g. Theorem 2.1)
%{\small\thmname{#1}% Theorem text (e.g. Theorem)
\thmnote{\nobreakspace\the\thm@notefont\normalfont\bfseries---\nobreakspace#3}}% Optional theorem note

\newtheoremstyle{unnumbered} % Theorem style name
{0pt}% Space above
{0pt}% Space below
{\normalfont}% Body font
{}% Indent amount
{\bf\scshape}% Theorem head font --- {\small\bf}
{.\;}% Punctuation after theorem head
{0.25em}% Space after theorem head
{\small\thmname{#1}\thmnumber{\@ifnotempty{#1}{}\@upn{#2}}% Theorem text (e.g. Theorem 2.1)
%{\small\thmname{#1}% Theorem text (e.g. Theorem)
\thmnote{\nobreakspace\the\thm@notefont\normalfont\bfseries---\nobreakspace#3}}% Optional theorem note


\newcounter{dummy} 
\numberwithin{dummy}{chapter}


\theoremstyle{blacknumbox}
\newtheorem{definitionT}[dummy]{Definizione}
\newtheorem{theoremT}[dummy]{Teorema}
\newtheorem{corollarioT}[dummy]{Corollario}
\newtheorem{lemmaT}[dummy]{Lemma}

% Per gli unnumbered tolgo il \nobreakspace subito dopo {\small\thmname{#1} perché altrimenti c'è uno spazio tra Teorema e il ".", lo spazio lo voglio solo se sono numerati per distanziare Teorema e "(2.1)"
\theoremstyle{unnumbered}
\newtheorem*{NBT}{NB}
\newtheorem*{ossT}{Osservazione}
\newtheorem*{ricalgT}{Richiamo di Algebra}
\newtheorem*{pseudocodiceT}{Pseudocodice}
\newtheorem*{dimT}{Dimostrazione}

\RequirePackage[framemethod=default]{mdframed} % Required for creating the theorem, definition, exercise and corollary boxes

\usepackage{lipsum}



% orange box
\newmdenv[skipabove=7pt,
skipbelow=7pt,
rightline=false,
leftline=true,
topline=false,
bottomline=false,
linecolor=orange,
backgroundcolor=orange!5,
innerleftmargin=5pt,
innerrightmargin=5pt,
innertopmargin=5pt,
leftmargin=0cm,
rightmargin=0cm,
linewidth=2pt,
innerbottommargin=5pt]{oBox}

% green box
\newmdenv[skipabove=7pt,
skipbelow=7pt,
rightline=false,
leftline=true,
topline=false,
bottomline=false,
linecolor=green,
backgroundcolor=green!5,
innerleftmargin=5pt,
innerrightmargin=5pt,
innertopmargin=5pt,
leftmargin=0cm,
rightmargin=0cm,
linewidth=2pt,
innerbottommargin=5pt]{gBox}

% blue box
\newmdenv[skipabove=7pt,
skipbelow=7pt,
rightline=false,
leftline=true,
topline=false,
bottomline=false,
linecolor=blue,
backgroundcolor=blue!5,
innerleftmargin=5pt,
innerrightmargin=5pt,
innertopmargin=5pt,
leftmargin=0cm,
rightmargin=0cm,
linewidth=2pt,
innerbottommargin=5pt]{bBox}

% dim box
\newmdenv[skipabove=7pt,
skipbelow=7pt,
rightline=false,
leftline=true,
topline=false,
bottomline=false,
linecolor=black,
backgroundcolor=grey245!0,
innerleftmargin=5pt,
innerrightmargin=5pt,
innertopmargin=5pt,
leftmargin=0cm,
rightmargin=0cm,
linewidth=2pt,
innerbottommargin=5pt]{dimBox}

\newenvironment{theorem}{\begin{gBox}\begin{theoremT}}{\end{theoremT}\end{gBox}}	
\newenvironment{definition}{\begin{bBox}\begin{definitionT}}{\end{definitionT}\end{bBox}}	
\newenvironment{nb}{\begin{oBox}\begin{NBT}}{\end{NBT}\end{oBox}}	
\newenvironment{oss}{\begin{oBox}\begin{ossT}}{\end{ossT}\end{oBox}}
\newenvironment{ricalg}{\begin{oBox}\begin{ricalgT}}{\end{ricalgT}\end{oBox}}
\newenvironment{pseudocode}{\begin{oBox}\begin{pseudocodiceT}}{\end{pseudocodiceT}\end{oBox}}
\newenvironment{corollario}{\begin{oBox}\begin{corollarioT}}{\end{corollarioT}\end{oBox}}
\newenvironment{lemma}{\begin{oBox}\begin{lemmaT}}{\end{lemmaT}\end{oBox}}
\newenvironment{dimostrazione}{\begin{dimBox}\begin{dimT}}{\[\qed\]\end{dimT}\end{dimBox}}

%----------------------------------------------------------------------------------------
%	INDICE INIZIALE
%----------------------------------------------------------------------------------------

\setcounter{secnumdepth}{3} % DI DEFAULT LE SUBSUBSECTION NON SONO NUMERATE, COSÌ SÌ
\setcounter{tocdepth}{2} % FISSA LA PROFONDITÀ DELLE COSE MOSTRATE NELL'INDICE

\usepackage[hidelinks]{hyperref} % Rende l'indice interattivo e hidelinks nasconde il bordo rosso dai riferimenti

%% Questo fa si che la numerazione dei capitoli venga resettata se si inizia una nuova "Parte"
% \makeatletter
% \@addtoreset{chapter}{part}
% \makeatother 

\newcommand{\nocontentsline}[3]{} % QUESTO COMANDO E QUELLO DOPO SERVONO PER AVERE IL COMANDO \tocless DA METTERE PRIMA DI UNA SEZIONE CHE NON VOGLIO FAR APPARIRE NELL'INDICE
\newcommand{\tocless}[2]{\bgroup\let\addcontentsline=\nocontentsline#1{#2}\egroup}



%%%%%%%%%%%%%%%%%%%%%%%%%%%%%%%%%%%%%%%%%
% Template Eserciziario
%
% Autore:
% Teo Bucci
%
%%%%%%%%%%%%%%%%%%%%%%%%%%%%%%%%%%%%%%%%%

%% \input{../template_dispense.tex}

%----------------------------------------------------------------------------------------
%	SETUP ESERCIZI / SOLUZIONI
%----------------------------------------------------------------------------------------

\usepackage{titlesec}
\usepackage{titletoc}
\usepackage{etoolbox}

\renewcommand{\thesection}{\arabic{section}}
\renewcommand{\thesubsection}{\arabic{subsection}}
\renewcommand{\thechapter}{\arabic{chapter}}

\titleformat{\section}{\normalfont\LARGE\bfseries}{}{0em}{} % \LARGE anziche' \Large
\titleformat{\subsection}{\normalfont\large\bfseries}{}{0em}{}

\newcommand{\Esercizio}[1]{\subsection{Esercizio~\thechapter.\thesubsection~#1}}
\newcommand{\Soluzione}{\subsection{Soluzione~\thechapter.\thesubsection}}

\newcommand{\ParteEsercizi}{\section{Esercizi}}
\newcommand{\ParteSoluzioni}{\section{Soluzioni}}

% MI ASSICURO QUESTA COSA IMPORTANTE

\setcounter{secnumdepth}{2}
\setcounter{tocdepth}{1}

\usepackage{stackengine} % per avere \addstackgap[5pt]{tabular}


\usepackage{chngcntr}
\counterwithout{figure}{chapter}
\counterwithout{equation}{chapter}

%\allowdisplaybreaks[4] % Consente di rompere equazioni su più pagine

\begin{document}




%%%%%%%%%%%%%%%%%%%%%%%%%%%%%%%%%%%%%%%%%%%%%%%
%%%%%%%%%%%%%%%%%%%%%%%%%%%%%%%%%%%%%%%%%%%%%%%



% COPERTINA

\pagestyle{empty} % SWITCHA PER NON AVERE NUMERO PAGINA
\vspace*{\fill}
%Teo Bucci \\
\begin{center}
	{\large \textsc{Esercitazioni di}}\\
	\vspace*{0.4cm}
	{\Huge \textsc{Probabilità}}
\end{center}
\vspace*{\fill}
\newpage


%%%%%%%%%%%%%%%%%%%%%%%%%%%%%%%%%%%%%%%%%%%%%%%
%%%%%%%%%%%%%%%%%%%%%%%%%%%%%%%%%%%%%%%%%%%%%%%



%%%%%%%%%%%%%%%%%%%%%%%%%%%%%%%%%%%%%%%%%%%%%%%
%%%%%%%%%%%%%%%%%%%%%%%%%%%%%%%%%%%%%%%%%%%%%%%




%\thispagestyle{empty}
%\clearpage
%
%{\Large \textit{Appunti di Matematica Numerica}}
%
%\vspace*{\fill}
%
%\textcopyright \ Gli autori, tutti i diritti riservati
%
%Sono proibite tutte le riproduzioni senza autorizzazione %scritta degli autori.
%
%Revisione del \today
%
%Developed by Teo Bucci \\
%Compiled with \ensuremath\heartsuit \\
%
%\textbf{Prefazione}
%
%% PREFAZIONE
%
%Per segnalare eventuali errori o suggerimenti potete %contattare gli autori.
%
%\newpage
%\thispagestyle{empty} % PER NON AVERE COSE NELL'HEADER
%\clearpage % PER TERMINARE SUBITO LA PAGINA DOPO L'INDICE



%%%%%%%%%%%%%%%%%%%%%%%%%%%%%%%%%%%%%%%%%%%%%%%
%%%%%%%%%%%%%%%%%%%%%%%%%%%%%%%%%%%%%%%%%%%%%%%







% INDICE
\addtocontents{toc}{\protect\thispagestyle{empty}}
\tableofcontents
\newpage
%\thispagestyle{empty} % PER NON AVERE COSE NELL'HEADER
%\clearpage % PER TERMINARE SUBITO LA PAGINA DOPO L'INDICE


% PAGINA VUOTA
%$ $\\
%\newpage
%\pagestyle{fancy} % RISWITCHA PER RIAVERE IL NUMERO PAGINA
\setcounter{page}{1} % FA RIPARTIRE IL CONTATORE PAGINA DA 1
%\newpage



%%%%%%%%%%%%%%%%%%%%%%%%%%%%%%%%%%%%%%%%%%%%%%%
%%%%%%%%%%%%%%%%%%%%%%%%%%%%%%%%%%%%%%%%%%%%%%%
%%%%%%%%%%%%%%%%%%%%%%%%%%%%%%%%%%%%%%%%%%%%%%%
%%%%%%%%%%%%%%%%%%%%%%%%%%%%%%%%%%%%%%%%%%%%%%%

%           TEMPLATE


%           \chapter{Spazi campionari}
%           
%           \ParteEsercizi
%           
%           \Esercizio{(Nome di un esercizio speciale)}
%           \Esercizio{}
%           \Esercizio{}
%           
%           \ParteSoluzioni
%           
%           \Soluzione
%           \Soluzione
%           \Soluzione
%           
%           
%           \chapter{Indipendenza}
%           
%           \ParteEsercizi
%           
%           \Esercizio{}
%           \begin{enumerate}
%               \item ine
%               \item oh
%               \item $\!\!\!\!{^*}$ jhk b
%           \end{enumerate}
%           \Esercizio{$*$}
%           \Esercizio{}
%           
%           \ParteSoluzioni
%           
%           \Soluzione
%           \Soluzione
%           \Soluzione


\chapter{Spazi campionari, sigma-algebre, probabilità}
%!TEX root = ../main.tex
% aspell --mode=tex --lang=it check 01-spazi_campionari_sigma-algebre_probabilita.tex

\ParteEsercizi

\Esercizio{}

Determinare un adeguato spazio campionario per ciascuno dei seguenti esperimenti aleatori.
\begin{enumerate}
\item Il lancio di una moneta ripetuto $n$ volte.
\item Il lancio simultaneo di $n$ monete tutte uguali tra loro.
\item Il lancio di una moneta al minuto, senza mai fermarsi.
\item L'estrazione di una carta da un mazzo di $10$ e lancio di una moneta.
\item L'estrazione dei cinque numeri vincenti sulla ruota di Milano del Gioco del Lotto, ossia estrazione casuale di cinque tra i numeri naturali da $1$ a $90$.
\item L'estrazione, una dopo l'altra, delle prime $20$ carte di un mazzo da $40$.
\item La misurazione della frazione di tempo impiegata per risolvere il cubo di Rubik, rispetto a un tempo massimo assegnato e partendo da una configurazione casuale.
\end{enumerate}

\Esercizio{}

Una moneta viene lanciata due volte. Antonio vince se al primo lancio esce testa mentre Benedetto vince se al secondo lancio esce croce.
\begin{enumerate}
\item Determinare il più piccolo spazio campionario che descrive tutti i possibili esiti dell'esperimento.
\item Descrivere in termini dei sottoinsiemi dello spazio campionario i seguenti eventi:
\begin{enumerate}
\item Antonio vince,
\item Benedetto vince,
\item Antonio non vince,
\item Benedetto non vince,
\item Antonio e Benedetto vincono entrambi,
\item vince Antonio ma non Benedetto,
\item vince Benedetto ma non Antonio,
\item almeno uno dei due vince,
\item nessuno dei due vince,
\item vince soltanto uno dei due,
\item esce testa al primo lancio ed esce croce al primo lancio,
\item esce testa o croce al secondo lancio.
\end{enumerate}
\end{enumerate}
\Esercizio{}

Sia $( \Omega ,\mathcal{A})$ uno spazio di probabilità e siano $A$, $B$ e $C$ tre eventi appartenenti ad $\mathcal{A}$. Esprimere attraverso operazioni insiemistiche i seguenti eventi associati ad $A$, $B$ e $C$:
\begin{enumerate}
\item almeno un evento si verifica,
\item al più un evento si verifica,
\item nessun evento si verifica,
\item tutti gli eventi si verificano,
\item si verifica esattamente un evento,
\item due eventi su tre si verificano.
\end{enumerate}

Tradurre in relazioni insiemistiche le seguenti relazioni fra eventi:
\begin{enumerate}
\item $A$ implica $B$,
\item $A$ e $C$ si escludono a vicenda,
\item almeno un evento fra $B$ e $C$ si verifica certamente.
\end{enumerate}

Si costruisca un esempio (o anche più di uno!) di esperimento aleatorio, avendo cura di descrivere gli eventi $A$, $B$ e $C$ e gli eventi considerati nei punti precedenti.
\Esercizio{}

Sia $( \Omega ,\mathcal{A})$ uno spazio di probabilità e siano $A$, $B$, $C$ e $D$ quattro eventi appartenenti ad $\mathcal{A}$. Esprimere attraverso operazioni insiemistiche i seguenti eventi associati ad $A$, $B$, $C$ e $D$.
\begin{enumerate}
\item Esattamente tre eventi su quattro si verificano.
\item Si verifica solo $C$.
\item Si verifica solo $C$ oppure si verifica solo $D$.
\item Almeno un evento si verifica.
\end{enumerate}
\Esercizio{}

Si osservano i risultati del lancio di una moneta e di un dado.
\begin{enumerate}
\item Determinare uno spazio campionario $\Omega $ che descriva tutti gli esiti dell'esperimento.
\item Determinare la $\sigma $-algebra che descrive gli eventi verificabili a fine esperimento.
\end{enumerate}

Restringere il nostro interesse alla sola moneta, significa considerare solo alcuni degli eventi descritti dalla $\sigma $-algebra del punto $2$.
\begin{enumerate}
\item Di quali eventi si tratta? Formano una $\sigma $-algebra?
\item Nel caso tali eventi non formino una $\sigma $-algebra, si descriva la più piccola $\sigma $-algebra generata da essi.
\end{enumerate}
\Esercizio{}

Per $k=1,2,\mathbb{N}$, si consideri $\Omega =\{0,1\}^{k}$ lo spazio degli esiti di $1$, $2$ o infinite prove di Bernoulli. In tutti e tre i casi, si consideri l'evento $E=$"successo alla prima prova", e si descriva esplicitamente la $\sigma $-algebra $\mathcal{A} =\sigma ( E)$.
\Esercizio{}

Si consideri $\Omega =\{0,1\}^{\mathbb{N}}$, lo spazio degli esiti di infinite prove di Bernoulli. Per $k=1,2,\dotsc $, si considerino gli eventi:
\begin{center}
$E_{k} =$ successo alla prova $k$,
\end{center}
e si consideri la $\sigma $-algebra $\mathcal{A} =\sigma ( E_{k} \mid k=1,2,\dotsc )$. Si individuino i seguenti eventi e se ne mostri l'appartenenza ad $\mathcal{A}$:
\begin{enumerate}
\item solo insuccessi,
\item solo la terza prova dà un successo,
\item nelle prove pari ci sono solo successi,
\item solo successi da una qualche prova in poi,
\item infiniti successi,
\item solo un numero finito di successi,
\item solo un successo.
\end{enumerate}
\Esercizio{}

Si consideri $\Omega =\{1,\dotsc ,6\}^{k}$, lo spazio degli esiti di infiniti lanci di un dado. Per $l=1,\dotsc ,6$ e $k=1,2,\dotsc $ si considerino gli eventi.
\begin{center}
$E_{k}^{l} =$ faccia $l$ al lancio $k$,
\end{center}
e si consideri la $\sigma $-algebra $\mathcal{A} =\sigma \left( E_{k}^{l} \mid l=1,\dotsc ,6,\ k=1,2,\dotsc \right)$. Si individuino i seguenti eventi e se ne mostri l'appartenenza ad $\mathcal{A}$:
\begin{enumerate}
\item esce sempre $1$ nei primi n lanci,
\item esce sempre $5$,
\item solo il tredicesimo lancio dà un $3$,
\item solo risultati dispari nei lanci dispari e risultati pari nei lanci pari,
\item sempre $3$ dal $33$-esimo lancio in poi,
\item sempre $3$ da un qualche lancio in poi,
\item sempre risultati dispari da un qualche lancio in poi,
\item infiniti $6$,
\item solo un numero finito di $2$,
\item solo un $4$.
\end{enumerate}
\Esercizio{}

Chiara e Marco acquistano assieme uno dei $50$ biglietti di una pesca di beneficenza. Ci sono $50$ premi di cui $7$ piacciono a Chiara, $5$ a Marco e $1$ solo ad entrambi. Uno di questi premi sarà casualmente associato al loro biglietto.
\begin{enumerate}
\item Determinare il più piccolo spazio campionario che descrive i possibili esiti della pesca di beneficenza.
\item Descrivere in termini di sottoinsiemi dello spazio campionario gli eventi:
\begin{enumerate}
\item il premio piacerà a Chiara,
\item il premio piacerà a Marco,
\item il premio piacerà a entrambi,
\item il premio piacerà ad almeno uno dei due,
\item a nessuno dei due piacerà il premio,
\item il premio piacerà a uno solo dei due.
\end{enumerate}
\item Dopo aver introdotto un opportuno spazio di probabilità per descrivere questo esperimento aleatorio, valutare la probabilità di tali eventi.
\end{enumerate}
\Esercizio{}

Una ditta riceve richieste di forniture, che possono essere urgenti oppure no, e richiedere la consegna in città oppure fuori città. Per una data richiesta è noto che:
\begin{itemize}
\item la probabilità che sia una consegna fuori città è $0.4$,
\item la probabilità che sia una consegna urgente è $0.3$,
\item la probabilità che sia una consegna non urgente in città è $0.4$.
\end{itemize}

Dopo aver introdotto un opportuno spazio di probabilità per descrivere questo esperimento aleatorio, calcolare:
\begin{enumerate}
\item la probabilità che sia una consegna urgente fuori città,
\item la probabilità che sia una consegna urgente in città.
\end{enumerate}
\Esercizio{}

Dato uno spazio di probabilità $( \Omega ,\mathcal{A} ,\P)$, si mostri che:
\begin{enumerate}
\item se un evento $A$ è quasi certo, ovvero $\P( A) =1$, allora $\P( A\cap B) =\P( B)$ per ogni $B\in \mathcal{A}$;
\item se un evento $A$ implica un evento $B$, ovvero $A\subset B$, allora $\P( A) \leq \P( B)$.
\end{enumerate}
\Esercizio{}

Sia $( \Omega ,\mathcal{A} ,\P)$ uno spazio di probabilità e siano $A$ e $B$ due eventi appartenenti ad $\mathcal{A}$, con probabilità $\P( A) =0.4$ e $\P( B) =0.7$, rispettivamente. Date le seguenti affermazioni dire quali sono certamente false, quali sono sempre vere, quali possono essere vere o false:
\begin{enumerate}
\item $\P( A\cup B) =0.4$,
\item $\P( A\cup B) =0.7$,
\item $\P( A\cup B) \geq 0.7$,
\item $\P( A\cup B) =1.1$,
\item $\P( A\cap B) =0.28$,
\item $\P\left( B\cap A\comp\right) \geq 0.3$,
\item $\P\left( A\cap B\comp\right) \leq 0.3$.
\end{enumerate}
\Esercizio{}

Si consideri $\Omega =\{0,1\}^{\mathbb{N}}$, lo spazio degli esiti di infinite prove di Bernoulli. Per $k=1,2,\dotsc $ si considerino gli eventi
\begin{center}
$E_{k} =$successo alla prova $k$,
\end{center}
e si consideri la $\sigma $-algebra $\mathcal{A} =\sigma ( E_{k} \mid k=1,2,\dotsc )$. Per ogni $n=1,2,\dotsc $ si consideri quindi
\begin{equation*}
A_{n} =\bigcap _{k=1}^{n} E_{k}
\end{equation*}
\begin{enumerate}
\item Si mostri che ogni $A_{n}$ è un evento in $A$.
\item Si dia l'interpretazione probabilistica degli eventi $A_{n}$.
\item Si mostri che $A_{n} \supset A_{n+1}$ per ogni $n$.
\item Si determini $\bigcap _{n} A_{n}$, si mostri che appartiene ad $A$ e se ne dia l'interpretazione probabilistica.
\item Si supponga ora che $\P( A_{n}) =\frac{1}{2^{n}}$ per ogni $n$. Quanto vale $\P\left(\bigcap\nolimits _{n} A_{n}\right)$?
\end{enumerate}
\Esercizio{}

Sia $\Omega =[ 0,1]$ e $\mathcal{A} :=\mathcal{B}([ 0,1])$ la $\sigma $-algebra di Borel. Si considerino, per ogni $n\in \mathbb{N}$, gli insiemi $A_{n} :=\left[ 0,\frac{1}{n}\right] \subset \Omega $.
\begin{enumerate}
\item Determinare gli insiemi $I$ e $U$, definiti come segue:\begin{equation*}
I:=\bigcap _{n\in \mathbb{N}} A_{n} \ \ \ \ \ \ \ \ U:=\bigcup _{n\in \mathbb{N}} A_{n} .
\end{equation*}
\item Si supponga che $\P( A_{n}) =\frac{1}{n}$ per ogni $n$. Quanto valgono $\P( I)$ e $\P( U)$?
\end{enumerate}
\Esercizio{$\star$}

Sia $( \Omega ,\mathcal{A} ,\P)$ uno spazio di probabilità e sia $A_{n}$ con $n\in \mathbb{N}$ una successione di eventi.
\begin{enumerate}
\item Si mostri che se $\P( A_{n}) =0$ per ogni $n\in \mathbb{N}$ allora $\P\left(\bigcup\limits _{n\in \mathbb{N}} A_{n}\right) =0$.
\item Si mostri che se $\P( A_{n}) =1$ per ogni $n\in \mathbb{N}$ allora $\P\left(\bigcap\limits _{n\in \mathbb{N}} A_{n}\right) =1$.
\end{enumerate}
\Esercizio{$\star$}

Dato $( \Omega ,\mathcal{A} ,\P)$ spazio di probabilità, data $A_{n}$ successione di eventi, si provi che
\begin{equation*}
\P\left(\liminf _{n} A_{n}\right) \leq \liminf _{n}\P( A_{n}) \leq \limsup _{n}\P( A_{n})\leq \mathbb{P}\left(\limsup _{n} A_{n}\right) .
\end{equation*}
Si deduca che se $A_{n}\rightarrow A$, ovvero $\liminf\nolimits _{n} A_{n} =\limsup _{n} A_{n} =A$, allora $\P( A_{n})\rightarrow \P( A)$.



\ParteSoluzioni










\Soluzione

\textbf{NB.} La scelta di $\Omega $ non è unica!
\begin{enumerate}
\item $\Omega =\{0,1\}^{n} .$
\item $\Omega =\{0,\dotsc ,n\}$.
\item $\Omega =\{0,1\}^{\mathbb{N}}$.
\item $\Omega =\{1,2,3,4,5,6,7,8,9,10\} \times \{t,c\}$.
\item Lo spazio delle combinazioni di $90$ elementi di classe $5$, ossia\begin{equation*}
\Omega =\{\omega \subseteq \{1,\dotsc ,90\} \mid |\omega |=5\}
\end{equation*}
\item Lo spazio delle disposizioni di $40$ elementi di classe $20$, ossia\begin{equation*}
\Omega =\{\omega =( \omega _{1} ,\dotsc ,\omega _{20}) \mid \omega _{i} \in \{1,\dotsc ,40\} \ \forall \ i,\ \omega _{i} \neq \omega _{j} \ \forall \ i\neq j\}
\end{equation*}
\item $\Omega =[ 0,1]$.
\end{enumerate}
\Soluzione

Indichiamo la coppia (esito del primo lancio, esito del secondo lancio). Poniamo inoltre
\begin{itemize}
\item $A=$ Antonio vince
\item $B=$ Benedetto vince
\end{itemize}

Vediamo la soluzione dell'esercizio.
\begin{enumerate}
\item $\Omega =\{( t,t) ,( t,c) ,( c,t) ,( c,c)\}$
\item 
\begin{enumerate}
\item $A=\{( t,t) ,( t,c)\}$
\item $B=\{( t,c) ,( c,c)\}$
\item $A\comp$
\item $B\comp$
\item $A\cap B=\{( t,c)\}$
\item $A\cap B\comp$
\item $A\comp \cap B$
\item $A\cup B$
\item $A\comp \cap B\comp =\{( c,t)\}$
\item $\left( A\cap B\comp\right) \cup \left( A\comp \cap B\right) =\{( t,t) ,( c,c)\}$
\item $\emptyset $
\item $\Omega $
\end{enumerate}
\end{enumerate}
\Soluzione
\begin{enumerate}
\item $A\cup B\cup C$
\item $\left( A\cap B\comp \cap C\comp\right) \cup \left( A\comp \cap B\cap C\comp\right) \cup \left( A\comp \cap B\comp \cap C\right) \cup \left( A\comp \cap B\comp \cap C\comp\right)$
\item $\left( A\comp \cap B\comp \cap C\comp\right)$
\item $( A\cap B\cap C)$
\item $\left( A\cap B\comp \cap C\comp\right) \cup \left( A\comp \cap B\cap C\comp\right) \cup \left( A\comp \cap B\comp \cap C\right)$
\item $\left( A\comp \cap B\cap C\right) \cup \left( A\cap B\comp \cap C\right) \cup \left( A\cap B\cap C\comp\right)$
\item $A\subset B$
\item $A\cap C=\emptyset $
\item $B\cup C=\Omega $
\end{enumerate}
\Soluzione
\begin{enumerate}
\item $\left( A\cap B\cap C\cap D\comp\right) \cup \left( A\cap B\cap C\comp \cap D\right) \cup \left( A\cap B\comp \cap C\cap D\right) \cup \left( A\comp \cap B\cap C\cap D\right)$
\item $A\comp \cap B\comp \cap C\cap D\comp$
\item $\left( A\comp \cap B\comp \cap C\cap D\comp\right) \cup \left( A\comp \cap B\comp \cap C\comp \cap D\right)$
\item $A\cup B\cup C\cup D$
\end{enumerate}
\Soluzione
\begin{enumerate}
\item $\Omega =\{t,c\} \times \{1,\dotsc ,6\}$
\item $A=2^{\Omega }$
\item Gli eventi $T=\{( t,k) \mid k=1,\dotsc ,6\}$ e $C=\{( c,k) \mid k=1,\dotsc ,6\}$. Non formano una $\sigma $-algebra.
\item $\sigma ( T,C) =\{T,C,\emptyset ,\Omega \}$.
\end{enumerate}
\Soluzione

$A=\sigma ( E) =\left\{E,E\comp ,\emptyset ,\Omega \right\}$.
\Soluzione
\begin{enumerate}
\item $\bigcup\limits _{k=1}^{\infty } E_{k}\comp$. Vediamo l'appartenenza ad $\mathcal{A}$: $E_{k} \in \mathcal{A} \forall k\implies  E_{k}\comp \in \mathcal{A} \forall k\implies  \bigcap\limits _{k=1}^{\infty } E_{k}\comp \in \mathcal{A}$
\item $E_{3} \cap \bigcap\limits _{k\neq 3} E_{k}\comp$
\item $\bigcap\limits _{k=1}^{\infty } E_{2k}$
\item $\bigcup\limits _{n=1}^{\infty }\bigcap\limits _{m\geq n} E_{m} =:\liminf _{n} E_{n}$

\begin{oss}
\begin{equation*}
\begin{aligned}
\omega \in \liminf _{n} E_{n} & \ \ \iff \ \ \omega \in \bigcup _{n=1}^{\infty }\bigcap _{m\geq n} E_{m}\\
 & \ \ \iff \ \ \exists \overline{n} \in \mathbb{N} :\ \omega \in \bigcap _{m\geq \overline{n}} E_{m}\\
 & \ \ \iff \ \ \exists \overline{n} \in \mathbb{N} :\ \omega \in E_{m} \forall m\geq \overline{n}
\end{aligned}
\end{equation*}
i.e. $\omega $ sta in tutti gli $E_{m}$ tranne al più un numero finito di essi.

i.e. ho successi in tutte le prove al più un numero finito di esse.
\end{oss}
\item $\bigcap\limits _{n=1}^{\infty }\bigcup\limits _{m\geq n} E_{m} =:\limsup _{n} E_{n}$

\begin{oss}
\begin{equation*}
\begin{aligned}
\omega \in \limsup _{n} E_{n} & \ \ \iff \ \ \omega \in \bigcap _{n=1}^{\infty }\bigcup _{m\geq n} E_{m}\\
 & \ \ \iff \ \ \forall n\in \mathbb{N} ,\ \omega \in \bigcup _{m\geq n} E_{m}\\
 & \ \ \iff \ \ \forall n\in \mathbb{N} \ \ \exists \overline{m} :\ \omega \in E_{\overline{m}}
\end{aligned}
\end{equation*}
i.e. $\omega $ sta in $E_{m}$ per infiniti $n\in \mathbb{N}$.

i.e. ho infiniti successi.
\end{oss}
\item $\left(\liminf\limits _{n} E_{n}\right)\comp =\underbrace{\left(\bigcup\limits _{n=1}^{\infty }\bigcap\limits _{m\geq n} E_{m}\right)\comp =\bigcap\limits _{n=1}^{\infty }\bigcup\limits _{m\geq n} E_{m}\comp}_{\text{Leggi di De Morgan}} =:\limsup\limits _{n} E_{n}\comp$
\item $\bigcup\limits _{l\in \mathbb{N}}\left( E_{l} \cap \bigcap\limits _{k\neq l} E_{k}\comp\right)$
\end{enumerate}
\Soluzione

Sia $D_{k} =E_{k}^{1} \cup E_{k}^{3} \cup E_{k}^{5} \in \mathcal{A}$ e sia $P_{k} =E_{k}^{2} \cup E_{k}^{4} \cup E_{k}^{6} =D_{k}\comp \in \mathcal{A}$.
\begin{enumerate}
\item $\bigcap\limits _{k=1}^{n} E_{k}^{1} \in \mathcal{A}$,
\item $\bigcap\limits _{k\geq 1} E_{k}^{5} \in \mathcal{A}$,
\item $E_{13}^{3} \cap \left(\bigcap\limits _{k\neq 13}\left[\left( E_{k}^{3}\right)\comp\right]\right) \in \mathcal{A}$,
\item $\left(\bigcap\limits _{k\geq 0} D_{2k+1}\right) \cap \left(\bigcap\limits _{k\geq 1} P_{2k}\right) \in \mathcal{A}$,
\item $\bigcap\limits _{k\geq 33} E_{k}^{3} \in \mathcal{A}$,
\item $\liminf _{k} E_{k}^{3} \in \mathcal{A}$,
\item $\liminf _{k} D_{k} \in \mathcal{A}$,
\item $\limsup _{k} E_{k}^{6} \in \mathcal{A}$,
\item $\left(\liminf _{k}\left( E_{k}^{2}\right)\right)\comp \in \mathcal{A}$,
\item $\bigcup _{n}\left( E_{n}^{4} \cap \left(\bigcap _{k\neq n}\left( E_{k}^{4}\right)\comp\right)\right) \in \mathcal{A}$.
\end{enumerate}
\Soluzione
\begin{enumerate}
\item Ricordiamo che lo spazio campionario $\Omega $ è definito come l'insieme dei possibili esiti (elementari). Intuitivamente:
\begin{enumerate}
\item se conduciamo un dato esperimento, $\Omega $ è l'insieme di tutti i possibili esiti di tale esperimento
\item condurre un esperimento si traduce quindi nell'estrarre un $\omega \in \Omega $.
\end{enumerate}

Inoltre, $\Omega $ non è unico, al più è unico l'$\Omega $ \textit{minimale}, ricordando che \textit{minimale}, e quindi \textit{elementare}, dipende da ciò che si stabilisce di voler controllare e descrivere. In questo caso l'esperimento che si fa deve tener conto di tutti i possibili esiti della pesca di beneficenza, i.e. $50$ possibili diversi premi. Si ha $\Omega =\{1,\dotsc ,50\} =$esiti elementari per Marco e Chiara.
\item Ricordiamo che:
\begin{enumerate}
\item un \textit{evento} è un fatto per il quale a \textit{fine} esperimento si può dire se si è verificato o no.
\item nel paradigma delle estrazioni da $\Omega $, un evento si verifica se pesco certi $\omega $, non si verifica se pesco altri $\omega $.
\end{enumerate}

Chiamo $C,M$ i seguenti eventi
\begin{enumerate}
\item $C=$il premio piace a Chiara$=\{1,\dotsc ,7\}$
\item $M=$il premio piace a Marco$=\{7,\dotsc ,11\}$ (ci deve essere un elemento comune)
\end{enumerate}

\begin{oss}
Non è necessario introdurre esplicitamente $\Omega $, possiamo fare tutto in maniera astratta. Le informazioni rilevanti sono le seguenti:\begin{equation*}
( \Omega ,\mathcal{A}) \ \ \ \ | \Omega | =50\ \ \ \ C,M\ \ \ \ \text{tali che} \ \ \ \ C,M\in \mathcal{A} ,| C| =7,| M| =5
\end{equation*}
\end{oss}
\begin{enumerate}
\item $C$
\item $M$
\item $C\cap M$
\item $C\cup M$
\item $( C\cup M)\comp$
\item $( C\cup M) \smallsetminus ( C\cap M)$
\end{enumerate}



\tikzset{every picture/.style={line width=0.75pt}} %set default line width to 0.75pt        

\begin{tikzpicture}[x=0.75pt,y=0.75pt,yscale=-1,xscale=1]
%uncomment if require: \path (0,125); %set diagram left start at 0, and has height of 125

%Shape: Circle [id:dp3285715962328033] 
\draw  [fill={rgb, 255:red, 155; green, 155; blue, 155 }  ,fill opacity=0.3 ] (200,60) .. controls (200,37.91) and (217.91,20) .. (240,20) .. controls (262.09,20) and (280,37.91) .. (280,60) .. controls (280,82.09) and (262.09,100) .. (240,100) .. controls (217.91,100) and (200,82.09) .. (200,60) -- cycle ;
%Shape: Circle [id:dp15381161335081517] 
\draw  [fill={rgb, 255:red, 155; green, 155; blue, 155 }  ,fill opacity=0.3 ] (260,60) .. controls (260,37.91) and (277.91,20) .. (300,20) .. controls (322.09,20) and (340,37.91) .. (340,60) .. controls (340,82.09) and (322.09,100) .. (300,100) .. controls (277.91,100) and (260,82.09) .. (260,60) -- cycle ;

% Text Node
\draw (184,22.4) node [anchor=north west][inner sep=0.75pt]    {$C$};
% Text Node
\draw (341,22.4) node [anchor=north west][inner sep=0.75pt]    {$M$};


\end{tikzpicture}


Per rispondere era necessario ricordare che le operazioni logiche tra insiemi si traducono in operazioni insiemistiche:\begin{align*}
\emptyset  & =\text{evento impossibile}\\
\Omega  & =\text{evento certo}\\
A\comp & =\text{contrario di} \ A\\
A\cup B & \iff A\ \text{oppure} \ B\\
A\cap B & \iff A\ \text{e} \ B\\
A\cap B=\emptyset  & \iff A\ \text{e} \ B\ \text{incompatibili}\\
A\subseteq B & \iff A\ \text{implica} \ B
\end{align*}
\item In questo caso l'insieme degli eventi $\Omega $ è un insieme discreto (finito). La probabilità è uniforme: gli eventi equiprobabili a priori. Pertanto si ha\begin{equation*}
\P( A) =\frac{| A| }{| \Omega | } =\frac{\text{casi favorevoli}}{\text{casi possibili}}
\end{equation*}

Allora
\begin{enumerate}
\item $\P( C) =\frac{| C| }{| \Omega | } =\frac{7}{50} =0.14$
\item $\P( M) =\frac{| M| }{| \Omega | } =\frac{5}{50} =0.1$
\item $\P( C\cap M) =\frac{1}{50} =0.02$
\item $\P( C\cup M) =\frac{11}{50} =0.22$
\item $\P\left(( C\cup M)\comp\right) =1-0.22=0.78$
\item Procediamo come segue

$\begin{aligned}
\P(( C\cup M) \smallsetminus ( C\cap M)) & \underbrace{=\P( C\cup M) -\overbrace{\P(( C\cup M) \cap ( C\cap M))}^{\P( C\cap M)}}_{\P( A\smallsetminus B) =\P( A) -\P( A\cap B)}\\
 & =\P( C\cup M) -\P( C\cap M)\\
 & =0.22-0.02=0.2
\end{aligned}$
\end{enumerate}
\end{enumerate}
\Soluzione

\textbf{Primo Metodo.}

Introduciamo lo spazio campionario $\Omega =\{U_{c} ,U_{f} ,N_{c} ,N_{f}\}$ per indicare gli urgenti $( U)$ e in non urgenti $( N)$ rispettivamente in città $(_{c})$ e fuori città $(_{f})$. Usiamo come $\sigma $-algebra $\mathcal{A} =2^{\Omega }$. Poniamo:
\begin{itemize}
\item $F=\{N_{f} ,U_{f}\} =$ consegna fuori città
\item $U=\{U_{c} ,U_{f}\} =$ consegna urgente
\end{itemize}

Dalle ipotesi del problema sappiamo che:
\begin{gather*}
\P( F) =\P(\{N_{f} ,U_{f}\}) =0.4\\
\P( U) =\P(\{U_{c} ,U_{f}\}) =0.3\\
\P(\{N_{c}\}) =0.4
\end{gather*}
Ricordiamo inoltre che:
\begin{gather*}
\P( A\cup B) =\P( A) +\P( B) -\P( A\cap B)\\
\implies  \mathbb{P}( A\cap B) =\P( A) +\P( B) -\P( A\cup B)
\end{gather*}
In questo modo possiamo calcolare:
\begin{enumerate}
\item la probabilità di consegna urgente fuori città

$\begin{aligned}
\P(\{U_{f}\}) & =\P( F\cap U)\\
 & =\P( F) +\P( U) -\P( F\cup U)\\
 & =\P( F) +\P( U) -\P(\{N_{f} ,U_{f} ,U_{c}\})\\
 & =\P( F) +\P( U) -\P( \Omega \smallsetminus \{N_{c}\})\\
 & =0.4+0.3-( 1-0.6)\\
 & =0.1
\end{aligned}$
\item la probabilità di consegna urgente in città

$\P(\{U_{c}\}) =\P( U) -\P(\{U_{f}\}) =0.3-0.1=0.2$
\end{enumerate}
\begin{oss}
Per avere i valori di probabilità di tutti i mattoncini possiamo anche calcolare
\begin{equation*}
\P(\{N_{f}\}) =1-\P(\{U_{f}\}) -\P(\{U_{c}\}) -\P(\{N_{c}\}) =1-0.1-0.2-0.4=0.3
\end{equation*}
In altri termini $\{N_{f}\} ,\{U_{f}\} ,\{N_{c}\} ,\{U_{c}\}$ rappresentano una partizione di $\Omega $. Sapendo le probabilità di questi eventi, ricaviamo le probabilità di qualsiasi evento.
\end{oss}


\textbf{Secondo Metodo.}

Con questo metodo mostriamo come non sia necessario introdurre $\Omega $: è possibile fissare un arbitrario spazio di probabilità $( \Omega ,\mathcal{A} ,\P)$ e definire la probabilità su specifici eventi in $\mathcal{A}$. Siano:
\begin{itemize}
\item $C=$ consegna in città
\item $U=$ consegna urgente
\end{itemize}

L'insieme $\Omega $ delle possibili consegne è unione dei seguenti eventi, tra loro incompatibili:
\[
	\begin{array}{lc}
			& \text{con probabilità}\\
		C\cap U & a\\
		C\comp \cap U & b\\
		C\cap U\comp & c\\
		C\comp \cap U\comp & d
	\end{array}
\]
Stiamo costruendo $\Omega $ come insieme delle possibili modalità di una richiesta. Stiamo partizionando $\Omega $.


\tikzset{every picture/.style={line width=0.75pt}} %set default line width to 0.75pt        

\begin{tikzpicture}[x=0.75pt,y=0.75pt,yscale=-1,xscale=1]
%uncomment if require: \path (0,124); %set diagram left start at 0, and has height of 124

%Shape: Square [id:dp05978656421370365] 
\draw   (260,10) -- (340,10) -- (340,90) -- (260,90) -- cycle ;
%Straight Lines [id:da27899061599265207] 
\draw    (340,50) -- (260,50) ;
%Straight Lines [id:da4092530719672025] 
\draw    (300,90) -- (300,10) ;

% Text Node
\draw (237,22.4) node [anchor=north west][inner sep=0.75pt]    {$C$};
% Text Node
\draw (237,62.4) node [anchor=north west][inner sep=0.75pt]    {$C\comp$};
% Text Node
\draw (274,100.4) node [anchor=north west][inner sep=0.75pt]    {$U$};
% Text Node
\draw (312,100.4) node [anchor=north west][inner sep=0.75pt]    {$U\comp$};


\end{tikzpicture}

I dati si traducono come segue:
\begin{itemize}
\item $0.4=\P\left( C\comp\right) =\P\left(\left( C\comp \cap U\right) \cup \left( C\comp \cap U\comp\right)\right)
 =\P\left( C\comp \cap U\right) +\P\left( C\comp \cap U\comp\right)
 =b+d$
\item $\begin{aligned}[t]
0.3=\P( U) =\P\left(\left( C\comp \cap U\right) \cup ( C\cap U)\right)
 =\P\left( C\comp \cap U\right) +\P( C\cap U)
 =b+a
\end{aligned}$
\item $0.4=\P\left( U\comp \cap C\right) =c$
\end{itemize}

Si ricava allora:
\begin{equation*}
a=0.2\ \ \ \ \ \ \ \ b=0.1\ \ \ \ \ \ \ \ c=0.4\ \ \ \ \ \ \ \ d=0.3
\end{equation*}
Quindi la probabilità che ci sia una consegna urgente in città è $\P( U\cap C) =a=0.2$ e la probabilità che ci sia una consegna urgente fuori città è $\P\left( U\cap C\comp\right) =b=0.1$.
\Soluzione
\begin{enumerate}
\item Scriviamo $B$ come unione disgiunta di insiemi: $B=( B\smallsetminus A) \cup ( A\cap B) =\left( B\cap A\comp\right) \cup ( A\cap B)$

Dimostriamo le due disuguaglianze.
\begin{enumerate}
\item $\P( B) =\overbrace{\P\left(\left( B\cap A\comp\right) \cup ( A\cap B)\right) =\underbrace{\P\left( B\cap A\comp\right)}_{\geq 0} +\P( A\cap B)}^{\text{perché gli insiemi sono disgiunti}} \geq \P( A\cap B)$
\item $\P( A\cap B) =\P( A) +\P( B) -\underbrace{\P( A\cap B)}_{\leq 1} \geq \underbrace{\P( A)}_{=1} +\P( B) -1=\P( B)$
\end{enumerate}

Quindi abbiamo $\P( A\cap B) =\P( B)$.
\item Scriviamo $B$ come unione di insiemi disgiunti: $B=A\cup ( B\smallsetminus A)$.

$\P( B) =\P( A\cup ( B\smallsetminus A)) =\P( A) +\underbrace{\P( B\smallsetminus A)}_{\geq 0} \geq \P( A)$
\end{enumerate}
\Soluzione

Facciamo due osservazioni per poter rispondere ai primi quesiti.
\begin{equation*}
\begin{cases}
\P( A\cup B) \geq \P( A) =0.4\\
\P( A\cup B) \geq \P( B) =0.7
\end{cases} \implies  \boxed{\P( A\cup B) \geq 0.7}
\end{equation*}
analogamente
\begin{equation*}
\begin{cases}
\P( A\cap B) \leq \P( A) =0.4\\
\P( A\cap B) \leq \P( B) =0.7
\end{cases} \implies  \boxed{\P( A\cap B) \leq 0.4}
\end{equation*}
Allora:
\begin{enumerate}
\item F
\item F/V (è plausibile)
\item V
\item F
\item V/F (è plausibile)
\item V, infatti\begin{equation*}
\P( B) =\P( A\cap B) +\P\left( B\cap A\comp\right)
\end{equation*}

pertanto

\begin{equation*}
\P\left( B\cap A\comp\right) =\underbrace{\P( B)}_{=0.7} -\underbrace{\P( A\cap B)}_{\leq 0.4} \geq 0.7-0.4=0.3
\end{equation*}
\item V, infatti\begin{equation*}
\begin{cases}
\P\left( A\cap B\comp\right) \leq \P( A) =0.4\\
\P\left( A\cap B\comp\right) \leq \P\left( B\comp\right) =1-\P( B) =1-0.7=0.3
\end{cases}\implies \mathbb{P}\left( A\cap B\comp\right) \leq 0.3
\end{equation*}
\end{enumerate}
\Soluzione
\begin{enumerate}
\item Proprietà delle $\sigma $-algebre
\item Solo successi nelle prime $n$ prove
\item Scriviamo le relazioni insiemistiche\begin{align*}
A_{n+1} :=\bigcap\limits _{k=1}^{n+1} E_{k} & =E_{k+1} \cap \underbrace{\bigcap\limits _{k=1}^{n} E_{k}}_{A_{n}} =E_{k+1} \cap A_{n} \subset A_{n}\\
\omega \in A_{n+1} & \iff \omega \in E_{k+1} \cap A_{n}\\
 & \iff \omega \in E_{k+1} \land \omega \in A_{n}\\
 & \text{i.e.} \ A_{n+1} \subset A_{n} \ \forall n
\end{align*}
\item Scriviamo le relazioni insiemistiche\begin{align*}
\bigcap\limits _{n\in \mathbb{N}} A_{n} =\bigcap\limits _{n\in \mathbb{N}}\bigcap\limits _{k=1}^{n} E_{k} =\bigcap\limits _{k\in \mathbb{N}} E_{k} & =\text{solo successi}\\
\omega \in \bigcap\limits _{n\in \mathbb{N}} A_{n} & \iff \omega \in A_{n} \ \forall n\in \mathbb{N}\\
 & \iff \omega \in \bigcap\limits _{k=1}^{n} E_{k} \ \forall n\in \mathbb{N}\\
 & \iff \omega \in E_{k} \ \forall n\in \mathbb{N} \ \forall k=1,\dotsc ,n\\
 & \iff \omega \in E_{k} \ \forall k
\end{align*}

$\bigcap\limits _{n\in \mathbb{N}} A_{n} \in \mathcal{A}$ per le proprietà delle $\sigma $-algebre.
\item Abbiamo $A_{n} \downarrow A$ con $A:=\bigcap\limits _{n\in \mathbb{N}} A_{n}$ infatti $\begin{cases}
A_{n+1} \subseteq A_{n} \ \forall n\\
\bigcap\nolimits _{n\in \mathbb{N}} A_{n} =A
\end{cases}$. Per la continuità di $\P$ dall'alto si ha $\P( A) =\P\left(\bigcap\limits _{n\in \mathbb{N}} A_{n}\right) =\lim\limits _{n}\P( A_{n}) =\lim\limits _{n}\frac{1}{2^{n}} =0$.
\end{enumerate}
\Soluzione
\begin{enumerate}
\item $\forall n\in \mathbb{N} \ \ \ \ A_{n} =\left[ 0,\frac{1}{n}\right)$

\tikzset{every picture/.style={line width=0.75pt}} %set default line width to 0.75pt        

\begin{tikzpicture}[x=0.75pt,y=0.75pt,yscale=-1,xscale=1]
%uncomment if require: \path (0,91); %set diagram left start at 0, and has height of 91

%Straight Lines [id:da5723837040292659] 
\draw    (142,27.5) -- (416.5,27.5) ;

% Text Node
\draw (159,12.4) node [anchor=north west][inner sep=0.75pt]  [font=\LARGE]  {$[$};
% Text Node
\draw (369,12.4) node [anchor=north west][inner sep=0.75pt]  [font=\LARGE]  {$)$};
% Text Node
\draw (269,12.4) node [anchor=north west][inner sep=0.75pt]  [font=\LARGE]  {$)$};
% Text Node
\draw (189,12.4) node [anchor=north west][inner sep=0.75pt]  [font=\LARGE]  {$)$};
% Text Node
\draw (158,52.4) node [anchor=north west][inner sep=0.75pt]    {$0$};
% Text Node
\draw (368,52.4) node [anchor=north west][inner sep=0.75pt]    {$1$};
% Text Node
\draw (268,52.4) node [anchor=north west][inner sep=0.75pt]    {$\frac{1}{2}$};
% Text Node
\draw (188,52.4) node [anchor=north west][inner sep=0.75pt]    {$\frac{1}{n}$};
% Text Node
\draw (229,12.4) node [anchor=north west][inner sep=0.75pt]  [font=\LARGE]  {$)$};
% Text Node
\draw (228,52.4) node [anchor=north west][inner sep=0.75pt]    {$\frac{1}{3}$};


\end{tikzpicture}


Osserviamo che gli insiemi $A_{n}$ sono inscatolati, i.e. $A_{n+1} \subseteq A_{n} \ \forall n$. Si ha:\begin{equation*}
I=\bigcap _{n\in \mathbb{N}} A_{n} \ =\{0\} \ \ \ \ \ \ \ \ U=\bigcup _{n\in \mathbb{N}} A_{n} \ =[ 0,1) =A_{1}
\end{equation*}
\item Abbiamo $A_{n} \downarrow I$, infatti $A_{n+1} \subseteq A_{n} \ \forall n$ e $\bigcap\nolimits _{n\in \mathbb{N}} A_{n} =I$.

Allora per la continuità dall'alto si ha $\P( I) =\lim\limits _{n}\P( A_{n}) =\lim\limits _{n}\frac{1}{n} =0$.

Infine $\P( U) =\P([ 0,1)) =\P( A_{1}) =1$.
\end{enumerate}
\Soluzione

Manca.
\Soluzione

Manca.

\chapter{Calcolo combinatorio}
%!TEX root = ../main.tex

\ParteEsercizi

Ci occuperemo di esperimenti aleatori che ammettono solo un numero \textbf{finito} $n< +\infty $ di risultati possibili. Sia $\Omega =\{\omega_{1} ,\dots ,\omega_{n}\}$ lo spazio campionario associato all'esperimento e $\Ac =\mathcal{P}(\Omega) =2^{\Omega }$.

Supponiamo che la natura dell'esperimento aleatorio ci suggerisca di assumere $p_{1} =p_{2} =\cdots =p_{n} =p$, i.e. di assegnare la stessa probabilità ad ogni evento elementare. In questo caso si parla di \textbf{spazio di probabilità uniforme}.

\begin{oss}
	In generale non è sempre lecito supporre che la probabilità sia uniforme.
\end{oss}

Dagli assiomi della probabilità segue che:

\begin{equation*}
	1=\PP(\Omega) =\underbrace{\sum\limits_{k=1}^{n}\PP(\{\omega_{k}\}) =\sum\limits_{k=1}^{n} p}_{\text{eventi equiprobabili}} =np=| \Omega | p\implies \boxed{p=\frac{1}{| \Omega | }}
\end{equation*}

Pertanto, quando abbiamo a che fare con esperimenti aleatori su spazi di probabilità uniformi procediamo nel seguente modo:
\begin{enumerate}
	\item individuiamo $\Omega $ (i.e. i possibili esiti dell'esperimento aleatorio) finito, $| \Omega | < +\infty \ \rightarrow \ \Ac =\mathcal{P}(\Omega) =2^{\Omega }$.
	\item individuiamo il sottoinsieme $A\subseteq \Omega $ che rappresenta l'evento.
	\item se abbiamo probabilità uniforme, cioè gli eventi sono equiprobabili,
	\begin{equation*}
		\boxed{\forall A\in \Ac \ \ \ \ \PP(A) =\frac{| A| }{| \Omega | } =\frac{\text{casi favorevoli}}{\text{casi possibili}}}
	\end{equation*}
\end{enumerate}
Il problema si riduce sostanzialmente a contare gli elementi di $\Omega $ e di $A$ (i.e. determinare $| \Omega | $ e $|A| $). Per farlo dobbiamo trovare uno \textbf{schema di scelte successive} che ci porti ad individuare univocamente un elemento di $\Omega $ (e di $A$). Questo schema di scelte successive è:
\begin{itemize}
	\item giustificato dal principio fondamentale del calcolo combinatorio;
	\item graficamente visualizzabile tramite un \textbf{diagramma ad albero}.
\end{itemize}
Il calcolo combinatorio è costituito da tecniche che ci permettono di \textbf{contare} il numero di elementi di un insieme \textbf{senza enumerarli} esplicitamente.

\textbf{Principio fondamentale del calcolo combinatorio.} Si realizzano $r$ esperimenti. Si supponga che il primo esperimento abbia $n_{1}$ esiti possibili; che per ognuno di essi il secondo abbia $n_{2} \ $esiti possibili; che per ognuno degli esiti dei primi due esperimenti il terzo esperimento abbia $n_{3}$ esiti possibili ecc. Allora, se sequenze distinte di esiti degli $r$ esperimenti producono esiti finali distinti, allora gli $r$ esperimenti hanno in tutto $n_{1} \cdot n_{2} \cdots n_{r}$ esiti possibili.

\begin{oss}
	Il diagramma ad albero è utile per rappresentare tutti i casi possibili: ci permette di avere un'elencazione grafica di tutti gli elementi di $\Omega $ (e/o di $A\in \Ac$).
\end{oss}

\begin{oss}
	Lo schema di scelte successive deve individuare univocamente gli elementi degli insiemi (come da principio).
\end{oss}

\textbf{Permutazioni, disposizioni e combinazioni.}

Sono concetti utili per contare gli elementi in un insieme.

\begin{definition}
	Le permutazioni di $n$ oggetti sono tutti i possibili ordinamenti di $n$ oggetti. Le permutazioni di $n$ oggetti sono pari a:
	\begin{equation*}
		\boxed{P_{n} =n!=n(n-1)(n-2) \cdots 1} \ \ \ \ (0!\coloneqq 1)
	\end{equation*}
	Ho $n$ scelte per il primo oggetto, $(n-1)$ scelte per il secondo ecc.
\end{definition}

\begin{oss}
	Vi sono
	\begin{equation*}
		\boxed{P_{n}^{n_{1} ,n_{2} ,\dots ,n_{r}} =\frac{n!}{n_{1} !\ n_{2} !\cdots n_{r} !}}
	\end{equation*}
	permutazioni di un insieme di $n$ oggetti, dei quali $n_{1}$ sono identici tra loro, $n_{2}$ sono identici tra loro e distinti dai precedenti, $\dots $, $n_{r}$ sono identici e distinti dai precedenti.
\end{oss}

\textit{Esempio.} Classifica di $n$ persone, ci sono $n!$ possibili ordinamenti delle persone.

Supponiamo di avere un insieme $E$ formato da $n$ oggetti \textbf{distinti}, da cui ne estraiamo $r$.
\begin{definition}
	Le \textbf{disposizioni con ripetizione di classe }$r$ sono le stringhe di $r$ elementi di $E$. Il modello è quello dell'\textit{estrazione con reimmissione} da un'urna contenente $n$ oggetti. Ogni volta lo rimetto nell'urna. Ogni volta ho $n$ modi di scegliere:
	\begin{equation*}
		\boxed{D'^{n}_{r} =\underbrace{n\cdot n\cdots n}_{r\ \text{volte}} =n^{r}}
	\end{equation*}
\end{definition}

\begin{definition}
	Le \textbf{disposizioni senza ripetizione di classe} $r$ sono le $r$-ple ordinate di elementi di $E$ senza ripetizione (deve essere $r\leq n$). Il modello è quello dell'\textit{estrazione senza reimmissione} da un'urna contenente $n$ oggetti. Ogni volta ho un modo in meno di scegliere. \textit{L'ordine conta}.
	\begin{equation*}
		\boxed{D_{r}^{n} =n(n-1)(n-2) \cdots (n-r+1) =\frac{n!}{(n-r) !}}
	\end{equation*}
\end{definition}

\begin{oss}
	Le permutazioni sono un caso particolare di disposizioni senza ripetizione: $n!=D_{n}^{n}$.
\end{oss}

\textit{Esempio.} I primi $r$ classificati in una gara di $n$ persone.

\begin{definition}
	Ogni sottoinsieme di $E$ di cardinalità $r\leq n$ è detto \textbf{combinazione di classe} $r$ di $E$. Il modello è quello di un'\textit{estrazione simultanea} di $r$ oggetti da un'urna che ne contiene $n$ $(r\leq n)$. Simile alle disposizioni, ma qui \textit{l'ordine non conta}.
	\begin{equation*}
		\boxed{C_{r}^{n} =\binom{n}{r} =\frac{n!}{r!(n-r) !}}
	\end{equation*}
\end{definition}

\begin{oss}
Per determinare $C_{r}^{n}$ basta osservare che: ogni fissata combinazione di classe $r$ dà luogo a $r!$ disposizioni di classe $r$, i.e. ho $r!$ possibili modi di permutare gli oggetti:
	\begin{equation*}
		D_{r}^{n} =r!\ C_{r}^{n}
	\end{equation*}
	Ora,
	\begin{equation*}
		D_{r}^{n} =n(n-1) \cdots (n-r+1) =\frac{n!}{(n-r) !}
	\end{equation*}
	Quindi:
	\begin{equation*}
		r!\ C_{r}^{n} =\frac{n!}{(n-r) !} \implies C_{r}^{n} =\frac{n!}{r!(n-r) !} =\binom{n}{r}
	\end{equation*}
\end{oss}

\Esercizio{}

Lanciate due volte un dado non truccato.
\begin{enumerate}
	\item Qual è la probabilità di un $6$ al primo lancio?
	\item Qual è la probabilità che la somma dei due risultati dia un $6$?
	\item A quanto è pari quest'ultima probabilità se lanciate tre volte il dado?
\end{enumerate}

\Esercizio{}

Trovate una ventiquattrore chiusa con combinazione di $6$ cifre. Con quale probabilità riuscirete ad aprirla al primo tentativo? E se la combinazione fosse di $6$ lettere scelte fra $A$, $B$, $C$ e $D$?

\Esercizio{}

Scommetto sui primi $5$ classificati di una gara con $40$ concorrenti, senza conoscerli.
\begin{enumerate}
	\item Qual è la probabilità di vincere?
	\item E se scommettessi sui nomi dei primi cinque, senza precisarne l'ordine?
\end{enumerate}

\Esercizio{(Paradosso dei compleanni)}

\begin{enumerate}
	\item Qual è la probabilità $p_{n}$ che, in un gruppo di $n$ persone selezionate in modo casuale (nate tutte in un anno non bisestile), almeno due di esse compiano gli anni lo stesso giorno? Quanto deve essere grande $n$ affinché $p_{n}  >\frac{1}{2}$?
	\item Qual è la probabilità $q_{n}$ che, nel gruppo di $n$ persone, ve ne sia almeno una che compie gli anni il vostro stesso giorno (posto che non siate nati il 29 febbraio)?Quanto deve essere grande $n$ affinché $q_{n}  >\frac{1}{2}$?
\end{enumerate}

\Esercizio{}

Tre amici si danno appuntamento nel bar della piazza centrale della città senza sapere che ci sono quattro bar. Qual è la probabilità che scelgano lo stesso bar? Tre bar differenti?

\Esercizio{}

Confrontate la probabilità di ottenere almeno un $6$ nel lancio di $6$ dadi con la probabilità di ottenere almeno due $6$ nel lancio di $12$ dadi.

\Esercizio{}

Una moneta viene lanciata $4$ volte. Quali sono le probabilità di ottenere $3$ teste e una croce e di ottenere $2$ teste e $2$ croci?

\Esercizio{}

Per compilare una colonna di una schedina del Totocalcio occorre scegliere, per ciascuna delle $13$ partite in esame, tra la vittoria della squadra di casa $(1)$, il pareggio $(X)$ o la vittoria della squadra in trasferta $(2)$. Si calcoli la probabilità di fare $13$ o $12$ al Totocalcio giocando per ogni partita una doppia (due possibili risultati per ogni partita). Si suppongano equiprobabili le possibili schedine.

\Esercizio{}

Giocate $6$ numeri al Superenalotto. Saranno estratti $6$ numeri senza ripetizione dai primi $90$ naturali, seguiti dall'estrazione di un $7^{o}$ numero jolly, diverso quindi dai precedenti. Qual è la probabilità di fare
\begin{enumerate}
	\item $6$ (indovinare i primi $6$ numeri estratti)?
	\item $5+1$ (indovinare $5$ dei primi $6$ numeri estratti e in più il numero jolly)?
	\item $6$ o $5+1$?
\end{enumerate}

Quale o quali spazi campionari avete introdotto per rispondere ai punti precedenti? Sono scelte coerenti?

\Esercizio{}

Si consideri l'estrazione di $k$ oggetti da un'urna contenente $n$ oggetti distinti, numerati da $1$ a $n$.
\begin{enumerate}

	\item Si consideri l'estrazione con reimmissione, per cui $k\in \NN$. Viene osservato il risultato ordinato delle $k$ estrazioni.
	\begin{enumerate}
		\item Individuare il più piccolo spazio campionario $\Omega $ che permette di descrivere l'esperimento aleatorio.
		\item * Dimostrare che i possibili risultati dell'esperimento aleatorio sono equiprobabili se e solo se le $k$ estrazioni sono indipendenti e per ciascuna singola estrazione gli $n$ possibili risultati sono equiprobabili.
		\item Calcolare la probabilità di estrarre un dato insieme di oggetti $\{x_{1} ,\dots ,x_{k}\}$, senza riguardo per l'ordine di estrazione.
	\end{enumerate}

	\item Si consideri l'estrazione senza reimmissione, per cui $1\leq k\leq n$. Viene osservato il risultato ordinato delle $k$ estrazioni.
	\begin{enumerate}
		\item Individuare il più piccolo spazio campionario $\Omega $ che permette di descrivere l'esperimento aleatorio.
		\item * Dimostrare che i possibili risultati dell'esperimento aleatorio sono equiprobabili se e solo se il risultato di ciascuna delle $k$ estrazioni, data la sequenza degli oggetti già estratti, è equiprobabile fra gli oggetti rimanenti.
		\item Calcolare la probabilità di estrarre un dato insieme di oggetti $\{x_{1} ,\dots ,x_{k}\}$, senza riguardo per l'ordine di estrazione.
	\end{enumerate}

	\item Si consideri l'estrazione simultanea, per cui $1\leq k\leq n$. Viene osservato il risultato dell'estrazione.
	\begin{enumerate}
		\item Individuare il più piccolo spazio campionario $\Omega $ che permette di descrivere l'esperimento aleatorio.
		\item Calcolare la probabilità di estrarre un dato insieme di oggetti $\{x_{1} ,\dots ,x_{k}\}$.
	\end{enumerate}

\end{enumerate}

\Esercizio{}

Si calcoli la probabilità di ottenere $2$ palline rosse estraendone $4$ da un'urna che contiene $3$ palline rosse e $7$ bianche. Si confrontino i casi di estrazioni con reimmissione, senza reimmissione e simultanea.

\Esercizio{}

Si estraggono, con o senza reimmissione, $7$ palline da un'urna contenente $4$ palline bianche e $8$ rosse. Qual è la probabilità di estrarne esattamente $3$ bianche?

\Esercizio{}

Da un'urna contenente $5$ palline bianche, $7$ rosse e $4$ blu, vengono estratte senza reimmissione $5$ palline. Con quale probabilità si ottengono $3$ bianche, $1$ rossa e $1$ blu?

\Esercizio{}

Si estraggono $2$ palline da un'urna contenente $5$ palline bianche, $6$ nere e $4$ rosse. Calcolare la probabilità che siano dello stesso colore. Distinguere il caso di estrazione simultanea o con reimmissione.

\Esercizio{}

Si consideri l'estrazione di $n$ palline da un'urna contenente $B$ palline bianche e $R$ palline rosse. Qual è la probabilità di estrarne esattamente $k$ rosse, con $0\leq k\leq n$? Si risponda nel caso di estrazione con reimmissione e nel caso di estrazione simultanea.

\Esercizio{}

A scommette con B che estrarrà 4 carte di 4 semi diversi da un mazzo di 40 carte. Qual è la probabilità che A vinca?

\Esercizio{}

Supponiamo di giocare a poker con un mazzo da $52$ carte, identificate dal seme (cuori $\varheartsuit $, quadri $\vardiamondsuit $, fiori $\clubsuit $, picche $\spadesuit $) e dal tipo (un numero da $2$ a $10$ oppure $J$, $Q$, $K$, $A$). Si calcolino:
\begin{enumerate}
	\item la probabilità di avere un full, ovvero un sottoinsieme di $5$ carte $\{X_{1} ,X_{2} ,X_{3} ,Y_{1} ,Y_{2}\}$ costituito dall'unione di un tris (un sottoinsieme di $3$ carte $X_{1} ,X_{2} ,X_{3}$ dello stesso tipo) e di una coppia (un sottoinsieme di $2$ carte $Y_{1} ,Y_{2}$ dello stesso tipo);
	\item la probabilità di avere una doppia coppia, ovvero un sottoinsieme di $5$ carte $\{X_{1} ,X_{2} ,Y_{1} ,Y_{2} ,Z\}$ costituito dall'unione di due coppie $\{X_{1} ,X_{2}\}$, $\{Y_{1} ,Y_{2}\}$ di tipi diversi, più una quinta carta $Z$ di tipo diverso dai tipi delle due coppie;
	\item la probabilità di avere una scala reale massima, ovvero le $5$ carte $10,J,Q,K,A$ dello stesso seme;
	\item la probabilità di avere una scala reale, ovvero una qualsiasi scala di $5$ carte dello stesso seme (ricordiamo che con l'asso possiamo fare la scala $A,2,3,4,5$ ma anche $10,J,Q,K,A$);
	\item la probabilità di avere una scala, ovvero una qualsiasi scala di $5$ carte non necessariamente dello stesso seme;
	\item la probabilità di avere colore, ovvero un sottoinsieme di $5$ carte dello stesso seme;
	\item la probabilità di avere poker, ovvero un sottoinsieme di $5$ carte in cui ci sono $4$ carte dello stesso tipo;
	\item la probabilità di avere un tris, ovvero un sottoinsieme di $5$ carte in cui ci sono $3$ carte dello stesso tipo e le altre due di tipo diverso tra loro e dalle prime tre;
	\item la probabilità di avere una coppia, ovvero un sottoinsieme di $5$ carte in cui ci sono $2$ carte dello stesso tipo e le altre tre di tipo diverso tra loro e dalle prime due.
\end{enumerate}

\Esercizio{}

Un mazzo di carte da scopone scientifico è costituito da $40$ carte, identificate dal seme (spade, coppe, bastoni, denari) e dal tipo (asso, $2$, $3$, $4$, $5$, $6$, $7$, fante, cavallo, re). Supponendo di pescare $10$ carte dal mazzo, calcolare la probabilità di estrarre:
\begin{enumerate}
	\item l'asso di bastoni,
	\item l'asso di spade e l'asso di coppe,
	\item almeno un asso fra l'asso di spade e l'asso di coppe.
\end{enumerate}

\Esercizio{}

Facendo uso del calcolo combinatorio, dimostrare che
\begin{equation*}
	(a+b)^{n} =\sum\limits_{k=0}^{n}\binom{n}{k} a^{k} b^{n-k}
\end{equation*}

\Esercizio{}

Dato un insieme finito $\Omega $ di cardinalità $| \Omega | $, dimostrare che il suo insieme delle parti $\mathcal{P}(\Omega) =2^{\Omega }$ ha cardinalità $2^{| \Omega | }$.

\ParteSoluzioni

\Soluzione

\begin{enumerate}
	\item Lo spazio degli eventi elementari è
	\begin{gather*}
		\Omega =\{(i,j) :i,j=1,\dots ,6\} =\{1,\dots ,6\}^{2} ,\ \ \ \ | \Omega | =36\\
		i=\text{primo lancio} \ \ \ \ j=\text{secondo lancio}
	\end{gather*}
	Come famiglia ($\sigma $-algebra) degli eventi possibili possiamo scegliere $\Ac \coloneqq \mathcal{P}(\Omega) =2^{\Omega }$. Per quanto riguarda l'assegnazione di una probabilità $\PP$ su $(\Omega ,\Ac)$ osserviamo che se assumiamo che i due non siano truccati e vogliamo che il nostro spazio di probabilità $(\Omega,\Ac,\PP)$ modellizzi questo fatto fisico, dobbiamo ammettere che tutti gli eventi elementari di $\Omega $ abbiano la stessa probabilità $\PP =\frac{1}{| \Omega | } =\frac{1}{36}$. Siamo pertanto in uno spazio campionario con probabilità uniforme. Allora $\forall A\in \Ac ,\PP(A) =\frac{| A| }{| \Omega | } =\frac{| A| }{36}$.

	Sia $A$ l'evento "avere un $6$ al primo lancio", allora
	\begin{gather*}
		A=\{(6,1) ,(6,2) ,(6,3)(6,4) ,(6,5) ,(6,6)\} ,\ \ \ \ | A| =6\\
		\implies \PP(A) =\frac{| A| }{| \Omega | } =\frac{6}{36} =\frac{1}{6} \approx 0.1667
	\end{gather*}

	\begin{oss}
		In questo caso $\Omega $ ed $A$ hanno pochi elementi e possiamo elencarli senza problemi. Avremmo anche potuto contare gli elementi di $A$ e $\Omega $ senza elencarli con un diagramma ad albero.
	\end{oss}

	\item Sia $B$ l'evento "la somma dei due risultati è $6$", allora
	\begin{gather*}
		B=\{(5,1) ,(2,4) ,(3,3) ,(4,2) ,(1,5)\} ,\ \ \ \ | B| =5\\
		\implies \PP(B) =\frac{| B| }{| \Omega | } =\frac{5}{36} \approx 0.1389
	\end{gather*}

	\begin{oss}
		I risultati della somma non sono equiprobabili. Infatti, se assumiamo che i due dadi non siano truccati, denotando con $\{i\}$ l'evento "la somma dei due dadi è $i$", per $i=1,\dots ,12$,
		\begin{equation*}
			\begin{array}{ l }
				\PP(\{2\}) =\PP(\{12\}) =1/36\\
				\PP(\{3\}) =\PP(\{11\}) =1/18\\
				\PP(\{4\}) =\PP(\{10\}) =1/12\\
				\PP(\{5\}) =\PP(\{9\}) =1/9\\
				\PP(\{6\}) =\PP(\{8\}) =5/36\\
				\PP(\{7\}) =1/6
			\end{array}
		\end{equation*}
		Se invece assumiamo che i possibili risultati della somma siano equiprobabili, dobbiamo porre
		\begin{equation*}
			\PP(\{i\}) =\frac{1}{11} \ \ \ \ 11=| \Omega | \ \text{con} \ \Omega =\{2,3,\dots ,12\}
		\end{equation*}
		$\Omega =$ spazio degli eventi elementari, somma degli esiti del lancio dei due dadi. Lo spazio di probabilità così costruito è matematicamente corretto, ma non ha nulla a che fare con la realtà fisica dell'esperimento.
	\end{oss}

	\item Lanciando $3$ volte il dado lo spazio campionario diventa
	\begin{equation*}
		\Omega =\{(i,j,k) :i,j,k=1,\dots ,6\} =\{1,\dots ,6\}^{3} \ \ \ \ | \Omega | =6^{3} =216
	\end{equation*}

	Sia $C$ l'evento "la somma dei dadi è $6$"
	\begin{gather*}
		\begin{aligned}
			C= & (\{1,1,4\} ,\{1,4,1\} ,\{4,1,1\} ,\\
			 & \{1,2,3\} ,\{1,3,2\} ,\{2,1,3\} ,\{2,3,1\} ,\{3,1,2\} ,\{3,2,1\} ,\\
			 & \{2,2,2\})
		\end{aligned}\\
		\implies \PP(C) =\frac{| C| }{| \Omega | } =\frac{10}{6^{3}} \approx 0.0463
	\end{gather*}

	\begin{oss}
		Avremmo potuto risolvere l'esercizio anche sfruttando il concetto di permutazioni, definendo
		\begin{gather*}
			C=\text{possibili permutazioni di} \ 
			\begin{cases}
				\{1,2,3\} & \rightarrow 3!=6\ \text{permutazioni}\\
				\{2,2,2\} & \rightarrow 1\ \text{permutazione}\\
				\{1,1,4\} & \rightarrow 3!/2!=3\ \text{permutazioni}
			\end{cases}\\
			\implies | C| =6+3+1=10
		\end{gather*}
	\end{oss}
	
\end{enumerate}

\Soluzione

Manca.

\Soluzione

Consideriamo i due eventi
\begin{itemize}
	\item $A=$ \event{vinco}, i.e. azzecco la cinquina vincente
	\item $B=$ \event{vinco senza precisare l'ordine}
\end{itemize}

Svolgiamo l'esercizio in $3$ diversi modi.

\textbf{[Metodo 1]}

Determiniamo innanzitutto l'insieme campionario $\Omega $: c'è una classifica di cui ci interessano solo i primi $5$:
\begin{equation*}
	\Omega =\{(\omega_{1} ,\dots ,\omega_{5}) :\omega_{i} \in \{1,\dots,40\} ,\omega_{i} \neq \omega_{j} \ \forall i\neq j\}
\end{equation*}
Tutte le possibili quintuple sono date dalla disposizione (l'ordine conta!) di $n=40$ persone in $r=5$ posti, i.e.
\begin{equation*}
	| \Omega | =D_{5}^{40} =\underbrace{40\cdot 39\cdots 36}_{5\ \text{numeri}} =78960960
\end{equation*}
Per il primo ci sono $40$ possibilità, per il secondo $39$, ecc. (visualizzabile con un diagramma ad albero.
\begin{enumerate}
	\item Ho esattamente $1$ possibilità (su tutte quelle possibili) di vincere.
	\begin{equation*}
		| A| =1\implies \PP(A) =| A| /| \Omega | =1/78960960\approx 1.3\times 10^{-8}
	\end{equation*}
	\item Se non preciso l'ordine ho $5!$ modi di scegliere la cinquina vincente, i.e. tutte le $5!$ permutazioni dei primi $5$ classificati.
	\begin{equation*}
		| B| =5!\implies \PP(B) =| B| /| \Omega | =5!/78960960\approx 1.5\times 10^{-6}
	\end{equation*}
\end{enumerate}

\textbf{[Metodo 2]}

Guardiamo tutta la classifica. In questo caso dobbiamo considerare le permutazioni di $40$ oggetti:
\begin{equation*}
	\Omega =\{(\omega_{1} ,\dots ,\omega_{40}) :\omega_{i} \in \{1,\dots,40\} ,\omega_{i} \neq \omega_{j} \ \forall i\neq j\} \ \ \ \ | \Omega | =40!
\end{equation*}
\begin{enumerate}
	\item Evento $A$.
	\begin{center}
	\addstackgap[5pt]{
		\begin{tabular}{cc|c}
			\hline 
			scelgo il numero $1$: & $1$ modo & \multirow{3}{*}{$1$ permutazione} \\
			$\vdots $ & $\vdots $ &   \\
			scelgo il numero $5$: & $1$ modo &   \\
			\hline 
			scelgo il numero $6$: & $35$ modi & \multirow{4}{*}{$35!$ permutazioni} \\
			scelgo il numero $7$: & $34$ modi &   \\
			$\vdots $ & $\vdots $ &   \\
			scelgo il numero $40$: & $1$ modo &   \\
			\hline
		\end{tabular}
		}
	\end{center}
	i.e. $| A| =35!\implies \PP(A) =| A| /| \Omega | =35!/40!\approx 1.3\times 10^{-8}$.
	\item Evento $B$. Simile a prima, ma abbiamo $5!$ modi di scegliere i primi $5$ arrivati:
	\begin{center}
	\addstackgap[5pt]{
		\begin{tabular}{cc|c}
			\hline 
			scelgo il numero $1$: & $5$ modi & \multirow{3}{*}{$5!$ permutazioni} \\
			$\vdots $ & $\vdots $ &   \\
			scelgo il numero $5$: & $1$ modo &   \\
			\hline 
			scelgo il numero $6$: & $35$ modi & \multirow{4}{*}{$35!$ permutazioni} \\
			scelgo il numero $7$: & $34$ modi &   \\
			$\vdots $ & $\vdots $ &   \\
			scelgo il numero $40$: & $1$ modo &   \\
			\hline
		\end{tabular}
		}
	\end{center}
	i.e. $|B| =5!\cdot 35!\implies \PP(B) = |B|/|\Omega| = 35!\cdot 5! / 40! \approx 1.5\times 10^{-6}$.
\end{enumerate}

\textbf{Metodo 3.}

Se dovessimo rispondere esclusivamente alla seconda domanda potremmo considerare lo spazio delle combinazioni di $n=40$ elementi in $r=5$ posti, i.e.
\begin{equation*}
	| \Omega | =C_{5}^{40} =\binom{40}{5} =\frac{40!}{35!\ 5!}
\end{equation*}
Se $B$ è l'evento "vinco senza precisare l'ordine", allora
\begin{equation*}
	| B| =1\implies \PP(B) =\frac{| B| }{| \Omega | } =\frac{1}{\binom{40}{5}} \approx 1.5\times 10^{-6}
\end{equation*}

\Soluzione

Dato che le persone sono selezionate in modo casuale possiamo considerare una distribuzione uniforme. Pertanto, una volta determinato $(\Omega ,\Ac)$ avremo che $\PP(A) =\frac{| A| }{| \Omega | }$.
\begin{enumerate}
	\item Costruiamo $\Omega $: abbiamo $n$ persone, ognuna delle quali ha $365$ possibilità di essere nata in un dato giorno $\implies \Omega =\{1,\dots,365\}^{n} ,| \Omega | =365^{n}$. Calcoliamo la probabilità che le $n$ persone siano nate tutte in giorni diversi, i.e. detto $A=$ \event{almeno due persone compiono gli anni lo stesso giorno}, calcoliamo $A\comp$ e avremo:
	\begin{equation*}
		\begin{array}{ l }
			\PP(A) =\PP\left(\Omega -A\comp\right) =1-\PP\left(A\comp\right)\\
			A\comp =\text{"tutte le persone sono nate in giorni diverse"}\\
			A\comp =\{\omega =(\omega_{1} ,\dots ,\omega_{n}) :\omega_{i} \neq \omega_{j} \ \forall i\neq j\}\\
			\left| A\comp\right| =365\cdot 364\cdots (365-n+1) =\frac{365!}{(365-n) !}\\
			\implies \PP\left(A\comp\right) =\frac{\left| A\comp\right| }{| \Omega | } =\frac{365!}{365^{n}(365-n) !}\\
			\implies \PP(A) =1-\PP\left(A\comp\right) =1-\frac{365!}{365^{n}(365-n) !}
		\end{array}
	\end{equation*}
	Si può calcolare $n$ per cui $p_{n}  >\frac{1}{2}$: $n=23$.
	\item Come nel punto precedente, possiamo considerare $\Omega =\{365\}^{n}$. Sia $A=$ \event{almeno una persona compie gli anni il mio stesso giorno}, anche in questo caso ci conviene calcolare la probabilità dell'evento complementare:
	\begin{equation*}
		\begin{array}{ l }
			\PP(A) =\PP\left(\Omega -A\comp\right) =1-\PP\left(A\comp\right)\\
			A\comp =\text{"nessuna persona compie gli anni il mio stesso giorno"}\\
			\left| A\comp\right| =364^{n}
		\end{array}
	\end{equation*}
	Ci sono $n$ persone e ognuna di esse \textit{non} compie gli anni lo stesso giorno del mio compleanno: per ognuna di esse abbiamo $364$ possibilità (i restanti giorni dell'anno).
	\begin{equation*}
		\begin{array}{ l }
			\implies \PP\left(A\comp\right) =\frac{\left| A\comp\right| }{| \Omega | } =\frac{364^{n}}{365^{n}} =\left(\frac{364}{365}\right)^{n}\\
			\implies \PP(A) =1-\PP\left(A\comp\right) =1-\left(\frac{364}{365}\right)^{n}
		\end{array}
	\end{equation*}
	Si può calcolare $n$ per cui $q_{n}  >\frac{1}{2}$: $n=253$.
\end{enumerate}

\Soluzione

Manca.

\Soluzione

Manca.

\Soluzione

Manca.

\Soluzione

Manca.

\Soluzione

Possiamo introdurre un unico spazio campionario per rispondere a tutti i punti. Pensiamo all'esperimento casuale come segue: lo spazio degli esiti è dato da tutte le giocate possibili, ossia scegliamo una combinazione di $90$ numeri di classe $6$:
\[
	\Omega =\{\omega \subset \{1,\dots,90\} :| \omega | =6\} \implies | \Omega | =C_{6}^{90} =\binom{90}{6} =622614630
\]
In altre parole, $\Omega $ è l'insieme delle "puntate secche" (ossia schedine con $6$ numeri giocati). La sequenza vincente è assegnata a priori.
\begin{enumerate}
	\item Sia $A=$ \event{indovino i primi $6$ numeri estratti}, i.e. azzecco i primi $6$ numeri della sequenza vincente (non mi interessa il jolly).
	\begin{equation*}
		| A| =1\implies \PP(A) =| A| /| \Omega | =1/\binom{90}{6} \approx 1.6\times 10^{-9}
	\end{equation*}

	\begin{oss}
		Si sarebbe potuto rispondere alla domanda anche introducendo $\Omega $ come lo spazio delle disposizioni anziché combinazioni (confronta con Es. 2.3.2.)
		\begin{gather*}
			\Omega =\{(\omega_{1} ,\dots ,\omega_{6}) :\omega_{i} \in \{1,\dots,90\} \ \forall i=1,\dots,6,\ \omega_{i} \neq \omega_{j} \ \forall i\neq j\}\\
			| \Omega | =D_{6}^{90} =\underbrace{90\cdot 89\cdot 88\cdot 87\cdot 86\cdot 85}_{6\ \text{termini}} =\frac{90!}{84!}
		\end{gather*}
	\end{oss}

	Sia $A=$ \event{indovino i primi $6$ numeri estratti}.
	\begin{equation*}
		| A| =6!
	\end{equation*}
	Dato che \textit{non} conta l'ordine, abbiamo $6!$ modi possibili (i.e. tutte le possibili permutazioni) di scegliere la $6$-pla vincente.
	\begin{equation*}
		\PP(A) =\frac{| A| }{| \Omega | } =\frac{6!}{\frac{90!}{84!}} =\frac{6!\ 84!}{90!} \approx 1.6\times 10^{-9}
	\end{equation*}
	Il ragionamento esposto sopra (schema di scelte successive formalmente giustificato dal principio di calcolo combinatorio) può essere graficamente visualizzato tramite uno schema di diagramma ad albero

	\tikzset{every picture/.style={line width=0.75pt}} %set default line width to 0.75pt        

	\begin{tikzpicture}[x=0.75pt,y=0.75pt,yscale=-1,xscale=1]
	%uncomment if require: \path (0,229); %set diagram left start at 0, and has height of 229

	%Straight Lines [id:da7855330926245441] 
	\draw    (120,100) -- (200,50) ;
	\draw [shift={(200,50)}, rotate = 327.99] [color={rgb, 255:red, 0; green, 0; blue, 0 }  ][fill={rgb, 255:red, 0; green, 0; blue, 0 }  ][line width=0.75]      (0, 0) circle [x radius= 3.35, y radius= 3.35]   ;
	\draw [shift={(120,100)}, rotate = 327.99] [color={rgb, 255:red, 0; green, 0; blue, 0 }  ][fill={rgb, 255:red, 0; green, 0; blue, 0 }  ][line width=0.75]      (0, 0) circle [x radius= 3.35, y radius= 3.35]   ;
	%Straight Lines [id:da040580577550131336] 
	\draw    (120,100) -- (200,70) ;
	\draw [shift={(200,70)}, rotate = 339.44] [color={rgb, 255:red, 0; green, 0; blue, 0 }  ][fill={rgb, 255:red, 0; green, 0; blue, 0 }  ][line width=0.75]      (0, 0) circle [x radius= 3.35, y radius= 3.35]   ;
	\draw [shift={(120,100)}, rotate = 339.44] [color={rgb, 255:red, 0; green, 0; blue, 0 }  ][fill={rgb, 255:red, 0; green, 0; blue, 0 }  ][line width=0.75]      (0, 0) circle [x radius= 3.35, y radius= 3.35]   ;
	%Straight Lines [id:da2306425536240584] 
	\draw    (120,100) -- (200,150) ;
	\draw [shift={(200,150)}, rotate = 32.01] [color={rgb, 255:red, 0; green, 0; blue, 0 }  ][fill={rgb, 255:red, 0; green, 0; blue, 0 }  ][line width=0.75]      (0, 0) circle [x radius= 3.35, y radius= 3.35]   ;
	\draw [shift={(120,100)}, rotate = 32.01] [color={rgb, 255:red, 0; green, 0; blue, 0 }  ][fill={rgb, 255:red, 0; green, 0; blue, 0 }  ][line width=0.75]      (0, 0) circle [x radius= 3.35, y radius= 3.35]   ;
	%Straight Lines [id:da2747219164087562] 
	\draw    (120,100) -- (200,130) ;
	\draw [shift={(200,130)}, rotate = 20.56] [color={rgb, 255:red, 0; green, 0; blue, 0 }  ][fill={rgb, 255:red, 0; green, 0; blue, 0 }  ][line width=0.75]      (0, 0) circle [x radius= 3.35, y radius= 3.35]   ;
	\draw [shift={(120,100)}, rotate = 20.56] [color={rgb, 255:red, 0; green, 0; blue, 0 }  ][fill={rgb, 255:red, 0; green, 0; blue, 0 }  ][line width=0.75]      (0, 0) circle [x radius= 3.35, y radius= 3.35]   ;
	%Straight Lines [id:da6865956571924272] 
	\draw    (200,130) -- (280,80) ;
	\draw [shift={(280,80)}, rotate = 327.99] [color={rgb, 255:red, 0; green, 0; blue, 0 }  ][fill={rgb, 255:red, 0; green, 0; blue, 0 }  ][line width=0.75]      (0, 0) circle [x radius= 3.35, y radius= 3.35]   ;
	\draw [shift={(200,130)}, rotate = 327.99] [color={rgb, 255:red, 0; green, 0; blue, 0 }  ][fill={rgb, 255:red, 0; green, 0; blue, 0 }  ][line width=0.75]      (0, 0) circle [x radius= 3.35, y radius= 3.35]   ;
	%Straight Lines [id:da7503928139510518] 
	\draw    (200,130) -- (280,100) ;
	\draw [shift={(280,100)}, rotate = 339.44] [color={rgb, 255:red, 0; green, 0; blue, 0 }  ][fill={rgb, 255:red, 0; green, 0; blue, 0 }  ][line width=0.75]      (0, 0) circle [x radius= 3.35, y radius= 3.35]   ;
	\draw [shift={(200,130)}, rotate = 339.44] [color={rgb, 255:red, 0; green, 0; blue, 0 }  ][fill={rgb, 255:red, 0; green, 0; blue, 0 }  ][line width=0.75]      (0, 0) circle [x radius= 3.35, y radius= 3.35]   ;
	%Straight Lines [id:da46418143484636754] 
	\draw    (200,130) -- (280,180) ;
	\draw [shift={(280,180)}, rotate = 32.01] [color={rgb, 255:red, 0; green, 0; blue, 0 }  ][fill={rgb, 255:red, 0; green, 0; blue, 0 }  ][line width=0.75]      (0, 0) circle [x radius= 3.35, y radius= 3.35]   ;
	\draw [shift={(200,130)}, rotate = 32.01] [color={rgb, 255:red, 0; green, 0; blue, 0 }  ][fill={rgb, 255:red, 0; green, 0; blue, 0 }  ][line width=0.75]      (0, 0) circle [x radius= 3.35, y radius= 3.35]   ;
	%Straight Lines [id:da03557774417074011] 
	\draw    (200,130) -- (280,160) ;
	\draw [shift={(280,160)}, rotate = 20.56] [color={rgb, 255:red, 0; green, 0; blue, 0 }  ][fill={rgb, 255:red, 0; green, 0; blue, 0 }  ][line width=0.75]      (0, 0) circle [x radius= 3.35, y radius= 3.35]   ;
	\draw [shift={(200,130)}, rotate = 20.56] [color={rgb, 255:red, 0; green, 0; blue, 0 }  ][fill={rgb, 255:red, 0; green, 0; blue, 0 }  ][line width=0.75]      (0, 0) circle [x radius= 3.35, y radius= 3.35]   ;
	%Straight Lines [id:da9336686808598069] 
	\draw    (350,100) -- (430,50) ;
	\draw [shift={(430,50)}, rotate = 327.99] [color={rgb, 255:red, 0; green, 0; blue, 0 }  ][fill={rgb, 255:red, 0; green, 0; blue, 0 }  ][line width=0.75]      (0, 0) circle [x radius= 3.35, y radius= 3.35]   ;
	\draw [shift={(350,100)}, rotate = 327.99] [color={rgb, 255:red, 0; green, 0; blue, 0 }  ][fill={rgb, 255:red, 0; green, 0; blue, 0 }  ][line width=0.75]      (0, 0) circle [x radius= 3.35, y radius= 3.35]   ;
	%Straight Lines [id:da46944720959200703] 
	\draw    (350,100) -- (430,70) ;
	\draw [shift={(430,70)}, rotate = 339.44] [color={rgb, 255:red, 0; green, 0; blue, 0 }  ][fill={rgb, 255:red, 0; green, 0; blue, 0 }  ][line width=0.75]      (0, 0) circle [x radius= 3.35, y radius= 3.35]   ;
	\draw [shift={(350,100)}, rotate = 339.44] [color={rgb, 255:red, 0; green, 0; blue, 0 }  ][fill={rgb, 255:red, 0; green, 0; blue, 0 }  ][line width=0.75]      (0, 0) circle [x radius= 3.35, y radius= 3.35]   ;
	%Straight Lines [id:da3073333322617353] 
	\draw    (350,100) -- (430,150) ;
	\draw [shift={(430,150)}, rotate = 32.01] [color={rgb, 255:red, 0; green, 0; blue, 0 }  ][fill={rgb, 255:red, 0; green, 0; blue, 0 }  ][line width=0.75]      (0, 0) circle [x radius= 3.35, y radius= 3.35]   ;
	\draw [shift={(350,100)}, rotate = 32.01] [color={rgb, 255:red, 0; green, 0; blue, 0 }  ][fill={rgb, 255:red, 0; green, 0; blue, 0 }  ][line width=0.75]      (0, 0) circle [x radius= 3.35, y radius= 3.35]   ;
	%Straight Lines [id:da1834195966154788] 
	\draw    (350,100) -- (430,130) ;
	\draw [shift={(430,130)}, rotate = 20.56] [color={rgb, 255:red, 0; green, 0; blue, 0 }  ][fill={rgb, 255:red, 0; green, 0; blue, 0 }  ][line width=0.75]      (0, 0) circle [x radius= 3.35, y radius= 3.35]   ;
	\draw [shift={(350,100)}, rotate = 20.56] [color={rgb, 255:red, 0; green, 0; blue, 0 }  ][fill={rgb, 255:red, 0; green, 0; blue, 0 }  ][line width=0.75]      (0, 0) circle [x radius= 3.35, y radius= 3.35]   ;

	% Text Node
	\draw (311,92.4) node [anchor=north west][inner sep=0.75pt]    {$\cdots $};
	% Text Node
	\draw (110,12.4) node [anchor=north west][inner sep=0.75pt]  [font=\footnotesize]  {$1^{o} \ \text{num estratto}$};
	% Text Node
	\draw (93,202.4) node [anchor=north west][inner sep=0.75pt]  [font=\footnotesize]  {$n_{1} =90\ \text{possibilità}$};
	% Text Node
	\draw (194,92.4) node [anchor=north west][inner sep=0.75pt]    {$\vdots $};
	% Text Node
	\draw (274,122.4) node [anchor=north west][inner sep=0.75pt]    {$\vdots $};
	% Text Node
	\draw (424,92.4) node [anchor=north west][inner sep=0.75pt]    {$\vdots $};
	% Text Node
	\draw (210,12.4) node [anchor=north west][inner sep=0.75pt]  [font=\footnotesize]  {$2^{o} \ \text{num estratto}$};
	% Text Node
	\draw (350,12.4) node [anchor=north west][inner sep=0.75pt]  [font=\footnotesize]  {$6^{o} \ \text{num estratto}$};
	% Text Node
	\draw (213,202.4) node [anchor=north west][inner sep=0.75pt]  [font=\footnotesize]  {$n_{2} =89\ \text{possibilità}$};
	% Text Node
	\draw (353,202.4) node [anchor=north west][inner sep=0.75pt]  [font=\footnotesize]  {$n_{6} =84\ \text{possibilità}$};

	\end{tikzpicture}

	L'evento $A=$ \event{indovino i $6$ numeri estratti} è visibile, sull'albero, come una particolare diramazione. Dato che però non ci interessa l'ordine di estrazione consideriamo tutte le possibili permutazioni di $A$, i.e. $| A| =6!$.
	\item Sia $B=$ \event{indovino $5$ dei $6$ numeri estratti e in più un numero jolly}. Determiniamo ogni esito possibile $B$ come segue:
	\begin{enumerate}
		\item quali sottoinsiemi di $5$ numeri dei $6$ azzecco: $\binom{6}{5} =\frac{6!}{5!\ 1!} =6$ modi,
		\item azzecco il jolly: $1$ modo.
	\end{enumerate}

	\begin{equation*}
		| B| =6\implies \PP(B) =\frac{| B| }{| \Omega | } =\frac{6}{\binom{90}{6}} \approx 9.6\times 10^{-9}
	\end{equation*}

	Alternativamente, possiamo pensare in quanti possiamo sostituire il numero "sbagliato" con il jolly:

	\tikzset{every picture/.style={line width=0.75pt}} %set default line width to 0.75pt        

	\begin{tikzpicture}[x=0.75pt,y=0.75pt,yscale=-1,xscale=1]
	%uncomment if require: \path (0,77); %set diagram left start at 0, and has height of 77

	%Straight Lines [id:da09449098783720666] 
	\draw    (50,60) -- (290,60) ;
	%Shape: Circle [id:dp40594633380564016] 
	\draw  [draw opacity=0][fill={rgb, 255:red, 155; green, 155; blue, 155 }  ,fill opacity=1 ] (60,40) .. controls (60,34.48) and (64.48,30) .. (70,30) .. controls (75.52,30) and (80,34.48) .. (80,40) .. controls (80,45.52) and (75.52,50) .. (70,50) .. controls (64.48,50) and (60,45.52) .. (60,40) -- cycle ;
	%Shape: Circle [id:dp6641428274341103] 
	\draw  [draw opacity=0][fill={rgb, 255:red, 0; green, 0; blue, 0 }  ,fill opacity=1 ] (100,40) .. controls (100,34.48) and (104.48,30) .. (110,30) .. controls (115.52,30) and (120,34.48) .. (120,40) .. controls (120,45.52) and (115.52,50) .. (110,50) .. controls (104.48,50) and (100,45.52) .. (100,40) -- cycle ;
	%Shape: Circle [id:dp7905252015750659] 
	\draw  [draw opacity=0][fill={rgb, 255:red, 155; green, 155; blue, 155 }  ,fill opacity=1 ] (140,40) .. controls (140,34.48) and (144.48,30) .. (150,30) .. controls (155.52,30) and (160,34.48) .. (160,40) .. controls (160,45.52) and (155.52,50) .. (150,50) .. controls (144.48,50) and (140,45.52) .. (140,40) -- cycle ;
	%Shape: Circle [id:dp33901503243236997] 
	\draw  [draw opacity=0][fill={rgb, 255:red, 155; green, 155; blue, 155 }  ,fill opacity=1 ] (180,40) .. controls (180,34.48) and (184.48,30) .. (190,30) .. controls (195.52,30) and (200,34.48) .. (200,40) .. controls (200,45.52) and (195.52,50) .. (190,50) .. controls (184.48,50) and (180,45.52) .. (180,40) -- cycle ;
	%Shape: Circle [id:dp9005911918848624] 
	\draw  [draw opacity=0][fill={rgb, 255:red, 155; green, 155; blue, 155 }  ,fill opacity=1 ] (220,40) .. controls (220,34.48) and (224.48,30) .. (230,30) .. controls (235.52,30) and (240,34.48) .. (240,40) .. controls (240,45.52) and (235.52,50) .. (230,50) .. controls (224.48,50) and (220,45.52) .. (220,40) -- cycle ;
	%Shape: Circle [id:dp4207926820924768] 
	\draw  [draw opacity=0][fill={rgb, 255:red, 155; green, 155; blue, 155 }  ,fill opacity=1 ] (260,40) .. controls (260,34.48) and (264.48,30) .. (270,30) .. controls (275.52,30) and (280,34.48) .. (280,40) .. controls (280,45.52) and (275.52,50) .. (270,50) .. controls (264.48,50) and (260,45.52) .. (260,40) -- cycle ;
	%Straight Lines [id:da9159657244031865] 
	\draw    (50,20) -- (50,60) ;
	%Straight Lines [id:da7573967278607023] 
	\draw    (90,20) -- (90,60) ;
	%Straight Lines [id:da7463985181158719] 
	\draw    (130,20) -- (130,60) ;
	%Straight Lines [id:da9290196388669603] 
	\draw    (170,20) -- (170,60) ;
	%Straight Lines [id:da06680789769911977] 
	\draw    (210,20) -- (210,60) ;
	%Straight Lines [id:da40034730485648806] 
	\draw    (250,20) -- (250,60) ;
	%Straight Lines [id:da4056303578838951] 
	\draw    (290,20) -- (290,60) ;
	%Shape: Circle [id:dp8217285666442] 
	\draw  [draw opacity=0][fill={rgb, 255:red, 0; green, 0; blue, 0 }  ,fill opacity=1 ] (330,20) .. controls (330,14.48) and (334.48,10) .. (340,10) .. controls (345.52,10) and (350,14.48) .. (350,20) .. controls (350,25.52) and (345.52,30) .. (340,30) .. controls (334.48,30) and (330,25.52) .. (330,20) -- cycle ;
	%Shape: Circle [id:dp043484895799565715] 
	\draw  [draw opacity=0][fill={rgb, 255:red, 155; green, 155; blue, 155 }  ,fill opacity=1 ] (330,50) .. controls (330,44.48) and (334.48,40) .. (340,40) .. controls (345.52,40) and (350,44.48) .. (350,50) .. controls (350,55.52) and (345.52,60) .. (340,60) .. controls (334.48,60) and (330,55.52) .. (330,50) -- cycle ;

	% Text Node
	\draw (361,10) node [anchor=north west][inner sep=0.75pt]   [align=left] {numero sbagliato};
	% Text Node
	\draw (361,40) node [anchor=north west][inner sep=0.75pt]   [align=left] {numero giusto};

	\end{tikzpicture}

	Il numero sbagliato può essere uno qualsiasi dei $6$ numeri che ho scelto, quindi ho $6$ modi possibili di sostituire il jolly al numero sbagliato.
	\item L'evento è dato dall'unione disgiunta:
	\begin{equation*}
		A\cup B\implies \PP(A\cup B) =\PP(A) +\PP(B) =\frac{7}{\binom{90}{6}} \approx 1.12\times 10^{-8}
	\end{equation*}
\end{enumerate}

\Soluzione

Poiché \textit{distinti}, possiamo enumerare gli oggetti dell'urna da $1$ a $n$. Introduciamo gli eventi $A_{j}^{m} =$ \event{esce l'oggetto $m$ all'estrazione $j$}:
\begin{equation*}
	A_{j}^{m} =\{\omega =(\omega_{1} ,\dots ,\omega_{k}) \in \Omega :\omega_{j} =m\} ,\ \ \ \ m\in \{1,\dots,n\} ,\ \ \ \ j\in \{1,\dots,k\}
\end{equation*}
\begin{enumerate}
	\item 
	\begin{enumerate}
		\item $\Omega $ è lo spazio campionario delle disposizioni \textit{con} ripetizione:
		\begin{align*}
			\Omega  & =\{1,\dots,n\}^{k}\\
			 & =\{\omega =(\omega_{1} ,\dots ,\omega_{k}) \in \Omega :\omega_{i} \in \{1,\dots,n\} ,i\in \{i,\dots,k\}\}\\
			 & \implies | \Omega | =n^{k}
		\end{align*}
		\item $(\impliedby)$ Ipotesi:
		\begin{enumerate}
			\item le $k$ estrazioni sono indipendenti, i.e.
			\begin{equation*}
				\PP\left(\bigcap_{j=1}^{k} A_{j}^{\omega_{j}}\right) =\prod_{j=1}^{k}\PP\left(A_{j}^{\omega_{j}}\right)
			\end{equation*}
			\item per ciascuna estrazione gli $n$ risultati possibili sono equiprobabili, i.e.
			\begin{equation*}
				\PP\left(A_{j}^{m}\right) =1/n\ \ \ \ \forall m,\forall j
			\end{equation*}
		\end{enumerate}
		Tesi da dimostrare: $\PP(\{\omega \}) =\frac{1}{| \Omega | } =\frac{1}{n^{k}} ,\forall \omega \in \Omega $.

		Abbiamo visto che i possibili evento $\omega \in \Omega $ sono della forma $\omega =(\omega_{1} ,\dots ,\omega_{k})$ con $\omega_{i} \in \{1,\dots,n\}$ e $i\in \{1,\dots,k\}$, allora
		\begin{align*}
			\PP(\{\omega \}) & =\PP(\{(\omega_{1} ,\dots ,\omega_{k})\}) & \\
			 & =\PP\left(\bigcap_{j=1}^{k} A_{j}^{\omega_{j}}\right) & \text{significato di} \ \omega \\
			 & =\prod_{j=1}^{k}\PP\left(A_{j}^{\omega_{j}}\right) & \text{indipendenza}\\
			 & =\prod_{j=1}^{k}\frac{1}{n} & \text{estrazioni equiprobabili}\\
			 & =\frac{1}{n^{k}} =\frac{1}{| \Omega | } & 
		\end{align*}

		$(\implies)$ esercizio.
		\item Sia $A=$ \event{estraggo gli oggetti $\{x_{1},\dots,x_{k}\}$ senza riguardo per l'ordine}. Dato che abbiamo la possibilità di ripetizioni, i casi possibili oscillano tra due estremi:
		\begin{enumerate}
			\item $\{x_{1},\dots,x_{k}\}$ è composto da oggetti tutti uguali: no permutazioni! (o meglio permutazioni di $k$ oggetti indistinguibili) $k!/k!=1$ possibilità. Quindi $| A| =1$.
			\item $\{x_{1},\dots,x_{k}\}$ è composto da oggetti tutti distinti: dato che non teniamo conto dell'ordine, dobbiamo tenere conto di tutte le possibili permutazioni. Quindi $| A| =k!$.
		\end{enumerate}

		Pertanto, $\frac{1}{n^{k}} \leq \PP(A) \leq \frac{k!}{n^{k}}$, nel \textit{mezzo} ci sono i casi con sottoclassi di oggetti uguali tra loro.
	\end{enumerate}
	\item 
	\begin{enumerate}
		\item $\Omega $ è lo spazio campionario delle disposizioni \textit{senza} ripetizione:
		\begin{align*}
			\Omega = & \{\omega =(\omega_{1} ,\dots ,\omega_{k}) \in \Omega :\omega_{i} \in \{1,\dots,n\} ,i\in \{i,\dots,k\}\\
			 & \ \ \ \ \omega_{i} \neq \omega_{j} \ \forall i\neq j\}\\
			\implies  & | \Omega | =n(n-1) \cdots (n-k+1) =\frac{n!}{(n-k) !}
		\end{align*}
		\item $(\implies)$ Analogamente al caso $1$, dobbiamo provare che
		\begin{equation*}
			\forall \omega \in \Omega \ \ \ \ \PP(\omega) =\frac{1}{| \Omega | } =\frac{1}{n(n-1) \cdots (n-k+1)}
		\end{equation*}

		Per ipotesi, sappiamo che ogni esito di ciascuna estrazione è equiprobabile \textit{condizionatamente} rispetto ai risultati delle passate estrazioni. Sfruttiamo il seguente risultato teorico \textit{(J.P. 3.3)} che connette indipendenza con probabilità condizionata per un numero finito di eventi.

		Sia $(\Omega,\Ac,\PP)$ uno spazio di probabilità e siano $A_{1} ,\dots ,A_{n} \in \Ac$ tale che $\PP(A_{1} \cap A_{2} \cap \cdots \cap A_{n-1})  >0$. Allora $\PP\left(\bigcap_{n} A_{n}\right) =\PP(A_{1})\PP(A_{2} |A_{1}) \cdots \PP(A_{n} |A_{1} \cap \cdots \cap A_{n-1})$.

		\begin{oss}
			Questo è esattamente il caso che ci interessa, perché quando estraiamo la $k$-esima pallina dobbiamo tenere conto del fatto che ne abbiamo estratte $k-1$ prima. Abbiamo allora:
			\begin{align*}
				\PP(\{\omega \}) & =\PP(\{\omega_{1} ,\dots ,\omega_{k}\})\\
				 & =\PP\left(\bigcap_{j=1}^{k} A_{j}^{\omega_{k}}\right)\\
				 & =\PP\left(A_{1}^{\omega_{1}}\right)\PP\left(A_{2}^{\omega_{2}} |A_{1}^{\omega_{1}}\right) \cdots \PP\left(A_{k}^{\omega_{k}} |A_{1}^{\omega_{1}} \cap \cdots \cap A_{k-1}^{\omega_{k-1}}\right)\\
				 & =\frac{\left| A_{1}^{\omega_{1}}\right| }{| \Omega | }\frac{\left| A_{2}^{\omega_{2}} \cap A_{1}^{\omega_{1}}\right| }{\left| A_{1}^{\omega_{1}}\right| } \cdots \frac{| A_{k}^{\omega_{k}} \cap \cdots \cap A_{1}^{\omega_{1}} |}{\left| A_{k-1}^{\omega_{k-1}} \cap \cdots \cap A_{1}^{\omega_{1}}\right| }\\
				 & =\frac{1}{n} \ \frac{1}{n-1} \cdots \frac{1}{n-(k-1)}\\
				 & =\frac{1}{n(n-1) \cdots (n-k+1)} =\frac{1}{| \Omega | }
			\end{align*}
		\end{oss}

		$(\impliedby)$ Esercizio.
		\item Sia $A$ come nel primo punto. Non abbiamo ripetizioni, perché stiamo considerando estrazioni senza reimmissione, dunque
		\begin{equation*}
			| A| =k!,\ \ \ \ \PP(A) =\frac{| A| }{| \Omega | } =\frac{k!(n-k) !}{n!} =\frac{1}{\binom{n}{k}}
		\end{equation*}
	\end{enumerate}
	\item 
	\begin{enumerate}
		\item $\Omega $ è lo spazio campionario delle combinazioni
		\begin{equation*}
			\Omega =\{\omega \subset \{1,\dots ,n\} :| \omega | =k\} ,\ \ \ \ | \Omega | =\binom{n}{k}
		\end{equation*}
		\item Sia $A$ come sopra, allora deve essere $\PP(A) =\frac{1}{\binom{n}{k}}$.
	\end{enumerate}
\end{enumerate}

\textbf{Osservazioni.}
\begin{itemize}
	\item Deve risultare che le probabilità $\PP(A)$ dei punti $2c$ e $3b$ siano uguali (confronta con es $3b$ e es $9a$ [OSS]).
	\item Importanza di questo esercizio.
	\begin{itemize}
		\item Serve per capire quando la probabilità uniforme è il modello corretto.
		\item La probabilità uniforme \textit{dipende} da $\Omega $, confronta punti $1b$ e $2b$.
	\end{itemize}
\end{itemize}

\Soluzione

\begin{equation*}
	\begin{drcases}
		7\ \text{blu}\\
		3\ \text{rosse}
	\end{drcases}
	\ 10\ \text{palline}\rightarrow \text{ne estraggo} \ 4
\end{equation*}
\begin{enumerate}
	\item \textit{Estrazione con reimmissione.}

	In questo caso abbiamo:
	\begin{equation*}
	\Omega =\{(\omega_{1} ,\dots ,\omega_{4}) :\omega_{i} \in \{1,\dots ,10\}\}
	\end{equation*}

	Spazio campionario delle disposizioni con ripetizione
	\begin{equation*}
	| \Omega | =10^{4}
	\end{equation*}

	Sia $A$ l'evento "$2$ palline rosse su $4$ estratte". Quanti modi ho di estrarre $2$ palline rosse e $2$ blu? Ho $3^{2} \cdot 7^{2}$ possibilità di scelta, ma poi dobbiamo considerare tutte le possibili permutazioni $\binom{4}{2} =\frac{4!}{2!2!}$.
	\begin{equation*}
		\implies | A| =\binom{4}{2} 3^{2} 7^{2}\implies \PP(A) =\frac{| A| }{| \Omega | } =\binom{4}{2}\frac{3^{2} 7^{2}}{10^{4}} \approx 0.2646.
	\end{equation*}
	\item \textit{Estrazione senza reimmissione.}

	Svolgiamo l'esercizio in due diversi modi, introducendo $\Omega $ in due diversi modi.
	\begin{enumerate}
		\item Metodo $1$
		\begin{equation*}
			\Omega =\{(\omega_{1} ,\dots ,\omega_{4}) :\omega_{i} \in \{1,\dots ,10\} ,\omega_{i} \neq \omega_{j} \ \forall i\neq j\}
		\end{equation*}

		Spazio campionario delle disposizioni senza ripetizione
		\begin{equation*}
			| \Omega | =10\cdot 9\cdot 8\cdot 7=\frac{10!}{6!}
		\end{equation*}

		Sia $A$ l'evento "$2$ palline rosse su $4$ estratte". Quanti modi abbiamo di estrarre $2$ palline rosse e $2$ blue? In totale abbiamo $(3\cdot 2) \cdot (7\cdot 6)$ possibilità di ottenere questa configurazione, però dobbiamo considerare tutti i possibili modi in cui le $2$ palline rosse possono disporsi su $4$ ($=$ palline estratte) posti i.e. dobbiamo considerare tutte le $\binom{4}{2}$ possibili configurazioni.

		\begin{oss}
			Possiamo pensare anche pensare il tutto come il modo di permutare le $4$ palline, tenendo conto che $2$ rosse sono indistinguibili e $2$ blu sono indistinguibili.
			\begin{gather*}
				\binom{4}{2} =\frac{4!}{2!2!} =\frac{4\cdot 3\cdot 2}{4} =6\ \text{possibili modi}\\
				\Downarrow \\
				| A| =\binom{4}{2} \cdot (3\cdot 2) \cdot (7\cdot 6)\\
				\PP A) =\frac{| A| }{| \Omega | } =\frac{\binom{4}{2} \cdot (3\cdot 2) \cdot (7\cdot 6)}{10\cdot 9\cdot 8\cdot 7} \approx 0.3
			\end{gather*}
		\end{oss}

		\item Metodo $2$
		\begin{equation*}
			\Omega =\{\omega \subset \{1,\dots ,10\} \ |\ | \omega | =4\}
		\end{equation*}
		Spazio campionario delle combinazioni
		\begin{equation*}
			| \Omega | =\binom{10}{4}
		\end{equation*}
		Sia $A$ l'evento "$2$ palline rosse su $4$ estratte". Quanti modi ho di estrarre $2$ palline rosse e $2$ blu? Abbiamo
		\begin{gather*}
			\binom{3}{2} \ \text{modi di scegliere le palline rosse}\\
			\binom{7}{2} \ \text{modi di scegliere le palline blu}\\
			\implies | A| =\binom{3}{2}\binom{7}{2}\\
			\PP(A) =\frac{| A| }{| \Omega | } =\frac{\binom{3}{2}\binom{7}{2}}{\binom{10}{4}} \approx 0.3
		\end{gather*}
	\end{enumerate}
	\item \textit{Estrazione simultanea}

	Questo caso si tratta esattamente come il metodo $2$ del punto precedente.
\end{enumerate}

\Soluzione

Manca.

\Soluzione

Manca.

\Soluzione

Manca.

\Soluzione

Manca.

\Soluzione

Manca.

\Soluzione

Introduciamo $\Omega $ come lo spazio delle combinazioni di $52$ elementi di classe $5$, i.e.
\begin{equation*}
	\Omega =\{\omega \subset \{1,\dots ,52\} \ |\ | \omega | =5\} ,\ \ \ \ | \Omega | =\binom{52}{5}
\end{equation*}
\begin{enumerate}
	\item Sia $A$ l'evento "esce un full" (esempio di full: $3\varheartsuit \ 3\spadesuit \ 3\vardiamondsuit \ 6\varheartsuit \ 6\clubsuit $)\begin{equation*}
	| A| =13\cdot \binom{4}{3} \cdot 12\cdot \binom{4}{2}
	\end{equation*}
	\begin{enumerate}
		\item ho $13$ possibilità per scegliere il tipo del tris (es. scelgo il $3$)
		\item ho $\binom{4}{3}$ possibilità di scegliere il seme (posso pensarlo come ad un'urna in cui ho $4$ semi e ne estraggo $3$, es. estratto $\varheartsuit ,\spadesuit ,\vardiamondsuit $)
		\item ho $12$ possibilità per scegliere il tipo della coppia (es. scelgo il $6$)

		\begin{oss}
			In questo caso considero $13\cdot 12$ possibilità per il tipo e NON $\binom{13}{2}$ perché i casi $(3\varheartsuit \ 3\spadesuit \ 3\vardiamondsuit \ 6\varheartsuit \ 6\clubsuit) \neq (6\varheartsuit \ 6\spadesuit \ 6\vardiamondsuit \ 3\varheartsuit \ 3\clubsuit)$, differenza con il punto successivo
		\end{oss}

		\item ho $\binom{4}{2}$ possibilità di scegliere il seme (es. $\varheartsuit ,\clubsuit $)
		\begin{equation*}
			\implies \PP(A) =\frac{| A| }{| \Omega | } =\frac{13\cdot \binom{4}{3} \cdot 12\cdot \binom{4}{2}}{\binom{52}{5}} \approx 0.0014
		\end{equation*}
	\end{enumerate}
	\item Sia $B$ l'evento "esce una doppia coppia" (esempio di doppia coppia: $3\varheartsuit \ 3\vardiamondsuit \ 7\clubsuit \ 7\varheartsuit \ 11\spadesuit $)
	\begin{equation*}
		| B| =\binom{13}{2}\binom{4}{2}\binom{4}{2} \cdot 11\cdot 4
	\end{equation*}
	\begin{enumerate}
		\item ho $\binom{13}{2}$ modi per scegliere i due tipi (es. escono il $3$ e il $7$)
		\item ho $\binom{4}{2}$ modi di scegliere i semi per la prima coppia
		\item ho $\binom{4}{2}$ modi di scegliere i semi per la seconda coppia
		\item mi rimangono $11$ possibilità tra cui scegliere il \textit{numero/figura} dell'ultima carta
		\item posso scegliere uno qualunque dei $4$ semi possibili per l'ultima carta
		\begin{equation*}
			\implies \PP(B) =\frac{| B| }{| \Omega | } =\frac{\binom{13}{2}\binom{4}{2}\binom{4}{2} \cdot 11\cdot 4}{\binom{52}{5}} \approx 0.0475
		\end{equation*}
	\end{enumerate}
	\item [punto e] Sia $C$ l'evento "esce una scala".
	\begin{gather*}
		| C| =10\cdot 4^{5}\\
		A\ 2\ 3\ 4\ 5\ 6\ 7\ 8\ 9\ 10\ |\ J\ Q\ K
	\end{gather*}
	\begin{enumerate}
		\item $10$ possibilità $\rightarrow $ blocco la prima carta: ho $10$ possibilità, non posso far partire la scala da $J,Q,K$.
		\item per ogni carta ho $4$ possibilità sui semi
		\begin{equation*}
			\implies \PP(C) =\frac{| C| }{| \Omega | } =\frac{10\cdot 4^{5}}{\binom{52}{5}}
		\end{equation*}
	\end{enumerate}
	\item [punto h] Sia $D$ l'evento "esce un tris" (esempio di tris: $3\varheartsuit \ 3\vardiamondsuit \ 3\spadesuit \ 4\vardiamondsuit \ 5\varheartsuit $)
	\begin{equation*}
		| D| =13\cdot \binom{4}{3} \cdot \binom{12}{2} \cdot 4\cdot 4
	\end{equation*}
	\begin{enumerate}
		\item ho $13$ possibilità per il tipo del tris (es. esce il $3$)
		\item ho $\binom{4}{3}$ possibilità per scegliere il seme del tris
		\item ho $\binom{12}{2}$ possibilità per scegliere il tipo delle altre carte
		\item ho $4$ possibilità per scegliere il seme della quarta e della quinta carta
		\begin{equation*}
			\implies \PP(D) =\frac{| D| }{| \Omega | } =\frac{13\cdot \binom{4}{3} \cdot \binom{12}{2} \cdot 4\cdot 4}{\binom{52}{5}}
		\end{equation*}
	\end{enumerate}
\end{enumerate}

\chapter{Probabilità condizionale e indipendenza}
%!TEX root = ../main.tex

\chapter{Probabilità condizionale e indipendenza}

\ParteEsercizi

\begin{defn}
	Sia $(\Omega,\Ac,\PP)$ uno spazio di probabilità. Siano $A,B\in \Ac$ con $\PP(B)  >0$. Si definisce \textbf{probabilità condizionata} di $A$ dato $B$, $\PP(A\mid B)$, la quantità
	\begin{equation*}
		\PP(A\mid B) =\frac{\PP(A\cap B)}{\PP(B)}
	\end{equation*}
\end{defn}
\textbf{Proprietà.}

Siano $A,B\in \Ac$ e sia $\{E_{n}\}_{n\in \NN} \subseteq \Ac$ una partizione discreta di $\Omega $:
\begin{enumerate}
	\item se $\PP(A) ,\PP(B)  >0$ allora $\PP(A\mid B) =\frac{\PP(B\mid A)\PP(A)}{\PP(B)}$.
	\item $\PP(A) =\sum\limits_{n}\PP(A\cap E_{n})$
	\item \textit{Probabilità totali}: se $\PP(E_{n})  >0,\forall n\in \NN$ allora $\PP(A) =\sum\limits_{n}\PP(A\mid E_{n})\PP(E_{n})$.
	\item \textit{Bayes}: se $\PP(A) ,\PP(E_{n})  >0,\forall n\in \NN$ allora $\PP(E_{n} \mid A) =\frac{\PP(A\mid E_{n})\PP(E_{n})}{\sum\limits_{n}\PP(A\mid E_{n})\PP(E_{n})}$.
\end{enumerate}
\begin{defn}
$A$ e $B$ sono indipendenti $(A\indep B)$ se $\PP(A\cap B) =\PP(A)\PP(B)$.
\end{defn}
\textbf{Proprietà.}
\begin{enumerate}
\item $A\indep B\iff \PP(A\mid B) =\PP(A)$.
\item $A\indep B\iff A\indep B\complementary \iff A\complementary \indep B\iff A\complementary \indep B\complementary$.
\end{enumerate}
\begin{defn}
	Sia $(\Omega,\Ac,\PP)$. Dati $\{A_{i}\}_{i\in I}$ gli eventi $A_{i}$ si dicono indipendenti se $\forall $ sottoinsieme di indici $J\subset I$ si ha
	\begin{equation*}
		\PP\left(\bigcap_{i\in I} A_{i}\right) =\prod_{i\in J}\PP(A_{i})
	\end{equation*}
\end{defn}

\Esercizio{}

Una roulette semplificata è formata da $12$ numeri che sono \emph{rosso} $(R)$ e \emph{nero} $(N)$ in base allo schema seguente:
\begin{equation*}
	\begin{array}{ c c c c c c c c c c c c }
		1 & 2 & 3 & 4 & 5 & 6 & 7 & 8 & 9 & 10 & 11 & 12\\
		R & R & N & N & R & N & N & R & N & N  & R  & R
	\end{array}
\end{equation*}
Siano
\begin{itemize}
	\item $A=$ \event{esce un numero pari},
	\item $B=$ \event{esce un numero rosso},
	\item $C=$ \event{esce un numero $\leq 3$},
	\item $D=$ \event{esce un numero $\leq 6$},
	\item $E=$ \event{esce un numero $\leq 8$},
	\item $F=$ \event{esce un numero dispari $\leq 3$}.
\end{itemize}
Calcolare le seguenti probabilità condizionali:
\begin{itemize}
	\item $\PP(A\mid C) ,\PP(C\mid A) ,\PP(B\mid C) ,\PP(A\mid F) ,\PP(D\mid F) ,\PP(A\mid B) ,\PP(B\mid A)$.
\end{itemize}
Stabilire quindi se:
\begin{itemize}
	\item gli eventi $A$, $B$ e $D$ sono a $2$ a $2$ indipendenti;
	\item $A,B,D$ costituiscono una famiglia di eventi indipendenti;
	\item $A,B,E$ costituiscono una famiglia di eventi indipendenti;
	\item $A,C,E$ costituiscono una famiglia di eventi indipendenti;
	\item $F$ è indipendente da $A$ e da $D$.
\end{itemize}

\Esercizio{}

Si consideri il lancio di un dado, ripetuto due volte.
\begin{enumerate}
	\item Si considerino i seguenti eventi:
	\begin{itemize}
		\item $A=$ \event{numero dispari sul primo lancio},
		\item $B=$ \event{numero dispari sul secondo lancio},
		\item $C=$ \event{la somma dei risultati dei due lanci è dispari}.
	\end{itemize}
	Gli eventi $A$, $B$ e $C$ sono indipendenti?
	\item Si considerino ora gli eventi:
	\begin{itemize}
		\item $E=$ \event{il risultato del secondo lancio è $1$, $2$ o $5$},
		\item $F=$ \event{il risultato del secondo lancio è $4$, $5$ o $6$},
		\item $G=$ \event{la somma dei risultati dei due lanci è $9$}.
	\end{itemize}
	\item Gli eventi $E$, $F$ e $G$ sono indipendenti?
\end{enumerate}

\Esercizio{}

I componenti prodotti da una ditta possono avere due tipi di difetti con percentuali del $3\%$ e del $7\%$ rispettivamente e in modo indipendente l'uno dall'altro. Qual è la probabilità che un componente
\begin{enumerate}
	\item presenti entrambi i difetti?
	\item sia difettoso?
	\item presenti il primo difetto, sapendo che è difettoso?
	\item presenti uno solo dei difetti, sapendo che è difettoso?
\end{enumerate}

\Esercizio{}

Si consideri una popolazione in cui una persona su $100$ abbia una certa malattia. Un test è disponibile per diagnosticare tale malattia. Si supponga che il test non sia perfetto, in quanto esso risulta positivo (ovvero indica la presenza della malattia) nel $5\%$ dei casi quando è effettuato su persone sane, mentre risulta negativo (indicando l'assenza della malattia) nel $2\%$ dei casi quando è effettuato su persone malate. Si calcolino le probabilità che
\begin{enumerate}
	\item una persona sia malata se il test risulta positivo;
	\item una persona sia sana se il test risulta negativo.
\end{enumerate}

\Esercizio{}

Si consideri un mazzo di carte $32$ carte da Poker, identificate dal seme (cuori $\varheartsuit $, quadri $\vardiamondsuit $, fiori $\clubsuit $, picche $\spadesuit $) e dal tipo (un numero da $7$ a $10$ oppure $J,Q,K,A$). Se sapete di avere in mano l'asso di cuori, qual è la probabilità di avere una scala reale massima, ovvero $10,J,Q,K,A$ dello stesso seme?

\Esercizio{}

Un'urna contiene tre carte: una di esse ha entrambi i lati neri, una entrambi i lati bianchi, l'ultima ha un lato nero e uno bianco. Una carta viene estratta e se ne guarda uno solo dei lati: è nero. Qual è la probabilità che il secondo lato sia nero?

\Esercizio{}

Nel Gioco del Lotto ad ogni estrazione settimanale $5$ numeri vengono estratti simultaneamente da un'urna che contiene $90$ palline numerate da $1$ a $90$. Si calcolino la probabilità di estrarre:
\begin{enumerate}
	\item il $15$,
	\item il $15$ sapendo che non è uscito nelle ultime $41$ estrazioni,
	\item almeno un $15$ su $42$ estrazioni.
\end{enumerate}

Si considerino ora le prime $42$ estrazioni.
\begin{enumerate}
	\item Se il $15$ è estratto una volta, con quale probabilità è uscito alla prima?
	\item Se il $15$ è estratto due volte, con quale probabilità è uscito alla prima e alla seconda?
\end{enumerate}

\Esercizio{(Paradosso di Monty Hall)}

In un gioco televisivo viene messo in palio $1$ milione di euro. Per vincerlo il concorrente deve indovinare quale fra tre pacchi è quello che contiene l'assegno. Il concorrente sceglie a caso un pacco.
\begin{enumerate}
	\item Quanto vale la probabilità che il pacco scelto contenga il premio?
\end{enumerate}
A questo punto sul banco son rimasti due pacchi ed il conduttore, che ne conosce il contenuto, ne apre uno vuoto, offrendo al concorrente la possibilità di cambiare il proprio pacco con quello rimanente. Calcolare la probabilità di vincere usando una delle seguenti strategie:
\begin{enumerate}
	\item conservando il pacco scelto inizialmente,
	\item cambiando pacco,
	\item giocando a testa o croce fra le due strategie.
\end{enumerate}
Da assidui spettatori sapete che, quando i due pacchi rimasti sul banco sono entrambi vuoti, quello di destra viene aperto con frequenza $p$.
\begin{enumerate}
	\item Quanto vale la probabilità che il pacco scelto inizialmente contenga il premio, sapendo che il conduttore ha aperto il pacco di destra?
\end{enumerate}
Per la variante domenicale del gioco vengono cambiate le regole: il conduttore non sa quale pacco contenga l'assegno e, dopo la scelta del concorrente, apre a caso uno dei due pacchi rimanenti. Se il conduttore trova l'assegno, questo viene devoluto in beneficenza, se non lo trova, il gioco procede offrendo al concorrente di scegliere nuovamente un pacco fra i due rimanenti.
\begin{enumerate}
	\item Quanto vale la probabilità che il pacco scelto inizialmente contenga il premio, sapendo che il conduttore ha aperto il pacco rimanente di destra e che questo si è rivelato vuoto?
\end{enumerate}

\Esercizio{fatto}

Un dado non truccato viene lanciato più volte.
\begin{enumerate}
	\item Quanti lanci devono essere fatti per avere una probabilità maggiore di $\frac{1}{2}$ di ottenere almeno un $6$?
	\item Calcolare la probabilità che esca sempre $6$.
	\item Calcolare la probabilità che esca sempre $6$ dal decimo lancio in poi.
	\item Calcolare la probabilità che escano infiniti $6$.
	\item Gli eventi \event{infiniti $6$} e \event{sempre $6$} sono compatibili? Sono indipendenti?
	\item Gli eventi \event{infiniti $6$} e \event{sempre $5$} sono compatibili? Sono indipendenti?
\end{enumerate}

\Esercizio{fatto}

Consideriamo infinite prove di Bernoulli indipendenti con probabilità di successo costante $0< p< 1$. Sia quindi $\Omega =\{0,1\}^{\NN}$, sia $\Ac =\sigma (E_{k} \mid k=1,2,\dots)$ con
\begin{equation*}
	E_{k} =\text{successo alla prova} \ k,
\end{equation*}
e sia $\PP$ tale che $\{E_{k}\}_{k\in \NN}$ risulti una famiglia di eventi indipendenti con $\PP(E_{k}) =p$ per ogni $k\in \NN$.
\begin{enumerate}
	\item Quanto vale la probabilità di ottenere solo insuccessi nelle prime $n$ prove?
	\item Quante prove dovete fare per ottenere almeno un successo con probabilità maggiore di $\frac{1}{2}$?
	\item Se nelle prime $n$ prove avete ottenuto almeno un successo, con quale probabilità l'$n$-esima prova è stata un successo?
	\item Quanto vale la probabilità di ottenere solo insuccessi?
	\item Quanto vale la probabilità di ottenere solo insuccessi dalla settima prova in poi?
	\item Quanto vale la probabilità di ottenere infiniti insuccessi?
	\item Quanto vale la probabilità di ottenere definitivamente solo successi? Quanto vale la probabilità di ottenere infiniti successi?
	\item Quanto vale la probabilità di ottenere infiniti successi e infiniti insuccessi?
	\item Gli eventi \event{solo insuccessi dalla settima prova in poi} e \event{$1$ successo nelle prime $10$ prove} sono indipendenti?
	\item Gli eventi \event{solo insuccessi dalla settima prova alla decima} e \event{$1$ successo nelle prime $10$ prove} sono indipendenti?
	\item Gli eventi \event{solo insuccessi dalla settima prova alla decima} e \event{$1$ successo nelle prime $5$ prove} sono indipendenti?
\end{enumerate}
Cosa sarebbe cambiato se avessimo realizzato in un altro spazio di probabilità $(\Omega,\Ac,\PP)$, diverso dallo spazio di Bernoulli, una successione di eventi $E_{n}$ indipendenti e tali che $\PP(E_{n}) =p$, ovvero se avessimo usato un differente spazio di probabilità per rappresentare infinite prove di Bernoulli indipendenti con probabilità di successo $0< p< 1$? Si motivi adeguatamente la risposta.

\Esercizio{fatto}

Un'urna contiene $r$ palline rosse e $b$ palline bianche. Una pallina è estratta a caso dall'urna (in modo tale che ogni pallina abbia la stessa probabilità di essere scelta). Quindi una seconda pallina è estratta ancora a caso dalle palline rimanenti nell'urna. E così via, sempre senza reimmissione. Si calcolino le probabilità che
\begin{enumerate}
	\item la prima pallina estratta sia rossa,
	\item la prima pallina estratta sia rossa e la seconda bianca,
	\item le due palline estratte abbiano colore diverso,
	\item la seconda pallina estratta sia rossa,
	\item la terza pallina estratta sia rossa.
\end{enumerate}

\Esercizio{}

Ci sono due urne, la prima urna contiene due dadi a sei facce equilibrati (non truccati), la seconda urna contiene due dadi a sei facce truccati nel modo seguente: ognuno dei dadi ha tre facce che indicano il numero $6$ e le rimanenti $3$ il numero $5$. Si lancia una moneta equilibrata e se viene testa si prendono i dadi dalla prima urna mentre se viene croce si prendono i dadi dalla seconda urna, poi, in ogni caso si lanciano i dadi.
\begin{enumerate}
	\item Calcolare la probabilità che la somma dei due dadi sia $11$.
	\item Sapendo di aver ottenuto un $11$ lanciando i due dadi, calcolare la probabilità di aver ottenuto croce lanciando la moneta.
\end{enumerate}

\Esercizio{}

Supponiamo di avere due urne ($A$ e $B$), contenenti dieci palline ciascuna, di colore bianco, rosso o nero, secondo questa composizione:
\begin{itemize}
	\item $6$ bianche, $3$ rosse e $1$ nera;
	\item $3$ bianche, $5$ rosse e $2$ nere.
\end{itemize}

Si consideri l'esperimento seguente. Si lancia una moneta non truccata e, se esce testa, si seleziona l'urna $A$, se esce croce, si seleziona l'urna $B$. Dall'urna scelta, si continua a estrarre \textbf{senza reimmissione} una coppia di palline finché non esce almeno una pallina nera.
\begin{enumerate}
	\item Qual è la probabilità che l'esperimento proceda oltre la prima estrazione?
	\item Sapendo che l'esperimento procede oltre la prima estrazione, è più probabile che sia uscita testa o croce?
\end{enumerate}
Se, invece, si effettuano estrazioni \textbf{con reimmissione}:
\begin{enumerate}
	\item Fissato $n\in \NN$, qual è la probabilità che il gioco proceda oltre la $n$-esima estrazione?
	\item Sapendo che il gioco procede oltre la $n$-esima estrazione, qual è la probabilità che sia uscita testa?
\end{enumerate}

\Esercizio{(Urna di Pólya). fatto}

Supponiamo che un'urna contenga $1$ pallina rossa e $1$ pallina bianca. Una pallina è estratta e se ne guarda il colore. Essa viene poi rimessa nell'urna insieme ad una pallina dello stesso colore. Il procedimento è detto di \textit{estrazione con rinforzo}. Sia $R_{i}$ l'evento \event{all'$i$-esima estrazione viene estratta una pallina rossa} e sia $B_{i}$ l'evento \event{all'$i$-esima estrazione viene estratta una pallina bianca}. Si calcolino:
\begin{enumerate}
	\item $\PP(R_{2})$ e $\PP(R_{3})$,
	\item sapendo che la seconda estratta è una pallina rossa, è più probabile che la prima pallina estratta sia stata rossa o che sia stata bianca?
\end{enumerate}

\Esercizio{fatto}

La signora Yellow è stata misteriosamente assassinata. Il Commissario Ferrero ha scoperto che tre giorni prima di essere uccisa la signora aveva minacciato di licenziamento Ambrogio, il maggiordomo. Interrogato da Ferrero, Ambrogio adduce in propria discolpa l'ultima indagine del Giornale della Sera: tra i maggiordomi minacciati di licenziamento, solo $1$ ogni $5000$ uccide la signora per cui lavora. Il Commissario Ferrero non è però persuaso dall'argomento di Ambrogio! Serve valutare la probabilità che una signora venga uccisa dal proprio maggiordomo alla luce di tutto quanto si sa in questo caso: la signora lo ha minacciato di licenziamento e la signora è stata effettivamente uccisa. Ferrero considera quindi gli eventi
\begin{itemize}
	\item $A:$ una signora minaccia di licenziamento il proprio maggiordomo,
	\item $B:$ una signora viene uccisa dal proprio maggiordomo,
	\item $C:$ una signora viene uccisa da qualcuno che non è il proprio maggiordomo.
\end{itemize}
Ferrero a questo punto chiede il vostro aiuto.
\begin{enumerate}
	\item Esprimere in funzione di $A$, $B$ e $C$ la probabilità fornita dal Giornale della Sera e calcolarla.
	\item Quali eventi fra $A$, $B$ e $C$ possono essere ritenuti incompatibili?
	\item Quali eventi fra $A$, $B$ e $C$ possono essere ritenuti indipendenti?
	\item Esprimere in funzione di $A$, $B$ e $C$ l'evento $D:$ una signora viene uccisa.
	\item Esprimere la probabilità condizionata $\PP(B\mid A,D)$ in funzione solo di $\PP(B\mid A)$ e $\PP(C)$.
\end{enumerate}
Negli archivi del Commissariato tuttavia Ferrero non trova $\PP(C)$. Trova però che $1$ donna ogni $100000$ muore assassinata.
\begin{enumerate}
	\item Limitare (dal basso o dall'alto) il valore di $\PP(B\mid A,D)$.
	\item È il caso che Ferrero continui ad indagare su Ambrogio?
\end{enumerate}

\Esercizio{fatto}

Un'urna contiene una pallina rossa ed una bianca. Le palline vengono estratte con reimmissione e con rinforzo delle sole rosse. Precisamente: una pallina viene estratta a caso, se è bianca viene semplicemente rimessa nell'urna, mentre se è rossa la pallina viene rimessa nell'urna assieme ad un'altra rossa. Si calcolino le seguenti probabilità
\begin{enumerate}
	\item che la prima pallina estratta sia bianca,
	\item che la seconda pallina estratta sia bianca, sapendo che la prima estratta è bianca,
	\item che la seconda pallina estratta sia bianca,
	\item che almeno una delle prime due estratte sia bianca,
	\item che la prima pallina sia bianca, sapendo che la seconda è bianca,
	\item che le prime $n$ palline estratte siano tutte bianche,
	\item che le prime $n$ palline estratte siano tutte rosse,
	\item che le palline estratte siano tutte bianche,
	\item che le palline estratte siano tutte dello stesso colore.
\end{enumerate}

\Esercizio{}

Siano $A$, $B$ e $C$ eventi indipendenti e si supponga $\PP(A\cap B) \neq 0$.
\begin{enumerate}
	\item Si mostri che $\PP(C\mid A\cap B) =\PP(C)$.
	\item Si mostri con un esempio che la proprietà non è vera sotto la sola ipotesi che $C$ sia indipendente sia da $A$ sia da $B$.
\end{enumerate}

\Esercizio{}

Sia $I$ un insieme arbitrario. Data una famiglia $(A_{i})_{i\in I}$ di eventi indipendenti, se per ogni $i$ si pone $B_{i} =A_{i}$ oppure $B_{i} =A_{i}\complementary$, allora anche $(B_{i})_{i\in I}$ risulta una famiglia di eventi indipendenti.

\ParteSoluzioni

\Soluzione

Osserviamo innanzitutto che lo spazio campionario è $\Omega =\{1,\dots,12\}$. Osserviamo anche che, per la risoluzione dell'esercizio \textit{non} è necessario specificare $\Omega $. Ci basta sapere (supponiamo) che la roulette sia non truccata e dunque possiamo considerare \textit{probabilità uniforme}. Ricordiamo che:
\begin{enumerate}
	\item 
	\begin{enumerate}
		\item $\PP(A\mid C) =\frac{\PP(A\cap C)}{\PP(C)} =\frac{1/12}{3/12} =\frac{1}{3}$
		\item $\PP(C\mid A) =\frac{\PP(A\cap C)}{\PP(A)} =\frac{1/12}{6/12} =\frac{1}{6}$
		\item $\PP(B\mid C) =\frac{\PP(B\cap C)}{\PP(C)} =\frac{2/12}{3/12} =\frac{2}{3}$
		\item $\PP(A\mid F) =\frac{\PP(A\cap F)}{\PP(F)} =0$ dato che $A\cap F=\emptyset $
		\item $\PP(D\mid F) =\underbrace{\frac{\PP(D\cap F)}{\PP(F)} =\frac{\PP(F)}{\PP(F)}}_{F\subset D\implies D\cap F=F} =1$
		\item $\PP(A\mid B) =\frac{3/12}{6/12}) =\frac{1}{2} =\PP(A)$
		\item $\PP(B\mid A) =\frac{1}{12} =\PP(B)$
	\end{enumerate}
	\item Dal punto precedente sappiamo già che $A,B$ sono indipendenti. Resta da verificare che $A$ e $D$, $B$ e $D$ sono indipendenti.
	\begin{gather*}
		\PP(A\mid D) =\frac{1}{2} =\PP(A)\\
		\PP(B\mid D) =\frac{1}{2} =\PP(B)
	\end{gather*}
	dunque sì.
	\item Gli eventi $A,B,D$ sono indipendenti?
	\begin{align*}
		\PP(A\cap B\cap D) & \questeq \PP(A)\PP(B)\PP(D)\\
		\frac{1}{2} & \neq \frac{1}{2} \cdotp \frac{1}{2} \cdotp \frac{1}{2} =\frac{1}{8}
	\end{align*}
	allora $A,B,D$ non sono indipendenti. L'indipendenza è un concetto più forte dell'indipendenza a coppie.
	\item
	\begin{gather*}
		\PP(A\mid E) =\frac{1}{2} =\PP(A) \implies A\indep E\\
		\PP(B\mid E) =\frac{1}{2} =\PP(B) \implies B\indep E\\
		\underbrace{\PP(A\cap B\cap E)}_{\frac{1}{6}} =\underbrace{\PP(A)\PP(B)\PP(E)}_{\frac{1}{2} \cdotp \frac{1}{2} \cdotp \frac{2}{3}} \implies A\indep B\indep E
	\end{gather*}
	\item $\dots$
\end{enumerate}

\Soluzione

Manca.

\Soluzione

Manca.

\Soluzione

Manca.

\Soluzione

Manca.

\Soluzione

Manca.

\Soluzione

Introdurre $\Omega =$ \event{spazio campionario delle combinazioni di $90$ elementi di classe $5$} andrebbe bene solo per rispondere al primo punto, agli altri punti abbiamo esperimenti ripetuti, quindi \textit{non} introduciamo $\Omega $. Dal meccanismo del gioco sappiamo solo che le diverse estrazioni sono tra loro indipendenti e in ciascuna estrazione gli esiti sono equiprobabili.

\textbf{Metodo 1}

Introduciamo gli eventi:
\begin{gather*}
	A=\event{estraggo il 15}\\
	A_{k} =\event{estraggo il $15$ alla $k$-esima estrazione}
\end{gather*}
\begin{enumerate}
	\item
	\begin{equation*}
		\PP(A) =\frac{\binom{89}{4}}{\binom{90}{5}} =\frac{5}{90} =\frac{1}{18} \approx 0.0556
	\end{equation*}

	$\binom{89}{4}\rightarrow $azzecco il numero $15$ e basta, quindi per gli altri $4$ numeri devo considerare tutte le possibilità.

$\binom{90}{5}\rightarrow $tutte le possibili cinquine che posso estrarre.
\item L'evento che ci interessa è\begin{equation*}
\left(A_{42} \mid \bigcap _{k=1}^{41} A_{k}\complementary\right)
\end{equation*}

sfruttiamo il fatto che \textit{le estrazioni sono indipendenti dal passato} e quindi, ricordando la proprietà $\PP(A\mid B) =\PP(A)$ se $A\indep B$, si ha
\begin{equation*}
\left(A_{42} \mid\bigcap _{k=1}^{41} A_{k}\complementary\right) =\underbrace{\PP(A_{42}) =\PP(A)}_{\text{primo punto}} \approx 0.0556
\end{equation*}

	Per calcolare $\PP$ potremmo usare la relazione (si osservi che gli $A_{j}$ sono indipendenti \textit{non disgiunti}): $\PP(A\cup B) =\PP(A) +\PP(B) -\PP(A\cap B)$, ma diventa più complicata avendo $42$ eventi.

	Dato che la proprietà di indipendenza la si sfrutta quando si considerano \textit{intersezioni} di insiemi procediamo come segue. Scriviamo
	\begin{gather*}
		\left(\bigcup_{k} A_{k}\right) =\left(\left(\bigcup_{k} A_{k}\right)\complementary\right)\complementary =\left(\bigcap_{k} A_{k}\complementary\right)\complementary ,\ \ \ \ \PP\left(A\complementary\right) =1-\PP(A) ,\\
		A_{k} \indep \text{allora} \ A_{k}\complementary \indep ,
	\end{gather*}
	quindi
	\begin{align*}
		\PP\left(\bigcup_{k=1}^{42} A_{k}\right) & =\PP\left(\bigcap_{k} A_{k}\complementary\right)\complementary =1-\PP\left(\bigcap_{k} A_{k}\complementary\right) =1-\prod_{k=1}^{42}\PP\left(A_{k}\complementary\right)\\
		 & =1-\prod_{k=1}^{42}\PP\left(A\complementary\right) =1-\left[\PP\left(A\complementary\right)\right]^{42}\\
		 & =1-\left(1-\frac{1}{18}\right)^{42} \approx 0.9093
	\end{align*}

	\begin{rem}
		Osserviamo (ma lo sapevamo già!) che questo punto e il precedente sono due cose diverse.
	\end{rem}
	\item Introduciamo i nuovi eventi $B_{k} =$ \event{il $15$ è estratto $k$ volte}.
	\begin{equation*}
		\PP(A_{1} \mid B_{1}) =\frac{\PP(A_{1} \cap B_{1})}{\PP(B_{1})}
	\end{equation*}
	Anche in questo caso dobbiamo \textit{scrivere gli eventi usando le informazioni che abbiamo}. Osserviamo allora che $B_{1}$, ovvero \event{il $15$ estratto $1$ volta} può essere scritto come
	\begin{equation*}
		B_{1} =\bigcup_{j=1}^{42}\left(A_{j} \cap \bigcap_{k\neq j} A_{k}\complementary\right) =
	\end{equation*}
	i.e. $B_{1}$ è unione disgiunta degli eventi \event{$15$ esce solo alla $k$-esima estrazione} e quindi
	\begin{equation*}
		\PP(B_{1}) =42\PP(A_{j})\PP\left(\bigcap_{k\neq j} A_{j}\complementary\right) =42\left(\frac{1}{18}\right)\left(\frac{17}{18}\right)^{41}
	\end{equation*}
	ci sono $42$ modi per scegliere $j$ (la $j$-esima estrazione in cui esce $15$); per un $j$ fissato stiamo usando il fatto che l'unione degli eventi è disgiunta.

	Calcoliamo ora il numeratore, ricordando che $A_{1}$ ci dice che il $15$ è estratto alla prima estrazione, quindi vorrà dire che nelle successive $41$ non è più estratto (questa è l'informazione che ci dà $B_{1}$)
	\begin{align*}
		\PP(A_{1} \cap B_{1}) & =\PP\left(A_{1} \cap \bigcap_{k=2}^{42} A_{k}\complementary\right)\overset{\indep }{=}\PP(A_{1})\prod_{k=2}^{42}\PP\left(A_{k}\complementary\right)\\
		 & =\PP(A_{1})\prod_{k=2}^{42}(1-\PP(A_{k})) =\\
		 & =\PP(A)(1-\PP(A))^{41} =\frac{1}{18}\left(1-\frac{1}{18}\right)^{41}
	\end{align*}
	Infine
	\begin{equation*}
		\PP(A_{1} \mid B_{1}) =\frac{\PP(A_{1} \cap B_{1})}{\PP(B_{1})} =\frac{\frac{1}{18}\left(1-\frac{1}{18}\right)^{41}}{42\left(\frac{1}{18}\right)\left(\frac{17}{18}\right)^{41}} =\frac{1}{42} \approx 0.0238
	\end{equation*}
	\item Osserviamo che
	\begin{gather*}
		B_{2} =\bigcup_{i,j=1}^{42}\left(A_{j} \cap A_{i} \cap \bigcap_{k\neq i,k\neq j} A_{k}\complementary\right)\\
		\PP(B_{2}) =\underbrace{\binom{42}{2}}_{\text{modi per} \ i,j}\PP(A_{j})\PP(A_{i})\PP\left(\bigcap_{k\neq i,k\neq j} A_{k}\complementary\right) =\binom{42}{2}\left(\frac{1}{18}\right)\left(\frac{1}{18}\right)\left(\frac{17}{18}\right)^{40}
	\end{gather*}
	Allora
	\begin{align*}
		\PP(A_{1} \cap A_{2} \mid B_{2}) & =\frac{\PP(A_{1} \cap A_{2} \cap B_{2})}{\PP(B_{2})} =\frac{\PP\left(A_{1} \cap A_{2} \cap \bigcap_{k=3}^{42} A_{k}\complementary\right)}{\PP(B_{2})}\\
		 & \overset{\indep }{=}\frac{\PP(A_{1})\PP(A_{2})\PP\left(\bigcap_{k=3}^{42} A_{k}\complementary\right)}{\PP(B_{2})}\\
		 & =\frac{\PP(A)\PP(A)(1-\PP(A))^{40}}{\PP(B_{2})}\\
		 & =\frac{\left(\frac{1}{18}\right)^{2}\left(\frac{17}{18}\right)^{40}}{\binom{42}{2}\left(\frac{1}{18}\right)\left(\frac{1}{18}\right)\left(\frac{17}{18}\right)^{40}} =\frac{1}{\binom{42}{2}} \approx 0.0012
	\end{align*}

	\begin{rem}
		Osserviamo che in tutto l'esercizio non abbiamo fatto altro che usare la probabilità che esca il $15$ in una singola estrazione, infatti sfruttando l'\textbf{indipendenza} degli eventi ci siamo sempre ricondotti a calcolare le probabilità richieste utilizzando $\PP(A)$. Pertanto se interpretiamo $\PP(A)$ come probabilità di successo, semplifichiamo!
	\end{rem}

Dato che la proprietà di indipendenza la si sfrutta quando si considerano \textit{intersezioni} di insiemi procediamo come segue. Scriviamo
\begin{gather*}
\left(\bigcup _{k} A_{k}\right) =\left(\left(\bigcup _{k} A_{k}\right)\complementary\right)\complementary =\left(\bigcap _{k} A_{k}\complementary\right)\complementary ,\ \ \ \ \PP\left(A\complementary\right) =1-\PP(A) ,\\
A_{k} \indep \text{allora} \ A_{k}\complementary \indep ,
\end{gather*}

quindi
\begin{align*}
\PP\left(\bigcup _{k=1}^{42} A_{k}\right) & =\PP\left(\bigcap _{k} A_{k}\complementary\right)\complementary =1-\PP\left(\bigcap _{k} A_{k}\complementary\right) =1-\prod _{k=1}^{42}\PP\left(A_{k}\complementary\right)\\
 & =1-\prod _{k=1}^{42}\PP\left(A\complementary\right) =1-\left[\PP\left(A\complementary\right)\right]^{42}\\
 & =1-\left(1-\frac{1}{18}\right)^{42} \approx 0.9093
\end{align*}

\begin{rem}
Osserviamo (ma lo sapevamo già!) che questo punto e il precedente sono due cose diverse.
\end{rem}
\item Introduciamo i nuovi eventi $B_{k} =$ \event{il $15$ è estratto $k$ volte}.
\begin{equation*}
\PP(A_{1} \mid B_{1}) =\frac{\PP(A_{1} \cap B_{1})}{\PP(B_{1})}
\end{equation*}

Anche in questo caso dobbiamo \textit{scrivere gli eventi usando le informazioni che abbiamo}. Osserviamo allora che $B_{1}$, ovvero \event{il $15$ estratto $1$ volta} può essere scritto come
\begin{equation*}
B_{1} =\bigcup _{j=1}^{42}\left(A_{j} \cap \bigcap _{k\neq j} A_{k}\complementary\right) =
\end{equation*}

i.e. $B_{1}$ è unione disgiunta degli eventi \event{$15$ esce solo alla $k$-esima estrazione} e quindi
\begin{equation*}
\PP(B_{1}) =42\PP(A_{j})\PP\left(\bigcap _{k\neq j} A_{j}\complementary\right) =42\left(\frac{1}{18}\right)\left(\frac{17}{18}\right)^{41}
\end{equation*}

ci sono $42$ modi per scegliere $j$ (la $j$-esima estrazione in cui esce $15$); per un $j$ fissato stiamo usando il fatto che l'unione degli eventi è disgiunta.

Calcoliamo ora il numeratore, ricordando che $A_{1}$ ci dice che il $15$ è estratto alla prima estrazione, quindi vorrà dire che nelle successive $41$ non è più estratto (questa è l'informazione che ci dà $B_{1}$)\begin{align*}
\PP(A_{1} \cap B_{1}) & =\PP\left(A_{1} \cap \bigcap _{k=2}^{42} A_{k}\complementary\right)\overset{\indep }{=}\PP(A_{1})\prod _{k=2}^{42}\PP\left(A_{k}\complementary\right)\\
 & =\PP(A_{1})\prod _{k=2}^{42}(1-\PP(A_{k})) =\\
 & =\PP(A)(1-\PP(A))^{41} =\frac{1}{18}\left(1-\frac{1}{18}\right)^{41}
\end{align*}

Infine
\begin{equation*}
\PP(A_{1} \mid B_{1}) =\frac{\PP(A_{1} \cap B_{1})}{\PP(B_{1})} =\frac{\frac{1}{18}\left(1-\frac{1}{18}\right)^{41}}{42\left(\frac{1}{18}\right)\left(\frac{17}{18}\right)^{41}} =\frac{1}{42} \approx 0.0238
\end{equation*}
\item Osserviamo che
\begin{gather*}
B_{2} =\bigcup _{i,j=1}^{42}\left(A_{j} \cap A_{i} \cap \bigcap _{k\neq i,k\neq j} A_{k}\complementary\right)\\
\PP(B_{2}) =\underbrace{\binom{42}{2}}_{\text{modi per} \ i,j}\PP(A_{j})\PP(A_{i})\PP\left(\bigcap _{k\neq i,k\neq j} A_{k}\complementary\right) =\binom{42}{2}\left(\frac{1}{18}\right)\left(\frac{1}{18}\right)\left(\frac{17}{18}\right)^{40}
\end{gather*}

Allora
\begin{align*}
\PP(A_{1} \cap A_{2} \mid B_{2}) & =\frac{\PP(A_{1} \cap A_{2} \cap B_{2})}{\PP(B_{2})} =\frac{\PP\left(A_{1} \cap A_{2} \cap \bigcap _{k=3}^{42} A_{k}\complementary\right)}{\PP(B_{2})}\\
 & \overset{\indep }{=}\frac{\PP(A_{1})\PP(A_{2})\PP\left(\bigcap _{k=3}^{42} A_{k}\complementary\right)}{\PP(B_{2})}\\
 & =\frac{\PP(A)\PP(A)(1-\PP(A))^{40}}{\PP(B_{2})}\\
 & =\frac{\left(\frac{1}{18}\right)^{2}\left(\frac{17}{18}\right)^{40}}{\binom{42}{2}\left(\frac{1}{18}\right)\left(\frac{1}{18}\right)\left(\frac{17}{18}\right)^{40}} =\frac{1}{\binom{42}{2}} \approx 0.0012
\end{align*}

\begin{rem}
Osserviamo che in tutto l'esercizio non abbiamo fatto altro che usare la probabilità che esca il $15$ in una singola estrazione, infatti sfruttando l'\textbf{indipendenza} degli eventi ci siamo sempre ricondotti a calcolare le probabilità richieste utilizzando $\PP(A)$. Pertanto se interpretiamo $\PP(A)$ come probabilità di successo, semplifichiamo!
\end{rem}

\end{enumerate}

\textbf{[Metodo 2]}

Possiamo ridurre il nostro esperimento aleatorio a $42$ prove di Bernoulli ripetute e \textbf{indipendenti}, ciascuna con probabilità di successo $p=1/18$, i.e. interpretiamo
\begin{itemize}
	\item $A=$ \event{successo}
	\item $A_{k} =$ \event{successo alla prova $k$}
	\item $p=\PP(A) =$ \event{probabilità di successo $\left(=\frac{1}{18}\right)$}.
\end{itemize}
Allora
\begin{enumerate}
	\item $\PP(A) =1/18$
	\item $\PP\left(A_{42} \mid \bigcup _{k=1}^{41} A_{k}\complementary\right) =\PP(A_{42}) =\PP(A) =p$
	\item $\PP\left(\bigcup _{k=1}^{42} A_{k}\right) =1-\PP\left(\bigcap _{k=1}^{42} A_{k}\complementary\right) =1-\prod _{k=1}^{42}\PP\left(A_{k}\complementary\right) =1-(1-p)^{42}$
	\item $B_{1} =$ solo un successo.

	$\PP(B_{1}) =42p(1-p)^{41} \implies \PP(A_{1} \mid B_{1}) =\frac{p(1-p)^{41}}{42p(1-p)^{41}} =\frac{1}{42}$
	\item $B_{2} =$ \event{solo due successi}.

	$\PP(B_{2}) =\binom{42}{2} p^{2}(1-p)^{40} \implies \PP(A_{1} \cap A_{2} \mid B_{2}) =\frac{p^{2}(1-p)^{40}}{\binom{42}{2} p^{2}(1-p)^{40}} =\frac{1}{\binom{42}{2}}$
\end{enumerate}

\Soluzione

Consideriamo l'evento
\begin{equation*}
	A=\event{il pacco scelto inizialmente contiene l'assegno}
\end{equation*}
\begin{enumerate}
	\item $\PP(A) =1/3$
	\item Introduciamo l'evento
	\begin{equation*}
	V_{\text{cons}} =\event{vinco conservando il pacco scelto inizialmente}
	\end{equation*}

	Per rispondere a questa domanda ci è utile ricorrere alla formula delle probabilità totali:\begin{thm}
	Sia $(\Omega ,\mathcal{A} ,\PP)$ uno spazio di probabilità. Si $A\in \mathcal{A}$ e $\{E_{k}\}_{k\in \mathbb{N}} \subset \mathcal{A}$ una partizione discreta di $\Omega $, allora
	\begin{equation*}
	\PP(A) =\sum\limits _{k\in \mathbb{N}}\PP(A\mid E_{k})\PP(E_{k})
	\end{equation*}
	\end{thm}

	La partizione di $\Omega $ la costruiamo usando \textit{le informazioni della prima scelta}, ovvero
	\begin{equation*}
	\Omega =A\cup A\complementary
	\end{equation*}

	Tutti i possibili eventi sono: o il pacco scelto inizialmente contiene l'assegno, o non lo contiene. Quindi
	\begin{equation*}
	\PP(V_{\text{cons}}) =\PP(V_{\text{cons}} \mid A)\PP(A) +\PP\left(V_{\text{cons}} \mid A\complementary\right)\PP\left(A\complementary\right) =\frac{1}{3}
	\end{equation*}
	\begin{enumerate}
	\item $\PP(V_{\text{cons}} \mid A) =1$, se ho scelto inizialmente il pacco contenente l'assegno e conservo il pacco, allora sono sicuro di vincere.
	\item $\PP(A) =\frac{1}{3}$
	\item $\PP\left(V_{\text{cons}} \mid A\complementary\right) =0$, se so che non ho scelto il pacco contenente l'assegno e lo conservo, allora sono sicuro di perdere.
	\item $\PP\left(A\complementary\right) =1-\PP(A) =1-\frac{1}{3} =\frac{2}{3}$
	\end{enumerate}
	\item Introduciamo l'evento
	\begin{equation*}
	V_{\text{camb}} =\event{vinco cambiando il pacco scelto inizialmente}
	\end{equation*}

	allora
	\begin{equation*}
	\PP(V_{\text{camb}}) =\underbrace{\PP(V_{\text{camb}} \mid A)}_{0}\underbrace{\PP(A)}_{\frac{1}{3}} +\underbrace{\PP\left(V_{\text{camb}} \mid A\complementary\right)}_{1}\underbrace{\PP\left(A\complementary\right)}_{\frac{2}{3}} =\frac{2}{3}
	\end{equation*}
\end{enumerate}

Dai punti appena visti si vede immediatamente che la probabilità di vittoria dipende dalla strategia! Possiamo rappresentare il risultato graficamente.

\tikzset{every picture/.style={line width=0.75pt}} %set default line width to 0.75pt        

\begin{tikzpicture}[x=0.75pt,y=0.75pt,yscale=-1,xscale=1]
%uncomment if require: \path (0,230); %set diagram left start at 0, and has height of 230


% Text Node
\draw    (149,16) -- (275,16) -- (275,41) -- (149,41) -- cycle  ;
\draw (152,20) node [anchor=north west][inner sep=0.75pt]   [align=left] {scelta di un pacco};
% Text Node
\draw    (77,74) -- (98,74) -- (98,99) -- (77,99) -- cycle  ;
\draw (80,78.4) node [anchor=north west][inner sep=0.75pt]    {$A$};
% Text Node
\draw    (203,74) -- (222,74) -- (222,99) -- (203,99) -- cycle  ;
\draw (206,78.4) node [anchor=north west][inner sep=0.75pt]    {$V$};
% Text Node
\draw    (330,74) -- (349,74) -- (349,99) -- (330,99) -- cycle  ;
\draw (333,78.4) node [anchor=north west][inner sep=0.75pt]    {$V$};
% Text Node
\draw (34,148) node [anchor=north west][inner sep=0.75pt]   [align=left] {perdo};
% Text Node
\draw (99,148) node [anchor=north west][inner sep=0.75pt]   [align=left] {vinco};
% Text Node
\draw (162,148) node [anchor=north west][inner sep=0.75pt]   [align=left] {vinco};
% Text Node
\draw (225,148) node [anchor=north west][inner sep=0.75pt]   [align=left] {perdo};
% Text Node
\draw (290,148) node [anchor=north west][inner sep=0.75pt]   [align=left] {vinco};
% Text Node
\draw (353,148) node [anchor=north west][inner sep=0.75pt]   [align=left] {perdo};
% Text Node
\draw (77,196) node [anchor=north west][inner sep=0.75pt]   [align=left] {qua è fondamentale sapere che l'altro pacco è vuoto!};
% Text Node
\draw (434,18.4) node [anchor=north west][inner sep=0.75pt]    {$\begin{array}{l}
A=\ \text{assegno}\\
V=\ \text{vuoto}\\
C=\ \text{cambio}\\
NC=\ \text{non cambio}
\end{array}$};
% Text Node
\draw (58,115.4) node [anchor=north west][inner sep=0.75pt]  [font=\scriptsize]  {$C$};
% Text Node
\draw (106,115.4) node [anchor=north west][inner sep=0.75pt]  [font=\scriptsize]  {$NC$};
% Text Node
\draw (183,115.4) node [anchor=north west][inner sep=0.75pt]  [font=\scriptsize]  {$C$};
% Text Node
\draw (231,115.4) node [anchor=north west][inner sep=0.75pt]  [font=\scriptsize]  {$NC$};
% Text Node
\draw (311,115.4) node [anchor=north west][inner sep=0.75pt]  [font=\scriptsize]  {$C$};
% Text Node
\draw (359,115.4) node [anchor=north west][inner sep=0.75pt]  [font=\scriptsize]  {$NC$};
% Connection
\draw    (81.43,99) -- (59.57,144) ;
% Connection
\draw    (92.86,99) -- (112.14,144) ;
% Connection
\draw    (218.21,99) -- (238.79,144) ;
% Connection
\draw    (206.79,99) -- (186.21,144) ;
% Connection
\draw    (333.96,99) -- (314.04,144) ;
% Connection
\draw    (345.39,99) -- (366.61,144) ;
% Connection
\draw    (185.17,41) -- (98,81.61) ;
% Connection
\draw    (212.11,41) -- (212.39,74) ;
% Connection
\draw    (239.48,41) -- (330,82.18) ;
% Connection
\draw    (198.6,169) -- (229.43,190.3) ;
\draw [shift={(231.9,192)}, rotate = 214.63] [fill={rgb, 255:red, 0; green, 0; blue, 0 }  ][line width=0.08]  [draw opacity=0] (10.72,-5.15) -- (0,0) -- (10.72,5.15) -- (7.12,0) -- cycle    ;
% Connection
\draw    (293.27,169) -- (267.55,190.1) ;
\draw [shift={(265.23,192)}, rotate = 320.63] [fill={rgb, 255:red, 0; green, 0; blue, 0 }  ][line width=0.08]  [draw opacity=0] (10.72,-5.15) -- (0,0) -- (10.72,5.15) -- (7.12,0) -- cycle    ;

\end{tikzpicture}

Ad esempio, per rispondere terzo punto ci interessano solo gli eventi in cui ala seconda scelta cambio pacco (i rami sinistri). Vediamo che su $3$ possibili situazioni quelle in cui vinciamo sono $2$, i.e. $\PP(V_{\text{camb}}) =\frac{2}{3}$.

Analogamente, se non cambiamo (rami destri), su tre possibili situazioni quella vittoriosa è solo una, i.e. $\PP(V_{\text{cons}}) =\frac{1}{3}$.

\begin{rem}
	Intuitivamente: il conduttore aprendo un pacco ci sta dando delle informazioni in più: è come se, con la sua informazione, ci dicesse ``il pacco rimasto (non il tuo) ha $\frac{2}{3}$ di probabilità di essere il pacco vincente''. Cioè se prima avevo una probabilità di $\frac{2}{3}$ \emph{distribuita su $2$ pacchi}, adesso questa probabilità è tutta \emph{concentrata sull'altro pacco}. Quindi mi conviene cambiare pacco perché questo ha una probabilità più alta $\left(\frac{2}{3}\right)$ rispetto al mio $\left(\frac{1}{3}\right)$ di contenere l'assegno.


	% Pattern Info
	 
	\tikzset{
	pattern size/.store in=\mcSize, 
	pattern size = 5pt,
	pattern thickness/.store in=\mcThickness, 
	pattern thickness = 0.3pt,
	pattern radius/.store in=\mcRadius, 
	pattern radius = 1pt}
	\makeatletter
	\pgfutil@ifundefined{pgf@pattern@name@_v47zenp0g}{
	\pgfdeclarepatternformonly[\mcThickness,\mcSize]{_v47zenp0g}
	{\pgfqpoint{0pt}{-\mcThickness}}
	{\pgfpoint{\mcSize}{\mcSize}}
	{\pgfpoint{\mcSize}{\mcSize}}
	{
	\pgfsetcolor{\tikz@pattern@color}
	\pgfsetlinewidth{\mcThickness}
	\pgfpathmoveto{\pgfqpoint{0pt}{\mcSize}}
	\pgfpathlineto{\pgfpoint{\mcSize+\mcThickness}{-\mcThickness}}
	\pgfusepath{stroke}
	}}
	\makeatother
	\tikzset{every picture/.style={line width=0.75pt}} %set default line width to 0.75pt        

	\begin{tikzpicture}[x=0.75pt,y=0.75pt,yscale=-1,xscale=1]
	%uncomment if require: \path (0,260); %set diagram left start at 0, and has height of 260

	%Shape: Brace [id:dp5635373611818795] 
	\draw   (197.5,58) .. controls (197.5,62.67) and (199.83,65) .. (204.5,65) -- (224.5,65) .. controls (231.17,65) and (234.5,67.33) .. (234.5,72) .. controls (234.5,67.33) and (237.83,65) .. (244.5,65)(241.5,65) -- (264.5,65) .. controls (269.17,65) and (271.5,62.67) .. (271.5,58) ;
	%Shape: Brace [id:dp17280187716588213] 
	\draw   (197.5,188) .. controls (197.5,192.67) and (199.83,195) .. (204.5,195) -- (224.5,195) .. controls (231.17,195) and (234.5,197.33) .. (234.5,202) .. controls (234.5,197.33) and (237.83,195) .. (244.5,195)(241.5,195) -- (264.5,195) .. controls (269.17,195) and (271.5,192.67) .. (271.5,188) ;
	%Shape: Rectangle [id:dp09840376922489535] 
	\draw  [color={rgb, 255:red, 155; green, 155; blue, 155 }  ,draw opacity=1 ] (37.5,21) -- (280,21) -- (280,114.56) -- (37.5,114.56) -- cycle ;
	%Shape: Rectangle [id:dp5883671807012183] 
	\draw  [color={rgb, 255:red, 155; green, 155; blue, 155 }  ,draw opacity=1 ] (37.5,148) -- (280,148) -- (280,241.56) -- (37.5,241.56) -- cycle ;
	%Shape: Rectangle [id:dp7587181690994056] 
	\draw  [color={rgb, 255:red, 155; green, 155; blue, 155 }  ,draw opacity=1 ] (364.5,148) -- (462,148) -- (462,241.56) -- (364.5,241.56) -- cycle ;

	% Text Node
	\draw    (151,29) -- (170,29) -- (170,54) -- (151,54) -- cycle  ;
	\draw (154,33.4) node [anchor=north west][inner sep=0.75pt]    {$1$};
	% Text Node
	\draw    (201,29) -- (220,29) -- (220,54) -- (201,54) -- cycle  ;
	\draw (204,33.4) node [anchor=north west][inner sep=0.75pt]    {$2$};
	% Text Node
	\draw    (251,29) -- (270,29) -- (270,54) -- (251,54) -- cycle  ;
	\draw (254,33.4) node [anchor=north west][inner sep=0.75pt]    {$3$};
	% Text Node
	\draw (47,80) node [anchor=north west][inner sep=0.75pt]   [align=left] {probabilità};
	% Text Node
	\draw (154,76.4) node [anchor=north west][inner sep=0.75pt]    {$\frac{1}{3}$};
	% Text Node
	\draw (230,76.4) node [anchor=north west][inner sep=0.75pt]    {$\frac{2}{3}$};
	% Text Node
	\draw (185,123.4) node [anchor=north west][inner sep=0.75pt]    {$\Downarrow $};
	% Text Node
	\draw    (151,159) -- (170,159) -- (170,184) -- (151,184) -- cycle  ;
	\draw (154,163.4) node [anchor=north west][inner sep=0.75pt]    {$1$};
	% Text Node
	\draw    (201,159) -- (220,159) -- (220,184) -- (201,184) -- cycle  ;
	\draw (204,163.4) node [anchor=north west][inner sep=0.75pt]    {$2$};
	% Text Node
	\draw  [pattern=_v47zenp0g,pattern size=6pt,pattern thickness=0.75pt,pattern radius=0pt, pattern color={rgb, 255:red, 0; green, 0; blue, 0}]  (251,159) -- (270,159) -- (270,184) -- (251,184) -- cycle  ;
	\draw (254,163.4) node [anchor=north west][inner sep=0.75pt]    {$3$};
	% Text Node
	\draw (47,210) node [anchor=north west][inner sep=0.75pt]   [align=left] {probabilità};
	% Text Node
	\draw (154,206.4) node [anchor=north west][inner sep=0.75pt]    {$\frac{1}{3}$};
	% Text Node
	\draw (230,206.4) node [anchor=north west][inner sep=0.75pt]    {$\frac{2}{3}$};
	% Text Node
	\draw    (381,159) -- (400,159) -- (400,184) -- (381,184) -- cycle  ;
	\draw (384,163.4) node [anchor=north west][inner sep=0.75pt]    {$1$};
	% Text Node
	\draw    (431,159) -- (450,159) -- (450,184) -- (431,184) -- cycle  ;
	\draw (434,163.4) node [anchor=north west][inner sep=0.75pt]    {$2$};
	% Text Node
	\draw (384,206.4) node [anchor=north west][inner sep=0.75pt]    {$\frac{1}{3}$};
	% Text Node
	\draw (435,206.4) node [anchor=north west][inner sep=0.75pt]    {$\frac{2}{3}$};
	% Text Node
	\draw (312,189.4) node [anchor=north west][inner sep=0.75pt]    {$\iff $};


	\end{tikzpicture}
\end{rem}
\begin{enumerate}
	\item Consideriamo l'evento
	\begin{equation*}
		V_{\text{tc}} =\event{vinco giocando a testa o croce tra le due strategie}
	\end{equation*}
	Supponiamo la moneta equa e, senza perdita di generalità, assegnamo a $T$ la strategia di cambio pacco. Osserviamo che il lancio della moneta è indipendente dagli eventi $V_{\text{camb}}$ e $V_{\text{cons}}$. Usiamo ancora la formula delle probabilità totali, in questo caso con $\Omega =T\cup C$.
	\begin{align*}
		\PP(V_{\text{tc}}) & =\PP(V_{\text{tc}} \mid T)\PP(T) +\PP(V_{\text{tc}} \mid C)\PP(C) \\
		 & =\PP(V_{\text{camb}} \mid T)\PP(T) +\PP(V_{\text{cons}} \mid C)\PP(C) \\
		 & \indepeq \PP(V_{\text{camb}})\PP(T) +\PP(V_{\text{cons}})\PP(C) \\
		 & =\frac{2}{3} \cdotp \frac{1}{2} +\frac{1}{3} \cdotp \frac{1}{2} =\frac{1}{3} +\frac{1}{6} =\frac{1}{2}
	\end{align*}
	\item Introduciamo gli eventi
	\begin{align*}
		D&=\event{il conduttore apre il pacco di destra}\\
		P_{d} &=\event{il pacco di destra contiene il premio}\\
		P_{s} &=\event{il pacco di sinistra contiene il premio}
	\end{align*}
	Dal testo sappiamo che, poiché in tal caso il vincitore ha il pacco vincente,
	\begin{equation*}
		\boxed{\PP(D\mid A) =p}
	\end{equation*}
	$p$ è la probabilità che il conduttore apra il pacco di destra, sapendo che il vincitore ha il pacco vincente (cioè i due pacchi rimasti sono entrambi vuoti). Noi vogliamo calcolare
	\begin{equation*}
		\boxed{\PP(A\mid D) =?}
	\end{equation*}
	ovvero la probabilità che il pacco scelto inizialmente contenga il premio, sapendo che il conduttore ha aperto il pacco di destra. Per calcolarla usiamo la formula di Bayes.
	\begin{thm}
		Sia $(\Omega,\Ac,\PP)$ uno spazio di probabilità. Siano $A\in \Ac$, $\{E_{k}\}_{k\in \NN} \subset \Ac$ una partizione discreta di $\Omega $. Se $\PP(A) ,\PP(E_{k})  >0,\forall k\in \NN$, allora
		\begin{equation*}
			\PP(E_{k} \mid A) =\frac{\PP(A\mid E_{k})\PP(E_{k})}{\sum_{k}\PP(A\mid E_{k})\PP(E_{k})}
		\end{equation*}
	\end{thm}

	\begin{rem}
		La prima idea sarebbe stata quella di usare la formula di Bayes nella forma $\PP(A\mid D) =\frac{\PP(D\mid A)\PP(A)}{\PP(D)}$. Questa formula è però inutile nel nostro caso perché \textbf{non} sappiamo quanto vale $\PP(D)$. Utilizziamo quindi la formula sopra proposta.
	\end{rem}

	La partizione di $\Omega $ è data da $\{A,P_{d} ,P_{s}\}$, questi $3$ eventi ci dicono che pacco sta il premio. Allora
	\begin{align*}
		\PP(A\mid D) & =\frac{\overbrace{\PP(D\mid A)}^{p}\overbrace{\PP(A)}^{\frac{1}{3}}}{\underbrace{\PP(D\mid A)}_{p}\underbrace{\PP(A)}_{\frac{1}{3}} +\underbrace{\PP(D\mid P_{d})}_{0}\underbrace{\PP(P_{d})}_{\frac{1}{3}} +\underbrace{\PP(D\mid P_{s})}_{1}\underbrace{\PP(P_{s})}_{\frac{1}{3}}}\\
		 & =\frac{\frac{p}{3}}{\frac{p}{3} +\frac{1}{3}} =\frac{p}{p+1}
	\end{align*}

	\begin{rem}
		In questo caso l'informazione che abbiamo è una info \textit{storica}: dai dati che abbiamo raccolto abbiamo una informazione in più sul gioco e questa informazione \textit{non} dipende dalla scelta del conduttore. La probabilità $\PP$ si \textit{aggiorna} con questa nuova informazione (Teorema di Bayes).
	\end{rem}

	Analizziamo alcuni casi di valori di $p$
	\begin{enumerate}
		\item $\boxed{p=0}$ i.e. se il concorrente ha il pacco vincente, il pacco di destra non viene mai aperto. In altri termini, se il pacco di destra viene aperto, allora il concorrente non ha il pacco vincente. Questo è quello che ci dice il risultato trovato $\PP(A\mid D) =0$.
		\item $\boxed{p=1}$ i.e. se il concorrente ha il pacco vincente, allora certamente viene aperto il pacco di destra. Quindi, in tal caso, osservando l'apertura del pacco di destra siamo più propensi a credere che il concorrente abbia il premio $\PP(A\mid D) =\frac{1}{2}$.
		\item $\boxed{p=\frac{1}{2}}$ i.e. se il concorrente ha il pacco vincente, allora viene aperto indifferentemente uno dei due pacchi. Non abbiamo ulteriori informazioni, quindi non aggiorneremo le nostre ipotesi \textit{a priori}.
	\end{enumerate}
\end{enumerate}

\Soluzione

\Soluzione

\Soluzione

Introduciamo gli eventi
\begin{itemize}
	\item $R_{k} =$ \event{la $k$-esima pallina estratta è rossa}.
	\item $B_{k} =$ \event{la $k$-esima pallina estratta è bianca}.
\end{itemize}
Allora
\begin{enumerate}
	\item $\PP(R_{1}) =\frac{r}{r+b}$
	\item $\PP(R_{1} \cap B_{2}) =\PP(B_{2} \mid R_{1})\PP(R_{1}) =\frac{b}{r+b-1} \cdotp \frac{r}{r+b} =\frac{rb}{(r+b)(r+b-1)}$
	\item $
	\begin{aligned}
		\PP((B_{1} \cap R_{2}) \cup (R_{1} \cap B_{2})) & =\underbrace{\PP(B_{1} \cap R_{2})}_{\underbrace{\PP(R_{1} \mid B_{1})}_{\frac{r}{r+b-1}}\underbrace{\PP(B_{1})}_{\frac{b}{r+b}}} +\underbrace{\PP(R_{1} \cap B_{2})}_{\frac{rb}{(r+b)(r+b-1)}}\\
		 & \ \ \ \ \ \ \ \ \ \ \ \ \ \ \ \ -\underbrace{\PP((B_{1} \cap R_{2}) \cap (R_{1} \cap B_{2}))}_{0}\\
		 & =\frac{2rb}{(r+b-1)(r+b)}
	\end{aligned}$
	\item $
	\begin{aligned}
		\PP(R_{2}) & =\PP(R_{2} \mid B_{1})\PP(B_{1}) +\PP(R_{2} \mid R_{1})\PP(R_{1})\\
		 & =\frac{rb}{(r+b-1)(r+b)} +\frac{(r-1) b}{(r+b-1)(r+b)} =\frac{r}{b+r}
	\end{aligned}$
	\item $
	\begin{aligned}
		\PP(R_{3}) & =\PP(R_{3} \mid R_{1} \cap R_{2})\PP(R_{1} \cap R_{2}) +\\
		 & \ \ \ \ \ \ \ \ +\PP(R_{3} \mid B_{1} \cap B_{2})\PP(B_{1} \cap B_{2}) +\\
		 & \ \ \ \ \ \ \ \ +\PP(R_{3} \mid B_{1} \cap R_{2})\PP(B_{1} \cap R_{2}) +\\
		 & \ \ \ \ \ \ \ \ +\PP(R_{3} \mid R_{1} \cap B_{2})\PP(R_{1} \cap B_{2})
	\end{aligned}$
	Dove $\PP(R_{3} \mid R_{1} \cap R_{2})\PP(R_{1} \cap R_{2}) =\PP(R_{3} \mid R_{1} \cap R_{2})\PP(R_{2} \mid R_{1})\PP(R_{1})$ e così anche per gli altri pezzi. Si trova infine $\PP(R_{3}) =\frac{r}{b+r}$.
\end{enumerate}

\Soluzione

\Soluzione

\Soluzione

\Soluzione

\begin{enumerate}
	\item $\PP(B\mid A) =\frac{1}{5000} =0.0002\ =\ 0.02\%$
	\item $B\cap C=\emptyset $
	\item $A\indep C$
	\item $D=B\cup C$
	\item $\PP(B\mid A,D) =\frac{\PP(B,A,D)}{\PP(A,D)} =\frac{\PP(B,A)}{\PP(B,A) +\PP(C,A)} =\frac{1}{1+\frac{\PP(C,A)}{\PP(B,A)}} =\frac{1}{1+\frac{\PP(C\mid A)}{\PP(B\mid A)}} =\frac{1}{1+\frac{\PP(C)}{\PP(B\mid A)}}$
	\item Poiché $\PP(C) \leq \PP(D) =\frac{1}{100000}$, vale
	\begin{equation*}
	\PP(B\mid A,D) \geq \frac{1}{1+\frac{\PP(D)}{\PP(B\mid A)}} =\frac{1}{1+\frac{5}{100}} =\frac{20}{21} =0.9524=95.24\%
	\end{equation*}
	\item Sì è il caso che Ferrero continui ad indagare su Ambrogio.
\end{enumerate}

\Soluzione

Definiamo l'evento
\begin{equation*}
	B_{k} = \event{la $k$-esima pallina estratta è bianca}
\end{equation*}
Allora
\begin{enumerate}
	\item $\PP(B_{1}) =\frac{1}{2}$
	\item $\PP(B_{2} \mid B_{1}) =\frac{1}{2}$. Si rinforzano solo le rosse: dato che la prima estratta è bianca, alla seconda estrazione mi troverò ancora davanti un'urna con $1B$ e $1R$.
	\item Con la formula delle probabilità totali e ricordando $\Omega =B_{1} \cup B_{1}\complementary$ (alla prima deve essere estratta $B$ o $R$)
	\begin{equation*}
		\PP(B_{2}) =\underbrace{\PP(B_{2} \mid B_{1})}_{\frac{1}{2}}\underbrace{\PP(B_{1})}_{\frac{1}{2}} +\PP\left(B_{2} \mid B_{1}\complementary\right)\underbrace{\PP\left(B_{1}\complementary\right)}_{\frac{1}{2}}
	\end{equation*}

	Ora, $\PP\left(B_{2} \mid B_{1}\complementary\right) =\frac{1}{3}$, infatti se alla prima estrazione è uscita $R$, alla seconda estrazione nell'urna ci sono $2B$ e $1R$. Allora $\PP(B_{2}) =\frac{1}{4} +\frac{1}{6} =\frac{5}{12}$.

	\tikzset{every picture/.style={line width=0.75pt}} %set default line width to 0.75pt        

	\begin{tikzpicture}[x=0.75pt,y=0.75pt,yscale=-1,xscale=1]
	%uncomment if require: \path (0,188); %set diagram left start at 0, and has height of 188

	%Shape: Brace [id:dp8335788707753122] 
	\draw   (292,157) .. controls (296.67,157) and (299,154.67) .. (299,150) -- (299,123.67) .. controls (299,117) and (301.33,113.67) .. (306,113.67) .. controls (301.33,113.67) and (299,110.34) .. (299,103.67)(299,106.67) -- (299,77.33) .. controls (299,72.66) and (296.67,70.33) .. (292,70.33) ;

	% Text Node
	\draw (77.25,80.4) node [anchor=north west][inner sep=0.75pt]    {$\Omega $};
	% Text Node
	\draw (147.25,47.4) node [anchor=north west][inner sep=0.75pt]  [font=\scriptsize]  {$R\left(\frac{1}{2}\right)$};
	% Text Node
	\draw (147.25,119.4) node [anchor=north west][inner sep=0.75pt]  [font=\scriptsize]  {$B\left(\frac{1}{2}\right)$};
	% Text Node
	\draw (223.25,24.4) node [anchor=north west][inner sep=0.75pt]  [font=\scriptsize]  {$R\left(\frac{2}{3}\right)$};
	% Text Node
	\draw (223.25,66.4) node [anchor=north west][inner sep=0.75pt]  [font=\scriptsize]  {$B\left(\frac{1}{3}\right)$};
	% Text Node
	\draw (223.25,102.4) node [anchor=north west][inner sep=0.75pt]  [font=\scriptsize]  {$R\left(\frac{1}{2}\right)$};
	% Text Node
	\draw (223.25,144.4) node [anchor=north west][inner sep=0.75pt]  [font=\scriptsize]  {$B\left(\frac{1}{2}\right)$};
	% Text Node
	\draw (271.22,33.52) node  [font=\scriptsize]  {$\frac{1}{3}$};
	% Text Node
	\draw    (271.22, 75.52) circle [x radius= 10.61, y radius= 16.26]   ;
	\draw (271.22,75.52) node  [font=\scriptsize]  {$\frac{1}{6}$};
	% Text Node
	\draw (271.22,111.52) node  [font=\scriptsize]  {$\frac{1}{4}$};
	% Text Node
	\draw    (271.22, 154.52) circle [x radius= 10.61, y radius= 16.26]   ;
	\draw (271.22,154.52) node  [font=\scriptsize]  {$\frac{1}{4}$};
	% Text Node
	\draw (312,98.73) node [anchor=north west][inner sep=0.75pt]    {$\PP(B_{2}) =\frac{1}{6} +\frac{1}{4} =\frac{5}{12}$};
	% Connection
	\draw    (179.25,54.2) -- (220.25,41.8) ;
	% Connection
	\draw    (179.25,63.88) -- (220.25,74.13) ;
	% Connection
	\draw    (95.25,84.55) -- (144.25,66.09) ;
	% Connection
	\draw    (95.25,94.36) -- (144.25,121.73) ;
	% Connection
	\draw    (179.25,127.59) -- (220.25,118.41) ;
	% Connection
	\draw    (179.25,137.26) -- (220.25,150.74) ;

	\end{tikzpicture}

	\item $\PP(B_{1} \cup B_{2}) =\underbrace{\PP(B_{1})}_{\frac{1}{2}} +\underbrace{\PP(B_{2})}_{\frac{5}{12}} -\underbrace{\PP(B_{1} \cap B_{2})}_{\frac{1}{4}} =\frac{6+5-3}{12} =\frac{2}{3}$

	$\PP(B_{1} \cap B_{2}) =\PP(B_{2} \mid B_{1})\PP(B_{1}) =\frac{1}{2} \cdotp \frac{1}{2} =\frac{1}{4}$

	Alternativamente:
	\begin{align*}
		\PP(B_{1} \cup B_{2}) & =1-\PP\left(B_{1}\complementary \cap B_{2}\complementary\right)\\
		 & =1-\PP\left(B_{2}\complementary \mid B_{1}\complementary\right)\PP\left(B_{1}\complementary\right) =1-\frac{1}{3} =\frac{2}{3}
	\end{align*}
	\item $\PP(B_{1} \mid B_{2}) =\frac{\PP(B_{2} \mid B_{1})\PP(B_{1})}{\PP(B_{2})} =\frac{3}{5}$
	\item $
	\begin{aligned}
		\PP\left(\bigcap\limits_{k=1}^{n} B_{k}\right) & \overset{\text{Thm} \ 3.3\ JP}{=}\PP(B_{1})\PP(B_{2} \mid B_{1})\PP(B_{3} \mid B_{1} \cap B_{2}) \cdots \PP(B_{n} \mid B_{1} \cap \cdots \cap B_{n-1})\\
		 & =\left(\frac{1}{2}\right)^{n}
	\end{aligned}$
	
	Stiamo sfruttando il fatto che c'è rinforzo solo delle rosse e \textit{non} che abbiamo prove ripetute indipendenti!
	\item $
	\begin{aligned}
		\PP\left(\bigcap\limits_{k=1}^{n} B_{k}\complementary\right) & =\PP\left(B_{1}\complementary\right)\PP\left(B_{2}\complementary \mid B_{1}\complementary\right) \cdots \PP\left(B_{n}\complementary \mid B_{1}\complementary \cap \cdots \cap B_{n-1}\complementary\right)\\
		 & =\frac{1}{\cancel{2}} \cdotp \frac{\cancel{2}}{\cancel{3}} \cdots \frac{1}{n+1} =\frac{1}{n+1}
	\end{aligned}$

	Stiamo sfruttando il rinforzo delle rosse.
	\item $\underbrace{\PP\left(\bigcap\limits_{k=1}^{\infty } B_{k}\right) =\lim\limits_{n\rightarrow \infty }\PP\left(\bigcap\limits_{k=1}^{n} B_{k}\right)}_{\bigcap_{k=1}^{n} B_{k} \ \downarrow \ \bigcap_{k=1}^{\infty } B_{k}} =\liminf\limits_{n\rightarrow \infty }\left(\frac{1}{2}\right)^{n} =0$

	Abbiamo sfruttato la continuità dall'alto di $\PP$.
	\item $
	\begin{aligned}
		\PP\left(\left(\bigcap\limits_{k=1}^{\infty } B_{k}\right) \cup \left(\bigcap\limits_{k=1}^{\infty } B_{k}\complementary\right)\right) & =\underbrace{\PP\left(\bigcap\limits_{k=1}^{\infty } B_{k}\right)}_{0} +\PP\left(\bigcap\limits_{k=1}^{\infty } B_{k}\complementary\right) & \text{(unione disgiunta)}\\
		 & =\lim\limits_{n\rightarrow \infty }\PP\left(\bigcap\limits_{k=1}^{n} B_{k}\complementary\right) & \text{(continuità)}\\
		 & =\lim\limits_{n\rightarrow \infty }\frac{1}{n+1} =0 & 
	\end{aligned}$

\end{enumerate}

\Soluzione

\chapter{Variabili aleatorie discrete}
%!TEX root = ../main.tex

\ParteEsercizi

Richiami

VA: Sia $( \Omega ,\mathcal{A} ,\mathbb{P})$ uno spazio di probabilità. Una funzione $X:( \Omega ,\mathcal{A})\rightarrow (\mathbb{R} ,\mathcal{B})$ si dice variabile aleatoria reale se è misurabile, i.e. $\forall B\in \mathcal{B} ,X^{-1}( B) \in \mathcal{A}$.

LEGGE: la legge di una variabile aleatoria reale $X$ è la misura di probabilità $P^{X}( B) =\mathbb{P}( X\in B) =\mathbb{P}\left( X^{-1}( B)\right) =\mathbb{P}(\{\omega \in \Omega :X( \omega ) \in B\})$, $\forall B\in \mathcal{B}$.
\begin{oss}
La legge è univocamente determinata dalla funzione di ripartizione, ossia $F:\mathbb{R}\rightarrow \mathbb{R}$ tale che $F( t) =\mathbb{P}( X\leq t) =P^{X}(( -\infty ,t])$. La funzione di ripartizione è definita su tutto $\mathbb{R}$ e si definisce solo per le VA reali.
\end{oss}
\begin{oss}
Somme, prodotti, limiti di VA reali sono VA reali.
\end{oss}
Funzioni boreliane: una funzione $f:(\mathbb{R} ,\mathcal{B})\rightarrow (\mathbb{R} ,\mathcal{B})$ si dice boreliana se è misurabile.
\begin{oss}
Sono boreliane le funzioni continue e generalmente continue (i.e. continue a tratti).
\end{oss}
\begin{oss}
Composizione di VA reali con funzioni boreliane sono VA reali.
\end{oss}
VA discrete: una VA reale $X$ si dice \textbf{discreta} se assume con probabilità $1$ valori in un insieme $S$ al più numerabile $\mathbb{P}( X\in S) =1$. $\iff $Una VA reale $X$ si dice discreta se e solo se $P^{X}$ è discreta, ossia $\exists S\subset \mathbb{R}$ al più numerabile e $p_{X} :S\rightarrow [ 0,1]$ tale che $P^{X}( B) =\sum\limits _{x\in B\cap S} p_{X}( x)$, $\forall B\in \mathcal{B}$.

La funzione $p_{X}$ si dice \textbf{densità discreta} di $X$. Si dice \textbf{supporto} di $P^{X}$ l'insieme $S_{X} =\{x\in \mathbb{R} :p_{X}( x)  >0\}$.
\begin{oss}
Posso, ma non per forza, pensare $S=S_{X}$.
\end{oss}
Graficamente


\tikzset{every picture/.style={line width=0.75pt}} %set default line width to 0.75pt        

\begin{tikzpicture}[x=0.75pt,y=0.75pt,yscale=-1,xscale=1]
%uncomment if require: \path (0,132); %set diagram left start at 0, and has height of 132

%Straight Lines [id:da6279508270671126] 
\draw    (178,114) -- (383.5,114) ;
%Straight Lines [id:da5575359122441383] 
\draw    (198,114) -- (198,75) ;
\draw [shift={(198,114)}, rotate = 270] [color={rgb, 255:red, 0; green, 0; blue, 0 }  ][fill={rgb, 255:red, 0; green, 0; blue, 0 }  ][line width=0.75]      (0, 0) circle [x radius= 3.35, y radius= 3.35]   ;
%Straight Lines [id:da6000000943512647] 
\draw    (238,114) -- (238,96) ;
\draw [shift={(238,114)}, rotate = 270] [color={rgb, 255:red, 0; green, 0; blue, 0 }  ][fill={rgb, 255:red, 0; green, 0; blue, 0 }  ][line width=0.75]      (0, 0) circle [x radius= 3.35, y radius= 3.35]   ;
%Straight Lines [id:da5096851903596566] 
\draw    (298,114) -- (298,49) ;
\draw [shift={(298,114)}, rotate = 270] [color={rgb, 255:red, 0; green, 0; blue, 0 }  ][fill={rgb, 255:red, 0; green, 0; blue, 0 }  ][line width=0.75]      (0, 0) circle [x radius= 3.35, y radius= 3.35]   ;
%Straight Lines [id:da19644762271772676] 
\draw    (324,114) -- (324,29) ;
\draw [shift={(324,114)}, rotate = 270] [color={rgb, 255:red, 0; green, 0; blue, 0 }  ][fill={rgb, 255:red, 0; green, 0; blue, 0 }  ][line width=0.75]      (0, 0) circle [x radius= 3.35, y radius= 3.35]   ;
%Straight Lines [id:da9962709356508239] 
\draw    (350,114) -- (350,80) ;
\draw [shift={(350,114)}, rotate = 270] [color={rgb, 255:red, 0; green, 0; blue, 0 }  ][fill={rgb, 255:red, 0; green, 0; blue, 0 }  ][line width=0.75]      (0, 0) circle [x radius= 3.35, y radius= 3.35]   ;

% Text Node
\draw (390,103.4) node [anchor=north west][inner sep=0.75pt]    {$\mathbb{R}$};


\end{tikzpicture}

\begin{oss}
Se $X$ è VA reale con funzione di ripartizione $F_{X}$ costante a tratti, allora $X$ è discreta. $S$ sarà l'insieme dei punti di discontinuità e $p_{X}$ l'ampiezza dei salti.
\end{oss}
Esempio. Sia $X$ la VA che indica il numero di teste ottenute in un lancio di $3$ monete non truccate. Si può calcolare
\begin{equation*}
\mathbb{P}( X\leq x) =F_{X}( x) =\begin{cases}
0 & \text{se} \ x< 0\\
\frac{1}{8} & \text{se} \ 0\leq x< 1\\
\frac{1}{2} & \text{se} \ 1\leq x< 2\\
\frac{7}{8} & \text{se} \ 2\leq x< 3\\
1 & \text{se} \ x\geq 3
\end{cases}
\end{equation*}


\tikzset{every picture/.style={line width=0.75pt}} %set default line width to 0.75pt        

\begin{tikzpicture}[x=0.75pt,y=0.75pt,yscale=-1,xscale=1]
%uncomment if require: \path (0,266); %set diagram left start at 0, and has height of 266

%Shape: Axis 2D [id:dp7785871975299155] 
\draw  (80,221.59) -- (531.5,221.59)(148.58,24) -- (148.58,221.59) (524.5,216.59) -- (531.5,221.59) -- (524.5,226.59) (143.58,31) -- (148.58,24) -- (153.58,31)  ;
%Straight Lines [id:da18293161685682802] 
\draw [line width=1.5]    (149.5,201.59) -- (206.15,201.59) ;
\draw [shift={(209.5,201.59)}, rotate = 0] [color={rgb, 255:red, 0; green, 0; blue, 0 }  ][line width=1.5]      (0, 0) circle [x radius= 4.36, y radius= 4.36]   ;
\draw [shift={(149.5,201.59)}, rotate = 0] [color={rgb, 255:red, 0; green, 0; blue, 0 }  ][fill={rgb, 255:red, 0; green, 0; blue, 0 }  ][line width=1.5]      (0, 0) circle [x radius= 4.36, y radius= 4.36]   ;
%Straight Lines [id:da48480951914110326] 
\draw [line width=1.5]    (209.5,141.59) -- (266.15,141.59) ;
\draw [shift={(269.5,141.59)}, rotate = 0] [color={rgb, 255:red, 0; green, 0; blue, 0 }  ][line width=1.5]      (0, 0) circle [x radius= 4.36, y radius= 4.36]   ;
\draw [shift={(209.5,141.59)}, rotate = 0] [color={rgb, 255:red, 0; green, 0; blue, 0 }  ][fill={rgb, 255:red, 0; green, 0; blue, 0 }  ][line width=1.5]      (0, 0) circle [x radius= 4.36, y radius= 4.36]   ;
%Straight Lines [id:da7961634069824368] 
\draw [line width=1.5]    (329.5,61.59) -- (479.5,61.59) ;
\draw [shift={(329.5,61.59)}, rotate = 0] [color={rgb, 255:red, 0; green, 0; blue, 0 }  ][fill={rgb, 255:red, 0; green, 0; blue, 0 }  ][line width=1.5]      (0, 0) circle [x radius= 4.36, y radius= 4.36]   ;
%Straight Lines [id:da8883646431391186] 
\draw [line width=1.5]    (269.5,81.59) -- (326.15,81.59) ;
\draw [shift={(329.5,81.59)}, rotate = 0] [color={rgb, 255:red, 0; green, 0; blue, 0 }  ][line width=1.5]      (0, 0) circle [x radius= 4.36, y radius= 4.36]   ;
\draw [shift={(269.5,81.59)}, rotate = 0] [color={rgb, 255:red, 0; green, 0; blue, 0 }  ][fill={rgb, 255:red, 0; green, 0; blue, 0 }  ][line width=1.5]      (0, 0) circle [x radius= 4.36, y radius= 4.36]   ;
%Straight Lines [id:da7731345693466043] 
\draw [line width=1.5]    (74.5,221.59) -- (146.15,221.59) ;
\draw [shift={(149.5,221.59)}, rotate = 0] [color={rgb, 255:red, 0; green, 0; blue, 0 }  ][line width=1.5]      (0, 0) circle [x radius= 4.36, y radius= 4.36]   ;
%Straight Lines [id:da8859133785362521] 
\draw  [dash pattern={on 0.84pt off 2.51pt}]  (209.5,141.59) -- (209.5,221.5) ;
%Straight Lines [id:da7507588586070726] 
\draw  [dash pattern={on 0.84pt off 2.51pt}]  (269.5,81.59) -- (269.5,221.5) ;
%Straight Lines [id:da9788753181983849] 
\draw  [dash pattern={on 0.84pt off 2.51pt}]  (329.5,61.59) -- (329.5,221.5) ;
%Straight Lines [id:da37404258980558325] 
\draw  [dash pattern={on 0.84pt off 2.51pt}]  (209.5,141.59) -- (150.5,141.59) ;
%Straight Lines [id:da8671546504037952] 
\draw  [dash pattern={on 0.84pt off 2.51pt}]  (269.5,81.59) -- (150.5,81.59) ;
%Straight Lines [id:da178027804620988] 
\draw  [dash pattern={on 0.84pt off 2.51pt}]  (329.5,61.59) -- (149.5,61.59) ;

% Text Node
\draw (101,191.9) node [anchor=north west][inner sep=0.75pt]    {$1/8$};
% Text Node
\draw (101,131.9) node [anchor=north west][inner sep=0.75pt]    {$1/2$};
% Text Node
\draw (101,71.9) node [anchor=north west][inner sep=0.75pt]    {$7/8$};
% Text Node
\draw (124,51.9) node [anchor=north west][inner sep=0.75pt]    {$1$};
% Text Node
\draw (204,231.9) node [anchor=north west][inner sep=0.75pt]    {$1$};
% Text Node
\draw (144,231.9) node [anchor=north west][inner sep=0.75pt]    {$0$};
% Text Node
\draw (325,231.9) node [anchor=north west][inner sep=0.75pt]    {$3$};
% Text Node
\draw (265,231.9) node [anchor=north west][inner sep=0.75pt]    {$2$};


\end{tikzpicture}

$S$ è l'insieme dei punti di discontinuità di $F$: infatti la VA assume i valori in $S=\{0,1,2,3\}$.

$p_{X}( x) =\mathbb{P}( X=x)$, per esempio $p_{X}( 2) =\mathbb{P}( X=2) =F_{X}( 2) -F_{X}( 2)^{-} =\frac{7}{8} -\frac{1}{2} =\frac{3}{8}$ che è l'ampiezza del salto da $1$ a $2$, i.e. $p_{X}( x)$ rappresenta l'ampiezza dei salti.
\begin{oss}
$\Omega \ \text{discreto} \implies \mathrm{Im}( X) \ \text{discreta} \implies X\ \text{discreta} .$
\end{oss}
Possiamo riassumere quanto detto sopra nella seguente
\begin{theorem}
Sia $X:\Omega \rightarrow S\subset \mathbb{R}$ una VA discreta che assume, con probabilità $1$, valori in $S=\{x_{k} :k\in I\}$, $I\subset \mathbb{Z}$. Allora
\end{theorem}
\begin{enumerate}
\item $0\leq p_{X}( x) \leq 1,\forall x\in \mathbb{R}$ e $p_{X}( x) =0,\forall x\notin S$
\item $\sum\limits _{k\in I} p_{X}( k) =1$
\item se $F_{X}$ è la funzione di ripartizione di $X$ allora\begin{equation*}
F_{X}( x) =\sum\limits _{k:x_{k} \leq x} p_{X}( x_{k}) ,\ \ \ \ \forall x\in \mathbb{R}
\end{equation*}
\item se i punti di $S$ possono essere numerati in modo tale che $x_{h} < x_{k}$ se $h< k$, allora\begin{equation*}
p_{X}( x_{k}) =F_{X}( x_{k}) -F_{X}( x_{k-1}) ,\ \ \ \ \forall k\in I
\end{equation*}
\item se $B\subset \mathbb{R}$ allora\begin{equation*}
\mathbb{P}( X\in B) =\sum\limits _{k:x_{k} \in B} p_{X}( x_{k})
\end{equation*}
\end{enumerate}
\begin{oss}
$( 3)$ e $( 4)$ ci dicono che: se i punti di $S$ possono essere numerati in modo tale che $x_{h} < x_{k}$ per $h< k$, allora la funzione di ripartizione di una VA discreta è una funzione a gradini, che i gradini sono situati nei punti dell'insieme $S$ (i.e. $S$ è l'insieme dei punti di discontinuità di $F_{X}$) e che l'altezza del gradino corrispondente al punto $x_{k} \in S$ è proprio $p_{X}( x_{k})$ (i.e. $p_{X}$ è l'ampiezza dei salti della funzione di ripartizione $F_{X}$ nei punti di discontinuità).

$( 5)$ ci dice che è possibile costruire la probabilità di ogni evento $B\in \mathcal{A}$ a partire dalla densità di probabilità $p_{X}$.
\end{oss}
\Esercizio{fatto}

Si consideri la probabilità uniforme nell'intervallo $[ 0,1$] su $(\mathbb{R} ,\mathcal{B})$. Qui si definiscano le funzioni
\begin{itemize}
\item $X( \omega ) =\Ind_{[ 0,p]}( \omega ) ,\ \omega \in \mathbb{R} ,$
\item $Y( \omega ) =\Ind_{[ 0,p]}( \omega ) +\omega \Ind_{( -\infty ,0) \cup ( 1,\infty )}( \omega ) ,\ \omega \in \mathbb{R} ,$
\end{itemize}

dove $0< p< 1$ è un parametro fissato. Si ricordi che la notazione $\Ind_{A}$ indica la funzione indicatrice dell'insieme $A\in \mathcal{B}$, ossia
\begin{equation*}
\Ind_{A}( \Omega ) :=\begin{cases}
1,\ \omega \in A,\\
0,\ \omega \notin A,
\end{cases} \ \omega \in \mathbb{R} .
\end{equation*}
\begin{enumerate}
\item Si mostri che $X$ e $Y$ sono variabili aleatorie reali discrete. Si dimostri inoltre che $X=Y$ q.c.
\item Si determinino legge $P^{X} ,P^{Y}$ e valore atteso $\mathbb{E}[ X] ,\mathbb{E}[ Y]$.
\item Si esibisca un'altra variabile aleatoria $X' $ con la stessa legge di $X$, ma definita su uno spazio $\Omega ' $ discreto.
\end{enumerate}
\Esercizio{}

Sia $\Omega =[ 0,1]$ e si consideri la $\sigma $-algebra di Borel $A=\mathcal{B}([ 0,1])$. Si definisca $\mathbb{P}$ la probabilità uniforme su $( \Omega ,\mathcal{A})$, caratterizzata dalla proprietà:
\begin{equation*}
\mathbb{P}(( a,b]) =b-a,\ \text{per ogni} \ a< b,\ \text{con} \ a,b\in [ 0,1] .
\end{equation*}
\begin{enumerate}
\item Provare che $\mathbb{P}([ a,b]) =\mathbb{P}([ a,b)) =\mathbb{P}(( a,b)) =b-a$ per ogni $a< b$, con $a,b\in [ 0,1]$.
\item Mostrare che $\mathbb{P}(\{\Omega \}) =0$ per ogni $\omega \in \Omega $.
\end{enumerate}

Si definiscano le seguenti funzioni, per ogni $\omega \in \Omega $
\begin{equation*}
X( \omega ) \ =\ \Ind_{[ 0,p]}( \omega ) ,\ \ \ \ Y( \omega ) =\Ind_{( 0,p)}( \omega ) ,\ \ \ \ Z( \omega ) =\Ind_{[ 0,p]}( \omega ) +\Ind_{J}( \omega ) ,
\end{equation*}
dove $0< p< 1$ è un parametro fissato e $J=\left\{\frac{1}{n}\right\}_{n\in \mathbb{N}} =\left\{1,\frac{1}{2} ,\frac{1}{3} ,\dotsc \right\}$.
\begin{enumerate}
\item Si mostri che $X$, $Y$ e $Z$ sono variabili aleatorie reali discrete. Si dimostri inoltre che $X=Y=Z$ q.c.
\item Si determinino legge $P^{X} ,P^{Y} ,P^{Z}$ e valore atteso $\mathbb{E}[ X] ,\mathbb{E}[ Y] ,\mathbb{E}[ Z]$.
\end{enumerate}
\Esercizio{}

Si consideri la funzione reale di variabile reale
\begin{equation*}
F( x) =\begin{cases}
0, & x< 0,\\
1/2, & 0\leq x< 1,\\
2/3, & 1\leq x< \ 2,\\
11/12, & 2\leq x< \ 3,\\
1, & x\geq 3.
\end{cases}
\end{equation*}
\begin{enumerate}
\item Si mostri che $F$ è la funzione di ripartizione di una probabilità $\mathbb{P}$ su $(\mathbb{R} ,\mathcal{B})$.
\item Si mostri che $\mathbb{P}$ è una probabilità discreta su $(\mathbb{R} ,\mathcal{B})$, trovando i punti $x_{k}$ dove $\mathbb{P}$ è concentrata e le corrispondenti probabilità $p_{k}$, o equivalentemente determinando la densità (discreta) $p:\mathbb{R}\rightarrow [ 0,1]$ definita come segue:\begin{equation*}
p( x) =P(\{x\}) =\begin{cases}
p_{k} & \text{se} \ x=x_{k}\\
0 & \text{altrimenti}
\end{cases}
\end{equation*}

per ogni $x\in \mathbb{R}$.
\item Quanto valgono $\mathbb{P}(( 1/2,+\infty )) ,\mathbb{P}(( 2,4]) ,\mathbb{P}(( 1,2)) ,\mathbb{P}(( -\infty ,3))$?
\item Si introduca una variabile aleatoria reale e discreta $X$ di legge $P$, definita su uno spazio $\Omega $ \textit{discreto}.
\item Si introduca una variabile aleatoria reale e discreta $X$ di legge $P$, definita su uno spazio $\Omega $ \textit{continuo}.
\item Determinare la densità discreta di $X$, il suo valore atteso e la sua varianza.
\item Determinare la probabilità degli eventi: $X >\frac{1}{2} ,2< X\leq 4,1< X< 2$ e $X< 3$.
\item Determinare i quartili di $X$.
\item Mostrare che $Y=( X-2)^{2}$ è una variabile aleatoria discreta. Determinarne legge e valore atteso.
\end{enumerate}
\Esercizio{}

Si consideri la funzione
\begin{equation*}
F( x) =\sum\limits _{i=1}^{+\infty }\frac{1}{2^{i}}\Ind_{\left[\frac{1}{i} ,+\infty \right)}( x) ,\ \ \ \ x\in \mathbb{R}
\end{equation*}
\begin{enumerate}
\item Si mostri che si tratta di una funzione di ripartizione.
\end{enumerate}

Sia dunque $X$ una variabile aleatoria discreta con funzione di ripartizione $F$.
\begin{enumerate}
\item Determinare la densità discreta di $X$.
\item 3. Determinare la probabilità dei seguenti eventi:
\begin{enumerate}
\item $X\geq 1$;
\item $X\geq 1$;
\item $X\leq 0$;
\item $0\leq X< \frac{1}{2}$.
\end{enumerate}
\end{enumerate}
\Esercizio{fatto}

Consideriamo infinite prove di Bernoulli indipendenti con probabilità di successo $0< p< 1$. Sia quindi $\Omega =\{0,1\}^{\mathbb{N}}$, sia $\mathcal{A} =\sigma ( E_{k} \ |\ k=1,2,\dotsc )$ con
\begin{equation*}
E_{k} =\text{successo alla prova} \ k,
\end{equation*}
e sia $\mathbb{P}$ tale che $\{E_{k}\}_{k\in \mathbb{N}}$ risulti una famiglia di eventi indipendenti con $\mathbb{P}( E_{k}) =p$ per ogni $k$. Indicato con $\omega =( \omega _{k})_{k=1}^{\infty }$ il generico esito dello spazio campionario $\Omega $, definiamo le funzioni
\begin{equation*}
X_{n} :\Omega \rightarrow \mathbb{R} ,\ \ \ \ X_{n}( \omega ) =\omega _{n} ,\ \ \ \ n\in \mathbb{N} .
\end{equation*}
\begin{enumerate}
\item Si mostri che $X_{n} =\Ind_{E_{n}}$.
\item Si mostri che le $X_{n}$ sono variabili aleatorie reali discrete e se ne interpreti il significato probabilistico.
\item Si determini la legge delle $X_{n}$.
\item Per ogni $n$, si mostri che $Y_{n} =\sum _{k=1}^{n} X_{k}$ è una variabile aleatoria discreta, se ne dia il significato probabilistico e se ne determini la legge.
\item Si mostri che $Z=\min\{n\in \mathbb{N} :X_{n} =1\} =$ "numero di prove necessarie per il primo successo", con la convenzione $\min \emptyset =+\infty $, è una variabile aleatoria discreta e se ne calcoli la legge.
\item Si mostri che $W=$ "numero di insuccessi prima del primo successo" è una variabile aleatoria discreta e se ne calcoli la legge.
\item Si mostri che $V=\liminf\limits _{n} X_{n}$ e $U=\limsup\limits _{n} X_{n}$ sono variabili aleatorie discrete e se ne calcolino le leggi.
\end{enumerate}

Cosa sarebbe cambiato se avessimo realizzato in un altro spazio di probabilità $( \Omega ' ,\mathcal{A} ' ,\mathbb{P} ' )$, diverso dallo spazio di Bernoulli, una successione di eventi $E_{n} ' $ indipendenti e tali che $\mathbb{P} ' ( E_{n} ' ) =p$, ed avessimo poi posto $X_{n} ' =\Ind_{E_{n} ' }$, ovvero se avessimo usato un differente spazio di probabilità per rappresentare infinite prove di Bernoulli indipendenti con probabilità di successo $0< p< 1$?
\Esercizio{$\star$}

Sia $X$ una variabile aleatoria a valori in $\{0,1,2,\dotsc \}$. Si mostri che
\begin{equation*}
\mathbb{E}[ X] =\sum\limits _{n=0}^{\infty }\mathbb{P}( X >n) .
\end{equation*}
\Esercizio{(Distribuzione binomiale). fatto}

Si consideri una variabile aleatoria binomiale $X$ di parametri $n$ e $p$, con $n\in \mathbb{N}$ e $0< p< 1$. Scriviamo $X\sim B( n,p)$.
\begin{enumerate}
\item Calcolare il valore atteso $\mathbb{E}[ X]$.
\item Calcolare la varianza di $X$.
\item Calcolare le mode di $X$, ovvero i punti di massimo della densità discreta di $X$.
\end{enumerate}
\Esercizio{(Distribuzione di Poisson). fatto}

Si consideri una variabile aleatoria di Poisson $X$ di parametro $\lambda  >0$. Scriviamo $X\sim P( \lambda )$.
\begin{enumerate}
\item Calcolare il valore atteso $\mathbb{E}[ X]$.
\item Calcolare la varianza di $X$.
\item Calcolare le mode di $X$.
\item Dato $k\in \{1,2,\dotsc \}$, quale $\lambda  >0$ massimizza $\mathbb{P}( X=k)$?
\end{enumerate}
\Esercizio{(Distribuzione geometrica). fatto}

Si consideri $X\sim \mathcal{G}( p)$, variabile aleatoria geometrica di parametro $p$, $0< p< 1$. Si ha pertanto
\begin{equation*}
\mathbb{P}( X=k) =p( 1-p)^{k-1} ,\ \ \ \ k\in \mathbb{N} .
\end{equation*}
\begin{enumerate}
\item Mostrare due diversi spazi $( \Omega ,\mathcal{A} ,\mathbb{P})$, uno discreto e uno continuo, su cui è possibile definire $X$.
\item Calcolare le mode di $X$.
\item Calcolare la funzione di ripartizione $F$ di $X$.
\item Calcolare il valore atteso $\mathbb{E}[ X]$.
\item (Proprietà di assenza di memoria) Mostrare che $\mathbb{P}( X >i+j\ |\ X >i) =\mathbb{P}( X >j)$ per ogni $i,j\in \mathbb{N}$.
\item (*) Si inverta il risultato appena ottenuto: se $T$ è una variabile aleatoria reale discreta a valori in $\mathbb{N}$ tale che $\mathbb{P}( T >i+j\ |\ T >i) =\mathbb{P}( T >j)$ per ogni $i,j\in \mathbb{N}$, allora $T\sim \mathcal{G}( q$) con $q=\mathbb{P}( T=1)$.
\end{enumerate}
\Esercizio{}

Data una variabile aleatoria $X$ di Poisson di media $3$, si considerino
\begin{equation*}
Y=\min( 3,X) ,\ \ \ \ W=e^{X/3} .
\end{equation*}
\begin{enumerate}
\item Si mostri che $Y$ e $W$ sono variabili aleatorie reali e discrete.
\item Si calcolino le loro distribuzioni.
\item Si calcolino le loro medie.
\end{enumerate}
\Esercizio{}

Data una variabile aleatoria $X$ con distribuzione geometrica di media $3$, si considerino
\begin{equation*}
Y=\min( 3,X) ,\ \ \ \ W=e^{X/3} .
\end{equation*}
\begin{enumerate}
\item Si mostri che $Y$ e $W$ sono variabili aleatorie reali e discrete.
\item Si calcolino le loro distribuzioni.
\item Si calcolino le loro medie.
\end{enumerate}
\Esercizio{fatto}

In una certa provincia montuosa si può supporre che il numero $X$ di frane al mese sia una variabile aleatoria con legge di Poisson di parametro $\lambda =2.3$.
\begin{enumerate}
\item Calcolare la probabilità che ci siano almeno due frane in un dato mese.
\item Calcolare mediana e quantile di ordine $0.6$ del numero di frane mensili.
\item Quanto dovrebbe valere il parametro $\lambda $ affinché la probabilità che in un mese non ci siano frane sia superiore a $1/2$?
\end{enumerate}
\Esercizio{}

Al buio cerco la chiave del mio ufficio in un mazzo di $10$ chiavi tutte della stessa fattura. Ovviamente metto da parte le chiavi provate. Sia $X$ il numero di chiavi che devo provare per aprire l'ufficio.
\begin{enumerate}
\item Qual è la legge di $X$?
\item Qual è il numero atteso di tentativi da fare?
\item Quanto vale la probabilità di controllare almeno $8$ chiavi?
\item Sapendo che al primo tentativo non ho trovato la chiave giusta, con quale probabilità non la trovo neanche al secondo?
\item Come cambiano le risposte ai punti precedenti se stupidamente non metto da parte le chiavi già provate prima di procedere a provarne una nuova?
\end{enumerate}
\Esercizio{}

Nel Gioco del Lotto ad ogni estrazione settimanale $5$ numeri vengono estratti simultaneamente da un'urna che contiene $90$ palline numerate da $1$ a $90$. Fissato un numero, ad esempio il $67$, sia $p$ la probabilità che esca in una singola estrazione.
\begin{enumerate}
\item Quanto vale p?
\item Qual è la probabilità che dopo $30$ estrazioni il $67$ non sia ancora uscito?
\item Supponiamo che nelle prime $100$ estrazioni il $67$ non sia ancora uscito, qual è la probabilità che esca dopo la $130$ esima estrazione?
\item Qual è la probabilità che esca almeno $6$ volte nelle prime $50$ estrazioni?
\item Se nelle prime $10$ estrazioni il $67$ è comparso $2$ volte, con quale probabilità è comparso alle prime $2$ estrazioni?
\end{enumerate}
\Esercizio{fatto}

Un'urna contiene una pallina rossa ed una blu. Una pallina viene estratta a caso. Se è blu il gioco termina. Se è rossa la pallina viene rimessa nell'urna insieme ad un'altra rossa.
\begin{enumerate}
\item Supponiamo che la procedura sopra descritta venga ripetuta fino ad aver fatto $10$ estrazioni o alla prima estrazione di una pallina blu, se si presenta prima della decima estrazione. Sia $X$ il numero di estrazioni effettuate. Determinare la distribuzione di $X$ e il suo valore atteso.
\item Supponiamo ora che il gioco termini solo quando compare la prima pallina blu e sia Y il numero di estrazioni in questo caso.
\begin{enumerate}
\item Determinare la probabilità di effettuare almeno n estrazioni prima che il gioco finisca.
\item Determinare la probabilità che il gioco non finisca mai.
\item Determinare la distribuzione di $Y$ e il suo valore atteso.
\end{enumerate}
\end{enumerate}
\Esercizio{fatto}

I punti $N$ realizzati dai Caprica Buccaneers in una partita di Piramid hanno distribuzione di Poisson di media $\lambda $. Vi propongono di scommettere contro i Caprica Buccaneers alla prossima partita: a fronte di una vostra puntata unitaria riceverete un ricavo pari a
\begin{equation*}
R=\frac{7}{3^{N}} ,
\end{equation*}
guadagnando quindi $G=R-1$.
\begin{enumerate}
\item Trovare la distribuzione del ricavo $R$ in funzione di $\lambda $.
\item Trovare il ricavo atteso in funzione di $\lambda $.
\end{enumerate}

Si supponga che la scommessa proposta sia equa.
\begin{enumerate}
\item Quanti punti a partita segnano mediamente i Caprica Buccaneers?
\item Con quale probabilità otterrete il ricavo massimo?
\item Con quale probabilità non perderete soldi?
\item Qual è il ricavo più probabile?
\end{enumerate}







\ParteSoluzioni






\Soluzione
\begin{enumerate}
\item Mostriamo innanzitutto che\begin{equation*}
X,Y:(\mathbb{R} ,\mathcal{B})\rightarrow (\mathbb{R} ,\mathcal{B})
\end{equation*}

sono VA reali. Secondo la definizione dobbiamo mostrare che\begin{equation*}
\forall B\in \mathcal{B} ,\ \ \ \ X^{-1}( B) \in \mathcal{B} ,\ \ \ \ Y^{-1}( B) \in \mathcal{B} ,
\end{equation*}

i.e. $X,Y$ sono misurabili.\begin{oss}
Se $( \Omega ,\mathcal{A} ,\mathbb{P})$ è uno spazio di probabilità, $X=\Ind_{A}$ è misurabile $\iff A\in \mathcal{A}$. Infatti
\begin{equation*}
\Ind_{A} :( \Omega ,\mathcal{A})\rightarrow \{0,1\} \subset \mathbb{R}
\end{equation*}
e
\begin{equation*}
\begin{drcases}
\Ind_{A}^{-1}\{0\} =A\comp\\
\Ind_{A}^{-1}\{1\} =A
\end{drcases} \ \in \mathcal{A} \iff A\in \mathcal{A}
\end{equation*}
\end{oss}

Quindi per verificare che $X$ è una VA ci basterà verificare che $[ 0,p] \in \mathcal{B}$. Questo è vero perché $[ 0,p]$ è chiuso. Quindi $X$ è VA reale.

In maniera simile deduciamo che $Y$ è VA reale perché somma di $X$ e del prodotto di funzioni misurabili.

Per dimostrare che $X,Y$ sono VA discrete, dobbiamo dimostrare che la legge immagine di $X,Y$ $\left( P^{X} ,P^{Y}\right)$ è discreta, i.e. $\exists S\subset \mathbb{R}$ al più numerabile e $p_{X} :S\rightarrow [ 0,1]$ tale che $P^{X}( B) =\sum\limits _{x\in B\cap S} p_{X}( x)$. Osserviamo però che\begin{equation*}
\mathrm{Im}( X) =\{0,1\} \implies X\ \text{discreta}
\end{equation*}

Inoltre\begin{equation*}
p_{X}( x) =\mathbb{P}( X=x) =\begin{cases}
p, & x=1\\
1-p, & x=0
\end{cases}
\end{equation*}

Infatti\begin{equation*}
\begin{aligned}
\text{se} \ x=1:\mathbb{P}( X=1) & =\mathbb{P}(\Ind_{[ 0,p]} =1)\\
 & =\mathbb{P}(\{\omega \in \Omega :\Ind_{[ 0,p]}( \omega ) =1\})\\
 & =\mathbb{P}\left(\Ind_{[ 0,p]}^{-1}( 1)\right)\\
 & =\mathbb{P}([ 0,p])\\
 & =p\ \left(\text{su} \ \mathbb{R} \ \text{ho prob. uniforme}\right)\\
 & \\
\text{se} \ x=0:\mathbb{P}( X=0) & =\mathbb{P}(\Ind_{[ 0,p]} =0)\\
 & =\mathbb{P}(\{\omega \in \Omega :\Ind_{[ 0,p]}( \omega ) =0\})\\
 & =\mathbb{P}\left(\Ind_{[ 0,p]}^{-1}( 0)\right)\\
 & =\mathbb{P}(( -\infty ,p) \cup ( p,+\infty ))\\
 & =\mathbb{P}(\mathbb{R} \smallsetminus [ 0,p])\\
 & =1-\mathbb{P}([ 0,p])\\
 & =1-p
\end{aligned}
\end{equation*}

Possiamo allora determinare la legge di $X$\begin{gather*}
\forall B\in \mathcal{B} ,P^{X}( B) =\mathbb{P}( X\in B) =\sum\limits _{x\in B\cap \{0,1\}} p_{X}( x) =\\
=\begin{cases}
p, & B=\{1\}\\
1-p, & B=\{0\}\\
0 & \text{altrimenti}
\end{cases} \iff \boxed{X\sim B( p)}
\end{gather*}

$X$ è una VA reale discreta Bernoulliana di parametro $p$.

Stabilire se $Y$ è discreta è meno immediato, infatti $\mathrm{Im}( Y) =( -\infty ,0] \cup [ 1,+\infty )$.

\begin{oss}
Sappiamo che se $Y$ assume valori discreti, allora $Y$ è una VA discreta, ma non è necessariamente vero il viceversa.
\end{oss}

Quindi da $\mathrm{Im}( Y)$ non possiamo dedurre nulla.

Se però proviamo che $X=Y$ q.c. allora, in particolare, avranno la stessa legge $X\sim Y\sim B( n) \implies Y$ discreta.

Proviamo allora che $X=Y$ q.c. Dobbiamo mostrare che $\mathbb{P}( X=Y) =1$.\begin{equation*}
\begin{aligned}
\mathbb{P}( X=Y) & =\mathbb{P}(\{\omega \in \Omega :X( \omega ) =Y( \omega )\})\\
 & =\mathbb{P}(\{\omega \in \Omega :X( \omega ) -Y( \omega ) =0\})\\
 & =\mathbb{P}(\{\omega \in \Omega :\omega \Ind_{( -\infty ,0) \cup ( 1,+\infty )}( \omega ) =0\})
\end{aligned}
\end{equation*}

Abbiamo che\begin{equation*}
\begin{aligned}
\omega \in ( -\infty ,0) \cup ( 1,+\infty ) & \implies \omega \Ind_{( -\infty ,0) \cup ( 1,+\infty )}( \omega ) \neq 0\\
\omega \in [ 0,1] & \implies \omega \Ind_{( -\infty ,0) \cup ( 1,+\infty )}( \omega ) =0
\end{aligned}
\end{equation*}

i.e.\begin{equation*}
\{\omega \in \Omega :\omega \Ind_{( -\infty ,0) \cup ( 1,+\infty )}( \omega ) =0\} =[ 0,1]
\end{equation*}

Quindi\begin{equation*}
\mathbb{P}( X=Y) =\mathbb{P}(\{\omega \in \Omega :\omega \Ind_{( -\infty ,0) \cup ( 1,+\infty )}( \omega ) =0\}) =\mathbb{P}([ 0,1]) =1
\end{equation*}

i.e. $X=Y$ q.c.
\item \begin{theorem}
Sia $X$ una VA reale definita su $( \Omega ,\mathcal{A} ,\mathbb{P})$ e sia $h:\mathbb{R}\rightarrow \mathbb{R}$ boreliana. Allora
\begin{equation*}
\mathbb{E}[| h( x)| ] =\int _{\Omega }| h( X( \omega ))| d\mathbb{P}( \omega ) < +\infty \ \ \iff \ \ \int _{\mathbb{R}}| h( x)| P^{X}( dx) < +\infty 
\end{equation*}
i.e. $h\circ X=h( X) \in \mathcal{L}^{1}(\mathbb{P}) \iff h\in \mathcal{L}^{1}\left( P^{X}\right)$.

Nel caso precedente, oppure se $h\geq 0$, valgono\begin{equation*}
\mathbb{E}[ h( X)] =\int _{\mathbb{R}} h( x) P^{X}( dx) \ \ \ \ \mathbb{E}[ X] =\int _{\mathbb{R}} xP^{X}( dx)
\end{equation*}

Se, in particolare, $X$ è discreta con densità $p_{X}$\begin{equation*}
\mathbb{E}[ h( X)] =\sum\limits _{x\in S} h( x) p_{X}( x) \ \ \ \ \mathbb{E}[ X] =\sum\limits _{x\in S} xp_{X}( x)
\end{equation*}
\end{theorem}

Abbiamo già calcolato la legge di $X$ e $Y$: $X\sim Y\sim B( p)$.\begin{equation*}
\mathbb{E}[ X] =\sum\limits _{x\in S} xp_{X}( x)\overset{S=\{0,1\}}{=}\underbrace{\sum\limits _{x\in \{0,1\}} xp_{X}( x) =0( 1-p) +1p}_{\text{ricordando la legge di } X} =p
\end{equation*}

Poiché $X=Y$ q.c., $X$ e $Y$ hanno lo stesso valore atteso.
\item Basta considerare l'identità (vedi esercizio $5$)\begin{gather*}
\Omega '=\{0,1\} ,\ \ \ \ \mathcal{A} '=2^{\Omega '} ,\ \ \ \ \mathbb{P} =P^{X} ,\ \ \ \ \tilde{X}( \omega ) :=\omega \\
\begin{aligned}
\tilde{X} =I:( \Omega ',\mathcal{A} ') & \rightarrow ( \Omega ',\mathcal{A} ')\\
\omega  & \mapsto \omega 
\end{aligned}
\end{gather*}

\begin{oss}
In generale è sempre possibile costruire uno spazio di probabilità $( \Omega ,\mathcal{F} ,\mathbb{P})$ e una VA $X$ su di esso che ha $p( \cdotp )$ come densità, i.e. tale che $p_{X}( x) =p( x)$. Infatti basta prendere $\Omega =S,\mathcal{F} =\mathcal{P}( S)$ e $p$ l'unica misura di probabilità su $S$ tale che $\mathbb{P}(\{x_{k}\}) =p( x_{k})$ con $k\in I$. È immediato verificare che la VA discreta $X( \omega ) =\omega ,\forall \omega \in \Omega $ ha densità $p( \cdotp )$.
\end{oss}
\end{enumerate}
\Soluzione

Manca.
\Soluzione

Manca.
\Soluzione

Manca.
\Soluzione
\begin{enumerate}
\item Fissiamo $\omega \in \Omega $. Dobbiamo mostrare che $X_{n}\left( \omega \right) =\Ind_{E_{n}}\left( \omega \right)$. Ricordiamo che $E_{n} =\left\{\omega \in \Omega :\omega _{n} =1\right\}$. Abbiamo $\forall \omega \in \Omega $:\begin{equation}
X_{n}\left( \omega \right) =\omega _{n} =\begin{cases}
1, & \omega _{n} =1\\
0, & \omega _{n} =0
\end{cases}
\end{equation}

e\begin{equation}
\Ind_{E_{n}}\left( \omega \right) =\begin{cases}
1, & \omega \in E_{n} \iff \omega _{n} =1\\
0, & \omega \notin E_{n} \iff \omega _{n} =0
\end{cases}
\end{equation}

Da (1) e (2) si allora che\begin{equation*}
\forall \omega \in \Omega ,\ \ \ \ X_{n}\left( \omega \right) =\Ind_{E_{n}}\left( \omega \right) ,\ \ \ \ \text{i.e.} \ \ \ \ X_{n} =\Ind_{E_{n}}
\end{equation*}

\begin{oss}
Questo ci dice che la VA $X_{n}$ rappresenta l'esito della $n$-esima prova, i.e. $1$ se otteniamo un successo, $0$ altrimenti.
\end{oss}
\item Sono funzioni a valori reali. In particolare $\forall n,\mathrm{Im}( X_{n}) =\{0,1\} \implies $ sono discrete. Resta da verificare che sono VA, ovvero che sono misurabili, i.e.\begin{gather*}
\begin{aligned}
X_{n}( \Omega ,\mathcal{A}) & \rightarrow (\mathbb{R} ,\mathcal{B})\\
\omega  & \mapsto X_{n}( \omega ) =\omega _{n}
\end{aligned}\\
X_{n}^{-1}( B) \in \mathcal{A} ,\forall B\in \mathcal{B}
\end{gather*}

Ci basta verificarlo su insiemi della forma $B=( -\infty ,t]$\begin{equation*}
\begin{aligned}
X_{n}^{-1}(( -\infty ,t]) & =\{\omega \in \Omega :X_{n}( \omega ) \in ( -\infty ,t]\}\\
 & =\{\omega \in \Omega :\Ind_{E_{n}}( \omega ) \in ( -\infty ,t]\}\\
 & =\Ind_{E_{n}}^{-1}( -\infty ,t] =\begin{cases}
\emptyset , & t< 0\\
E_{n}\comp , & 0\leq t< 1\\
\Omega , & t\geq 1
\end{cases}
\end{aligned}
\end{equation*}

Nell'ultimo passaggio bisogna ricordare che\begin{equation*}
\Ind_{E_{n}} \in \{0,1\} ,\ \ \ \ \Ind_{E_{n}}( \omega ) =\begin{cases}
1, & \omega \in E_{n}\\
0, & \omega \notin E_{n}
\end{cases}
\end{equation*}

Dato che $\emptyset ,E_{n}\comp ,\Omega \in \mathcal{A}$ abbiamo che $X_{n}$ è misurabile.

\textit{Significato:} esito dell'$n$-esima prova, può essere successo o insuccesso.
\item Le $X_{n}$ sono VA discrete, quindi hanno una densità discreta\begin{gather*}
x\in \{0,1\}\\
\mathbb{P}( X_{n} =1) =\mathbb{P}(\{\omega \in \Omega :X_{n}( \omega ) =1\}) =\mathbb{P}\left( X_{n}^{-1}(\{1\})\right) =\mathbb{P}( E_{n}) =p\\
\mathbb{P}( X_{n} =0) =\mathbb{P}(\{\omega \in \Omega :X_{n}( \omega ) =0\}) =\mathbb{P}\left( X_{n}^{-1}(\{0\})\right) =\mathbb{P}\left( E_{n}\comp\right) =1-p
\end{gather*}

quindi

\begin{gather*}
p_{X_{n}} :\{0,1\}\rightarrow [ 0,1]\\
p_{X_{n}}( x) =\mathbb{P}( X_{n} =x) =\begin{cases}
p, & x=1\\
1-p, & x=0
\end{cases}
\end{gather*}

Infine\begin{equation*}
P^{X}( B) =\sum\limits _{x\in B\cap \{0,1\}} p_{X_{n}}( x) =\begin{cases}
p, & B=\{1\}\\
1-p, & B=\{0\}\\
0 & \text{altrimenti}
\end{cases}
\end{equation*}

i.e.\begin{equation*}
\boxed{X_{n} \sim B( p)}
\end{equation*}
\item $Y_{n} =\sum\limits _{k=1}^{n} X_{k}$

È una VA reale perché è somma di VA reali.

È discreta perché $\mathrm{Im}( Y_{n}) =\{0,\dotsc ,n\}$.

Descrive il numero di successi in $n$ prove di Bernoulli.

Densità discreta:\begin{gather*}
p_{Y_{n}} :\{0,\dotsc ,n\}\rightarrow [ 0,1]\\
\begin{aligned}
p_{Y_{n}}( 0) & =\mathbb{P}( Y_{n} =0) =\mathbb{P}( X_{1} +\cdots +X_{n} =0) =\mathbb{P}\left(\bigcap _{j=1}^{n} E_{j}\comp\right) =( 1-p)^{n}\\
p_{Y_{n}}( 1) & =\mathbb{P}( Y_{n} =1) =\mathbb{P}( X_{1} +\cdots +X_{n} =1) =\mathbb{P}\left(\bigcup _{i=1}^{n} E_{i}\bigcap _{j\neq i}^{n} E_{j}\comp\right)\\
 & =\sum\limits _{i=1}^{n}\mathbb{P}( E_{i})\prod _{j\neq i}\mathbb{P}\left( E_{j}\comp\right) =np( 1-p)^{n-1}\\
p_{Y_{n}}( k) & =\mathbb{P}( Y_{n} =k) =\mathbb{P}( X_{1} +\cdots +X_{n} =k) =\mathbb{P}\left(\bigcup\limits _{J\subset \{1..n\} ,| J| =k}\left[\bigcap _{j\in J} E_{j} \cap \bigcap _{j\notin J}^{n} E_{j}\comp\right]\right)\\
 & =\sum\limits _{J\subset \{1..n\} ,| J| =k}\prod _{j\in J} \mathbb{P}( E_{j})\prod _{j\notin J}\mathbb{P}\left( E_{j}\comp\right) =\binom{n}{k} p^{k}( 1-p)^{n-k}
\end{aligned}
\end{gather*}

i.e.\begin{equation*}
\forall k,\ \ \ \ p_{Y_{n}}\left( k\right) =\binom{n}{k} p^{k}\left( 1-p\right)^{n-k} ,\ \ \ \ Y_{n} \sim B\left( n,p\right)
\end{equation*}

la somma di $n$ Bernoulliane di parametro $p$ è una Binomiale $B\left( n,p\right)$.\begin{oss}
Questo risultato non è vero in generale. La somma di VA bernoulliane è binomiale solo se queste sono indipendenti e con la stessa probabilità di successo $p$.
\end{oss}

\textit{Interpretazione:} la distribuzione binomiale $B\left( n,p\right)$ descrive la probabilità del numero di successi in $n$ prove di Bernoulli indipendenti e con la stessa probabilità di successo $p$.
\item $Z$ è VA (perché minimo di VA) a valori in $\mathrm{Im}\left( Z\right) =\mathbb{N} \cup \left\{+\infty \right\} \implies $ VA discreta. Densità discreta: $p_{Z} :\mathbb{N} \cup \left\{+\infty \right\}\rightarrow \left[ 0,1\right]$. Sia $k\in \mathbb{N}$ fissato\begin{equation*}
p_{Z}( k) =\underbrace{\mathbb{P}( Z=k) =\mathbb{P}\left(\bigcap\limits _{j=1}^{k-1} E_{j}\comp \cap E_{k}\right)}_{Z:=\min\{n\in \mathbb{N} :X_{n} =1\}} =p( 1-p)^{k-1}
\end{equation*}

$Z$ è il numero minimo di prove per dare un successo.\begin{equation*}
\begin{aligned}
\overbrace{p_{Z}( +\infty ) =\mathbb{P}( Z=+\infty )}^{\text{i.e. sempre insuccessi}} & =\mathbb{P}\left(\bigcap _{j\in \mathbb{N}} E_{j}\comp\right) =\lim _{n\rightarrow \infty }\mathbb{P}\left(\bigcap _{j=1}^{n} E_{j}\comp\right)\\
 & =\lim _{n\rightarrow \infty }( 1-p)^{n} =0
\end{aligned}
\end{equation*}

\begin{oss}
Potevamo calcolare $p_{Z}( +\infty )$ anche nel seguente modo
\begin{equation*}
\begin{aligned}
p_{Z}( +\infty ) & =1-\sum\limits _{k=1}^{\infty } p_{Z}( k) =1-p\sum\limits _{k=1}^{\infty }( 1-p)^{k-1} =1-p\sum\limits _{k=0}^{\infty }( 1-p)^{k}\\
 & =1-p\frac{1}{1-( 1-p)} =1-\frac{p}{p} =0
\end{aligned}
\end{equation*}
dove abbiamo usato la serie geometrica
\begin{equation*}
\sum\limits _{n=0}^{\infty } q^{n} =\frac{1}{1-q} ,\ \ \ \ | q| < 1.
\end{equation*}
\end{oss}

Quindi\begin{equation*}
\forall k,\ \ \ \ p_{Z}( k) =p( 1-p)^{k-1} ,\ \ \ \ \boxed{Z\sim \mathcal{G}( p)}
\end{equation*}

\textit{Interpretazione:} la distribuzione geometrica descrive la probabilità che il primo successo richieda l'esecuzione di $k$ prove.
\item $W$ è il numero di insuccessi prima del primo successo. Ricordiamo che $Z$ era uguale al numero di prove necessarie per il primo successo, per esempio avremo\begin{equation*}
\underbrace{\overbrace{0,0,0}^{W=3} ,1}_{Z=4} ,\dotsc \implies W=Z-1\ \text{q.c.}
\end{equation*}

Quindi $W$ è una VA discreta.\begin{gather*}
p_{W} :\{0,1,\dotsc \}\rightarrow [ 0,1]\\
p_{W}( k) =\mathbb{P}( W=k) =\mathbb{P}( Z-1=k) =\underbrace{\mathbb{P}( Z=k+1) =p( 1-p)^{k}}_{Z\sim \mathcal{G}( p)}
\end{gather*}

Allora anche $W$ è una geometrica $W\sim \mathcal{G}( p)$.
\item $V:=\liminf _{n} X_{n}$, $U:=\limsup _{n} X_{n}$ sono VA perché, rispettivamente, $\liminf _{n}$ e $\limsup _{n}$ di VA. In particolare\begin{align*}
V & =\liminf _{n} X_{n} =\liminf _{n}\Ind_{E_{n}} =\Ind_{\liminf\limits _{n} E_{n}}
\end{align*}

l'ultimo passaggio è giustificato dal fatto che\begin{equation*}
\begin{aligned}
\Ind_{\liminf _{n} E_{n}} & =\Ind_{\bigcup\limits _{n}\bigcap\limits _{k\geq n} E_{k}} & \text{(definizione)}\\
 & =\sup\limits _{n\geq 0}\Ind_{\bigcap\limits _{k\geq n} E_{k}} & \Ind_{A\cup B} =\sup \{\Ind_{A} ,\Ind_{B}\}\\
 & =\sup _{n\geq 0}\inf_{k\geq n}\Ind_{E_{k}} & \Ind_{A\cap B} =\inf\{\Ind_{A} ,\Ind_{B}\}
\end{aligned}
\end{equation*}

Analogamente,\begin{equation*}
U=\limsup _{n} X_{n} =\Ind_{\limsup\limits _{n} E_{n}}
\end{equation*}

Allora\begin{equation*}
\mathrm{Im}( V) =\mathrm{Im}( U) =\{0,1\}
\end{equation*}

Ricaviamoci la densità discreta di $V$ (il procedimento per $U$ è simile)\begin{gather*}
p_{V} :\{0,1\}\rightarrow [ 0,1]\\
p_{V}( 1) =\mathbb{P}( V=1) =\mathbb{P}\left(\liminf _{n} X_{n} =1\right) =\mathbb{P}\left(\liminf _{n} E_{n}\right)
\end{gather*}

Ora\begin{equation*}
\liminf _{n} E_{n} =\bigcup _{n\geq 0}\underbrace{\bigcap\limits _{k\geq n} E_{k}}_{A_{n} \ } \ \ \ \implies \ \ \ \ A_{n} \uparrow \bigcup _{n\geq 0} A_{n}
\end{equation*}

Quindi\begin{equation*}
\mathbb{P}\left(\liminf _{n} E_{n}\right) =\mathbb{P}\left(\bigcup _{n\geq 0} A_{n}\right)
\end{equation*}

Osserviamo ora che\begin{equation*}
\forall n\ \text{fissato} ,\ \mathbb{P}\left( A_{n}\right) =\lim _{m\rightarrow +\infty }\mathbb{P}\left(\bigcap\limits _{k=n}^{m} E_{k}\right) =\lim _{m\rightarrow +\infty } p^{m-n+1} =0
\end{equation*}

Allora\begin{gather*}
\mathbb{P}\left(\liminf _{n} E_{n}\right) =\mathbb{P}\left(\bigcup _{n\geq 0} A_{n}\right) =0\\
\begin{aligned}
\implies p_{V}( 1) & =0\\
p_{V}( 0) & =\mathbb{P}( V=0) =1-\mathbb{P}( V=1) =1
\end{aligned}
\end{gather*}

Concludiamo che\begin{equation*}
p_{V}\left( k\right) =\mathbb{P}\left( V=k\right) =\begin{cases}
0, & k=1\\
1, & k=0
\end{cases} \ \ \ \ V\sim B\left( 0\right) ,\ \ \ \ V=0\ \text{q.c.}
\end{equation*}

Analogamente si mostra che\begin{equation*}
U\sim B( 1) ,\ \ \ \ U=1\ \text{q.c.}
\end{equation*}
\end{enumerate}
\begin{oss}
Alcune considerazioni Es. 1 ed Es. 5.

In quegli esercizi abbiamo visto \textit{diversi possibili modi} in cui introdurre una VA con distribuzione di Bernoulli.

Nell'Es. 1 abbiamo considerato uno spazio di probabilità continuo
\begin{equation*}
( \Omega ,\mathcal{A} ,\mathbb{P}) =(\mathbb{R} ,\mathcal{B} ,\mathbb{P}_{\text{unif.}})
\end{equation*}
Abbiamo introdotto le VA
\begin{gather*}
\begin{aligned}
X:(\mathbb{R} ,\mathcal{B} ,\mathbb{P}_{\text{unif.}}) & \rightarrow (\mathbb{R} ,\mathcal{B})\\
\omega  & \mapsto X( \omega ) :=\Ind_{[ 0,p]}( \omega )\\
Y:(\mathbb{R} ,\mathcal{B} ,\mathbb{P}_{\text{unif.}}) & \rightarrow (\mathbb{R} ,\mathcal{B})\\
\omega  & \mapsto Y( \omega ) :=\Ind_{[ 0,p]}( \omega ) +\omega \Ind_{( -\infty ,0) \cup ( 1,\infty )}( \omega )
\end{aligned}\\
\ \\
\implies X\sim B( p) ,Y\sim B( p)
\end{gather*}
Negli Es. 5 e 1.c abbiamo considerato uno spazio di probabilità discreto
\begin{gather*}
( \Omega ,\mathcal{A} ,\mathbb{P}) =\left(\{0,1\} ,\sigma (\{0\}) ,\tilde{\mathbb{P}}\right)\\
\tilde{\mathbb{P}}(\{0\}) =1-p,\ \ \ \ \tilde{\mathbb{P}}(\{1\}) =p
\end{gather*}
Abbiamo introdotto la VA
\begin{gather*}
\begin{aligned}
Z:\left(\{0,1\} ,\sigma (\{0\}) ,\tilde{\mathbb{P}}\right) & \rightarrow (\mathbb{R} ,\mathcal{B})\\
\omega  & \mapsto Z( \omega )
\end{aligned}\\
\\
\implies Z\sim B( p)
\end{gather*}
\end{oss}
\Soluzione

Manca.
\Soluzione

Richiami**************
\begin{itemize}
\item Nell'Es. 5 abbiamo visto che una VA $X\sim B( n,p)$ rappresenta il numero di successo ottenuti in $n$ prove di Bernoulli, aventi probabilità di successo $p$.
\item La distribuzione binomiale è caratterizzata da due parametri:
\begin{itemize}
\item $n$ i.e. il numero di prove effettuate
\item $p$ i.e. la probabilità di successo della singola prova di Bernoulli $( 0\leq p\leq 1)$.
\end{itemize}
\item Formalmente $X\sim B( n,p)$ si scrive come somma di $n$ VA $\Bot $ $X_{i} \sim B( p)$ (bernoulliane di parametro $p$)\begin{equation*}
X=\sum\limits _{i=1}^{n} X_{i}
\end{equation*}
\item La distribuzione di probabilità di $X\sim B( n,p)$, dimostrata formalmente nel punto (4) di Es. 5, è la probabilità di avere $k$ successi in $n$ prove, con probabilità di successo $p$\begin{equation*}
\boxed{p_{X}( k) =\mathbb{P}( X=k) =\binom{n}{k} p^{k}( 1-p)^{n-k}}
\end{equation*}
\end{itemize}

Vamonos

Richiami Teoria
\begin{equation*}
\mathbb{E}[ X] :=\int _{\Omega } Xd\mathbb{P} ,\ \ \ \ X:( \Omega ,\mathcal{A} ,\mathbb{P})\rightarrow ( E,\mathcal{E}) \ \text{VA}
\end{equation*}
Se $X$ è una VA discreta con densità $p_{X} :S\rightarrow [ 0,1]$, allora
\begin{equation*}
\mathbb{E}[ X] =\sum\limits _{x\in S} xp_{X}( x)
\end{equation*}
Infatti:
\begin{equation*}
\mathbb{E}[ X] =\underbrace{\int _{\Omega } X( \omega ) d\mathbb{P}( \omega ) =\int _{\mathbb{R}} xdP^{X}( x)}_{\text{Teorema "astratto-concreto"}}\overset{\text{discreta}}{=}\sum\limits _{x\in S} xp_{X}( x) =\sum\limits _{x\in S} x\mathbb{P}( X=x)
\end{equation*}
Vamonos ora! Hay carramba!
\begin{enumerate}
\item Per calcolare $\mathbb{E}[ X]$, con $X\sim B( n,p)$ possiamo procedere in due modi.

\textbf{[Metodo 1]}

Utilizziamo la definizione di $\mathbb{E}[ X] =\sum\limits _{x\in S} xp_{X}( x)$ e la forma esplicita della densità binomiale:\begin{equation*}
p_{X}( x) =\mathbb{P}( X=x) =\binom{n}{x} p^{x}( 1-p)^{n-x}
\end{equation*}Avremo\begin{equation*}
\begin{aligned}
\mathbb{E}[ X] & =\sum\limits _{x\in S} xp_{X}( x) & \\
 & =\sum\limits _{x=0}^{n} x\binom{n}{x} p^{x}( 1-p)^{n-x} & \\
 & =\sum\limits _{x=1}^{n} x\binom{n}{x} p^{x}( 1-p)^{n-x} & ( j+1=x)\\
 & =\sum\limits _{j=0}^{n-1}( j+1)\binom{n}{j+1} p^{j+1}( 1-p)^{n-( j+1)} & \\
 & =\sum\limits _{j=0}^{n-1}( j+1)\frac{n!}{( j+1) !( n-( j+1)) !} p^{j+1}( 1-p)^{n-( j+1)} & \\
 & =\sum\limits _{j=0}^{n-1}\frac{n( n-1) !}{j!(( n-j) -1)) !} p^{j} p( 1-p)^{( n-1) -j)} & \\
 & =np\sum\limits _{j=0}^{n-1}\frac{( n-1) !}{j!(( n-j) -1)) !} p^{j}( 1-p)^{( n-1) -j)} & \\
 & =np\underbrace{\sum\limits _{j=0}^{n-1}\binom{n-1}{j} p^{j}( 1-p)^{( n-1) -j)}}_{=1} & \\
 & =np & 
\end{aligned}
\end{equation*}

i.e. se $X\sim B( n,p)$, allora $\boxed{\mathbb{E}[ X] =np}$.

\textbf{[Metodo 2]}

Utilizziamo il fatto che\begin{equation*}
X=\sum\limits _{i=1}^{n} X_{i} ,\ \ \ \ X_{i} \sim B( p) ,\ \ \ \ X_{i} \Bot 
\end{equation*}

e la linearità di $\mathbb{E}$\begin{equation*}
\mathbb{E}[ X] =\mathbb{E}\left[\sum\limits _{i=1}^{n} X_{i}\right]\overset{\text{lin.}}{=}\sum\limits _{i=1}^{n}\mathbb{E}[ X_{i}] =\sum\limits _{i=1}^{n} p=np
\end{equation*}
\item Ricordiamo che TEORIA OVER HEREEEEE

Se $X\in L^{2}(\mathbb{P})$, i.e. $\mathbb{E}\left[ X^{2}\right] < +\infty $, allora\begin{equation*}
\mathrm{Var}[ X] :=\mathbb{E}( X-\mathbb{E}[ X])^{2} =\mathbb{E}\left[ X^{2}\right] -(\mathbb{E}[ X])^{2}
\end{equation*}

Usiamo\begin{equation*}
\mathrm{Var}[ X] =\mathbb{E}\left[ X^{2}\right] -(\mathbb{E}[ X])^{2}\overset{( 1)}{=}\mathbb{E}\left[ X^{2}\right] -( np)^{2}
\end{equation*}

Resta da calcolare $\mathbb{E}\left[ X^{2}\right]$, lo facciamo sfruttando il fatto che\begin{gather*}
X=\sum\limits _{i=1}^{n} X_{i} ,\ \ \ \ X\sim B( p) ,\forall i=0,\dotsc ,n,\ \ \ \ X_{i} \Bot X_{j} ,\forall i\neq j\\
\mathbb{E}\left[ X^{2}\right] =\mathbb{E}\left[\left(\sum\limits _{i=1}^{n} X_{i}\right)^{2}\right] =\underbrace{\mathbb{E}\left[\sum\limits _{i=1}^{n} X_{i}^{2}\right]}_{\text{(a)}} +2\underbrace{\mathbb{E}\left[\sum\limits _{i< j} X_{i} X_{j}\right]}_{\text{(b)}}
\end{gather*}

Ora\begin{equation*}
\text{(a)} =\mathbb{E}\left[\sum\limits _{i=1}^{n} X_{i}^{2}\right]\overset{\text{lin.}}{=}\sum\limits _{i=1}^{n}\mathbb{E}\left[ X_{i}^{2}\right]\overset{\text{(a1)}}{=}\sum\limits _{i=1}^{n} p=np
\end{equation*}

Il passaggio (a1) è giustificato da\begin{equation*}
X_{i} \sim B( p) ,\forall i,\ \ \ \ \mathbb{E}\left[ X_{i}^{2}\right] :=\sum\limits _{k\in \{0,1\}} k^{2}\mathbb{P}( X_{i} =k) =1\cdotp \mathbb{P}( X_{i} =1) =p
\end{equation*}

Poi\begin{equation*}
\begin{aligned}
\text{(b)} & =\mathbb{E}\left[\sum\limits _{i< j} X_{i} X_{j}\right]\overset{\text{lin.}}{=}\sum\limits _{i< j} \mathbb{E}[ X_{i} X_{j}]\overset{\text{(b1)}}{=}\sum\limits _{i< j}\mathbb{P}( E_{i} \cap E_{j})\\
 & \overset{\Bot }{=}\sum\limits _{i< j}\mathbb{P}( E_{i})\mathbb{P}( E_{j}) =\sum\limits _{i< j} p^{2} =p^{2}\sum\limits _{i< j} 1\\
 & \overset{\text{(b2)}}{=} p^{2}\frac{n( n-1)}{2}
\end{aligned}
\end{equation*}

Il passaggio (b1) è giustificato da\begin{equation*}
\begin{aligned}
X_{i} ,X_{j} \sim B( p) \implies \mathbb{E}[ X_{i} X_{j}] & =\sum\limits _{k\in \{0,1\}} k\mathbb{P}( X_{i} X_{j} =k)\\
 & =\mathbb{P}( X_{i} X_{j} =1)\\
 & =\mathbb{P}( E_{i} \cap E_{j})
\end{aligned}
\end{equation*}

mentre il risultato della sommatoria (b2) si può vedere graficamente

% Pattern Info
 
\tikzset{
pattern size/.store in=\mcSize, 
pattern size = 5pt,
pattern thickness/.store in=\mcThickness, 
pattern thickness = 0.3pt,
pattern radius/.store in=\mcRadius, 
pattern radius = 1pt}\makeatletter
\pgfutil@ifundefined{pgf@pattern@name@_m49hw7hay}{
\pgfdeclarepatternformonly[\mcThickness,\mcSize]{_m49hw7hay}
{\pgfqpoint{-\mcThickness}{-\mcThickness}}
{\pgfpoint{\mcSize}{\mcSize}}
{\pgfpoint{\mcSize}{\mcSize}}
{\pgfsetcolor{\tikz@pattern@color}
\pgfsetlinewidth{\mcThickness}
\pgfpathmoveto{\pgfpointorigin}
\pgfpathlineto{\pgfpoint{\mcSize}{0}}
\pgfpathmoveto{\pgfpointorigin}
\pgfpathlineto{\pgfpoint{0}{\mcSize}}
\pgfusepath{stroke}}}
\makeatother
\tikzset{every picture/.style={line width=0.75pt}} %set default line width to 0.75pt        

\begin{tikzpicture}[x=0.75pt,y=0.75pt,yscale=-1,xscale=1]
%uncomment if require: \path (0,189); %set diagram left start at 0, and has height of 189

%Shape: Axis 2D [id:dp937276347290068] 
\draw  (162,156.3) -- (329,156.3)(178.7,6) -- (178.7,173) (322,151.3) -- (329,156.3) -- (322,161.3) (173.7,13) -- (178.7,6) -- (183.7,13)  ;
%Shape: Polygon [id:ds1432415461068106] 
\draw  [pattern=_m49hw7hay,pattern size=6pt,pattern thickness=0.75pt,pattern radius=0pt, pattern color={rgb, 255:red, 0; green, 0; blue, 0}] (298.7,156.3) -- (178.7,156.3) -- (298.7,36.3) -- cycle ;
%Straight Lines [id:da9906900184134042] 
\draw  [dash pattern={on 0.84pt off 2.51pt}]  (178.7,36.3) -- (298.7,36.3) ;

% Text Node
\draw (352,64.4) node [anchor=north west][inner sep=0.75pt]    {$\sum\limits _{i< j} 1=\sum\limits _{i=1}^{n}\sum\limits _{i< j} 1=\frac{n( n-1)}{2}$};
% Text Node
\draw (180.7,159.7) node [anchor=north west][inner sep=0.75pt]  [font=\scriptsize]  {$i$};
% Text Node
\draw (290.7,159.7) node [anchor=north west][inner sep=0.75pt]  [font=\scriptsize]  {$n$};
% Text Node
\draw (304.7,23.7) node [anchor=north west][inner sep=0.75pt]  [font=\scriptsize]  {$i=j$};
% Text Node
\draw (146.7,30.7) node [anchor=north west][inner sep=0.75pt]  [font=\scriptsize]  {$n-1$};
% Text Node
\draw (164.7,141.7) node [anchor=north west][inner sep=0.75pt]  [font=\scriptsize]  {$j$};


\end{tikzpicture}


Mettendo insieme i pezzi\begin{gather*}
\begin{aligned}
\mathrm{Var}[ X] & =\mathbb{E}\left[ X^{2}\right] -(\mathbb{E}[ X])^{2}\\
 & =\mathbb{E}\left[\sum\limits _{i=1}^{n} X_{i}^{2}\right] +2\mathbb{E}\left[\sum\limits _{i< j} X_{i} X_{j}\right] -(\mathbb{E}[ X])^{2}\\
 & =np+2p^{2}\frac{n( n-1)}{2} -( np)^{2}\\
 & =np( 1+p( n-1) -np)\\
 & =np( 1+pn-p-np)\\
 & =np( 1-p)
\end{aligned}\\
\\
\boxed{\mathrm{Var}[ X] =np( 1-p)}
\end{gather*}
\item Teoria. Ricordiamo che

Mode: sono i valori che con maggior probabilità vengono assunti da $X$, ossia che massimizzano $p_{X}( k) =\mathbb{P}( X=k)$.

Cerchiamo i punti di massimo locale, i.e. i $k$ per cui\begin{equation*}
\boxed{p_{X}( k+1) \leq p_{X}( k) \ \ \ \ \text{e} \ \ \ \ p_{X}( k-2) \leq p_{X}( k)}
\end{equation*}

Abbiamo\begin{equation*}
\begin{aligned}
p_{X}( k+1) \leq p_{X}( k) & \iff \binom{n}{k+1}\cancel{p^{k}} p( 1-p)^{n-k-1} \leq \binom{n}{k}\cancel{p^{k}}( 1-p)^{n-k}\\
 & \iff \frac{\cancel{n!} p}{( k+1) !( n-k-1) !}\frac{\cancel{( 1-p)^{n-k}}}{( 1-p)} \leq \frac{\cancel{n!}}{k!( n-k) !}\cancel{( 1-p)^{n-k}}\\
 & \iff \frac{p}{( k+1)\cancel{k!}\cancel{( n-k-1) !}( 1-p)} \leq \frac{1}{\cancel{k!}( n-k)\cancel{( n-k-1) !}}\\
 & \iff \frac{p( n-k)}{( k+1)( 1-p)} \leq 1\\
 & \vdots \\
 & \iff k\geq p( n+1) -1\\
 & \\
p_{X}( k-2) \leq p_{X}( k) & \iff \binom{n}{k-1} p^{k-1}( 1-p)^{n-k+1} \leq \binom{n}{k} p^{k}( 1-p)^{n-k}\\
 & \vdots \\
 & \iff k\leq p( n+1)
\end{aligned}
\end{equation*}

Allora la condizione si verifica per\begin{equation*}
\begin{cases}
k\geq p( n+1) -1\\
k\leq p( n+1)
\end{cases} \ \ \ \ \iff \ \ \ \ p( n+1) -1\leq k\leq p( n+1)
\end{equation*}

Quindi\begin{equation*}
\text{Moda}( X) =\begin{cases}
p( n+1) -1\ \text{e} \ p( n+1) , & p( n+1) \in \mathbb{N}\\
\lfloor p( n+1)\rfloor , & p( n+1) \notin \mathbb{N}
\end{cases}
\end{equation*}
\end{enumerate}
\Soluzione

Ricordiamo che: per $\lambda  >0$ la densità di Poisson di parametro $\lambda $ è
\begin{equation*}
p( k) :=\begin{cases}
e^{-\lambda }\frac{\lambda ^{k}}{k!} , & k\in \{0,1,2,\dotsc \}\\
0, & k\notin \{0,1,2,\dotsc \}
\end{cases}
\end{equation*}
Una VA con questa densità è detta VA di Poisson di parametro $\lambda $. Scriviamo $X\sim \mathcal{P}( \lambda )$.
\begin{enumerate}
\item \begin{equation*}
\begin{aligned}
\mathbb{E}[ X] & :=\int _{\Omega } X( \omega ) d\mathbb{P}( \omega ) =\int _{\mathbb{R}} xdP^{X}( x) & \\
 & =\sum\limits _{k=0}^{\infty } kp_{X}( k) & \\
 & =\sum\limits _{k=1}^{\infty } ke^{-\lambda }\frac{\lambda ^{k}}{k!} & \text{(} k\ \text{dà contributo nullo)}\\
 & =e^{-\lambda } \lambda \sum\limits _{k=1}^{\infty }\frac{\lambda ^{k-1}}{( k-1) !} & \\
 & =\lambda \sum\limits _{x=0}^{\infty } e^{-\lambda }\frac{\lambda ^{x}}{x!} & k-1=:x\\
 & =\lambda  & 
\end{aligned}
\end{equation*}
\item \begin{equation*}
\mathrm{Var}[ X] =\mathbb{E}[ X-\mathbb{E}[ X]]^{2} =\mathbb{E}\left[ X^{2}\right] -(\mathbb{E}[ X])^{2}\overset{( 1)}{=}\mathbb{E}\left[ X^{2}\right] -\lambda ^{2}
\end{equation*}

Per esercizio, calcolare $\mathbb{E}\left[ X^{2}\right]$ e concludere:\begin{equation*}
X\sim \mathcal{P}( \lambda ) \implies \mathrm{Var}[ X] =\lambda 
\end{equation*}
\item Si procede come in Es. 7.
\item hahahah
\end{enumerate}
\Soluzione
\begin{oss}
$\mathbb{P}( X=k) =p( 1-p)^{k-1}$ rappresenta la probabilità che il primo successo richieda l'esecuzione di $k$ prove $\Bot $, ognuna con probabilità di successo $p$. Se la probabilità di successo è $p$ in ogni prova, allora la probabilità che alla $k$ esima prova si ottenga il primo successo è $\mathbb{P}( X=k) =p( 1-p)^{k-1}$.
\end{oss}
\begin{enumerate}
\item Abbiamo\begin{gather*}
X:( \Omega ,\mathcal{F} ,\mathbb{P})\rightarrow \mathbb{N} \subset \mathbb{R}\\
X\sim \mathcal{G}( p) \ \ \ \ \text{i.e.} \ \ \ \ \mathbb{P}( X=k) =p( 1-p)^{k-1} =:\tilde{p}( k) ,\ k\in \mathbb{N}
\end{gather*}

\textbf{Spazio di probabilità discreto:}

$( \Omega ,\mathcal{A} ,\mathbb{P})$ con $\Omega =\mathbb{N} ,\mathcal{A} =2^{\mathbb{N}}$ e $\mathbb{P}$ l'unica misura di probabilità su $\Omega $ t.c.\begin{equation}
\mathbb{P}( k) =\tilde{p}( k) =p( 1-p)^{k-1}
\end{equation}

Con questa scelta di $( \Omega ,\mathcal{A} ,\mathbb{P})$ si ha che $X( \cdotp ) :=I( \cdotp )$\begin{align*}
X:\left(\mathbb{N} ,2^{\mathbb{N}} ,\mathbb{P}\right) & \rightarrow \mathbb{N}\\
\omega  & \rightarrow X( \omega ) :=\omega ,\ \ \ \ X\sim \mathcal{G}( p)
\end{align*}

Verifica: dobbiamo mostrare che $\forall k\in \mathbb{N} ,p_{X}( k) =\tilde{p}( k)$\begin{align*}
p_{X}( k) & =\mathbb{P}( X=k)\\
 & =\mathbb{P}(\{\omega \in \mathbb{N} :X( \omega ) =k\})\\
 & =\mathbb{P}(\{\omega \in \mathbb{N} :\omega =k\})\\
 & =\mathbb{P}( k)\\
 & =\tilde{p}( k) =p( 1-p)^{k-1}
\end{align*}

\begin{oss}
Dato che le v.c. discrete sono caratterizzate dalla probabilità sugli atomi, nel senso che
\begin{equation*}
\forall B\in \mathcal{A} \ \ \ \ \mathbb{P}( X\in B) =\sum\limits _{k:\ x_{k} \in B\cap S} p_{X}( x_{k})
\end{equation*}
caratterizzare $\mathbb{P}$ sugli atomi, come fatto in (3), significa sostanzialmente dire che $\mathbb{P}$ è l'unica misura di probabilità su $\left(\mathbb{N} ,2^{\mathbb{N}}\right)$ t.c. $\mathbb{P} =P^{X}$, dove $P^{X}( \cdotp ) =\mathbb{P}( X\in \cdotp )$ è la legge di $X$.

La misura che dobbiamo mettere su $\left(\mathbb{N} ,2^{\mathbb{N}}\right)$ è proprio la legge immagine di una v.c. geometrica di parametro $p$. Poi basta considerare su questo spazio di probabilità la v.c. $X=I$ e abbiamo immediatamente $X\sim \mathcal{G}( p)$.
\end{oss}

\textbf{Spazio di probabilità continuo:}

Abbiamo\begin{equation*}
( \Omega ,\mathcal{F} ,\mathbb{P}) =\left(\mathbb{R} ,\mathcal{B}(\mathbb{R}) ,P^{X}\right)
\end{equation*}

e\begin{equation*}
X( \omega ) =\omega \ \ \forall \omega \in \Omega 
\end{equation*}

L'idea è la seguente: sia $( \Omega ,\mathcal{F} ,\mathbb{P})$ un generico spazio di probabilità e\begin{align*}
X:( \Omega ,\mathcal{F} ,\mathbb{P}) & \rightarrow \left(\mathbb{R} ,\mathcal{B}(\mathbb{R}) ,P^{X}\right)\\
\omega  & \rightarrow X( \omega )
\end{align*}

con\begin{equation*}
X\sim \mathcal{G}( p)
\end{equation*}

cioè\begin{equation*}
\forall B\in \mathcal{B}(\mathbb{R}) ,\ \ P^{X}( B) =\mathbb{P}( X\in B)
\end{equation*}

$P^{X}$ è la legge immagine di $X$.

Se quindi come spazio di probabilità consideriamo proprio $\left(\mathbb{R} ,\mathcal{B}(\mathbb{R}) ,P^{X}\right)$ e\begin{align*}
X=I:\left(\mathbb{R} ,\mathcal{B}(\mathbb{R}) ,P^{X}\right) & \rightarrow \left(\mathbb{R} ,\mathcal{B}(\mathbb{R}) ,P^{X}\right)\\
\omega  & \rightarrow X( \omega ) =\omega 
\end{align*}

abbiamo costruito uno spazio di probabilità continuo dove è possibile definire $X$.

Resta ora da capire come è fatta $P^{X}$.
\end{enumerate}


\chapter{Variabili aleatorie continue}
%!TEX root = ../main.tex


\ParteEsercizi
\Esercizio{}

Si consideri lo spazio di probabilità $( \Omega ,\mathcal{A} ,\PP) =(\mathbb{R} ,\mathcal{B} ,m_{0,1})$, dove $m_{0,1} =m( B\cap ( 0,1]) ,\forall B\in \mathcal{B}$, ed $m$ è la misura di Lebesgue sui boreliani. Si considerino inoltre le funzioni $X,Y,Z:\Omega =\mathbb{R}\rightarrow \mathbb{R}$ definite sull'intervallo $( 0,1)$ nel modo seguente:
\begin{equation*}
X( \omega ) =\frac{1}{\omega } ,\ \ \ \ Y( \omega ) =\frac{1}{\sqrt{\omega }} ,\ \ \ \ Z( \omega ) =\log \omega ,
\end{equation*}
per ogni $\omega \in ( 0,1)$. Per quanto riguarda il valore di $X,Y,Z$ su $\mathbb{R} \setminus ( 0,1)$, si sa solamente che $X,Y,Z$ sono costanti su tale insieme.
\begin{enumerate}
\item Mostrare che $X$, $Y$ e $Z$ sono variabili aleatorie ben definite q.c.
\item Determinare le distribuzioni di $X$, $Y$ e $Z$, mostrando in particolare che si tratta di variabili aleatorie continue.
\item Calcolare le densità di probabilità di $X$, $Y$ e $Z$.
\item Calcolare i valori attesi di $X$, $Y$ e $Z$.
\item Esibire un'altra variabile aleatoria $\tilde{X}$ con la stessa legge di $X$, ma definita su un altro spazio di probabilità $\left(\tilde{\Omega } ,\tilde{\mathcal{A}} ,\tilde{\PP}\right)$.
\end{enumerate}
\Esercizio{}

Data una variabile aleatoria reale continua $X$ con densità $f$ simmetrica rispetto a $\mu \in \mathbb{R}$,

si mostri che
\begin{enumerate}
\item $\mu $ è una mediana,
\item Se $\EE[| X| ] < +\infty $, allora $\mu =\EE[ X]$.
\end{enumerate}
\Esercizio{(Distribuzione di Cauchy)}

Data una variabile aleatoria di Cauchy $X$, con densità
\begin{equation*}
f( x) =\frac{1}{\pi \left( 1+( x-\alpha )^{2}\right)} ,\ \ \ \ \alpha \in \mathbb{R} ,
\end{equation*}
si calcolino:
\begin{enumerate}
\item la mediana,
\item $\EE[ X]$,
\item $\EE\left[ X^{2}\right]$.
\end{enumerate}
\Esercizio{$\star$}

Data $X$ variabile aleatoria positiva e continua, denotata con $F$ la sua funzione di \ ripartizione, si mostri che
\begin{equation*}
\EE[ X] =\int _{0}^{+\infty }( 1-F( x)) dx.
\end{equation*}
\Esercizio{(Distribuzione esponenziale)}

Si consideri $X\sim \mathcal{E}( \lambda )$, variabile esponenziale di parametro $\lambda  >0$, quindi con densità


\begin{equation*}
f( x) =\lambda e^{-\lambda x} \Ind_{[ 0,+\infty )}( x) ,\ \ \ \ x\in \mathbb{R} .
\end{equation*}
\begin{enumerate}
\item Calcolare la moda di $X$.
\item Calcolare il valore atteso $\EE[ X]$ e la varianza $\mathrm{Var}( X)$.
\item Calcolare la funzione di ripartizione $F_{X}$ di $X$.
\item (Proprietà di assenza di memoria) Mostrare che $\PP( X >s+t \mid X >s) =\PP( X >t)$ per ogni $s,t >0$.
\item * Si inverta il risultato appena ottenuto: se $T$ è una variabile aleatoria reale strettamente positiva tale che $\PP( T >s+t\mid T >s) =\PP( T >t)$ per ogni $s,t >0$, allora $T\sim \mathcal{E}( \nu )$ con $\nu =-\log(\PP( T >1))$.
\end{enumerate}
\Esercizio{(Distribuzione Gamma)}

Si consideri $X\sim \Gamma ( \alpha ,\lambda )$, variabile aleatoria di distribuzione Gamma di parametri $\alpha  >0$ e $\lambda  >0$. Si ha quindi
\begin{equation*}
f( x) =\frac{\lambda ^{\alpha }}{\Gamma ( \alpha )} x^{\alpha -1} e^{-\lambda x} \Ind_{( 0,+\infty )}( x) ,
\end{equation*}
dove $\Gamma $ è la funzione Gamma (di Eulero) definita da
\begin{equation*}
\Gamma ( \alpha ) \coloneqq \int _{0}^{+\infty } t^{\alpha -1} e^{-t} dt,\ \ \ \ \forall \alpha  >0.
\end{equation*}
Si ricordi che
\begin{equation*}
\Gamma ( \alpha +1) =\alpha \Gamma ( \alpha ) ,\ \ \ \ \forall \alpha  >0,\ \ \ \ \ \ \ \ \ \ \ \ \Gamma ( 1) =1,
\end{equation*}
da cui $\Gamma ( n+1) =n!,\ \forall n\in \mathbb{N}$. Infine $\Gamma ( 1/2) =\sqrt{\pi }$.
\begin{enumerate}
\item Si verifichi che $f$ è una densità di probabilità.
\item Si calcolino le mode di $f$.
\item Si calcolino i momenti $\EE\left[ X^{k}\right] ,\ k\in \mathbb{Z} ,\ \alpha +k >0$, e la varianza $\mathrm{Var}( X)$.
\item Dato $c >0$, si mostri che $Y=cX\sim \Gamma ( \alpha ,\lambda /c)$.
\item Per $\alpha =n/2,\ n\in \mathbb{N}$, si determini la relazione fra i punti percentuali di una $\Gamma \left(\frac{n}{2} ,\lambda \right)$ e quelli di una $\chi ^{2}( n) =\Gamma \left(\frac{n}{2} ,\frac{1}{2}\right)$.
\item Per $\alpha =n\in \mathbb{N}$, si mostri che la funzione di ripartizione di $X$ è data da
\end{enumerate}
\begin{equation*}
F( t) =\left( 1-\sum\limits _{k=0}^{n-1} e^{-\lambda t}\frac{( \lambda t)^{k}}{k!}\right) \Ind_{( 0,+\infty )}( t) .
\end{equation*}
\Esercizio{(Distribuzione normale o gaussiana)}

Sia $X\sim \mathcal{N}\left( \mu ,\sigma ^{2}\right)$.
\begin{enumerate}
\item Si calcolino media e varianza di $X$.
\item Si esprimano i punti percentuali $c_{\alpha }$ di $X$ in funzione di quelli di una normale standard.
\end{enumerate}

Si consideri $Y$ di \textit{legge lognormale} di parametri $\mu $ e $\sigma ^{2}$, ovvero $Y=e^{X}$.
\begin{enumerate}
\item Calcolare la distribuzione della variabile aleatoria $Y$, i momenti (ossia $\EE[ Y_{k}] ,k\in \mathbb{Z}$) e la varianza.
\item Si calcoli $\EE\left[ Xe^{X}\right]$.
\end{enumerate}

Si consideri $Z$ di \textit{legge normale standard}, ossia $Z\sim \mathcal{N}( 0,1)$.
\begin{enumerate}
\item Calcolare la distribuzione della variabile aleatoria $Z^{2}$.
\item Calcolare la legge della variabile aleatoria $| Z| $, il valore atteso, la varianza e i quantili.
\item Si calcoli $\EE\left[ Z^{4}\right]$ e $\EE\left[ e^{tZ^{2}}\right] ,t\in \mathbb{R}$.
\end{enumerate}
\Esercizio{}

Sia $X$ una variabile aleatoria con funzione di ripartizione $F$.
\begin{enumerate}
\item Si calcoli la legge di $Y=| X| $.
\item Nel caso in cui $F$ ammetta derivata continua, si mostri che $Y=| X| $ è continua e se ne calcoli la densità.
\item Nel caso $X\sim \mathcal{N}( 0,1)$, si confronti il risultato ottenuto nei punti $( 1)$ e $( 2)$ con quanto trovato nel punto $( 6)$ dell'esercizio $7$.
\end{enumerate}
\Esercizio{}
\begin{enumerate}
\item Dati $U\sim U(( 0,1))$ e $\lambda  >0$, si mostri che $X=-\frac{1}{\lambda }\log U\sim \mathcal{E}( \lambda )$.
\item Data $X\sim F$ con $F$ invertibile, si mostri che $Y=F( X) \sim U(( 0,1))$.
\item Date $U\sim U(( 0,1))$ e una funzione di ripartizione $F$ invertibile, si mostri che $X=F^{-1}( U) \sim F$.
\item * Data $X\sim F$, con $F$ continua, si mostri che $Y=F( X) \sim U(( 0,1))$. [Suggerimento: sia $t\in \left( 0,1\right)$ e $G\left( t\right) =\inf\left\{x:F\left( x\right) \geq t\right\}$. Utilizzando la continuità di $F$, si mostri che $F\left( G\left( t\right)\right) =t$. Si noti infine che $F\left( x\right) \geq t\iff x\geq G\left( t\right)$.]
\item * Date $U\sim U\left(\left( 0,1\right)\right)$ e una funzione di ripartizione $F$ continua, definita $G\left( t\right) =\inf\left\{x:F\left( x\right) \geq t\right\}$, si mostri che $X=G\left( U\right)$ è una variabile aleatoria di legge $F$.
\item * Date $U\sim U(( 0,1))$ e una distribuzione discreta $P_{0}$ su $\left(\mathbb{R} ,\mathcal{B}\right)$, si trovi una trasformazione misurabile $G$ tale che $X=G\left( U\right) \sim P_{0}$.
\end{enumerate}
\subsubsection{}
\begin{enumerate}
\item Allora
\begin{equation*}
U\sim U\left(\left( 0,1\right)\right) \implies F_{U}\left( x\right) =\begin{cases}
0, & x< 0\\
x, & 0\leq x< 1\\
1, & x\geq 1
\end{cases}
\end{equation*}

a partire da questa ci troviamo
\begin{gather*}
F_{X}\left( t\right) \coloneqq \PP\left( X\leq t\right) =\PP\left( -\frac{1}{\lambda }\log U\leq t\right)\\
-\frac{1}{\lambda }\log U\leq t\iff \log U\geq -\lambda t\iff U\geq e^{-\lambda t}\\
=\PP\left( U\geq e^{-\lambda t}\right) =1-\underbrace{\PP\left( U< e^{-\lambda t}\right)}_{F_{U}\left( e^{-\lambda t}\right)} =\\
=1-\begin{cases}
0, & e^{-\lambda t} < 0\ MAI\\
e^{-\lambda t} , & 0\leq e^{-\lambda t} < 1\iff t >0\\
1, & e^{-\lambda t} \geq 1\iff t\leq 0
\end{cases}\\
=\begin{cases}
1-e^{-\lambda t} , & t >0\\
0, & t\leq 0
\end{cases} =\left( 1-e^{-\lambda t}\right) \Ind_{\left( 0,+\infty \right)}\left( t\right)
\end{gather*}

Quindi se $U\sim U\left(\left( 0,1\right)\right)$ e $X=-\frac{1}{\lambda }\log U$ allora $F_{X}\left( t\right) =\left( 1-e^{-\lambda t}\right) \Ind_{\left( 0,+\infty \right)}\left( t\right)$, che è la funzione di ripartizione di un'esponenziale, ovvero $X\sim \mathcal{E}\left( \lambda \right)$.

A partire dalla distribuzione uniforme si può ottenere qualunque altra distribuzione.
\item 
\end{enumerate}
\Esercizio{}

Data una variabile aleatoria continua, non negativa $X$, con densità di probabilità $f_{X}$ continua su $\left( 0,+\infty \right)$, definiamo "tasso di fallimento" la funzione
\begin{equation*}
h_{X}( t) =\lim _{\varepsilon \rightarrow 0^{+}}\frac{\PP( t< X< t+\varepsilon \mid X >t)}{\varepsilon } ,\ \ \ \ t >0.
\end{equation*}
Interpretando $X$ come un tempo di attesa, il tasso di fallimento rappresenta la probabilità istantanea di arrivo, sapendo che l'attesa è durata fino al tempo $t$.
\begin{enumerate}
\item Dimostrare che vale la seguente relazione:\begin{equation*}
h_{X}( t) =\frac{f_{X}( t)}{1-F_{X}( t)} ,\ \ \ \ t >0,
\end{equation*}

dove $F_{X}$ indica la funzione di ripartizione di $X$.
\item Dimostrare che $F_{X}( t) =\left( 1-\exp\left( -\int _{0}^{t} h_{X}( s) ds\right)\right) \Ind_{( 0,+\infty )}( t)$.
\item Si prenda $X$ con legge esponenziale di parametro $\lambda $. Determinare $h_{X}$.
\item Si prenda $Y$ con legge Weibull di parametri $\alpha  >0$ e $\lambda  >0$, vale a dire
\begin{equation*}
f_{Y}( t) =\alpha \lambda ^{\alpha } t^{\alpha -1} e^{-( \lambda t)^{\alpha }} \Ind_{( 0,+\infty )}( t) .
\end{equation*}

Determinare $F_{Y}$ e $h_{Y}$.
\item Discutere il significato dei risultati ottenuti nei punti $( 3)$ e $( 4)$ e il loro legame.
\end{enumerate}
\Esercizio{}

Il raggio $R$ di un certo tipo di particella inquinante, espresso in micron, è una variabile aleatoria con densità di probabilità
\begin{equation*}
f( x) =\begin{cases}
c\ x\ e^{-x^{2}} , & \text{per} \ x >0,\\
0, & \text{per} \ x\leq 0.
\end{cases}
\end{equation*}
\begin{enumerate}
\item Determinare il valore della costante reale $c$.
\item Calcolare la funzione di ripartizione di $R$.
\item Calcolare la mediana e la media di $R$.
\item Calcolare la probabilità che una particella abbia un raggio superiore a $2$ micron.
\item Calcolare la legge della variabile aleatoria $X$ che vale $1$ se $R< 2$ e $0$ altrimenti.
\item Supponendo che le particelle inquinanti siano delle sfere, calcolare la probabilità che una particella abbia un volume superiore ad $1$ micron cubo.
\item Detto $V$ il volume di una particella, mostrare che si tratta di una variabile aleatoria continua e calcolarne la densità.
\end{enumerate}
\Esercizio{}

Sia $X$ una variabile aleatoria non negativa con densità di probabilità
\begin{equation*}
f_{X}( x) =\begin{cases}
\frac{2x}{\lambda } e^{-x^{2} /\lambda } , & \text{per} \ x\geq 0,\\
0, & \text{per} \ x< 0.
\end{cases}
\end{equation*}
\begin{enumerate}
\item Determinare i possibili valori della costante $\lambda \in \mathbb{R}$.
\item Determinare il tasso di fallimento di $X$.
\item Determinare la legge di $X^{2}$ e riconoscerla.
\item Calcolare la media e la varianza di $e^{-X^{2} /\lambda }$.
\item Calcolare $\EE\left[\frac{1}{X}\right]$.
\item Stabilire per quali $\alpha \in \mathbb{R}$ si ha che $Y=X^{\alpha } \in L^{1}$.
\end{enumerate}
\Esercizio{}

Si consideri $X$ variabile aleatoria di densità:
\begin{equation*}
f_{X}\left( x\right) =\begin{cases}
\lambda \ e^{-\left( x-\lambda \right)} , & x >\lambda ,\\
0, & x\leq \lambda .
\end{cases}
\end{equation*}
\begin{enumerate}
\item Determinare il valore della costante reale $\lambda $ e calcolare la funzione di ripartizione.
\item Posta $Y=e^{X}$, si determini la distribuzione di $Y$, e si calcolino $\EE\left[ Y\right]$ e $\EE\left[ Y^{2}\right]$.
\item Si calcolino $\PP\left( X >3\right)$ e $\PP\left( X^{3}  >27\right)$.
\item Si calcoli la media di $3X^{2} -1$.
\item Stimare la probabilità che $X$ assuma valori distanti dalla media più di $2$ unità, utilizzando la disuguaglianza di Chebychev.
\item Stabilire per quali $\alpha ,\beta \in \mathbb{R}$ si ha che $Y=X^{\alpha } \in L^{1}$ e $Z=\frac{1}{\left( X-1\right)^{\beta }} \in L^{1}$.
\end{enumerate}
\Esercizio{}

Siano $X\sim U\left(\left( 0,1\right)\right)$ e $Y\sim \Gamma \left( \alpha ,\lambda \right) ,\alpha ,\lambda  >0$.
\begin{enumerate}
\item Si stabilisca per quali $\beta \in \mathbb{R}$ la variabile aleatoria $Z=\frac{1}{\ \left( 1-X\right)^{\beta }} \in L^{1}$.
\item Si determini, al variare dei parametri $\alpha $, $\lambda $, per quali $\gamma \in \mathbb{R}$ la variabile aleatoria $W=Y^{\gamma } \in L^{1}$.

Si confronti questo risultato con la condizione imposta su $\alpha $ e $k$ nel punto $\left( 3\right)$ dell'esercizio $6$.
\end{enumerate}
\Esercizio{}

\textit{Ricavo di un'opzione "call" europea.} Un agente finanziario sottoscrive un contratto che gli dà il diritto (ma non l'obbligo) di acquistare un certo titolo ad una data futura fissata, ad un prezzo $k >0$ anch'esso fissato. Detto $S$ il valore di tale titolo alla data fissata, esso è al momento sconosciuto, ma si ritiene che i suoi possibili valori abbiano distribuzione lognormale, ovvero che siano dati da
\begin{equation*}
S=\exp X,\ \ \ \ X\sim \mathcal{N}\left( \mu ,\sigma ^{2}\right) ,\ \ \ \ \mu \in \mathbb{R} ,\sigma  >0.
\end{equation*}
Se il valore $S$ supererà $k$, l'agente eserciterà il suo diritto ricavando la differenza $S-k$, altrimenti non lo eserciterà e avrà ricavo nullo. Sia quindi $R$ il ricavo dell'agente.
\begin{enumerate}
\item Calcolare la probabilità di un ricavo positivo, $\PP\left( R >0\right)$.
\item Calcolare la probabilità di un ricavo nullo, $\PP\left( R=0\right)$.
\item Calcolare la legge del ricavo $R$. Si tratta di una variabile aleatoria continua? Discreta? Perché?
\item Calcolare il ricavo atteso $\EE[ R]$.
\item Calcolare $\lim\limits _{k\rightarrow 0}\EE[ R]$.
\end{enumerate}
\Esercizio{}

Petyr Baelish vende a Lord Varys il diritto di comprare fra un anno $50$ kg di acciaio di Valyria al prezzo di $100$ monete d'oro. Petyr Baelish non conosce il costo $X$ di $50$ kg di acciaio di Valyria fra un anno, ma ritiene che i suoi possibili valori abbiano una distribuzione (approssimativamente) normale di media $\mu =100$ e varianza $\sigma ^{2} =81$. Ovviamente Lord Varys eserciterà il suo diritto solo se sarà $X >100$. Sia quindi $Y$ la differenza che Petyr Baelish dovrà pagare fra un anno.
\begin{enumerate}
\item Scrivere $Y$ in funzione di $X$.
\item Determinare la funzione di ripartizione di $Y$, in termini della funzione di ripartizione della legge normale standard, e tracciarne un grafico qualitativo.
\item Si tratta di una variabile aleatoria continua? Discreta? Perché?
\item Determinare la probabilità che Petyr Baelish non debba aggiungere soldi fra un anno.
\item Determinare il primo e il terzo quartile di $Y$.
\item Determinare il valore atteso e la varianza di $Y$.
\item Se Petyr Baelish vende tale diritto di acquisto ad un prezzo pari al valore atteso di $Y$, con quale probabilità prevede di avere un guadagno positivo?
\end{enumerate}
\Esercizio{}

Un'industria produce su commissione delle sbarre d'acciaio cilindriche, il cui diametro dovrebbe essere di $4$ cm, ma che tuttavia sono accettabili se hanno diametro compreso fra $3.95$ cm e $4.05$ cm. Il cliente, nel controllare le sbarre fornitegli, constata che il $5\%$ sono di diametro inferiore al minimo tollerato ed il $12\%$ di diametro superiore al massimo tollerato.
\begin{enumerate}
\item Supponendo che le misure dei diametri seguano una legge normale, determinarne media e deviazione standard.
\item Mantenendo la media precedentemente calcolata, determinare quale dovrebbe essere il valore delle deviazione standard affinché la percentuale di sbarre con diametro superiore al massimo tollerato sia minore del $5\%$.
\end{enumerate}
\Esercizio{}

Sia $( \Omega ,\mathcal{A} ,\PP)$ uno spazio di probabilità fissato. Data una variabile aleatoria reale continua $X$, si mostri che
\begin{enumerate}
\item $\{X\in \mathbb{Q}\}$ è un evento, ossia $\{X\in \mathbb{Q}\} \in \mathcal{A}$,
\item $\PP( X\in \mathbb{Q}) =0$.
\end{enumerate}
\Esercizio{}

Si mostri che, se $X$ è una variabile aleatoria continua, la variabile aleatoria $W=\min\{X,1\}$ può essere continua o no, a seconda della legge di $X$.
\Esercizio{}

Data $X$ variabile aleatoria continua di densità $f_{X}$, si determini la legge di $Y\ =aX+b$ per $a\neq 0$. In particolare si riconosca la legge di $Y$ per $X$ di legge $\mathcal{N}\left( \mu ,\sigma ^{2}\right)$, $U([ c,d])$, $\mathcal{E}( \lambda )$.
\Esercizio{}

Data una variabile aleatoria $X\sim U([ -2,-1] \cup [ 1,2])$, calcolarne densità di probabilità $f$, mediane $m$ e, se esiste, media $\mu $.
\Esercizio{$\star$}

Data una variabile aleatoria $X\sim U(( 0,2\pi ))$ e un angolo fissato $\vartheta \in \mathbb{R}$, si calcoli la legge di $Y=\sin( X+\vartheta )$.
\Esercizio{}

Data $X\sim \mathcal{E}( \lambda )$, $\lambda  >0$, se ne consideri il suo arrotondamento per eccesso $Y$, ovvero
\begin{equation*}
Y=\sum\limits _{k=1}^{\ +\infty } k\Ind_{( k-1,k]}( X) =\lceil X\rceil .
\end{equation*}
Se quindi $X$ rappresenta un tempo d'attesa, $Y$ rappresenta il corrispondente tempo d'attesa per un osservatore stroboscopico. Si mostri che $Y$ è una variabile aleatoria e se ne calcoli la distribuzione.
\Esercizio{}

Data $X\sim \mathcal{N}( 2,5)$, mostrare che esiste un unico $c$ tale che $\PP(| X| < c) =0.4$. Trovare un valore approssimato di $c$ con metodi di analisi numerica.
\Esercizio{}

Aldo possiede un vecchio cronometro che, una volta avviato, si arresta dopo un tempo casuale $X$, che si può considerare una variabile aleatoria con legge esponenziale di media $10$ minuti.
\begin{enumerate}
\item Dopo aver fatto partire il cronometro Aldo evita di guardarlo per $5$ minuti al termine dei quali lo osserva e annota l'ora indicata $Y$. Trovare la legge di $Y$.
\item $Y$ è una variabile aleatoria discreta? È continua?
\item Aldo gioca contro Bruno nel modo seguente: se al momento dell'arresto del cronometro il numero di minuti interamente trascorsi è un numero pari allora vince Aldo (ad esempio se $X=2.5$ oppure $X=0.15$); se invece è dispari allora vince Bruno (ad esempio se $X=5$ oppure $X=7.4$). Calcolare la probabilità di vittoria per Aldo.
\end{enumerate}
\Esercizio{}

Data la funzione $f:\mathbb{R}\rightarrow \mathbb{R}$ definita come
\begin{equation*}
f( x) \coloneqq \begin{cases}
0 & x< 0\\
\frac{1}{2} & 0\leq x< \frac{1}{2}\\
2x-1 & \frac{1}{2} \leq x< 1\\
\frac{1}{4} & 1\leq x\leq 3\\
0 & x >3,
\end{cases}
\end{equation*}
\begin{enumerate}
\item dimostrare che $f$ è una funzione di densità di probabilità,
\item determinare la funzione di ripartizione $F$ avente $f$ come funzione di densità,
\item calcolare il quantile di ordine $3$ di una variabile casuale $X$ avente $f$ come funzione di densità di probabilità,
\item calcolare media e varianza di una variabile casuale $X$ avente $f$ come funzione di densità di probabilità,
\item costruire uno spazio di probabilità e una variabile casuale (reale continua) $X$ avente $f$ come funzione di densità.
\end{enumerate}


\ParteSoluzioni

\chapter{Vettori aleatori discreti}
%!TEX root = ../main.tex

\ParteEsercizi

\Esercizio{}

Date due variabili aleatorie reali X e Y di varianza finita e non nulla, definite sul medesimo spazio di probabilità, si consideri il coefficiente di correlazione $\rho _{X,Y} =\frac{\mathrm{Cov} (X,Y)}{\sigma _{X} \sigma _{Y}}$. Si mostri che:

(a) se $X$ e $Y$ sono indipendenti, allora $\rho _{X,Y} =0$

(b) $\rho _{aX+b,cY+d} =\rho _{X,Y}$ per ogni $a,c >0,b,d\in \mathbb{R}$

(c) $\rho _{X,aX+b} =a/|a|$, per ogni $a\neq 0$

(d) $\mathrm{Var}\left(\frac{1}{\sigma _{Y}} Y-\frac{\rho _{X,Y}}{\sigma _{X}} X\right) =1-\rho _{X,Y}^{2}$

(e) $| \rho _{X,Y}| =1$ implica $Y=aX+b$ per qualche $a\neq 0,b\in \mathbb{R}$. Suggerimento: utilizzare il punto d
\Esercizio{$\star$}

Dato un vettore aleatorio $X=( X_{1} ,\dots ,X_{n}) ,$ in cui ciascuna variabile aleatoria $X_{i}$ ha varianza finita, si mostri che $X\in \mathrm{Im} (\mathrm{C} )+\EE [\mathrm{X} ]$ q.c., ovvero $\PP (\mathrm{X} \in \mathrm{Im} (\mathrm{C} )+\EE [\mathrm{X} ])=1$, dove $\mathrm{C}$ è la

matrice varianza di $X$. Suggerimento: si supponga inizialmente che $\EE [X]=0$ e, in tal caso, si mostri che $X\in (\mathrm{Ker} (C))^{\perp } =\mathrm{Im} (C)$ q.c.
\Esercizio{}

Date $X_{1} ,\dots ,X_{n}$ variabili aleatorie reali i.i.d. con momento secondo finito, introdotti
\begin{gather*}
\mu =\EE[ X_{k}] ,\ \ \sigma ^{2} =\mathrm{Var}( X_{k}) ,\\
\overline{X}_{n} =\frac{1}{n}\sum _{k=1}^{n} X_{k} ,\ \ S_{n}^{2} =\frac{1}{n-1}\sum _{k=1}^{n}( X_{k} -\overline{X}_{n})^{2} ,\ \ W_{n}^{2} =\frac{1}{n}\sum _{k=1}^{n}( X_{k} -\mu )^{2}
\end{gather*}
si mostri che:

(a) $\EE[\overline{X}_{n}] =\mu ,\mathrm{Var}(\overline{X}_{n}) =\sigma ^{2} /n$

(b) $\sum _{k=1}^{n} X_{k}^{2} =\sum _{k=1}^{n}( X_{k} -\overline{X}_{n})^{2} +n\overline{X}_{n}^{2}$

(c) $\EE\left[ S_{n}^{2}\right] =\EE\left[ W_{n}^{2}\right] =\sigma ^{2}$
\Esercizio{}

Siano $X_{1} ,\dots ,X_{n}$ variabili aleatorie indipendenti con funzioni di ripartizione $F_{X_{1}} ,\dots ,F_{X_{n}}$

(a) Qual è la funzione di ripartizione di $X_{(n)} =\max\{X_{1} ,\dots ,X_{n}\} ?$

(b) Qual è la funzione di ripartizione di $X_{(1)} =\min\{X_{1} ,\dots ,X_{n}\} ?$

(c) $M=\min\{X_{2} ,X_{3}\}$ e $S=X_{1} +X_{7}$ sono indipendenti?

Le variabili aleatorie $X_{(1)}$ e $X_{(n)}$ sono indipendenti? Si consideri, ad esempio, il caso in cui $n=2$ e

$X_{1} ,X_{2}$ sono variabili aleatorie i.i.d. bernoulliane di parametro $p\in (0,1)$

(d) Come sono distribuite $X_{(1)}$ e $X_{(2)}$?

(e) $X_{(1)}$ e $X_{(2)}$ sono indipendenti?
\Esercizio{}

Siano $X,\ Y,\ Z$ variabili aleatorie reali. Mostrare che:

1. Se $Y=c$ q.c., per qualche costante $c\in \mathbb{R} ,$ allora $X$ ed $Y$ sono indipendenti.

$2^{*} .$ Se $\PP (Z\in B)\in \{0,1\}$ per ogni boreliano $B,$ allora $Z$ è q.c. costante.

$3^{*} .$ Nel caso $Y=h(X)$, per qualche funzione boreliana $h:\mathbb{R}\rightarrow \mathbb{R} ,X$ e $Y$ sono indipendenti se e solo se $Y=c$ q.c., per qualche costante $c\in \mathbb{R}$
\Esercizio{}

Un perito elettrotecnico deve costruire un sistema costituito da tre componenti in serie. Egli pesca i tre componenti da una scatola in cui vi sono tre componenti nuovi, due usati ma funzionanti e due difettosi. Siano $X$ ed $Y$ rispettivamente il numero di componenti nuovi e di componenti usati ma funzionanti tra quelli pescati dalla scatola.

(a) Determinare la legge congiunta di $X$ ed $Y$ e le leggi marginali.

(b) Le variabili $X$ ed $Y$ sono indipendenti?

(c) Calcolare $\EE [\mathrm{X} ],\EE [\mathrm{Y} ],\EE [\mathrm{XY} ].$ Scrivere la matrice varianza di $(X,Y)$ e determinare $\rho _{X,Y}$

(d) Calcolare la legge, il valore atteso e la varianza del numero di componenti pescati funzionanti.

(e) Calcolare la probabilità che l'apparecchio funzioni.
\Esercizio{}

Siano $X$ ed $Y$ variabili aleatorie i.i.d. bernoulliane di parametro $p=\frac{1}{2}$ e siano $u=X+Y$

e $V=|X-Y|$

(a) Mostrare che $(U,V)$ è un vettore aleatorio e determinarne la legge.

(b) Calcolare la probabilità che $V$ sia minore di $U$

(c) Calcolare la covarianza di $U$ e $V$ e la matrice varianza di $(U,V)$

(d) $U$ e $V$ sono indipendenti?
\Esercizio{}

Siano $X$ ed $Y$ due variabili aleatorie con legge congiunta parzialmente data da:
\begin{equation*}
\begin{array}{ c|c|c|c|c }
X\backslash Y & -1 & 5 & 10 & p_{X}\\
\hline
0 &  & 0.12 &  & 0.4\\
\hline
5 &  &  &  & \\
\hline
p_{Y} & 0.3 &  &  & 1
\end{array}
\end{equation*}
(a) Completare la tabella in modo che $X$ ed $Y$ siano indipendenti.

(b) Calcolare $\PP (X< Y)$

(c) Calcolare il valore atteso del vettore $(X,Y)$ e $\EE [XY]$

(d) Calcolare $\PP (|XY|\geq 5)$ e $\PP (X+Y >5)$

(e) Siano $U=|XY|$ e $V=X+Y$. Calcolare la legge congiunta di $U$ e $V$ e le leggi marginali.
\Esercizio{}

Sia $X$ una variabile aleatoria discreta con legge uniforme sull'insieme discreto $\{-1,1\}$

(a) Determinare la legge di $X$.

Sia ora $Y$ un'altra variabile aleatoria discreta, indipendente da $X,$ ma con la stessa legge di $X$. Si introduca $Z=XY$

(b) Calcolare la legge congiunta di $X$ e $Z$ e le leggi marginali.

(c) $X,Y$ e $Z$ sono indipendenti a coppie?

(d) $X,Y$ e $Z$ sono mutuamente indipendenti?
\Esercizio{}

Date $X$ e $Y$ variabili aleatorie indipendenti di legge geometrica di parametro $1/2$, si definisca $Z=\min \{X,Y\}$ e si calcolino:

(a) $\PP (Z\leq k)$ per $k\in \mathbb{N}$

(b) la distribuzione di $Z$

(c) $\PP (X=Y)$

(d) $\PP (X >Y)$
\Esercizio{}

Siano $T_{1}$ e $T_{2}$ variabili aleatorie indipendenti di legge geometrica di parametro, rispettivamente, $p_{1}$ e $p_{2} ,$ con cui a tempo discreto si descrive la durata aleatoria di due apparecchiature.

(a) Scrivere la legge congiunta di $T_{1}$ e $T_{2}$

(b) Calcolare la probabilità degli eventi $\{T_{1} =T_{2}\}$ e $\{T_{1} \geq T_{2}\}$

(c) Trovare la legge della durata del sistema composto dalle due apparecchiature collegate in serie.

(d) Trovare la legge della durata del sistema composto dalle due apparecchiature collegate in parallelo.

(e) Trovare la legge congiunta di $U=\min\{T_{1} ,T_{2}\}$ e $V=\max\{T_{1} ,T_{2}\}$

(f) Trovare la legge congiunta di $U$ e $W=V-U$

(g) Trovare la legge di $W$

(h) $U$ e $W$ sono indipendenti?
\Esercizio{}

Siano $X,\ Z,\ W$ variabili aleatorie indipendenti con $Z$ e $W$ entrambe con legge di Poisson di parametro $\lambda  >0$ e $X\sim \mathrm{Be} (p),p\in (0,1)$. Definiamo la variabile aleatoria $Y=XZ+W$

(a) Determinare le leggi di $(X,Y)$ e $Y$

(b) Calcolare $\EE [Y]$ e $\operatorname{Var} (Y)$

(c) Calcolare $\mathrm{Cov} (X,Y)$. Le variabili aleatorie $X$ e $Y$ sono indipendenti?
\Esercizio{}

Siano $N,X_{1} ,X_{2} ,\dots $ variabili aleatorie indipendenti, $N$ abbia legge di Poisson di parametro $\lambda  >0,$ e ciascuna delle $X_{k}$ abbia legge di Bernoulli di parametro $p\in (0,1).$ Si consideri la somma aleatoria (per addendi e per numero di addendi)
\begin{equation*}
S=\begin{cases}
0, & \text{se} \ N=0\\
X_{1} +\cdots +X_{N} , & \text{se} \ N\neq 0
\end{cases}
\end{equation*}
(a*) Si mostri che $S$ e $N-S$ sono variabili aleatorie.

(b) Qual è la legge di $S$?

(c) Qual è la legge di $N-S$?

(d) La somma $S$ è indipendente da $N$?

(e) Determinare la densità discreta congiunta di $S$ e $N-S$. La somma $S$ è indipendente da $N-S$ ?
\Esercizio{}

Sia $X$ una variabile aleatoria discreta, con $\mathrm{Im} (X)=\{1,2,\dots \},$ che soddisfa
\begin{equation*}
\PP (X\geq k)=\frac{1}{k^{\alpha }} ,\ \ k\geq 1
\end{equation*}
dove $\alpha  >0$ è un parametro reale.

(a) Calcolare la densità discreta di $X$.

(b) Calcolare la funzione di ripartizione di $X,$ disegnandone un grafico qualitativo.

(c) Per $\alpha =1$, calcolare il valore atteso di $X$

Si supponga d'ora in poi $\alpha =1.$ Sia $Y$ un'altra variabile aleatoria, indipendente da $X,$ ma con la stessa legge.

(d) Calcolare la densità discreta di $M=\min \{X,Y\}$

(e) Calcolare la densità discreta congiunta di $X$ e $X+Y$.
\Esercizio{}

Si consideri l'estrazione di $n$ palline da un'urna contenente $B$ palline bianche e $R$ palline rosse ($B\geq 1$ e $R\geq 1$). Sia $X_{k}$ la variabile aleatoria bernoulliana che indica se alla $k$-esima estrazione esce una pallina rossa e sia $Y=\sum _{k=1}^{n} X_{k}$ il numero di palline rosse sulle $n$ estratte. Si risponda ai seguenti quesiti, sia nel caso di estrazione con reimmissione sia nel caso senza reimmissione (in quest'ultimo caso $n\leq B+R$).

(a) Qual è la legge di $Y$?

(b) Qual è il valore atteso delle $X_{k} ?$

(c) Qual è il valore atteso di $Y?$

(d) Le $X_{k}$ sono indipendenti?

(e) $Y$ e $X_{1}$ sono indipendenti?

(f) Calcolare $\mathrm{Cov}( X_{1} ,X_{2})$

$(\mathrm{g} )$ Calcolare $\mathrm{Var} (Y)$ per $n=2$
\Esercizio{}

Consideriamo infinite prove di Bernoulli indipendenti con probabilità di successo $p\in (0,1)$. Siano quindi $\Omega =\{0,1\}^{\mathbb{N}}$ e $\mathcal{A} =\sigma ( E_{k} \mid k=1,2,\dots )$,
\begin{equation*}
E_{k} =\ \text{successo alla prova } k
\end{equation*}
e sia $\PP$ tale che $\{E_{k}\}_{k\in \mathbb{N}}$ risulti una famiglia di eventi indipendenti con $\PP( E_{k}) =p$ per ogni $k$. Indicato con $\omega =( \omega _{k})_{k=1}^{\infty }$ il generico esito dello spazio campionario $\Omega $, definiamo infine le variabili aleatorie
\begin{equation*}
X_{k} :\Omega \rightarrow \mathbb{R} ,\ \ X_{k} (\omega )=\omega _{k} ,\ \ k\in \mathbb{N}
\end{equation*}
(a) Le variabili aleatorie $X_{k}$ sono indipendenti?

(b) Le variabili aleatorie $Y=\sum _{k=1}^{2} X_{k}$ e $N=\sum _{k=5}^{7} X_{k}$ sono indipendenti?

(c) Le variabili aleatorie $\overline{X}_{n} =\frac{1}{n}\sum _{k=1}^{n} X_{k}$ e $S_{n}^{2} =\frac{1}{n-1}\sum _{k=1}^{n}( X_{k} -\overline{X}_{n})^{2}$ sono indipendenti?

[Suggerimento: per il punto (b) dell'esercizio 3, si ha che $S_{n}^{2} =\frac{1}{n-1}\sum _{k=1}^{n} X_{k}^{2} -\frac{n}{n-1}\overline{X}_{n}^{2} ;$ notare poi che $X_{k}^{2} =X_{k}$.

(d) Se nelle prime $10$ prove vengono registrati $2$ successi, con quale probabilità almeno un successo si è verificato nelle prime $2$ prove?

(e) Se nelle prime $10$ prove viene registrato almeno un successo, con quale probabilità le prime $8$ prove danno esattamente un successo?

(f) Introdotte anche $Z=$ "numero di prove necessarie per il primo successo" e $W=$ "numero di insuccessi prima del primo successo", calcolare le leggi congiunte delle coppie $(Z,W),(Y,Z)$ $(Y,W).$ Sono coppie di variabili aleatorie indipendenti?

(g) Determinare il coefficiente di correlazione $\rho _{Z,W}$ e la covarianza $\mathrm{Cov} (Z,W)$

cosa sarebbe cambiato se avessimo realizzato una successione di variabili aleatorie $X_{n}$ i.i.d., $X_{n} \sim \mathrm{Be} (p)$, in un altro spazio di probabilità $(\Omega ,\mathcal{A} ,\PP )$?
\Esercizio{}

Esibire due diversi spazi $(\Omega ,\mathcal{A} ,\PP )$, uno discreto e uno continuo, su cui è possibile definire un vettore aleatorio $( X_{1} ,X_{2}) ,$ con $X_{1}$ e $X_{2}$ indipendenti, $X_{1} \sim \mathrm{Bi}( n_{1} ,p)$, $X_{2} \sim \mathrm{Bi}( n_{2} ,p)$, dove $n_{1} ,n_{2} \in \mathbb{N}$ e $p\in (0,1)$.
\Esercizio{}

Alberto usa le lenti a contatto, ma è parecchio maldestro e, quando al mattino cerca di mettersele, spesso nell'operazione ne perde una, se non entrambe. Detto quindi $X$ il numero di lenti a contatto perse al mattino da Alberto, sappiamo che per un qualche valore del parametro $p$ il numero casuale $X$ ha distribuzione
\begin{equation*}
p_{X} (k)=\PP (X=k)=\begin{cases}
p & k=0\\
1-p-p^{2} & k=1\\
p^{2} & k=2
\end{cases}
\end{equation*}
1. Stabilire i possibili valori del parametro $p$.

Occupiamoci ora di quel che può capitare in due giorni, considerando indipendenti, e distribuiti entrambi come $X,$ i totali $X_{1}$ e $X_{2}$ di lenti perse alla prima e alla seconda mattina. Quando Alberto perde (almeno) una lente, esce in ritardo nel vano tentativo di cercarla. Siano $Y=$ "numero di lenti a contatto perse da Alberto in due giorni", $\mathrm{Z} =$ "numero di mattine su due giorni in cui Alberto è in ritardo". Trovare:

2. La distribuzione di $Y$.

3. La distribuzione di $Z$.

\ParteSoluzioni



\chapter{Vettori aleatori continui}
%!TEX root = ../main.tex

% Introduzione

\section{Introduzione}
Vogliamo descrivere uno ad uno gli strumenti che serviranno a svolgere gli esercizi di questo foglio di calcolo.

\subsection{Integrali multipli}
Rinfreschiamo dal corso di Analisi 2 le seguenti definizioni.

\begin{definition}[Domini normali]$\\$
Siano $f_1(x), f_2(x)$ due funzioni continue in $[a,b]\subseteq\RR$ e tali che che $f_1(x)\leq f_2(x)$ $\ \forall x\in[a,b]$. Allora l'insieme
\[
D\coloneqq\{(x,y)\in\RR^2\ :\ x\in[a,b],\ f_1(x)\leq y\leq f_2(x)   \}
\]
è misurabile ed è chiamato dominio normale rispetto all'asse $x$.
\fg{0.3}{9_1}
Siano $f_3(y), f_4(y)$ due funzioni continue in $[c,d]\subseteq\RR$ e tali che che $f_3(y)\leq f_4(y)$ $\ \forall y\in[c,d]$. Allora l'insieme
\[
E\coloneqq\{(x,y)\in\RR^2\ :\ f_3(y)\leq x\leq f_4(y),\ y\in[c,d]   \}
\]
è misurabile ed è chiamato dominio normale rispetto all'asse $y$.
\fg{0.3}{9_2}
\end{definition}
Osserviamo che esistono anche domini normali rispetto entrambi gli assi.

\begin{definition}[Domini normali regolari]$\\$
Il dominio normale $D$ è detto normale regolare se $f_1,f_2\in\Cu([a,b])$ con $f_1(x)<f_2(x)$ $\ \forall x\in[a,b]$.

Analogamente, il dominio normale $E$ è detto normale regolare se $f_3,f_4\in\Cu([c,d])$ con $f_3(y)<f_4(y)$ $\ \forall y\in[c,d]$.
\end{definition}

\begin{definition}[Domini regolari]$\\$
Un dominio $\Omega\subseteq\RR^2$ si dice regolare se è l'unione finita di domini normali regolari privi di punti interni comuni.
\end{definition}
Osserviamo che per costruzione un dominio regolare è chiuso e limitato, quindi è anche compatto.

Siamo pronti per enunciare
\begin{theorem}[Di Fubini]$\\$
\label{introth1}
Sia $D$ un dominio normale rispetto all'asse $x$ in $[a,b]$. Allora data $f:D\to\RR$ continua si ha
\[
\iint_{D}f(x,y)\dxy=\int_{a}^{b} \left(\int_{f_1(x)}^{f_2(x)}f(x,y)\dy  \right)\dx
\]
Sia $E$ un dominio normale rispetto all'asse $y$ in $[c,d]$. Allora data $f:R\to\RR$ continua si ha
\[
\iint_{E}f(x,y)\dxy=\int_{c}^{d} \left(\int_{f_3(y)}^{f_4(y)}f(x,y)\dx  \right)\dy
\]
\end{theorem}

\begin{theorem}[Di Jacobi]$\\$
\label{introth2}
Siano $\Omega,\Omega'\subseteq\RR^2$ due domini regolari. Sia $g:\Omega'\to\Omega$ il campo vettoriale che associa ad ogni coppia $(u,v)\in\Omega'$ la coppia $(x,y)\in\Omega$. Se
\begin{enumerate}
\item [(i)] $g$ è biunivoca
\item [(ii)] $g\in\Cu(\Omega')$
\item [(iii)] $g$ tale che $\det J_g(u,v)\neq0$ per ogni coppia  $(u,v)\in\Omega'$
\end{enumerate}
allora data una funzione $f:\Omega\to\RR$ continua si ha che
\begin{gather*}
\begin{aligned}
\iint_\Omega f(x,y)\dxy&=\iint_{\Omega'}f\left(g^{-1}(u,v)\right)\cdot |\det J_{g^{-1}}(u,v)| \text{ d}u\text{d}v \\
&=\iint_{\Omega'}f\left(g^{-1}(u,v)\right)\cdot \frac{1}{|\det J_{g}(u,v)|} \text{ d}u\text{d}v
\end{aligned}
\end{gather*}
\end{theorem}
Vediamo subito un esempio: calcoliamo l'integrale doppio di $$f(x,y)=e^{2x-y} \text{ su } \Omega=\{(x,y)\in\RR^2\ :\ 2x\leq y\leq2x+5,\ -3x\leq y\leq 15/2-3x  \}$$
Prima di partire riscriviamo meglio l'insieme di integrazione
\[
\Omega=\{(x,y)\in\RR^2\ :\ 0\leq y-2x\leq5,\ 0\leq y+3x\leq 15/2  \}
\]
Allora prendendo $g$ tale che
\[
(u,v)=(y-2x,y+3x)=g(x,y)
\]
si rispettano tutte le ipotesi del teorema di Jacobi (a voi i conti). Allora è ben definita l'inversa di $g$
\[
g^{-1}(u,v)=((v-u)/5,(3u+2v)/5)=(x,y)
\]
con il suo jacobiano
\[
J_{g^{-1}}(u,v)=\begin{pmatrix}
\frac{\partial ((v-u)/5)}{\partial u} &\frac{\partial ((v-u)/5)}{\partial v}  \\
\frac{\partial ((3u+2v)/5)}{\partial u} &\frac{\partial ((3u+2v)/5)}{\partial v}  \\
\end{pmatrix} =\begin{pmatrix}
-1/5 & 1/5 \\
3/5 & 2/5 \\
\end{pmatrix}
\]
tale che $\det J_{g^{-1}}(u,v)=-1/5$. Allora
\begin{gather*}
\begin{aligned}
\iint_\Omega e^{2x-y}\dxy&=\iint_{[0,0.5]\times[0,15/2]} e^{-u}\cdot\frac{1}{5}\text{ d}u\text{d}v=\\
&=\frac{1}{5}\int_0^{0.5}e^{-u}\text{ d}u\int_0^{15/2}1\text{ d}v=\\
&=\cdots =\\
&=\frac{3}{2}(1-e^{-5})
\end{aligned}
\end{gather*}
\fg{0.7}{9_3}
Ora interpreteremo questo importante teorema in funzione delle nozioni probabilistiche del corso.

\subsection{Formule di Jacobi in Probabilità}
Nel capitolo sulle variabili aleatori continue abbiamo già visto come applicare la formula di Jacobi, anche se abbiamo anche detto che nel caso 1 - dimensionale non si riesce a comprendere la vera portata di tale enunciato, in quanto è più semplice passare per il calcolo della funzione di ripartizione. Per completezza, ecco il teorema
\begin{theorem}[Di Jacobi per le variabili continue]$\\$
\label{introth3}
Sia $X$ variabile aleatoria assolutamente continua con densità $f_X$. Inoltre sia $g:\RR\to\RR$ tale che
\begin{enumerate}
\item [(i)] $g$ è iniettiva
\item [(ii)] $g\in\Cu(\RR)$
\item [(iii)] $g$ tale che $g'(x)\neq0$ per ogni $x\in\RR$
\end{enumerate}
Allora $Y=g(X)$ è una variabile aleatoria assolutamente continua con densità
\[
f_Y(y)=\begin{cases}
f_X(g^{-1}(y))\cdot|(g^{-1}(y))'|   &\text{ se }y\in g(\RR) \\
0&\text{ altrove, cioè dove }y\not\in g(\RR)
\end{cases}
\]
Attenzione, delle volte se $X$ prende valori in un certo $S\subseteq\RR$ aperto, allora è comodo definire $g:S\to\RR$ perché potrebbero esserci $g$ non iniettive o derivabili in $\RR$ ma iniettive o derivabili in $S$.
\end{theorem}
Si veda l'esercizio 7 per un esempio pratico.

Ora vediamo il caso $n$ - dimensionale
\begin{theorem}[Di Jacobi per i vettori continui]$\\$
\label{introth4}
Sia $\SDP$ uno spazio di probabilità e $X:\Omega\to\RR^n$ un vettore aleatorio assolutamente continuo con densità $f_X:\RR^n\to\RR$. Inoltre sia $g:\RR^n\to\RR^n$ una funzione tale che
\begin{enumerate}
\item [(i)] $g$ è iniettiva
\item [(ii)] $g\in\Cu(\RR^n)$
\item [(iii)] $g$ tale che $\det J_g(x)\neq0$ per ogni $x\in\RR^n$
\end{enumerate}
Allora $g:\RR^n\to g(\RR^n)$ è biiettiva, ha inversa $g^{-1}:g(\RR^n)\to\RR^n$ anch'essa di classe $\Cu$, e la variabile aleatoria $Y\coloneqq g\circ X:\Omega\to\RR^n$ è assolutamente continua con densità
\[
f_Y(y)=\begin{cases}
f_X(g^{-1}(y))\cdot|\det J_{g^{-1}}(y)|   &\text{ se }y\in g(\RR^n) \\
0&\text{ altrove, cioè dove }y\not\in g(\RR^^n)
\end{cases}
\]
dove
\[
J_{g^{-1}}(y)=\frac{1}{J_g(g^{-1}(y))}
\]
Attenzione: $g$ non cambia le dimensioni del vettore; inoltre delle volte se $X$ prende valori in un certo $S\subseteq\RR^n$ aperto (cioè $\PP(X\in S)=1$, $X\in S$ quasi certamente), allora è comodo definire $g:S\to\RR^n$ perché potrebbero esserci $g$ non iniettive o derivabili in $\RR^n$ ma iniettive o derivabili in $S$.
\end{theorem}
Si veda l'esercizio 9 per un esempio pratico.

\subsection{Altri teoremi utili}
Richiamiamo qui altri teoremi che ci servanno per svolgere gli esercizi.

\begin{theorem}[Calcolo delle leggi marginali]
\label{introth5-}
$\\$Se il vettore aleatorio $(X,Y)$ è continuo allora $X$ e $Y$ sono variabili aleatorie continue (il viceversa non vale).
\end{theorem}

\begin{theorem}[Calcolo delle leggi marginali]
\label{introth5}
$\\$Se $(X,Y)$ è un vettore aleatorio assolutamente continuo con densità $\fXY$ allora $X,Y$ sono variabili aleatorie assolutamente continue, con densità
\begin{gather*}
\begin{aligned}
&f_X(x)=\int_\RR \fXYxy\text{ d}y&\forall x\in\RR \\
&f_Y(y)=\int_\RR \fXYxy\text{ d}x&\forall y\in\RR
\end{aligned}
\end{gather*}
\end{theorem}

Tale enunciato non può essere invertito, a meno che non si aggiunga l'ipotesi di indipendenza, cioè
\begin{theorem}[Fattorizzazione della legge congiunta]
\label{introth6}
$\\$$X\indep Y\ \Longleftrightarrow\ \fXYxy=f_X(x)\cdot f_Y(y)$
\end{theorem}

\begin{theorem}[Condizione necessaria sui supporti per l'indipendenza]
\label{introth7}
$\\$Siano $X$ e $Y$ due variabili aleatorie continue e indipendenti con densità rispettivamente $f_X$ su $S_X$ e $f_Y$ su $S_Y$. Sia $(X,Y)$ il vettore aleatorio continuo con densità $\fXY$ su $S_{(X,Y)}$. Allora deve valere che
\begin{equation*}
S_{(X,Y)}=S_X\cdot S_Y
\end{equation*}
\end{theorem}

\begin{theorem}[Regola del valore atteso per vettori continui]
\label{introth8}
$\\$Sia $(X,Y)$ vettore aleatorio continuo su $(\Omega,\mathcal{A},\PP)$ con legge $P^{(X,Y)}$ e densità $\fXY$. Allora $\forall h:\RR^2\to\RR$ borelliana e $\forall h:\RR^2\to[0,+\infty)$
\begin{enumerate}
\item [i)] $h\in L^1(P^{(X,Y)})\ \Longleftrightarrow\ h\cdot \fXY\in L^1(m_2)$
\item [ii)] $\EE[h(X,Y)]=\int_{\RR^2}h(x,y)\fXYxy \text{ d}x\text{d}y$
\end{enumerate}
\end{theorem}

\begin{theorem}[Risultato notevole (esercizio 4 del foglio 5)]
\label{introth9}
$\\$Data $X$ variabile aleatoria positiva e continua, denotata $F$ la sua funzione di ripartizione, allora
\[
\EE[X]=\int_{0}^{+\infty}(1-F(x))\dx
\]
\end{theorem}

\begin{theorem}[Coefficiente di correlazione unitario]
\label{introth10}
$\\$Siano $X,Y\in L^2$ tali che $\Var(X),\Var(Y)>0$. Allora
\[
|\rho_{X,Y}|=1\ \Longleftrightarrow\ \exists a\neq 0,\ \exists b\in\RR\ :\ Y=aX+b
\]
\end{theorem}

\begin{theorem}[Alcune proprietà di valore atteso, varianza e covarianza]
\label{introth11}
$\\$Siano $X,Y\in L^2$ e $a,b,c,d\in\RR$. Allora valgono:
\begin{enumerate}
\item [(a)] $\Cov(X,X)=\Var(X)$
\item [(b)] $X\indep Y\ \Longrightarrow\ \Cov(X,Y)=0\ \Longrightarrow\ \rho_{X,Y}=0$
\item [(c)] $\EE(aX+b)=a\EE[X]$
\item [(d)] $X\indep Y\ \Longrightarrow\ \EE[XY]=\EE[X] \EE[Y]$
\item [(e)] $\Var(aX+b)=a^2\Var(X)$
\item [(f)] $\Cov(aX+b,cY+d)=\Cov(aX,cY)=ac\Cov(X,Y)$
\item [(g)] $\rho_{aX+b,cY+d}=\rho_{X,Y}$
\item [(h)] $\Var(X+Y)=\Var(X)+\Var(Y)+2\Cov(X,Y)$
\item [(i)] $X\indep Y\ \Longrightarrow\ \Var(X+Y)=\Var(X)+\Var(Y)$
\end{enumerate}
\end{theorem}

\begin{theorem}[Bilinearità della covarianza]
\label{introth12}
$\\$Dati $X,Y,Z,W\in L^2$ e $a,b,c,d\in\RR$ si ha
\[
\Cov(aX+bY,cW+dZ)=ac\Cov(X,W)+ad\Cov(X,Z)+bc\Cov(Y,W)+bd\Cov(Y,Z)
\]
\end{theorem}

\begin{theorem}[Valore atteso e varianza per trasformazioni affini]
\label{introth13}
$\\$Sia $X:\Omega\to\RR^n$ un vettore aleatorio. Allora la trasformazione affine $Y\coloneqq AX+b$ con $A\in\RR^{m\times n},\ b\in\RR^{m}$ è un vettore aleatorio in $\RR^m$ con
\begin{gather*}
\begin{aligned}
\EE[Y]&=A\cdot\EE[X]+b \\
\Var(Y)&=A\cdot \Var(X)\cdot A^T
\end{aligned}
\end{gather*}
\end{theorem}

Ora anticipiamo due risultati notevoli che verranno poi dimostrati nelle pagine a seguire.
\begin{theorem}[Risultato notevole (esercizio 13 del foglio 7)]
\label{introth14}
$\\$Date $X$ e $Y$ di legge congiunta continua con densità $\fXY$, si consideri $Z=X+Y$. Allora $Z$ èuna variabile aleatoria continua con densità
\[
f_Z(z)=\int_{-\infty}^{+\infty} \fXY(z-y,y)\dy
\]
\end{theorem}
\begin{theorem}[Risultato notevole (esercizio 2 del foglio 7)]
\label{introth15}
$\\$Due variabili aleatorie $X$ e $Y$ sono variabili aleatorie continue e indipendenti se e solo se sono congiuntamente continue con densità fattorizzabile, cioè esistono due funzioni $h_1:\RR\to\RR$ e $h_2:\RR\to\RR$ tali che $\fXYxy=h_1(x)\ h_2(y)$ q.o.
\end{theorem}

\subsection{Qualche integrale notevole}
 Vogliamo infine ricordare i seguenti integrali (risolti integrando \textit{pp}: per parti), in modo tale da rendere più fluidi i futuri passaggi.
\begin{enumerate}
\item [$(\alpha)$]
\begin{gather*}
\begin{aligned}
\int te^{-t}\text{ d}t&\overset{\underset{\textit{pp}}{}}{=}\begin{Bmatrix}
\text{fattore finito }&f=t \\ \text{fattore differenziale }&g=-e^{-t}
\end{Bmatrix}=  \\
&=fg-\int f'g\text{ d}t=\\
&=-te^{-t}-\int-e^{-t}\text{ d}t=\\
&=-te^{-t}+\int e^{-t}\text{ d}t=\\
&=-te^{-t}-e^{-t}=\\
&=-(t+1)e^{-t}
\end{aligned}
\end{gather*}
\item [$(\alpha^1)$]
\begin{equation*}
\int_0^{+\infty} te^{-t}\text{ d}t\overset{\underset{(\alpha)}{}}{=}\left[-(t+1)e^{-t}\right]_0^{+\infty}=1
\end{equation*}
\item [$(\beta)$]
\begin{gather*}
\begin{aligned}
\int t^2e^{-t}\text{ d}t&\overset{\underset{\textit{pp}}{}}{=}-t^2e^{-t}-\int-2te^{-t}\text{ d}t=\\
&=-t^2e^{-t}+2\int te^{-t}\text{ d}t=\\
&\overset{\underset{(\alpha)}{}}{=}-t^2e^{-t}-2(t+1)e^{-t}=\\
&=-(t^2+2t+2)e^{-t}
\end{aligned}
\end{gather*}
\item [$(\beta^1)$]
\begin{equation*}
\int_0^{+\infty} t^2e^{-t}\text{ d}t\overset{\underset{(\beta)}{}}{=}\left[-(t^2+2t+2)e^{-t}\right]_0^{+\infty}=2
\end{equation*}
\item [$(\gamma)$]
\begin{gather*}
\begin{aligned}
\int t^3e^{-t}\text{ d}t&\overset{\underset{\textit{pp}}{}}{=}-t^3e^{-t}-\int-3t^2e^{-t}\text{ d}t=\\
&=-t^3e^{-t}+3\int t^2e^{-t}\text{ d}t=\\
&\overset{\underset{(\beta)}{}}{=}-t^3e^{-t}-3(t^2+2t+2)e^{-t}=\\
&=-(t^3+3t^2+6t+6)e^{-t}
\end{aligned}
\end{gather*}
\item [$(\gamma^1)$]
\begin{equation*}
\int_0^{+\infty} t^3e^{-t}\text{ d}t\overset{\underset{(\gamma)}{}}{=}\left[-(t^3+3t^2+6t+6)e^{-t}\right]_0^{+\infty}=6
\end{equation*}

\end{enumerate}

% Fine introduzione

\newpage

\ParteEsercizi

\Esercizio{} % es 1 manca sol
Si esibisca un esempio di due variabili aleatorie continue $X$ e $Y$ con legge congiunta continua. \\
Si esibisca poi un esempio di due variabili aleatorie continue $X$ e $Y$ con legge congiunta non continua.

\Esercizio{$^\ast$}
Si mostri che due variabili aleatorie $X$ e $Y$ sono variabili aleatorie continue e indipendenti se e solo se sono congiuntamente continue con densità fattorizzabile, cioè esistono due funzioni $h_1:\RR\to\RR$ e $h_2:\RR\to\RR$ tali che $\fXYxy=h_1(x)\ h_2(y)$ q.o.

\Esercizio{}
Date $X,Y$ indipendenti ed entrambe di legge $\mathcal{E}(\lambda)$, $\lambda>0$, si calcoli $\PP(Y>X)$.

\Esercizio{}
Sia $(X,Y)$ un vettore aleatorio con densità
\begin{equation*}
\fXYxy=\begin{cases} x(y-x)e^{-y},&0<x<y, \\ 0,&\text{altrove}.\end{cases}
\end{equation*}
\begin{enumerate}
\item [(a)] Calcolare le leggi di $X$ e di $Y$.
\item [(b)] $X$ e $Y$ sono indipendenti?
\item [(c)] Calcolare $\PP(X\leq 2,Y\leq 3)$.
\item [(d)] Calcolare il coefficiente di correlazione $\rho_{X,Y}$.
\item [(e)] Trovare una diversa densità congiunta avente le stesse densità marginali.
\end{enumerate}

\Esercizio{}
Sia $(X,Y)$ un vettore aleatorio di densità
\[
\fXYxy=\begin{cases}
xe^{-(x+y)} &\text{se }x>0,y>0 \\
0 &\text{altrove}
\end{cases}
\]
\begin{enumerate}
\item [(a)] Quali sono le leggi di $X$ e $Y$? Le variabili aleatorie $X$ e $Y$ sono indipendenti?
\item [(b)] Calcolare $\Var(X+Y)$.
\item [(c)] Si calcolino i valori attesi di $U=\min(X,Y)$ e $V=\max(X,Y)$.
\end{enumerate}

\Esercizio{}
Il vettore aleatorio $(X,Y)$ ha legge uniforme sul quadrato $Q$ di vertici $(0,2),(2,0),(4,2),(2,4)$.
\begin{enumerate}
\item [(a)] Qual è la legge di $X$?
\item [(b)] Quanto vale $\EE[X]$?
\item [(c)] Quanto vale $\PP(X<Y)$?
\item [(d)] Quanto vale $\PP(X\leq 2,Y\leq 1)$?
\item [(e)] $X$ e $Y$ sono indipendenti?
\end{enumerate}

\Esercizio{}
Una misura di resistenza in un circuito elettrico eseguita con uno strumento di risoluzione pari a 1 ohm dà lettura di 12 ohm. La resistenza $R$ del circuito risulta pertanto descritta da una variabile aleatoria con distribuzione uniforme tra 11.5 e 12.5 ohm. Sia $M=1/R$ la conduttanza del circuito. 
\begin{enumerate}
\item [(a)] $R$ e $M$ hanno legge congiunta continua?
\item [(b)] Calcolare i valori attesi di $R$ e $M$.
\item [(c)] Calcolare la matrice varianza di $(R,M)$.
\item [(d)] Calcolare il coefficiente di correlazione lineare fra $R$ e $M$.
\item [(e)] Determinare la distribuzione di $M$.
\end{enumerate}

\Esercizio{ (Problema dell'ago di Buffon)}
Un ago di lunghezza $l>0$ viene lanciato su un pavimento decorato con linee parallele distanti $d>0$. Qual è la probabilità $p$ che l'ago intersechi almeno una linea?

\Esercizio{}
Siano $X$ e $Y$ variabili aleatorie indipendenti con legge $\Uc([0,1])$.
\begin{enumerate}
\item [(a)] Si scriva la densità di probabilità del vettore aleatorio $(X,Y)$.
\end{enumerate}
Si consideri il vettore aleatorio $(U,V)=(X+1,X+Y)$.
\begin{enumerate}
\item [(b)] Calcolarne il valore atteso e la matrice varianza.
\item [(c)] Se ne calcoli la legge.
\item [(d)] Il vettore ha componenti indipendenti?
\end{enumerate}
Si consideri ora il vettore aleatorio $(U,V)=(X+1,X+Y-XY+X^2Y)$.
\begin{enumerate}
\item [(e)] Calcolarne il valore atteso.
\item [(f)] Se ne calcoli la legge.
\item [(g)] Il vettore ha componenti indipendenti?
\end{enumerate}

\Esercizio{} % es 10 manca sol
Verificare che se $X_1,\dots,X_n$ sono variabili aleatorie esponenziali indipendenti di parametri $\lambda_1,\dots,\lambda_n$ rispettivamente, allora
\[
X_{(1)}=\min\{X_1,\dots,X_n  \}\sim\Ec(\lambda_1,\dots,\lambda_n)
\]

\Esercizio{} % es 11 manca sol
Siano $X_1,\dots,X_n$ sono variabili aleatorie continue i.i.d. con funzione di ripartizione $\Cc^1$ a tratti. 
\begin{enumerate}
\item [(a)] Mostrare che $X_{(n)}=\max\{ X_1,\dots,X_n \}$ è una variabile aleatoria continua e determinarne la densità.
\item [(b)] Mostrare che $X_{(1)}=\min\{ X_1,\dots,X_n \}$ è una variabile aleatoria continua e determinarne la densità.
\end{enumerate}

\Esercizio{}
Un componente elettronico è formato da tre elementi indipendenti in serie, ciascuno dei quali ha un tempo di vita esponenziale di parametro $\lambda=0.3,\mu=0.1,\gamma=0,2$ rispettivamente.
\begin{enumerate}
\item [(a)] Indichiamo con $T$ il tempo di vita del componente. Qual è la legge di $T$?
\item [(b)] Per aumentare l'affidabilità e ridurre gli interventi di sostituzione, viene proprosto di aggiungere un componente identico in parallelo. Qual è la legge del tempo di vita $S$ del nuovo complesso?
\end{enumerate}

\Esercizio{}
Date $X$ e $Y$ di legge congiunta continua con densità $\fXY$, si consideri $Z=X+Y$. Si mostri che $Z$ è una variabile aleatoria continua con densità
\[
f_Z(z)=\int_{-\infty}^{+\infty} \fXY(z-y,y)\dy
\]
Si mostri questo risultato procedendo nei due seguenti modi:
\begin{enumerate}
\item [(a)] Si calcoli la funzione di ripartizione di $Z$ e si determini la densità.
\item [(b)] Si calcoli la legge di $(Z,Y)$ e si trovi la legge marginale di $Z$.
\end{enumerate}

\Esercizio{}
Siano $X$ e $Y$ due variabili aleatorie indipendenti entrambe con distribuzione uniforme su $[0,1]$.
\begin{enumerate}
\item [(a)] Determinare media, varianza e legge di $Z=X+Y$.
\item [(b$^*$)] Si trovi la legge di $T=X+Y-\Ind_{\{ X+Y-1 \}}$.
\end{enumerate}

\Esercizio{}
Siano $X$ e $Y$ due variabili aleatorie indipendenti entrambe con distribuzione uniforme su $[0,1]$.
\begin{enumerate}
\item [(a)] Determinare la distribuzione congiunta del vettore $(U,V)$ ove $U=XY$ e $V=\dfrac{Y}{X}$.
\item [(b)] Determinare le marginali di $U$ e $V$.
\end{enumerate}

\Esercizio{}
Siano $X\sim\Gamma(\alpha,\lambda)$ e $Y\sim\Gamma(\beta,\lambda)$ due variabili aleatorie indipendenti; determinare la densità della variabile aleatoria $U=\dfrac{X}{Y}$.

\Esercizio{} % es 17 manca sol
Calcolo delle densità della $t$ d Student e della $F$ di Fisher.
\begin{enumerate}
\item [(a)] Siano $Z\sim\Nc(0,1)$ e $X\sim\chi^2(n)$ due variabili aleatorie indipendenti; determinare la densità di $T=\dfrac{Z}{\sqrt{\frac{X}{n}}}$.
\item [(b)] Siano $X\sim\chi^2(m)$ e $Y\sim\chi^2(n)$ due variabili aleatorie indipendenti; determinare la densità di $F=\dfrac{X/m}{Y/n}$.
\end{enumerate}

\Esercizio{} % es 18 manca sol
Siano $X\sim\Uc([-5,3])$ e $Y\sim\Uc([3,5])$ due variabili aleatorie indipendenti.
\begin{enumerate}
\item [(a)] Determinare media, varianza e legge di $Z=X+Y$.
\item [(b)] Si calcoli il coefficiente di correlazione $\rho_{X,Z}$.
\end{enumerate}

\Esercizio{}
Sia $(X,Y)$ un vettore aleatorio continuo con densità
\[
\fXYxy=\begin{cases} \dfrac{1}{2}(x+y)e^{-(x+y)} &x,y>0 \\ 0&\text{altrove} \end{cases}
\]
\begin{enumerate}
\item [(a)] Si calcoli la media di $(X+Y)^{-1}$.
\item [(b)] Si determini la legge di $X+Y$.
\item [(c)] Si calcoli la media di $X+Y$.
\item [(d)] Si calcolino le marginali di $X,Y$. $X\indep Y$?
\item [(e)] Si calcoli $\Cov(X,Y)$.
\item [(f)] Si calcoli $\PP(X\geq 1,Y\geq 2)$.
\end{enumerate}

\Esercizio{} % es 20 manca sol
Due numeri $X,Y$ vengono scelti a caso indipendentemente con distribuzione uniforme su $[0,1]$.
\begin{enumerate}
\item [(a)] Calcolare $\PP(|X-Y|>1/2)$.
\item [(b)] Sia $Z$ la variabile aleatoria che misura la distanza fra $X$ e $Y$. Qual è la legge di $Z$? Qual è la distanza media fra $X$ e $Y$?
\end{enumerate}

\Esercizio{} % es 21 manca punto a
Sia assegnata una successione $N,X_1,X_2,\dots$ di variabili aleatorie indipendenti, tale che $N\sim\Gc(p), p\in(0,1)$, e ciascuna delle $X_i\sim\Ec(\lambda),\lambda>0$. Si ponga $Y=\min\{X_1,\dots,X_N  \}$.
\begin{enumerate}
\item [(a$^*$)] Si mostri che $Y$ è una variabile aleatoria.
\item [(b)] Qual è la legge di $Y$?
\item [(c)] Qual è il valor atteso di $Y$? 
\end{enumerate}

\Esercizio{} % es 22 manca sol
Si mostri che, date due variabili aleatorie indipendenti $X$ e $Y$ in $L^1$, allora necessariamente si ha anche $XY$ in $L^1$. Si mostri però che, se $X$ e $Y$ non sono indipendendenti, allora $XY$ può non essere integrabile.

\Esercizio{$^\ast$} % es 23 manca sol
Date $n$ variabili aleatorie reali $X_1,\dots,X_n$, si definiscono \emph{statistiche d'ordine}
\begin{gather*}
\begin{aligned}
X_{(1)}&=\min\{X_1,\dots,X_n  \}, \\
X_{(2)}&=\text{secondo più piccolo valore di }X_1,\dots,X_n, \\
&\ \vdots \\
X_{(n)}&=\max\{X_1,\dots,X_n  \}.
\end{aligned}
\end{gather*}
\begin{enumerate}
\item [(a)] Si trovi una formula per $X_{(2)}$ nel caso $n=3$ e una formula per $X_{(3)}$ nel caso $n=4$.
\item [(b)] Si trovi una formula per $X_{(k)}$ nel caso $n\in\NN,1\leq k\leq n$.
\item [(c)] Si mostri chele statistiche d'ordine sono variabili aleatorie e che $X_{(1)}\leq X_{(2)}\leq \cdots \leq X_{(n)}$ q.c.
\item [(d)] Sia $B\in\Bc^n$ un boreliano di $\RR^n$ tale che: se $(y_1,\dots,y_n)\in B$ allora $y_1\leq y_2\leq \cdots\leq y_n$. In altre parole, $B\subset\{(y_1,\dots,y_n)\in\RR^n\ :\ y_1\leq y_2\leq \cdots\leq y_n  \}$. Si esprima l'evento $((X_{(1)},\dots,X_{(n)})\in B)$ in termini di $X_1,\dots,X_n$.
\item [(e)] Supponiamo ora che $X_1,\dots,X_n$ siano i.i.d. con densità continua $f$. Si mostri che $X_{(1)},\dots,X_{(n)}$ hanno legge congiuntamente continua con densità
\[
f_{(X_{(1)},\dots,X_{(n)})}(y_1,\dots,y_n)=\begin{cases}n!\displaystyle\prod_{i=1}^n f(y_i) &y_1\leq y_2\leq \cdots\leq y_n \\ 0&\text{altrimenti}   \end{cases}
\]
\end{enumerate}

\Esercizio{}
Le lunghezze $X$ e $Y$ dei cateti di un triangolo rettangolo sono generate a caso, in modo tale che si tratti di variabili esponenziali indipendenti di stesso parametro $\lambda>0$. Sia $A$ l'area del triangolo, $Z=\frac{Y}{X}$ il rapporto dei cateti e $\alpha$ l'angolo (misurato in radianti) opposto al cateto $Y$.
\begin{enumerate}
\item Calcolare il valore atteso dell'area $A$, in funzione di $\lambda$.
\item Calcolare la funzione di ripartizione di $Z$, notando che non dipende da $\lambda$ e disegnarne il grafico.
\item Determinare se $Z$ è assolutamente continua e in tal caso calcolarne la densità e disegnarne il grafico.
\item Calcolare la media di $Z$.
\item Calcolare la funzione di ripartizione di $\alpha$, notando che non dipende da $\lambda$.
\item Determinare se $\alpha$ è assolutamente continua e in tal caso calcolarne la densità.
\end{enumerate}

\ParteSoluzioni

\Soluzione{}
Manca

\Soluzione{}
La dimostrazione si articola in due parti:
\begin{enumerate}
\item [$(\Rightarrow)$] Questo caso è banale, perché basta prendere $h_1(x)=f_X(x)$ e $h_2(y)=f_Y(y)$.

\item [$(\Leftarrow)$] Sappiamo per ipotesi che $\fXYxy=h_1(x)\ h_2(y)$. Allora grazie al teorema (\ref{introth6}) calcoliamo le due marginali come
\begin{gather*}
\begin{aligned}
&f_X(x)=\int_\RR h_1(x)\ h_2(y) \dy=h_1(x)\int_\RR h_2(y) \dy \\
&f_Y(y)=\int_\RR h_1(x)\ h_2(y) \dx=h_2(y)\int_\RR h_1(x) \dx
\end{aligned}
\end{gather*}
Sfruttando il fatto che per definizione di densità deve essere $\int_\RR\fXYxy\dxy=1$, possiamo calcolare
\begin{gather*}
\begin{aligned}
f_X(x)\ f_Y(y)&=h_1(x)\int_\RR h_2(y) \dy\cdot h_2(y)\int_\RR h_1(x) \dx=\\
&=h_1(x)\ h_2(y) \cdot \int_\RR h_1(x)\ h_2(y) \dxy=\\
&=h_1(x)\ h_2(y) \cdot \underbrace{\int_\RR \fXYxy \dxy}_{=1}=\\
&=h_1(x)\ h_2(y)=\\
&=\fXYxy
\end{aligned}
\end{gather*}
e quindi concludere che $X$ e $Y$ sono indipendenti.
\end{enumerate}
L'enunciato è quindi dimostrato.

\Soluzione{}
Il vettore aleatorio $(X,Y):\Omega\to\mathbb{R}^2$ è assolutamente continuo; la sua densità $\fXYxy$, essendo $X$ e $Y$ indipendenti, è data, grazie al  teorema (\ref{introth6}), dal prodotto tra le due densità marginali
\begin{gather*}
\begin{aligned}
&f_X(x)=\lambda e^{-\lambda x}\ \Ind_{[0,+\infty)}(x) \\
&f_Y(y)=\lambda e^{-\lambda y}\ \Ind_{[0,+\infty)}(y)
\end{aligned}
\end{gather*}
Allora per calcolare $\PP(Y>X)$ basta semplicemente vedere l'evento $Y>X$ come $(X,Y)\in T$, con $T$ zona del primo quadrante del piano $X\times Y$ che sta sopra la bisettrice $Y=X$, e poi integrare di conseguenza:
\begin{gather*}
\begin{aligned}
\PP(Y>X)&=\PP((X,Y)\in T)= \\
&=\int_T \fXYxy \text{ d}x\text{d}y= \\
&=\int_0^{+\infty} \int_{x}^{+\infty}\lambda^2e^{-\lambda x}e^{-\lambda y}\text{ d}x\text{d}y= \\
&=\int_0^{+\infty}\lambda e^{-\lambda x} \left(\int_{x}^{+\infty} \lambda e^{-\lambda y}  \text{ d}y  \right)\text{ d}x= \\
&=\int_0^{+\infty}\lambda e^{-\lambda x} \left[ e^{-\lambda y}  \right]_x^{+\infty}\text{ d}x=\\
&=\underbrace{\int_0^{+\infty}\lambda e^{-2\lambda x}\text{ d}x}_{\text{riconduco a }\mathcal{E}(2\lambda)}=\\
&=\frac{1}{2}\underbrace{\int_0^{+\infty}2\lambda e^{-2\lambda x}\text{ d}x}_{=1}=\\
&=\frac{1}{2}
\end{aligned}
\end{gather*}
Esiste un secondo modo, più teorico ma meno calcoloso, per risolvere l'esercizio. Infatti possiamo osservare che
\begin{equation*}
1=\PP(X<Y)+\PP(X=Y)+\PP(X>Y)
\end{equation*}
Diamo un'occhiata ai tre termini:
\begin{enumerate}
\item $\PP(X<Y)$ è la nostra incognita.
\item $\PP(X=Y)=\PP((X,Y)\in r)$ con $r$ bisettrice del primo quadrante del piano $X\times Y$; ma
\begin{enumerate} 
\item [(i)] il vettore $(X,Y)$ è assolutamente continuo rispetto la misura di Lebesgue su $\mathbb{R}^2$
\item [(ii)] la retta $r$ è un sottospazio di $\mathbb{R}^2$ di dimensione $1$ e di conseguenza ha misura di Lebesgue $m_2=0$
\end{enumerate}
$\Longrightarrow \PP(X=Y)=0$.
\item $\PP(X>Y)=\PP(X<Y)$ perché il ruolo di $X$ e $Y$ è interscambiabile dato che hanno la stessa distribuzione.
\end{enumerate}
Allora $1=2\ \PP(X<Y)\ \Longrightarrow\ \PP(Y>X)=\frac{1}{2}$.

\Soluzione{}
\begin{enumerate}
\item [(a)] Calcolare le leggi di $X$ e di $Y$.

Conoscendo la densità congiunta del vettore aleatorio assolutamente continuo $(X,Y)$, per calcolare le leggi marginali usiamo il teorema (\ref{introth5}). Prima di procedere però è sempre meglio capire come è fatto il supporto della densità congiunta, così da capire quale sarà il supporto delle marginali:
\[
S_{(X,Y)}\coloneqq\{(x,y)\in\RR^2\ \big|\ \fXYxy>0   \}=\{(x,y)\in\RR^2\ \big|\ 0<x<y   \}
\]
\fg{0.3}{7_17}
Dal grafico della densità congiunta deduciamo un'importante informazione su $X$ e $Y$: $X,Y\geq 0$ quasi certamente, cioè $f_X(x)=0$ per ogni $x<0$ e $f_Y(y)=0$ per ogni $y<0$. Quindi fissiamo $x\geq 0$ e procediamo con il calcolo della prima legge marginale:
\begin{gather*}
\begin{aligned}
f_X(x)&=\int_\RR \fXYxy\text{ d}y= \\
&=\int_x^{+\infty}x(y-x)e^{-y}\text{ d}y= \\
&=\begin{Bmatrix}
\text{cambio di variabile}\\t\coloneqq y-x 
\end{Bmatrix}= \\
&=x\int_0^{+\infty}te^{-t-x}\text{ d}t=\\
&=xe^{-x}\underbrace{\int_0^{+\infty}te^{-t}\text{ d}t}_{=1 \text{ per } (\alpha^1)\text{ o per }(\alpha^2)}=xe^{-x}
\end{aligned}
\end{gather*}
con $(\alpha^2)$ che sintetizza la seguente osservazione: data $T\sim\mathcal{E}(\lambda)$ con $\lambda=1$ si ha per definizione
\begin{equation*}
\mathbb{E}[T]\coloneqq\int_0^{+\infty}\lambda te^{-t}\text{ d}t=\cdots=\frac{1}{\lambda}=\frac{1}{1}=1
\end{equation*}

Ora fissiamo $y\geq 0$ e calcoliamo la seconda legge marginale:
\begin{gather*}
\begin{aligned}
f_Y(y)&=\int_\RR \fXYxy\text{ d}x= \\
&=\int_0^y x(y-x)e^{-y}\text{ d}x= \\
&=e^{-y}\left( y\int_0^y x\text{ d}x-\int_0^yx^2\text{ d}x  \right)=\\
&=e^{-y}\left( y\left[\frac{x^2}{2}    \right]_0^y-\left[\frac{x^3}{3}    \right]  \right)=\\
&=e^{-y}\left( \frac{y^3}{2}  -\frac{y^3}{3} \right)=\\
&=\frac{1}{6}\ y^3e^{-y}
\end{aligned}
\end{gather*}
Ricapitolando, abbiamo ottenuto
\begin{gather*}
\begin{aligned}
&f_X(x)=xe^{-x}\ \Ind_{[0,+\infty)}(x) \\
&f_Y(y)=\frac{1}{6}\ y^3e^{-y}\ \Ind_{[0,+\infty)}(y)
\end{aligned}
\end{gather*}
Ricordando che
\begin{oss}
$\\$Sia $X\sim\Gamma(\alpha,\lambda)$ con $\alpha,\lambda>0$. Allora
\begin{gather*}
\begin{aligned}
f_X(x)&=\frac{\lambda^\alpha}{\Gamma(\alpha)}\ x^{\alpha-1}e^{-\lambda x}\ \ \ \ \ x\in\RR \\
\Gamma(\alpha)&=(\alpha-1)! \\
\EE[X]&=\frac{\alpha}{\lambda} \\
\Var(X)&=\frac{\alpha}{\lambda^2}
\end{aligned}
\end{gather*}
(Quando si ha \textit{densità}$=$\textit{polinomio}$\cdot$\textit{esponenziale} è quasi sempre una $\Gamma$!)
\end{oss}
riconosciamo che $X\sim\Gamma(2,1)$ e $Y\sim\Gamma(4,1)$.
\item [(b)] $X$ e $Y$ sono indipendenti?

Abbiamo già visto con il teorema (\ref{introth6}) che se $X\indep Y$ allora la densità congiunta del vettore $(X,Y)$ fattorizza nelle due marginali. Questo è sufficiente per concludere che in questo caso $X\not\indep Y$.$\\$In realtà si potrebbe usare anche il teorema (\ref{introth7}).
Infatti abbiamo
\begin{gather*}
\begin{aligned}
S_X&=[0,+\infty) \\
S_Y&=[0,+\infty) \\
S_{(X,Y)}&=\{(x,y)\in\RR^2\ \big|\ 0<x<y   \}
\end{aligned}
\end{gather*}
e quindi anche in questo caso concludiamo che $X\not\indep Y$.

\item [(c)] Calcolare $\PP(X\leq 2,Y\leq 3)$.

Se le variabili $X,Y$ fossero state indipendenti sarebbe stato banale il calcolco:
\begin{equation*}
\PP(X\leq 2, Y\leq 3)=\PP(X\leq 2)\cdot \PP(Y\leq 3)
\end{equation*}
Tuttavia non lo sono, quindi dobbiamo ricondurre l'evento $X\leq 2, Y\leq 3$ alla coppia $(X,Y)$, cioè
\begin{gather*}
\begin{aligned}
\PP(X\leq 2, Y\leq 3)&=\PP((X,Y)\in(-\infty,2]\times(-\infty,3])=\\
&=\int_{-\infty}^2\left(\int_{-\infty}^3\fXYxy\text{ d}y   \right)\text{d}x=
\end{aligned}
\end{gather*}
che per 
\fg{0.4}{7_18}
diventa
\begin{gather*}
\begin{aligned}
&=\int_0^2\left(\int_x^3x(y-x)e^{-y}\text{ d}y   \right)\text{d}x=\int_0^2x\left(\underbrace{\int_x^3ye^{-y}\text{ d}y}_{(\alpha)}-x\int_x^3e^{-y}\text{ d}y   \right)\text{d}x=\\
&=\int_0^2x\left(  \left[-(y+1)e^{-y}     \right]_x^3-x\left[-e^{-y}     \right]_x^3    \right)\text{d}x=\int_0^2 x\left(-4e^{-3}+(x+1)e^{-x}+xe^{-3}-xe^{-x}   \right)\text{d}x=\\
&=\int_0^2 x\left(-4e^{-3}+e^{-x}+xe^{-3}   \right)\text{d}x=-4e^{-3}\int_0^2 x\text{ d}x+\underbrace{\int_0^2xe^{-x}\text{ d}x}_{(\alpha)}+e^{-3}\int_0^2{x^2}\text{ d}x=\\
&=-4e^{-3}\left[\frac{x^2}{2}   \right]_0^2+\left[-(x+1)e^{-x}   \right]_0^2+e^{-3}\left[\frac{x^3}{3}   \right]_0^2=-8e^{-3}-3e^{-2}+1+\frac{8}{3}\ e^{-3}=\\
&=1-3e^{-2}-\frac{16}{3}\ e^{-3}
\end{aligned}
\end{gather*}

\item [(d)] Calcolare il coefficiente di correlazione $\rho_{X,Y}$.

Ricordiamo che
\begin{oss}
$\\$Il coefficiente di correlazione lineare si calcola con
\begin{equation*}
\rho_{X,Y}\coloneqq\frac{\Cov(X,Y)}{\sqrt{\Var(X)\cdot \Var(Y)}}=\frac{\EE[XY]-\EE[X]\cdot\EE[Y]}{\sqrt{\Var(X)\cdot \Var(Y)}}
\end{equation*}
\end{oss}

Al volo possiamo già dire che $\EE[X]=2=\Var(X)$ e $\EE[Y]=4=\Var(Y)$. Invece per $\EE[XY]$ sfruttiamo il teorema (\ref{introth8})
\begin{gather*}
\begin{aligned}
\EE[XY]&\overset{\underset{\text{ii)}}{}}{=}\int_{\RR^2}xy\fXYxy \text{ d}x\text{d}y=\int_0^{+\infty}x^2\left(\int_x^{+\infty}y(y-x)e^{-y}\text{ d}y   \right)\text{d}x=\\
&=\int_0^{+\infty}x^2\left(\underbrace{\int_x^{+\infty}y^2e^{-y}\text{ d}y}_{(\beta)}   \right)\text{d}x-\int_0^{+\infty}x^3\left(\underbrace{\int_x^{+\infty}ye^{-y}\text{ d}y}_{(\alpha)}   \right)\text{d}x=\\
&=\int_0^{+\infty}x^2\left[-(y^2+2y+2)e^{-y} \right]_x^{+\infty}\text{d}x-\int_0^{+\infty}x^3\left[-(y+1)e^{-y} \right]_x^{+\infty}\text{d}x=\\
&=\int_0^{+\infty}x^2(x^2+2x+2)e^{-x}\text{ d}x-\int_0^{+\infty}x^3(x+1)e^{-x}\text{ d}x=\\
&=\int_0^{+\infty}x^4e^{-x}   \text{ d}x+2\int_0^{+\infty}x^3e^{-x}   \text{ d}x+2\int_0^{+\infty}x^2e^{-x}   \text{ d}x-\int_0^{+\infty}x^4e^{-x}   \text{ d}x-\int_0^{+\infty}  x^3e^{-x} \text{ d}x=\\
&=\underbrace{\int_0^{+\infty}x^3e^{-x}   \text{ d}x}_{=6\text{ per }(\gamma^1)}+2\underbrace{\int_0^{+\infty}x^2e^{-x}   \text{ d}x}_{=2\text{ per }(\beta^1)}=6+4=10
\end{aligned}
\end{gather*}
E finalmente
\begin{equation*}
\rho_{X,Y}=\frac{10-2\cdot 4}{\sqrt{2\cdot 4}}=\frac{2}{2\sqrt{2}}=\frac{1}{\sqrt{2}}=\frac{\sqrt{2}}{2}
\end{equation*}
\item [(e)] Trovare una diversa densità congiunta avente le stesse densità marginali.

Banalmente basta prendere la densità data dal prodotto delle marginali, cioè
\begin{equation*}
\widetilde{f}_{(X,Y)}(x,y)=xe^{-x}\ \Ind_{[0,+\infty)}(x)\cdot \frac{1}{6}\ y^3e^{-y}\ \Ind_{[0,+\infty)}(y)
\end{equation*}
\end{enumerate}

\Soluzione{}
\begin{enumerate}
\item [(a)] Quali sono le leggi di $X$ e $Y$? Le variabili aleatorie $X$ e $Y$ sono indipendenti?

Per calcolare le due marginali usiamo il teorema (\ref{introth5}). Quindi, fissato $x>0$, si ha
\begin{gather*}
\begin{aligned}
f_X(x)&=\int_\RR\fXYxy\dy=\\
&=\int_{0}^{+\infty} xe^{-(x+y)}\dy=\\
&=\left[-xe^{-(x+y)}  \right]_0^{+\infty}=\\
&=xe^{-x}
\end{aligned}
\end{gather*}
Quindi $f_X(x)=xe^{-x}\ \Ind_{(0,+\infty)}(x)$ e perciò $X\sim\Gamma(2,1)$.

Allo stesso modo fissiamo $y>0$ e calcoliamo
\begin{gather*}
\begin{aligned}
f_Y(y)&=\int_\RR\fXYxy\dy=\\
&=\int_{0}^{+\infty} xe^{-(x+y)}\dx=\\
&=e^{-y}\underbrace{\int_{0}^{+\infty} xe^{-x}\dx}_{=1\text{ per }(\alpha^1)}=\\
&=e^{-y}
\end{aligned}
\end{gather*}
Quindi $f_Y(y)=e^{-y}\ \Ind_{(0,+\infty)}(y)$ e perciò $Y\sim\Gamma(1,1)\sim\Ec(1)$.

Dato che la congiunta fattorizza nelle due marginali, allora per il teorema (\ref{introth6}) le variabili $X$ e $Y$ sono indipendenti.
\item [(b)] Calcolare $\Var(X+Y)$.

Ricordando che $\Var(\Gamma(\alpha,\lambda))=\alpha/\lambda^2$ e $\Var(\Ec(\lambda))=1/\lambda^2$ possiamo facilmente calcolare
\[
\Var(X+Y)\overset{\underset{\indep}{}}{=}\Var(X)+\Var(Y)=2+1=3
\]
\item [(c)] Si calcolino i valori attesi di $U=\min(X,Y)$ e $V=\max(X,Y)$.

Iniziamo da $U$, osservando che per com'è definita si ha $Im(U)\in(0,+\infty)$; allora fissatto $u>0$ abbiamo
\begin{gather*}
\begin{aligned}
F_U(u)&=\PP(U\leq u)=\\
&=\PP(\min(X,Y)\leq u)=         \\
&=1- \PP(\min(X,Y)> u)=        \\
&= 1-\PP(X>u,Y>u)=        \\
&\overset{\underset{\indep}{}}{=}1-\PP(X>u)\ \PP(Y>u)=         \\
&= 1-(1-F_X(u))(1-F_Y(u))        \\
&=F_X(u)+F_Y(u)-F_X(u)\ F_Y(u)         
\end{aligned}
\end{gather*}
Analogamente, fissato $v>0$, esprimiamo anche il $\max$ in funzione delle due funzioni di ripartizione
\begin{gather*}
\begin{aligned}
F_V(v)&=\PP(V\leq v)=\\
&=\PP(\max(X,Y)\leq v)=         \\
&=\PP(X\leq v,Y\leq v)=        \\
&\overset{\underset{\indep}{}}{=}\PP(X\leq v)\ \PP(Y\leq v)=         \\
&=F_X(v)\ F_Y(v)      
\end{aligned}
\end{gather*}
Dobbiamo quindi calcolarci $F_X$ e $F_Y$ (o le copiamo, tanto sono note, oppure integriamo le $f$ ricavate al punto (a))
\begin{gather*}
\begin{aligned}
&F_X(x)=\int_0^x te^{-t}\dt=1-(x+1)e^{-x}\\  
&F_Y(y)=\int_0^y e^{-t}=1-e^{-y}
\end{aligned}
\end{gather*}
Riprendiamo il calcolo di $F_U$ e $F_V$
\begin{gather*}
\begin{aligned}
F_U(u)&=F_X(u)+F_Y(u)-F_X(u)\ F_Y(u)=\\
&=1-(u+1)e^{-u} + 1-e^{-u} -(1-(u+1)e^{-u})(1-e^{-u})=\\ 
&\overset{\underset{\cdots}{}}{=}1-(u+1)e^{-2u} \\
F_V(v)&=F_X(v)\ F_Y(v)=\\
&=(1-(v+1)e^{-v})(1-e^{-v})=\\
&\overset{\underset{\cdots}{}}{=}(v+1)e^{-2v}-(v+2)e^{-v}+1
\end{aligned}
\end{gather*}
Siamo pronti per calcolare i valori attesi di $U$ e $V$. In particolare faremo vedere per completezza due modi diversi per procedere: per il calcolo di $\EE[U]$ deriviamo $F_U$ trovando $f_U$ e poi usiamo la definizione di $\EE$; per il calcolo di $\EE[V]$ usiamo il risultato notevole (\ref{introth9}). Allora:
\begin{gather*}
\begin{aligned}
f_U(u)&=(F_U(u))'=(2u+1)e^{-2u}\ \Ind_{(0,+\infty)}(u) \\
\Longrightarrow\ \EE[U]&=\int_{0}^{+\infty} u\cdot (2u+1)e^{-2u}\du =\\
&\overset{\underset{pp}{}}{=}\left[(2u^2+u)(-\frac{1}{2}e^{-2u})   \right]_0^{+\infty} -\int_{0}^{+\infty} (4u+1)(-\frac{1}{2}e^{-2u})\du \\
&\overset{\underset{pp}{}}{=}0+\frac{1}{2}\left\{\left[(4u+1)(-\frac{1}{2}e^{-2u})  \right]_0^{+\infty}+2\int_{0}^{+\infty} e^{-2u}\du   \right\}=\\
&=\frac{1}{4}+\left[ -\frac{1}{2}e^{-2u} \right]_0^{+\infty}=\\
&=\frac{1}{4}+\frac{1}{2}=\\
&=\frac{3}{4} \\
\EE[V]&=\int_{0}^{+\infty} (1-F_V(v))\dv=\\
&=\int_{0}^{+\infty} (v+2)e^{-v}-(v+1)e^{-2v}\dv=\\
&\overset{\underset{pp}{}}{=}\left[-(v+2)e^{-v}+\frac{1}{2}(v+1)e^{-2v}  \right]_0^{+\infty}+\int_{0}^{+\infty} e^{-v}\dv-\int_{0}^{+\infty} \frac{1}{2}e^{-2v}\dv=\\
&=2-\frac{1}{2}+\left[-e^{-v}  \right]_0^{+\infty}  + \left[\frac{1}{4}e^{-2v}  \right]_0^{+\infty}=\\
&=2-\frac{1}{2}+1-\frac{1}{4}=\\
&=\frac{9}{4}
\end{aligned}
\end{gather*}
\end{enumerate}

\Soluzione{}
Prima di tutto, il testo dice che $(X,Y)\sim\Uc(Q)$ e dunque
\[
\fXYxy=\frac{1}{m(Q)}\ \Ind_Q(x,y)=\frac{1}{8}\ \Ind_Q(x,y)
\]
\fg{0.5}{7_7}
\begin{enumerate}
\item [(a)] Qual è la legge di $X$?

Per calcolare $f_X$ sfruttiamo il teorema (\ref{introth5}):
\[
f_X(x)=\int_\RR\fXYxy\dy
\]
Pe risolvere tale integrale usiamo il teorema di Fubini (\ref{introth1}): vedendo $Q$ come dominio normale rispetto all'asse $x$, ossia compreso tra le funzioni $f_1(x)=|x-2|$ e $f_2(x)=4-|x-2|$, fissato $0\leq x\leq4$, possiamo scrivere
\begin{gather*}
\begin{aligned}
f_X(x)&=\int_{|x-2|}^{4-|x-2|}\frac{1}{8}\dy=\\
&=\left[\frac{1}{8} y  \right]_{|x-2|}^{4-|x-2|}=\\
&=\frac{1}{2}-\frac{1}{4}|x-2|
\end{aligned}
\end{gather*}
Allora
\[
f_X(x)=\left(\frac{1}{4}(2-|x-2|)  \right)\Ind_{[0,4]}(x)
\]

\item [(b)] Quanto vale $\EE[X]$?

Abbiamo
\begin{gather*}
\begin{aligned}
\EE[X]&=\int_{0}^{+\infty} x\ f_X(x)\dx=\\
&=\int_0^4 \frac{1}{4}x(2-|x-2|)\dx=\\
&=\int_0^4 \frac{1}{2}x\dx-\int_0^4 \frac{1}{4}x|x-2|\dx=\\
&=\left[\frac{1}{4}x^2\right]_0^4-\left(\int_0^2 -\frac{1}{4}x(x-2)\dx+\int_2^4 \frac{1}{4}x(x-2)\dx  \right)=\\
&=4+\frac{1}{4}\left[ \frac{x^3}{3}-x^2 \right]_0^2-\frac{1}{4}\left[\frac{x^3}{3}-x^2  \right]_2^4=\\
&=2
\end{aligned}
\end{gather*}

\item [(c)] Quanto vale $\PP(X<Y)$?

L'evento $(X<Y)$ corrisponde all'evento $(Y>X)$, cioè alla zona di piano appartenente a $Q$ e soprastante la bisettrice del primo quadrante:
\fg{0.4}{7_8}
Siccome la probabilità è uniforme, la probabilità di "cadere" in metà quadrato è 
\[
\PP(X<Y)=\frac{1}{2}
\]
oppure
\[
\PP(X<Y)=\frac{m(X<Y)}{m(Q)}=\frac{4}{8}=\frac{1}{2}
\]

\item [(d)] Quanto vale $\PP(X\leq 2,Y\leq 1)$?

Analogamente al punto precedente
\[
\PP(X\leq 2,Y\leq 1)=\frac{m(X\leq 2,Y\leq 1)}{m(Q)}=\frac{1/2}{8}=\frac{1}{16}
\]
\fg{0.4}{7_9}

\item [(e)] $X$ e $Y$ sono indipendenti?

Sulla scia del punto (a), vediamo $Q$ come dominio normale rispetto all'asse $y$, cioè per $|2-y|\leq x\leq 4-|2-y|$. Allora
\begin{gather*}
\begin{aligned}
f_Y(y)&=\int_\RR\fXYxy\dx=\\
&=\int_{|2-y|}^{4-|2-y|}\frac{1}{8}\dx=\\
&=\left[\frac{1}{8} x  \right]_{|2-y|}^{4-|2-y|}
\end{aligned}
\end{gather*}
Quindi
\[
f_Y(y)=\left(\frac{1}{4}(2-|y-2|)  \right)\Ind_{[0,4]}(y)
\]
e come possiamo notare $\fXY\neq f_X\cdot f_Y$ cioè $X$ e $Y$ non sono indipendenti.

\end{enumerate}

\Soluzione{}
Siano
\[
R\sim\Uc([a,b])\quad \text{con }a=11.5,\ b=12.5\qquad\qquad\qquad M=\frac{1}{R}
\]
rispettivamente resistenza e conduttanza di un dato circuito.

\begin{enumerate}
\item [(a)] $R$ e $M$ hanno legge congiunta continua?

Se per assurdo il vettore $(R,M)$ fosse assolutamente continuo allora dovrebbe esistere
\[
f_{(R,M)}:\RR^2\to[0,+\infty)
\]
tale che
\[
\PP((R,M)\in B)=\int_B f_{(R,M)}\dxy\qquad\forall B\in\Bc(\RR^2)
\]
Ma prendendo
\[
\overline{B}=\left\{(x,y)\in\RR^2\ :\ 11.5\leq x\leq 12.5,\ y=\frac{1}{x}  \right\}
\]
incorriamo in una contraddizione:
\begin{gather*} 
\PP((R,M)\in \overline{B})=0\qquad\text{perché l'insieme ha misura di Lebesgue }m_2=0 \\
\PP((R,M)\in \overline{B})=\PP\left(R=\frac{1}{M}\right)=1\qquad\text{proprio per come sono definite }R\text{ e }M
\end{gather*}
Abbiamo quindi sbagliato a presupporre che il vettore $(R,M)$ fosse assolutamente continuo.

\begin{oss} Abbiamo chiarito ancora una volta la non invertibilità del teorema (\ref{introth5-}. \end{oss}

\item [(b-c)] Calcolare i valori attesi di $R$ e $M$. Calcolare la matrice varianza di $(R,M)$.

Dato che $R$ è uniforme:
\[
\EE[R]=\frac{a+b}{2}=12 \qquad \Var(R)=\frac{(b-a)^2}{12}=\frac{1}{12}=0.08333
\]
Invece per $M$:
\begin{gather*}
\begin{aligned}
\EE[M]&=\EE\left[ \frac{1}{R} \right]=\int_\RR \frac{1}{x}\ f_R(x)\dx=\int_\RR \frac{1}{x}\ \frac{1}{b-a}\ \Ind_{[a,b](x)}\dx=\\
&=\int_{a}^{b} \frac{1}{x}\dx=\log\left( \frac{b}{a} \right)=0.08338\\
\EE[M^2]&=\int_\RR \frac{1}{x^2}\ f_R(x)\dx=\int_{a}^{b} \frac{1}{x^2}\dx=6.956\cdot 10^{-3}\\
\implies \Var(M)&=\EE[M^2]-\EE[M]^2=4.029\cdot 10^{-6}
\end{aligned}
\end{gather*}

Per concludere
\begin{gather*}
\begin{aligned}
\Cov(R,M)&=\EE[RM]-\EE[R]\,\EE[M]=1-12\cdot 0.08338 = -5.793\cdot 10^{-4}\\
\Var(R,M)&=
\begin{pmatrix}
0.08338 & -5.793\cdot 10^{-4} \\
-5.793\cdot 10^{-4} & 4.029\cdot 10^{-6} \\
\end{pmatrix}
\end{aligned}
\end{gather*}

\item [(d)] Calcolare il coefficiente di correlazione lineare fra $R$ e $M$.

Dalla definizione di coefficiente di correlazione lineare abbiamo
\[
\rho_{R,M}=\frac{\Cov(R,M)}{\sqrt{\Var(R)}\,\sqrt{\Var(M)}}=\frac{-5.793\cdot 10^{-4}}{\sqrt{0.08338}\,\sqrt{4.029\cdot 10^{-6}}}=-0.9998
\]
Nonostante sia parecchio invintante, NON possiamo approssimare $\rho_{R,M}=-1$, in quanto per il teorema (\ref{introth10}) $R$ e $M$ dovrebbero essere legate da una relazione lineare, cosa che invece non avviene.

\item [(e)] Determinare la distribuzione di $M$.

Come sempre (siamo nel caso uni - dimensionale !!), per rispondere alla domanda si passa per il calcolo della funzione di ripartizione di $M$. E come sempre, prima di procedere è utile capire com'è fatto il supporto di $M$:
\[
S_M=\left\{ t\in\RR\ :\ \frac{1}{b}\leq t\leq\frac{1}{a}  \right\}
\]
Allora
\begin{enumerate}
\item [(i)] se $t<\dfrac{1}{b}$ allora $F_M(t)=0$

\item [(ii)] se $t>\dfrac{1}{a}$ allora $F_M(t)=1$

\item [(iii)] se $\dfrac{1}{b}\leq t\leq\dfrac{1}{a}$ allora
\begin{gather*}
\begin{aligned}
F_M(t)&=\PP(M\leq t)=\PP\left(\frac{1}{R}\leq t\right)=\PP\left(R\geq\frac{1}{t}\right)=1-F_R\left(\frac{1}{t}\right)=\\
&=1-\int_{-\infty}^{1/t}\frac{1}{b-a}\ \Ind_{[a,b](x)}\dx=1-\left(\frac{1}{t}-a\right)=12.5-\frac{1}{t}
\end{aligned}
\end{gather*}
\end{enumerate}
Dunque
\[
F_M(t)=
\begin{cases}
0                               &\text{se }t<\dfrac{1}{b} \\
12.5-\dfrac{1}{t}        &\text{se } \dfrac{1}{b}\leq t\leq \dfrac{1}{a}  \\
1                               &\text{se }t>\dfrac{1}{a}
\end{cases}
\]
Notiamo infine che $F_M\in\Cz\cap\widetilde{\Cc}^1$ e quindi $M$ è variabile aleatoria assolutamente continua con densità
\[
f_M(t)=F_M'(t)=\frac{1}{t^2}\ \Ind_{\left(\frac{1}{b},\frac{1}{a}\right)}(t)
\]

\begin{oss} Avremmo potuto calcolare $f_M$ anche con la formula di Jacobi (\ref{introth3}), infatti:
\begin{itemize}
\item $g(x)=\frac{1}{x}$ iniettiva su $S=(a,b)$ e con immagine $g(S)=\left(\frac{1}{b},\frac{1}{a}\right)$;
\item $g\in\Cu(S)$;
\item $g'(x)=-\frac{1}{x^2}\neq 0\quad \forall x\in S$.
\end{itemize}
Allora
\[
f_M(t)=f_R(g(y))\ |g'(y)|\ \Ind_{\left(\frac{1}{b},\frac{1}{a}\right)}(y)=\frac{1}{y^2}\ \Ind_{\left(\frac{1}{b},\frac{1}{a}\right)}(y)
\]
\end{oss}

\end{enumerate}

\Soluzione{}
Mettiamo a fuoco la situazione: fissiamo due linee parallele a distanza $d$ sul pavimento e immaginiamo che un ago venga lanciato in modo casuale in tale zona di piano (per "periodicità dell'esperimento" questo è sufficiente)
\fg{0.5}{7_11}
Procediamo per passi graduali:
\begin{enumerate}

\item Riformulazione dell'esperimento in termini astratti.

Osserviamo che la posizione di caduta dell'ago è univocamente determinata dal punto in cui cade il suo baricentro e dall'inclinazione dell'ago rispetto alle linee del pavimento. Viene quindi naturale introdurre le variabili aleatorie
\begin{gather*}
\begin{aligned}
X&=\begin{matrix}
\text{"distanza del baricentro dell'ago dalla} \\
\text{linea del pavimento più vicina"}
\end{matrix} \\
\Theta &=\begin{matrix}
\text{"angolo (antiorario) formato dall'ago} \\ 
\text{rispetto alle linee del pavimento"}
\end{matrix}
\end{aligned}
\end{gather*}
\fg{0.5}{7_12}
Possiamo allora supporre
\begin{align*}
X&\sim\Uc\left(\left[0,\frac{d}{2}\right]\right) &\implies f_X(x)&=\frac{2}{d}\ \Ind_{\left[0,\frac{d}{2}\right]}(x) \\
\Theta&\sim\Uc\left(\left[0,\pi\right]\right) &\implies f_\Theta(\vartheta)&=\frac{1}{\pi}\ \Ind_{\left[0,\pi\right]}(\vartheta) \\
X&\indep\Theta&\implies f_{(X,\Theta)}(x,\vartheta)&=\frac{2}{\pi d}\ \Ind_{\left[0,\frac{d}{2}\right]}(x)\ \Ind_{\left[0,\pi\right]}(\vartheta)
\end{align*}

\item Caratterizzazione dell'evento di interesse, cioè ci chiediamo quando l'ago intersechi almeno una linea del pavimento.
\fg{0.5}{7_13}
Osservando il grafico capiamo che ciò si verifica quando
\[
\boxed{X\leq\frac{l}{2}\sin\Theta}
\]
Possono verificarsi due casi:
\begin{enumerate}

\item [(a)] $l\leq d$, cioè può esserci al più un'intersezione

\item [(b)] $l>d$, cioè possono esserci più di un'intersezione

\end{enumerate}

\medskip

\begin{enumerate}

\item [(a)] L'obiettivo è quello di calcolare $\PP\left(\frac{l}{2}\sin\Theta\geq X\right)$.

Detto 
\[
A\coloneqq\left\{(x,\vartheta)\in \left[0,\frac{d}{2}\right]\times \left[0,\pi\right]\ :\ \frac{l}{2}\sin\vartheta\geq x  \right\}
\]
\fg{0.4}{7_14}
abbiamo
\begin{gather*}
\begin{aligned}
\PP(A)&=\int_A f_{(X,\Theta)}(x,\vartheta)\dx\text{d}\vartheta=\\
&=\frac{2}{\pi d}\int_0^\pi \left(\int_0^{\frac{l}{2}\sin\vartheta}1\dx   \right)\text{d}\vartheta=\\
&=\frac{l}{\pi d}\int_0^\pi \sin\vartheta\text{ d}\vartheta=\\
&=\frac{2l}{\pi d}
\end{aligned}
\end{gather*}

\item [(b)] Siamo sempre interessati alla probabilità dell'evento $A$. Tuttavia in questo caso:
\fg{0.6}{7_15}

Denotiamo con $B$ il nuovo insieme di interesse. Allora possiamo vedere $B=B_1\cup B_2\cup B_3=A-V$:
\fg{0.6}{7_16}

Quindi
\begin{gather}
\begin{aligned}
\label{eqn_7_8}
\PP(A)&=\int_A f_{(X,\Theta)}(x,\vartheta)\dx\text{d}\vartheta=\\
&=\frac{2}{\pi d}\int_B 1\dx\text{d}\vartheta=\\
&=\frac{2}{\pi d}\ m_2(B)=\\
&=\frac{2}{\pi d}\ (m_2(A)-m_2(V))=\\
&=\frac{2}{\pi d}\left[\int_0^\pi \left(\int_0^{\frac{l}{2}\sin\vartheta}1\dx \right)\text{d}\vartheta - \int_{\vartheta_1}^{\vartheta_2} \left(\int_{\frac{d}{2}}^{\frac{l}{2}\sin\vartheta}1\dx \right)\text{d}\vartheta  \right]
\end{aligned}
\end{gather}

Calcoliamo separatamente i due integrali

\begin{gather*}
\begin{aligned}
\int_0^\pi \left(\int_0^{\frac{l}{2}\sin\vartheta}1\dx \right)\text{d}\vartheta &=\int_0^\pi \frac{l}{2}\sin\vartheta\text{ d}\vartheta=l \\
\int_{\vartheta_1}^{\vartheta_2} \left(\int_{\frac{d}{2}}^{\frac{l}{2}\sin\vartheta}1\dx \right)\text{d}\vartheta&=\int_{\vartheta_1}^{\vartheta_2} \left(\frac{l}{2}\sin\vartheta-\frac{d}{2}\right)\text{d}\vartheta= \\
&=\left[-\frac{l}{2}\cos\vartheta  \right]_{\vartheta_1}^{\vartheta_2}-\left[\frac{d}{2}\vartheta  \right]_{\vartheta_1}^{\vartheta_2}=\\
&=\frac{l}{2}\cos\vartheta_1-\frac{l}{2}\cos\vartheta_2-\frac{d}{2}\vartheta_2+\frac{d}{2}\vartheta_1 \\
&\overset{\underset{\text{oss}}{}}{=}l\sqrt{1-\left(\frac{d}{l}\right)^2}-\frac{d}{2}\left(\pi-2\arcsin \left(\frac{d}{l}\right) \right)
\end{aligned}
\end{gather*}

\begin{oss}$\\$
Sono state usate le seguenti proprietà trigonometriche
\[
\cos(\arcsin(x))=\sqrt{1-x^2}\qquad \cos(\alpha-\beta)=\cos\alpha\,\cos\beta+\sin\alpha\,\sin\beta
\]
\end{oss}

Riprendendo quindi l'equazione (\ref{eqn_7_8}) otteniamo
\[
\PP(A)=\frac{2}{\pi d}\left(l-l\sqrt{1-\left(\frac{d}{l}\right)^2}+\frac{\pi d}{2}-d\arcsin\left(\frac{d}{l}\right)  \right)=\frac{2l}{\pi d}\left(1-\sqrt{1-\left(\frac{d}{l}\right)^2}\right)+1-\frac{\pi }{2}\arcsin\left(\frac{d}{l}\right)
\]

\end{enumerate}

\end{enumerate}

In conclusione

\[
\PP(A)=
\begin{cases}
\dfrac{2l}{\pi d}    &\text{se }l\leq d \\
\dfrac{2l}{\pi d}\left(1-\sqrt{1-\left(\dfrac{d}{l}\right)^2}\right)+1-\dfrac{\pi }{2}\arcsin\left(\dfrac{d}{l}\right) &\text{se }l> d
\end{cases}
\]

\begin{oss}[Importanza storica dell'esperimento]$\\$
Consideriamo $l\leq d$. Abbiamo trovato che $p=\PP(A)=\dfrac{2l}{\pi d}$ e quindi
\[
\pi=\dfrac{2l}{p d}=2ap^{-1}
\]
con $a\coloneqq$ "rapporto tra lunghezza dell'ago e distanza tra le linee del pavimento".

Interpretando la probabilità in maniera frequentista come $p=f/t$ dove $f$ rappresenta il numero di esperimenti favorevoli (cioè ago interseca linea) e $t$ il numero di esperimenti totali, abbiamo che
\[
p=\displaystyle\lim_{t\to+\infty}\frac{f}{t}
\]
Se il numero di esperimenti è abbastanza grande si può allora ottenere un'approssimazione probabilistica di $\pi$:
\[
\pi=2ap^{-1}=2a\displaystyle\lim_{t\to+\infty}\frac{t}{f}\implies \pi\approx 2a\frac{t}{f}
\]
Si lasciano di seguito alcuni link con simulazioni: \\
\url{https://datagenetics.com/blog/may42015/index.html} \\
\url{https://www.maplesoft.com/support/help/Maple/view.aspx?path=MathApps%2FBuffonsNeedleProblem}
\end{oss}

\Soluzione{}
\begin{enumerate}
\item [(a)] Si scriva la densità di probabilità del vettore aleatorio $(X,Y)$.

Siccome $X,Y$ sono VAR assolutamente continue e indipendenti allora il vettore aleatorio $(X,Y)$ è assolutamente continuo e la sua legge congiunta è
\[
\fXYxy=f_X(x)\cdot f_Y(y)=\Ind_{[0,1]}(x)\cdot \Ind_{[0,1]}(y)=_{[0,1]^2}(x,y)
\]
cioè $(X,Y)\sim\Uc\left([0,1]^2\right)$.

\item [(b)] Si consideri il vettore aleatorio $(U,V)=(X+1,X+Y)$. Calcolarne il valore atteso e la matrice varianza.

Abbiamo
\[
\mathbb{E}\left[\begin{pmatrix}
U \\ V
\end{pmatrix}\right] =
\begin{pmatrix}
\mathbb{E}[U] \\ \mathbb{E}[V]
\end{pmatrix} = 
\begin{pmatrix}
\mathbb{E}[X+1] \\ \mathbb{E}[X+Y]
\end{pmatrix} = 
\begin{pmatrix}
\mathbb{E}[X]+1 \\ \mathbb{E}[X] +\mathbb{E}[Y]
\end{pmatrix} =
\begin{pmatrix}
1/2+1 \\ 1/2+1/2
\end{pmatrix} = 
\begin{pmatrix}
3/2 \\ 1
\end{pmatrix}
\]
\begin{gather*}
\begin{aligned}
\Var(U)&=\Var(X+1)=\Var(X)=1/12 \\
\Var(V)&=\Var(X+Y)=\Var(X)+\Var(Y)+\underbrace{2\Cov(X,Y)}_{0\text{ perché }X\indep Y}=1/6 \\
\Cov(U,V)&=\Cov(X+1,X+Y)=\\
&=\underbrace{\Cov(X,X)}_{\Var(X)}+\underbrace{\Cov(X,Y)}_{0}+\underbrace{\Cov(1,X)}_{0}+\underbrace{\Cov(1,Y)}_{0}=\Var(X)=1/12 \\
\Var(U,V)&=\begin{pmatrix}
 \dfrac{1}{12}& \dfrac{1}{12} \\
\, & \, \\
\dfrac{1}{12} &\dfrac{1}{6}  \\
\end{pmatrix}  
\end{aligned}
\end{gather*}
In alternativa si poteva usare il teorema (\ref{introth13}), dopo aver introdotto l'opportuna trasformazione affine $(U,V)=A\,(X,Y)+b$. Introdurre tale trasformazione non è utile solo per calcolare valore atteso e matrice varianza, ma anche per ricavare la legge del vettore trasformato. Vediamo nel prossimo punto come fare.

\item [(c)] Se ne calcoli la legge.

Vediamo il vettore $(U,V)$ come
\[
\begin{pmatrix}
U \\ V
\end{pmatrix}  
=
g \begin{pmatrix}
X \\ Y
\end{pmatrix} 
=
\underbrace{\begin{pmatrix}
1 & 0 \\
1 & 1 \\
\end{pmatrix}}_{A}\begin{pmatrix}
X \\ Y
\end{pmatrix}+\underbrace{\begin{pmatrix}
1 \\0
\end{pmatrix}}_{b} 
\]
In tal caso $g:\RR^2\to\RR^2$ è iniettiva, $\Cu$ e ha jacobiano $J_g=A$ con $\det A=1\neq 0$.

\begin{oss}$\\$
Se nel caso 1D $(mx+q)'=m$ nel caso $n$D $(AX+b)'=J_{Ax+b}=A$.
\end{oss}

Possiamo quindi applicare la formula di Jacobi (\ref{introth4}) con $g^{-1}:g(\RR^2)\to\RR^2$ tale che
\begin{gather*}
\begin{aligned}
f_{(U,V)}(u,v)&=\fXY(g^{-1}(u,v))\cdot |\det J_{g^{-1}}(u,v)|=\\
&=\Ind_{[0,1]^2}(g^{-1}(u,v))\cdot\frac{1}{|\det J_g(g^{-1}(u,v))|}=\\
&=\Ind_{g([0,1]^2)}(u,v)\cdot\frac{1}{|\det A|}=\\
&=\Ind_Q(u,v)
\end{aligned}
\end{gather*}
\fg{0.6}{7_19}
Quindi $(U,V)\sim\Uc(Q)$.

\item [(d)] Il vettore ha componenti indipendenti?

Confrontando $f_{(U,V)}$ con $f_U$ e $f_V$ capiamo che $f_{(U,V)}\neq f_U\cdot f_V$ quindi $U$ e $V$ non sono indipendenti.

\begin{oss}Del resto, quando l'insieme di arrivo ha linee di perimetro oblique le due variabili non possono essere indipendenti, perché tali linee avranno forma $y=mx+q$ e quindi apppunto $y$ dipende da $x$ e viceversa. 
\end{oss}

\item [(e)] Si consideri ora il vettore aleatorio $(U,V)=(X+1,X+Y-XY+X^2Y)$. Calcolarne il valore atteso.

Abbiamo
\begin{gather*}
\begin{aligned}
\EE[U]&=\EE[X+1]=3/2 \\
\EE[V]&=\EE[X]+\EE[Y]-\EE[X]\,\EE[Y]+\EE[X^2]\,\EE[Y]=11/12 \\
\implies \EE[(U,V)]&=(3/2,11/12)
\end{aligned}
\end{gather*}

\item [(f)] Se ne calcoli la legge.

Vediamo
\[
\begin{pmatrix}
U \\ V
\end{pmatrix}  
=
g \begin{pmatrix}
X \\ Y
\end{pmatrix}
=
\begin{pmatrix}
X+1 \\ X+Y-XY+X^2Y
\end{pmatrix}
\]
quindi 
\begin{itemize}
\item $g$ iniettiva;
\item $g\in\Cu$;
\item determinante dello jacobiano
$$
\det J_g(x,y)=\det \begin{pmatrix}
\dfrac{\partial (x+1)}{\partial x} &\dfrac{\partial (x+1)}{\partial y}  \\
\, & \, \\
\dfrac{\partial (x+y-xy+x^2y)}{\partial x}   &  \dfrac{\partial (x+y-xy+x^2y)}{\partial y}  \\
\end{pmatrix}
= \det
\begin{pmatrix}
1 & 0 \\
\, &\, \\
1-y+2xy & 1-x+x^2 \\
\end{pmatrix} 
$$
diverso da zero $\forall (x,y)\in S=[0,1]^2$.
\end{itemize}
La trasformazione $g$ soddisfa quindi le ipotesi del teorema di Jacobi (\ref{introth4}), e possiamo allora cercare $g^{-1}(u,v)$:
\[
\begin{cases} u=x+1 \\ v=x+y-xy+x^2y  \end{cases}\implies \begin{cases} x=u-1 \\ y=\frac{v-x}{1-x+x^2}\end{cases}\implies g^{-1}(u,v)=\begin{pmatrix}
u-1 \\ \, \\\dfrac{v-u+1}{u^2-3u+3}
\end{pmatrix}
\]
Allora
\begin{gather*}
\begin{aligned}
f_{(U,V)}(u,v)&=\fXY(g^{-1}(u,v))\cdot |\det J_{g^{-1}}(u,v)|=\\
&=\Ind_{[0,1]^2}(g^{-1}(u,v))\cdot\frac{1}{|\det J_g(g^{-1}(u,v))|}=\\
&=\Ind_{g([0,1]^2)}(u,v)\cdot\frac{1}{1-(u-1)+(u-1)^2}=\\
&=\Ind_{g(S)}(u,v)\cdot\frac{1}{u^2-3u+3}
\end{aligned}
\end{gather*}
Ora manca da calcolare $g(S)$:
\begin{gather*}
g(S)=g(\{(x,y)\in\RR^2\ :\ 0\leq x,y\leq 1 \}) \\
\implies \begin{cases}0\leq x\leq 1\implies 0\leq u-1\leq 1\implies 1\leq u \leq 2 \\ 0\leq y\leq 1\implies 0\leq \frac{v-u+1}{u^2-3u+3} \leq 1 \implies u-1\leq v\leq u^2-2u+2   \end{cases} \\
\implies g(S)=\{(u,v)\in\RR^2\ :\ 1\leq u \leq 2,\ u-1\leq v\leq u^2-2u+2 \}=Q\\
\, \\
\implies f_{(U,V)}(u,v)=\Ind_Q(u,v)\ \frac{1}{u^2-3u+3}
\end{gather*}
\fg{0.6}{7_20}

\item [(g)] Il vettore ha componenti indipendenti?

Per quanto detto al punto (d), a maggior ragione in questo caso, è impossibile che $U$ e $V$ siano indipendenti.

\end{enumerate}

\Soluzione{}
Manca

\Soluzione{}
Manca

\Soluzione{}
Siano 
\begin{align*}
E_1&\sim\Ec(\lambda)=\Ec(0.3) \\
E_2&\sim\Ec(\mu)=\Ec(0.1) \\
E_3&\sim\Ec(\gamma)=\Ec(0.2)
\end{align*}
Sia $T$ la variabile aleatoria che modellizza il tempo di vita del componente elettronico formato dal collegamento in serie dei tre sottocomponenti indipendenti che hanno un tempo di vita modellizzato rispettivamente da $E_1,E_2,E_3$.

\begin{enumerate}
\item [(a)] Qual è la legge di $T$?

Dato che gli elementi sono collegati in serie, l'apparecchio si guasta non appena se ne guasta uno, quindi
\[
T=\min\{E_1,E_2,E_3 \}
\]
Possiamo anche osservare che, poiché $E_i\geq 0$ q.c. $\forall i = 1:3$, si ha $T\geq 0$ q.c. Allora con l'obiettivo di determinare la legge di $T$, fissiamo un $t\geq 0$ e calcoliamo la sua funzione di ripartizione:
\begin{gather*}
\begin{aligned}
F_T(t)&=\PP(T\leq t)=1-\PP(T>t)= \\
&=1-\PP(\min\{E_1,E_2,E_3 \}>t)=\\
&=1-\PP(E_1>t,E_2>t,E_3>t)=\\
&\overset{\underset{\indep}{}}{=}1-\PP(E_1>t)\,\PP(E_2>t)\,\PP(E_3>t)=\\
&=1-(1-F_{E_1}(t))\,(1-F_{E_2}(t))\,(1-F_{E_3}(t))
\end{aligned}
\end{gather*}
Ricordando che
\begin{oss}$\\$
Se $X\sim\Ec(\lambda)$ allora
\[
F_X(x)=\left(1-e^{-\lambda x}\right)\Ind_{[0,+\infty)}(x)
\]
\end{oss}
otteniamo
\begin{gather*}
F_T(t)=1-\left(e^{-\lambda t}\right)\,\left(e^{-\mu t}\right)\,\left(e^{-\gamma t}\right)=1-e^{-(\lambda+\mu+\gamma)t} \\
\implies F_T(t)=\left(1-e^{-(\lambda+\mu+\gamma)t} \right)\Ind_{[0,+\infty)}(t) \\
\implies T\sim\Ec(\lambda+\mu+\gamma)=\Ec(0.6)
\end{gather*}

\item [(b)] Per aumentare l'affidabilità e ridurre gli interventi di sostituzione, viene proprosto di aggiungere un componente identico in parallelo. Qual è la legge del tempo di vita $S$ del nuovo complesso?

Siano $C_1$ e $C_2$ i due componenti identici collegati in parallelo che formano il componente complessivo $C_{\text{TOT}}$, rispettivamente di vita $T,P,S$. Sappiamo che $T\sim\Ec(0.6)\sim P$ e per costruzione è lecito supporre $T\indep P$.

Dato che $C_1$ e $C_2$ sono collegati in parallelo, $C_{\text{TOT}}$ si guasterà quando $C_1$ e $C_2$ saranno entrambi guasti, perciò
\[
S=\max\{T,P\}
\]
Analogamente al punto precedente, fissato $s\geq 0$ si ha:
\begin{gather*}
\begin{aligned}
F_S(s)&=\PP(S\leq s)=\\
&=\PP(\max\{T,P \}\leq s)=\\
&=\PP(T\leq s,P\leq s)=\\
&\overset{\underset{\indep}{}}{=}\PP(T\leq s)\,\PP(P\leq s)=\\
&=F_T(s)\,F_P(s)=\\
&=\left(1-e^{-0.6s}  \right)^2\\
\implies F_S(s)&=\left(1-e^{-0.6s}  \right)^2\Ind_{[0,+\infty)}(s)
\end{aligned}
\end{gather*}
Essendo $\Cc^\infty$, possiamo calcolare
\[
f_S(s)=F_S'(s)=\left(1.2\left(1-e^{-0.6s}  \right)e^{-0.6s}  \right)\Ind_{[0,+\infty)}(s)
\]
\end{enumerate}

\begin{oss}$\\$
Quanto abbiamo quadagnato rispetto al punto (a)?
Prima avevamo
\[
\EE[T]=\frac{1}{0.6}=1.6667
\]
Ora
\[
\EE[S]=\int_{0}^{+\infty} \left(1.2\left(1-e^{-0.6s}  \right)e^{-0.6s}  \right)\ds=2.5
\]
Fare una sostituzione ogni 30 mesi rispetto a farla ogni 20 mesi può essere molto più conveniente!
\end{oss}

\Soluzione{}
\begin{enumerate}
\item [(a)] Si calcoli la funzione di ripartiione di $Z$ e si determini la densità.

Per definizione
\[
F_Z(z)=\PP(Z\leq z)=\PP(X+Y\leq z)
\]
Come sempre, vediamo l'insieme $(X+Y\leq z)$ in funzione del vettore $(X,Y)$, cioè
\[
\PP(X+Y\leq z)=\PP((X,Y)\in R_Z)
\]
con 
\[
R_Z=\{(x,y)\in\RR^2\ :\ -\infty\leq x\leq z-y,\ -\infty\leq y\leq +\infty  \}
\]
Allora
\begin{gather*}
\begin{aligned}
F_Z(z)&=\PP((X,Y)\in R_Z)=\\
&=\int_{R_Z}\fXYxy\dxy =\\
&=\int_{-\infty}^{+\infty} \left(\int_{-\infty}^{z-y}\fXYxy\dx   \right)\dy
\end{aligned}
\end{gather*}
Arrivati a questo punto è conveniente applicare un cambio di variabili di integrazioni (per l'integrale interno): poniamo $w\coloneqq x+y$ in modo tale che l'integrale che va da $-\infty$ a $z-y$ diventi della sola funzione della $z$, dato che dovremo appunto ricavarci la densità di $Z$. Così facendo, per $x\to z-y$ si ha $w\to z$, mentre $\dx=\dw$, e possiamo quindi scambiare l'integrale più esterno con quello più interno:
\begin{gather*}
\begin{aligned}
F_Z(z)&=\int_{-\infty}^{+\infty} \left(\int_{-\infty}^{z-y}\fXYxy\dx   \right)\dy=\\
&=\int_{-\infty}^{+\infty} \left(\int_{-\infty}^{z}\fXY(w-y,y)\dw   \right)\dy=\\
&=\int_{-\infty}^{z}\left(\int_{-\infty}^{+\infty} \fXY(w-y,y)\dy    \right)\dw
\end{aligned}
\end{gather*}
Questo ci dice che $Z$ è una variabile aleatoria assolutamente continua di densità pari a $F_Z'(x)$, cioè l'integranda dell'integrale sopra valutata in $z$:
\begin{gather*}
\begin{aligned}
f_Z(z)&=F_Z'(x)=\\
&=\int_{-\infty}^{+\infty} \fXY(z-y,y)\dy
\end{aligned}
\end{gather*}
E con questo si conclude la dimostrazione.
\fg{0.5}{9_4}

\item [(b)] Si calcoli la legge di $(Z,Y)$ e si trovi la legge marginale di $Z$.

Il vettore $(Z,Y)$ può essere visto come una trasformazione affine del vettore $(X,Y)$, cioè
\[
\begin{pmatrix}
Z \\ Y
\end{pmatrix} =\begin{pmatrix}
X+Y \\ Y
\end{pmatrix} = \underbrace{\begin{pmatrix}
1 & 1 \\
0 &  1\\
\end{pmatrix}}_{A}\begin{pmatrix}
X \\ Y
\end{pmatrix}
\]
Allora la funzione
\[
g\begin{pmatrix}
x \\ y
\end{pmatrix}=A\begin{pmatrix}
x \\ y
\end{pmatrix}
\]
è $\Cu$, è biiettiva perché $\det(A)=1\neq0$ ed ha jacobiano $J_g(x,y)=\det(A)=1\neq0$; in particolare è ben definita l'inversa di $g$
\[
g^{-1}\begin{pmatrix}
u \\ v
\end{pmatrix} =A^{-1}\begin{pmatrix}
u \\ v
\end{pmatrix} = \begin{pmatrix}
1 & -1 \\
0 &  1\\
\end{pmatrix}\begin{pmatrix}
u \\ v
\end{pmatrix}=\begin{pmatrix}
u-v \\ v
\end{pmatrix}
\]
Allora grazie alla formula di Jacobi (\ref{introth4}) possiamo calcolare la congiunta
\[
f_{(Z,Y)}(z,y)=\fXY(g^{-1}(z,y))\ \frac{1}{\underbrace{|\det J_g(z,y)|}_{=1\ \forall z,y}}=\fXY(z-y,y)
\]
e dedurne, grazie al teorema (\ref{introth5}), che
\[
f_Z(z)=\int_{-\infty}^{+\infty} \fXY(z-y,y)\dy
\]
E con questo si conclude la dimostrazione.
\end{enumerate}

\Soluzione{}
\begin{enumerate}
\item [(a)] Determinare media, varianza e legge di $Z=X+Y$.

Abbiamo
\[
\EE[Z]=\EE[X]+\EE[Y]=\frac{1}{2}+\frac{1}{2}=1\qquad \Var(Z)\overset{\underset{\indep}{}}{=}\Var(X)+\Var(Z)=\frac{1}{12}+\frac{1}{12}=\frac{1}{6}
\]
Per calcolare la legge di $Z$ usiamo il risultato notevole dell'esercizio precedente: $Z$ è variabile aleatoria assolutamente conitnua con densità
\[
f_Z(z)=\int_\RR \fXY(z-y,y)\dy
\]
Osserviamo che
\begin{itemize}
\item $\fXYxy\overset{\underset{\indep}{}}{=}f_X(x)\cdot f_Y(y)\overset{\underset{\Uc}{}}{=}\Ind_{[0,1]^2}(x,y)$
\item $Z\in[0,2]$ q.c.
\end{itemize}
Allora
\[
f_Z(z)=
\begin{cases}
\int_\RR\Ind_{[0,1]^2}(z-y,y)\dy &z\in[0,2] \\
0 &\text{altrove}
\end{cases}
\]
Vediamo com'è fatta $\Ind_{[0,1]^2}(z-y,y)$: per definizione si ha
\[
\Ind_{[0,1]^2}(z-y,y)=
\begin{cases}
1 &(z-y,y)\in[0,1]^2 \\
0 &\text{altrove}
\end{cases}
\]
E $(z-y,y)\in[0,1]^2$ cosa vuol dire? Vuol dire $z-y\in[0,1]$ e $y\in[0,1]$, quindi
\[
\begin{cases}
0\leq y\leq 1 \\
0\leq z-y \leq 1
\end{cases}
\implies
\begin{cases}
0\leq y\leq 1 \\
-z\leq -y \leq 1-z
\end{cases}
\implies
\begin{cases}
0\leq y\leq 1 \\
z-1\leq y \leq z
\end{cases}
\]
Allora detto 
\[
R=\{(z,y)\in\RR^2\ :\ z\in[0,2],\ 0\leq y\leq 1,\ z-i\leq y\leq z  \}
\]

\fg{0.5}{7_21}

si ha che
\[
\Ind_{[0,1]^2}(z-y,y)=\Ind_R(z,y)
\]
e quindi
\[
f_Z(z)=
\begin{cases}
\int_\RR\Ind_R(z,y)\dy &z\in[0,2] \\
0 &\text{altrove}
\end{cases}
=
\begin{cases}
\int_0^z\dy=z &z\in[0,1] \\
\int_{z-1}^1\dy=2-z  &z\in[1,2] \\
0 &\text{altrove}
\end{cases}
\]
che volendo si può riscrivere in
\[
f_Z(z)=
\begin{cases}
\int_{\max(0,z-1)}^{\min(z,1)}\dy=\min(z,1)-\max(0,z-1)=\min(z,2-z) &z\in[0,2] \\
0 &\text{altrove}
\end{cases}
\]

\fg{0.5}{7_22}

\begin{oss} Tale legge vedremo nel capitolo successivo che è il risultato della convoluzione delle due caratteristiche. \end{oss}

\item [(b$^*$)] Si trovi la legge di $T=X+Y-\Ind_{\{ X+Y-1 \}}$.

Siamo nel caso 1D, perciò per trovare la legge di $T$ passiamo per il calcolo della funzione di ripartizione. E come sempre, prima di farlo, cerchiamo di capire com'è fatto il supporto di $T$: $X+Y\in[0,2]$ q.c. allora
\begin{itemize}
\item se $X+Y\in[0,1]$ allora $\Ind_{\{ X+Y-1 \}}=0$ e quindi $T\in[0,1]$
\item se $X+Y\in[1,2]$ allora $\Ind_{\{ X+Y-1 \}}=1$ e quindi $T\in[1,2]-1=[0,1]$
\end{itemize}
Deduciamo che $T\in[0,1]$ q.c.

Allora, fissato $t\in[0,1]$, la funzione di ripartizione è
\begin{gather*}
\begin{aligned}
F_T(t)&=\PP\left(X+Y-\Ind_{\{ X+Y-1 \}}\leq t  \right)=\\
&=\PP(X+Y-1\leq t,X+Y>1)+\PP(X+Y\leq t,X+Y\leq 1)=\\
&=\PP(1<X+Y\leq t+1)+\PP(X+Y\leq t)=\\
&=\int_1^{t+1}2-z\dz+\int_0^tz\dz=\\
&=t
\end{aligned}
\end{gather*}

Quindi
\[
F_T(t)=
\begin{cases}
0 &t<0 \\
t &t\in[0,1] \\
1 &t>1
\end{cases}
\implies T\sim\Uc([0,1])
\]

\end{enumerate}

\Soluzione{}
\begin{enumerate}
\item [(a)] Determinare la distribuzione congiunta del vettore $(U,V)$ ove $U=XY$ e $V=\dfrac{Y}{X}$.

Vediamo
\[
(U,V)=g(X,Y)=(XY,Y/X)
\]
con $g:S=(0,+\infty)^2\to(0,+\infty)^2$. Possiamo dire che
\begin{itemize}
\item $g$ è iniettiva in $S$
\item $g\in\Cu(S)$
\item $\det J_g(x,y)\neq 0\ \ \forall (x,y)\in S$ perché
\[
J_g(x,y)=
\begin{bmatrix}
y & x \\
-y/x^2 & 1/x \\
\end{bmatrix}
\implies \det J_g(x,y)=2y/x
\]
\end{itemize}
Cerchiamo dunque, se esiste, $g^{-1}$: presi $(u,v)\in(0,+\infty)^2$ si ha
\[
\begin{cases}u=xy \\ v=y/x \end{cases}
\implies
\begin{cases}x=u/y \\ y=vx\end{cases} 
\implies
\begin{cases}x=u/y \\ y^2=uv\end{cases} 
\implies
\begin{cases}x=u/y \\ y=\sqrt{uv} \end{cases} 
\implies
\begin{cases}x=\sqrt{u/v} \\ y=\sqrt{uv} \end{cases}
\]
Quindi $g^{-1}$ è ben definita, cioè $g^{-1}:g(S)=(0,+\infty)^2\to S=(0,+\infty)^2$ con $g^{-1}(u,v)=(\sqrt{u/v}, \sqrt{uv})$.

Allora grazie alla formula di Jacobi (\ref{introth4}) la congiunta vale
\begin{gather*}
\begin{aligned}
f_{(U,V)}(u,v)&=\fXY(g^{-1}(u,v))\cdot |\det J_{g^{-1}}(u,v)|=\\
&=\Ind_{[0,1]^2}(g^{-1}(u,v))\cdot\frac{1}{|\det J_g(g^{-1}(u,v))|}=\\
&=\Ind_{[0,1]^2}(\sqrt{u/v},\sqrt{uv})\cdot\frac{1}{2\sqrt{uv}/\sqrt{u/v}}=\\
&=\Ind_{[0,1]^2}(\sqrt{u/v},\sqrt{uv})\cdot\frac{1}{2v}
\end{aligned}
\end{gather*}
Vediamo com'è fatta $\Ind_{[0,1]^2}(\sqrt{u/v},\sqrt{uv})$:
\[
\begin{cases}0\leq\sqrt{u/v}\leq 1  \\ 0\leq \sqrt{uv}\leq 1 \end{cases}
\implies
\begin{cases}0\leq\sqrt{u}\leq \sqrt{v}  \\ 0\leq \sqrt{uv}\leq 1 \end{cases}
\implies
\begin{cases}0\leq u\leq v  \\ u\leq v\leq 1/u \end{cases}
\]
Quindi detto $I=\{(u,v)\in(0,+\infty)^2\ :\ u\in[0,1],\ v\in[u,1/u]  \}$ si ha
\[
f_{(U,V)}(u,v)=\frac{1}{2v}\ \Ind_I(u,v)=\frac{1}{2v}\ \Ind_{[0,1]}(u)\ \Ind_{[u,1/u]}(v)
\]

\item [(b)] Determinare le marginali di $U$ e $V$.

Usando il teorema (\ref{introth5}) abbiamo
\begin{gather*}
\begin{aligned}
u\in[0,1]&\implies f_{U}(u)=\int_{-\infty}^{+\infty} f_{(U,V)}(u,v)\dv=\int_u^{1/u}\frac{1}{2v}\dv=-\log u \\
&\implies  f_{U}(u)=-\log u \ \Ind_{[0,1]}(u) \\
v\in[0,1]&\implies f_V(v)=\int_{-\infty}^{+\infty} f_{(U,V)}(u,v)\du=\int_0^v \frac{1}{2v}\du=\frac{1}{2} \\
v\in[1,+\infty)&\implies f_V(v)=\int_{-\infty}^{+\infty} f_{(U,V)}(u,v)\du=\int_1^{1/v} \frac{1}{2v}\du=\frac{1}{2v^2} \\
&\implies f_V(v)=\frac{1}{2}\ \Ind_{[0,1]}(v)+\frac{1}{2v^2}\ \Ind_{[1,+\infty)}(v)
\end{aligned}
\end{gather*}

\end{enumerate}

\Soluzione{}
Dato che $X$ e $Y$ sono $>0$ q.c. allora $U>0$ q.c. 

Sia quindi $t>0$. Abbiamo
\begin{gather*}
\begin{aligned}
F_U(t)&=\PP(U\leq t)=\PP(X/Y\leq t)=\PP(Y\geq X/t)=\PP((X,Y)\in R)=\\
&=\left\{ R=\{(x,y)\in\RR^2\ :\ y\geq 0,\ 0\leq x\leq ty \} \right\}=\\
&=\int_R \fXYxy\dxy\overset{\underset{\indep}{}}{=}\int_R f_X(x)\cdot f_Y(y)\dxy=\\
&=\int_0^{+\infty}\left(\int_0^{ty}\frac{\lambda^{\alpha+\beta}}{\Gamma(\alpha)\,\Gamma(\beta)}\ x^{\alpha-1}\ y^{\beta-1}\ e^{-\lambda(x+y)}\dx\right)\dy
\end{aligned}
\end{gather*}
Per risolvere tale integrale procediamo in maniera analoga al punto (a) dell'esercizio 13, cioè "facciamo saltar fuori" la funzione di ripartizione di $t$ tramite un cambio di variabili: ponendo $z=x/y$ abbiamo
\begin{gather*}
\begin{aligned}
F_U(t)&=\int_0^{+\infty}\left(\int_0^{t}\frac{\lambda^{\alpha+\beta}}{\Gamma(\alpha)\,\Gamma(\beta)}\ (yz)^{\alpha-1}\ y^{\beta-1}\ e^{-\lambda(yz+y)}\ y\dz\right)\dy= \\
&=\int_0^t\left (\int_0^{+\infty}\frac{\lambda^{\alpha+\beta}}{\Gamma(\alpha)\,\Gamma(\beta)}\ y^{\alpha+\beta-1}\ z^{\alpha-1}\ e^{-\lambda(z+1)y}\dy\right)\dz
\end{aligned}
\end{gather*}
Moltiplicando e dividendo per un'opportuna quantità, ci riconduciamo alla densità di una $\Gamma(\alpha+\beta,\lambda(z+1))$ per risolvere l'integrale interno:
\begin{gather*}
\begin{aligned}
F_U(t)&=\int_0^t z^{\alpha-1}\ \frac{\Gamma(\alpha+\beta)}{(z+1)^{\alpha+\beta)}\,\Gamma(\alpha)\,\Gamma(\beta)}\ \left (\underbrace{\int_0^{+\infty}\frac{(\lambda(z+1))^{\alpha+\beta}}{\Gamma(\alpha+\beta)}\ y^{(\alpha+\beta)-1}\ e^{-(\lambda(z+1))y}\dy}_{=1}\right)\dz=\\
&=\frac{\Gamma(\alpha+\beta)}{\Gamma(\alpha)\,\Gamma(\beta)}\int_0^t\frac{z^{\alpha-1}}{(z+1)^{\alpha+\beta}}\dz
\end{aligned}
\end{gather*}
E quindi
\[
f_U(t)=\frac{\Gamma(\alpha+\beta)}{\Gamma(\alpha)\,\Gamma(\beta)}\ \frac{t^{\alpha-1}}{(t+1)^{\alpha+\beta}}\ \Ind_{(0,+\infty)}(t)
\]

\Soluzione{}
Manca

\Soluzione{}
Manca

\Soluzione{}
\begin{enumerate}
\item [(a)] Si calcoli la media di $(X+Y)^{-1}$.

Abbiamo
\begin{gather*}
\begin{aligned}
\EE\left[ \frac{1}{X+Y} \right]&=\int_{\RR^2}\frac{1}{x+y}\ \fXYxy \dxy=\\
&=\int_{\RR^2}\frac{1}{2}e^{-(x+y)}\dxy=\\
&=\frac{1}{2}\int_{0}^{+\infty} e^{-x}\dx\int_{0}^{+\infty} e^{-y}\dy=\frac{1}{2}
\end{aligned}
\end{gather*}

\item [(b)] Si determini la legge di $X+Y$.

Usando il risultato notevole dell'esercizio 13 (\ref{introth14}) abbiamo che $Z=X+Y$ è variabile aleatoria assolutamente continua con densità, per $z>0$, data da
\begin{gather*}
\begin{aligned}
f_Z(z)&=\int_\RR\fXY(z-y,y)\dy=\\
&=\int_0^z\fXY(z-y,y)\dy=\\
&=\int_0^z\frac{1}{2}ze^{-z}\dy=\frac{1}{2}z^2e^{-z}
\end{aligned}
\end{gather*}
Quindi $Z\sim\Gamma(3,1)$.

\item [(c)] Si calcoli la media di $X+Y$.

Il valore atteso di una $\Gamma$ è noto, perciò $\EE[X+Y]=3/1=3$.

\item [(d)] Si calcolino le marginali di $X,Y$. $X\indep Y$?

Grazie al teorema (\ref{introth5}):
\begin{gather*}
\begin{aligned}
f_X(x)&=\int_\RR \fXYxy\dy=\\
&=\int_{0}^{+\infty} \frac{1}{2}(x+y)e^{-(x+y)}\dy=\\
&=\frac{1}{2}xe^{-x}\underbrace{\int_{0}^{+\infty} e^{-y}\dy}_{1}+\frac{1}{2}e^{-x}\underbrace{\int_{0}^{+\infty} ye^{-y}\dy}_{1\text{ per }\left(\alpha^1\right)}=\frac{1}{2}(x+1)e^{-x}
\end{aligned}
\end{gather*}
Per simmetria $f_Y(y)=\dfrac{1}{2}(y+1)e^{-y}$, da cui segue che $X\not\indep Y$.

\item [(e)] Si calcoli $\Cov(X,Y)$.

Abbiamo
\begin{gather*}
\begin{aligned}
\EE[X]&=\int_{0}^{+\infty} \frac{1}{2}x(x+1)e^{-x}\dx=\frac{3}{2}=\EE[Y] \\
\EE[XY]&=\int_{0}^{+\infty} \frac{1}{2}xy(x+y)e^{-(x+y)}\dxy=2
\implies \Cov(X,Y)&=\EE[XY]-\EE[X]\,\EE[Y]=2-\frac{9}{4}-\frac{1}{4}
\end{aligned}
\end{gather*}

\item [(f)] Si calcoli $\PP(X\geq 1,Y\geq 2)$.

Abbiamo
\begin{gather*}
\begin{aligned}
\PP(X\geq 1,Y\geq 2)&=\int_1^{+\infty}\int_2^{+\infty} \frac{1}{2}(x+y)e^{-(x+y)}\dxy=\\
&=\int_1^{+\infty}\left[-\frac{1}{2}(x+y)e^{-(x+y)}-\frac{1}{2}e^{-(x+y)}  \right]_2^{+\infty}\dx=\\
&=\frac{5}{2}e^{-3}\approx 0.1245
\end{aligned}
\end{gather*}

\end{enumerate}

\Soluzione{}
Manca

\Soluzione{}
\begin{enumerate}
\item [(a$^*$)] Si mostri che $Y$ è una variabile aleatoria.

Manca % manca

\item [(b)] Qual è la legge di $Y$?

Con l'obiettivo di determinare la legge di $Y$, fissiamo $y\geq 0$ perché per costruzione $Y\geq 0$, e calcoliamo la sua funzione di ripartizione:
\[
F_Y(y)=\PP(Y\leq y)=1-\PP(Y>y)
\]
Concentriamoci su $\PP(Y>y)$:
\begin{gather*}
\begin{aligned}
\PP(Y>y)&=\PP\left(\min\{X_1,\dots,X_N  \}>y  \right)=\\
&=\PP\left(X_1>y,\dots,X_N>y,\bigcup_{n=1}^\infty (N=n)  \right)=\\
&=\PP\left(\bigcup_{n=1}^\infty(X_1>y,\dots,X_n>y, N=n)  \right)=\\
&=\sum_{n=1}^\infty \PP(X_1>y,\dots,X_n>y, N=n)=\\
&\overset{\underset{\indep}{}}{=}\sum_{n=1}^\infty \PP(X_1>y)\cdots\PP(X_n>y)\ \PP(N=n)=\\
&=\sum_{n=1}^\infty (1-F_{X_1}(y))\cdots (1-F_{X_n}(y))\ \PP(N=n)=\\
&=\sum_{n=1}^\infty e^{-n\lambda y}\ p(1-p)^{n-1}=\\
&=\sum_{n=0}^\infty e^{-(n+1)\lambda y}\ p(1-p)^{n}=\\
&=pe^{-\lambda y} \underbrace{\sum_{n=0}^\infty \left((1-p)e^{-\lambda y}  \right)^n}_{\text{serie geom.}}=\\
&=\frac{pe^{-\lambda y}}{1-(1-p)e^{-\lambda y}}
\end{aligned}
\end{gather*}
Quindi
\[
F_Y(y)=\left(1-\frac{pe^{-\lambda y}}{1-(1-p)e^{-\lambda y}}  \right)\Ind_{[0,+\infty)}(y)=\frac{1-e^{-\lambda y}}{1-(1-p)e^{-\lambda y}}\ \Ind_{[0,+\infty)}(y)
\]

\item [(c)] Qual è il valor atteso di $Y$? 

Dato che $Y$ è una variabile aleatoria continua e positiva possiamo usare il risultato notevole (\ref{introth9}):
\begin{gather*}
\begin{aligned}
\EE[Y]&=\int_0^{+\infty}(1-F_Y(y))\dy=\\
&=\int_{0}^{+\infty} \frac{pe^{-\lambda y}}{1-(1-p)e^{-\lambda y}}\dy=\\
&=\frac{p}{\lambda(1-p)}\int_{0}^{+\infty} \frac{\lambda(1-p)e^{-\lambda y}}{1-(1-p)e^{-\lambda y}}\dy=\\
&=\frac{p}{\lambda(1-p)}\left[\log |1-(1-p)e^{-\lambda y}|  \right]_0^{+\infty}=\\
&=\frac{p}{\lambda(1-p)}(-\log |p|)=\\
&=-\frac{1}{\lambda}\, \frac{p}{1-p}\, \log p
\end{aligned}
\end{gather*}

\end{enumerate}

\Soluzione{}
Manca

\Soluzione{}
Manca

\Soluzione{}
Siano $X\sim Y\sim\Ec(\lambda)$ con $X\indep Y$ tali che

\fg{0.4}{7_1}

\begin{enumerate}
\item Calcolare il valore atteso dell'area $A$, in funzione di $\lambda$.

Osservando che $A=XY/2$ abbiamo
\[
\EE[A]=\EE\left[ \frac{XY}{2} \right]=\frac{1}{2}\ \EE[XY]\overset{\underset{\indep}{}}{=}\frac{1}{2}\ \EE[X]\ \EE[Y]=\frac{1}{2\lambda^2}
\]

\item Calcolare la funzione di ripartizione di $Z$, notando che non dipende da $\lambda$, e disegnarne il grafico.

Dato che $X,Y>0$ q.c. si ha $Z>0$ q.c. quindi, fissato $z>0$, la funzione di ripartizione di $Z$ è
\[
F_Z(z)=\PP(Z\leq z)=\PP\left(\frac{Y}{X}\leq z \right)=\PP(Y\leq zX)
\] 
Come sempre, conviene vedere $(Y\leq zX)$ in funzione del vettore $(X,Y)$, quindi detto
\[
R_Z\coloneqq\{(x,y)\in\RR^2\ :\ x>0,\ 0<y<zx\}
\]

\fg{0.4}{7_6}

abbiamo 
\begin{gather*}
\begin{aligned}
F_Z(z)&=\PP((X,Y)\in R_Z)= \\
&=\int_{R_Z}\fXYxy\dxy= \\
&\overset{\underset{\indep}{}}{=}\int_{R_Z}f_X(x)\ f_Y(y)\dxy=\\
&=\int_{R_Z}\lambda^2e^{-\lambda(x+y)}\dxy=\\
&=\lambda^2\int_{0}^{+\infty} e^{-\lambda x} \left( \int_0^{zx}e^{-\lambda y}\dy  \right)\dx=\\
&=\lambda^2\int_{0}^{+\infty} e^{-\lambda x} \left[-\frac{1}{\lambda}\ e^{-\lambda y}  \right]_0^{zx}\dx=\\
&=\lambda\int_{0}^{+\infty} e^{-\lambda x} \left( -e^{-\lambda zx}+1  \right)\dx=\\
&=\lambda\int_{0}^{+\infty} \left(e^{-\lambda x} -e^{-\lambda (z+1)x}  \right)\dx=\\
&=\lambda\left[-\frac{1}{\lambda}\ e^{-\lambda x}+\frac{1}{\lambda(z+1)}\ e^{-\lambda (z+1)x}  \right]_0^{+\infty}=\\
&=\lambda\left(\frac{1}{\lambda}-\frac{1}{\lambda(z+1)} \right)=\\
&=1-\frac{1}{z+1}=\\
&=\frac{z}{z+1}
\end{aligned}
\end{gather*}
Quindi
\[
F_Z(z)=
\begin{cases}
0 &\text{se }z\leq 0 \\
\displaystyle\frac{z}{1+z} &\text{se }z>0
\end{cases}
\]
\fg{0.5}{7_2}

\item Determinare se $Z$ è assolutamente continua e in tal caso calcolarne la densità e disegnarne il grafico.

Dato che $F_Z\in\Cz\cap\widetilde{\mathcal{C}}^1$ allora $Z$ è una variabile aleatoria assolutamente continua con densità
\[
f_Z(z)=F_Z'(z)=
\begin{cases}
0 &\text{se }z\leq 0 \\
\displaystyle\frac{1}{(1+z)^2} &\text{se }z>0
\end{cases}
\]
\fg{0.5}{7_3}

\item Calcolare la media di $Z$.

Abbiamo
\begin{gather*}
\begin{aligned}
\EE[Z]&=\int_\RR z\ f_Z(z)\dz=\\
&=\int_{0}^{+\infty}\frac{z}{(1+z)^2}\dz=\\
&=\left\{ \frac{z}{(1+z)^2}=\frac{1}{1+z}-\frac{1}{(1+z)^2}  \right\}=\\
&=\int_{0}^{+\infty}\frac{1}{1+z}\dz-\int_{0}^{+\infty}\frac{1}{(1+z)^2}\dz=\\
&=\big[\log |1+z|  \big]_0^{+\infty}-\left[-\frac{1}{1+z}  \right]_0^{+\infty}=\\
&=+\infty
\end{aligned}
\end{gather*}

\item Calcolare la funzione di ripartizione di $\alpha$, notando che non dipende da $\lambda$.

Osserviamo che $\displaystyle\alpha=\arctan\left(\frac{Y}{X}\right)$ e che $\displaystyle \text{Im}(\alpha)=\left(0,\frac{\pi}{2}\right)$. Allora
\begin{itemize}
\item Per $t\leq 0$ si ha $F_\alpha(t)=0$
\item Per $t\geq\pi/2$ si ha $F_\alpha(t)=1$
\item Per $t\in(0,\pi/2)$ si ha
\begin{gather*}
\begin{aligned}
F_\alpha(t)&=\PP(\alpha\leq t)=\\
&=\PP\left(\arctan\left(\frac{Y}{Z}\right)\leq t\right)=\\
&=\PP\left(\frac{Y}{X}\leq\tan(t)\right)=\\
&=\PP(Z\leq\tan(t))=\\
&=F_Z(\tan(t))=\\
&=\frac{\tan(t)}{1+\tan(t)}
\end{aligned}
\end{gather*}
\end{itemize}

Quindi
\[
\begin{cases}
0 &\text{se }t\leq 0 \\
\displaystyle\frac{\tan(t)}{1+\tan(t)} &\text{se }\displaystyle t\in\left(0,\frac{\pi}{2}\right) \\
1 &\text{se }\displaystyle t\geq \frac{\pi}{2}
\end{cases}
\]
\fg{0.5}{7_4}

\item Determinare se $\alpha$ è assolutamente continua e in tal caso calcolarne la densità.

Analogamente al punto 3, osserviamo che $F_\alpha\in\Cz\cap\widetilde{\mathcal{C}}^1$ quindi $\alpha$ è una variabile aleatoria assolutamente continua con densità
\[
\begin{cases}
0 &\text{se }\displaystyle t\not\in\left(0,\frac{\pi}{2}\right) \\
\displaystyle\frac{1+\tan^2(t)}{(1+\tan(t))^2} &\text{se }\displaystyle t\in\left(0,\frac{\pi}{2}\right)
\end{cases}
\]
\fg{0.5}{7_5}
\end{enumerate}


\chapter{Funzioni caratteristiche. Vettori aleatori gaussiani}
%!TEX root = ../main.tex

Vediamo le principali definizioni e i principali teoremi che serviranno per affrontare gi esercizi di questo capitolo:
\begin{definition}$\\$
Sia $\PP$ una probabilità su $(\RR,\Bc)$, la funzione caratteristica (o \emph{trasformata di Fourier}) di $\PP$ è la funzione
\begin{gather*}
\begin{aligned}
\widehat{\PP}=\varphi: \RR^n &\to \CC\\
u&\mapsto\varphi (u)\coloneqq \int_{\RR^n}e^{i\langle u|x\rangle}\ \PP(\text{d}x)
\end{aligned}
\end{gather*}
\end{definition}
Dalla \emph{formula di Eulero} sappiamo che $e^{i \vartheta}=\cos(\vartheta) + i \sin (\vartheta)$ e $|e^{i \vartheta}|=1$ per ogni $\vartheta\in\RR$. Ne discende che
\begin{gather*}
\begin{aligned}
&e^{i\langle u|x\rangle}=\cos(\langle u|x\rangle) + i \sin (\langle u|x\rangle) \\
&|e^{i \langle u|x\rangle}|=1
\end{aligned}
\end{gather*}
per ogni $u,x\in\RR^n$.
\begin{definition}$\\$
\label{carmom}
Sia $X:\SDP\to(\RR^n,\Bc^n)$ vettore aleatorio (continuo, discreto, altro) di legge $P^X$. La funzione caratteristica di $X$ è $\widehat P^X =\varphi_X:\RR^n\to \CC$ definita come
\[
\varphi_X(u)=\int_{\RR^n}e^{i \langle u|X\rangle}\ P^X(\text{d}x)=\EE\left[e^{i \langle u|X\rangle}\right]=\int_\Omega e^{i \langle u|X\rangle}\dP
\]
\end{definition}
\emph{Esempio}. Sia $X\sim\delta_t$ con $t\in\RR$. Allora
\[
\varphi_X(u)=\EE\left[e^{iuX}\right]=e^{iut}
\]
\emph{Esempio}. Sia $X\sim\Pc(\lambda)$ con $\lambda\in\RR_+$. Allora
\begin{gather*}
\begin{aligned}
\varphi_X(u) &=\EE[e^{i u X}] =\sum_{k=0}^{{+\infty}}e^{i u k}\ e^{-\lambda}\ \frac{\lambda^k}{k!}=e^{-\lambda}\sum_{k=0}^{{+\infty}}\frac{(\lambda e^{i u})^k}{k!}=\\
&=\left\{ \sum_{k=0}^{+\infty}\frac{x^k}{k!}=e^x\ \ \ \forall x\in\RR  \right\}=e^{-\lambda}\ e^{\lambda e^{iu}}=e^{ \lambda (e^{i u}-1)}
\end{aligned}
\end{gather*}
\begin{theorem}[La funzione caratteristica di una probabilità caratterizza la probabilità]$\\$
\label{La funzione caratteristica di una probabilità caratterizza la probabilità}
La funzione caratteristica $\widehat{\PP}$ di una probabilità $\PP$ su $(\RR^n, \Bc^n)$ caratterizza $\PP$, ovvero
\[
\widehat{\PP}(u)=\widehat{\QQ}(u) \ \forall u \in \RR^n\ \Longleftrightarrow\ \PP(B)=\QQ(B) \ \forall B \in \Bc^n
\]
\end{theorem}
\begin{theorem}$\\$
\label{teo dei momenti}
Sia $X=(X_1,\dots,X_n):\SDP\to(\RR^n,\Bc^n)$ vettore aleatorio tale che per qualche $m\in\NN$ si abbia \\ $X_k \in L^m (\PP)$ $\forall k=1, \dots,  n$. Allora $\varphi_X\in \mathcal{C}^m(\RR^n,\CC)$ e
\[
\frac{\partial^m}{\partial u_{k_1} \cdots \partial u_{k_m}}\ \varphi(u)=i^m\ \EE\left[X_{k_1} \cdots X_{k_m}\ e^{i \langle u|x\rangle}\right] \qquad \forall u \in \RR^n
\]
\end{theorem}
\emph{Esempio}. Data $X\sim \Nc(0,1)$ si vuole calcolare $\varphi_X(u)$. \\
Prima di tutto osserviamo che $X\in L^m$ $\forall m\geq 1$. Sostituendo nella definizione di funzione caratteristica la densità della normale standard si ottiene
  \begin{equation*}
    \varphi_X (u) = \EE[e^{i u X}]=\int_{\RR}e^{iut}\ \frac{e^{-\frac{t^2}{2}}}{\sqrt{2\pi}}\dt
  \end{equation*}
Dalla formula di Eulero abbiamo $e^{iut}=\cos(ut)+i\sin(ut)$, quindi
\[
 \varphi_X (u) =\int_{\RR} \cos(ut)\ \frac{e^{-\frac{t^2}{2}}}{\sqrt{2\pi}}\dt + i\underbrace{\int_{\RR}\sin(ut)\ \frac{e^{-\frac{t^2}{2}}}{\sqrt{2\pi}}\dt}_{0\text{ perché dispari}}=\frac{1}{\sqrt{2\pi}}\int_{\RR} \cos(ut)\ e^{-\frac{t^2}{2}}\dt
\]
Per risolvere quest'integrale si usa un trucco: applicando il teorema (\ref{teo dei momenti}) si ottiene la derivata $\varphi_X' (u)$ in funzione di $\varphi_X (u)$, cioè
\begin{gather*}
\begin{aligned}
 \varphi_X' (u) & = i \ \EE\left[X e^{i u X}\right] = i \int_\RR t e^{iut}\ \frac{e^{-\frac{t^2}{2}}}{\sqrt{2\pi}}\dt= \\
&= i \left(\underbrace{ \int_\RR t \cos(ut)\ \frac{e^{-\frac{t^2}{2}}}{\sqrt{2\pi}}}_{0\text{ perché dispari}}\dt
    + i \int_\RR t \sin(ut)\ \frac{e^{-\frac{t^2}{2}}}{\sqrt{2\pi}}\dt \right)=\\
&= -\int_\RR t \sin(ut)\ \frac{e^{-\frac{t^2}{2}}}{\sqrt{2\pi}}\dt=\\
&\overset{\underset{pp}{}}{=}-\frac{1}{\sqrt{2 \pi}}\left\{ \left[ -e^{-\frac{t^2}{2}} \sin (ut) \right]_{-\infty}^{+\infty} +\int_{-\infty}^{+\infty}ue^{\frac{t^2}2} \cos (ut)\dt \right\}=\\
&=-u \varphi_X (u)
\end{aligned}
\end{gather*}
L'integrale si è così trasformato in un \emph{problema di Cauchy} ben posto:
  $$\begin{cases}
    \varphi_X'(u)=-u\varphi_X(u) \\ \varphi_X(0)=1
  \end{cases} \quad \Longrightarrow \quad \varphi_X(u)=e^{-\frac{u^2}{2}}$$
Siano $X:\SDP\to(\RR^n,\Bc^n)$ un vettore aleatorio e $Y:\SDP\to(\RR^m,\Bc^m)$ con $A\in\RR^{m\times m},b\in\RR^m$ la trasformazione affine $Y=AX+b$.
\begin{theorem}$\\$
\label{t_affine_t}
La funzione caratteristica del vettore $Y$ dato dalla trasformazione affine $Y=AX+b$ è
\[
\varphi_Y(u)=e^{i\langle u|b\rangle} \ \varphi_X\left(A^T  u\right) \ \ \ \forall u \in \RR^m
\]
\end{theorem}
\emph{Esempio}. Calcolare la funzione caratteristica di una normale generica.\\
Sia $X\sim N(\mu,\sigma^2)$. Allora basta vedere $X=\mu+\sigma Z$, con $Z=\dfrac{X-\mu}{\sigma}\sim \Nc(0,1)$. Allora
$$\varphi_X (u)=e^{i u \mu}\ \varphi_Z(\sigma u)=  e^{i u \mu}\ e^{-\sigma^2u^2/2} =e^{i u \mu-\sigma^2u^2/2}$$
\emph{Esempio}. Calcolare la funzione caratteristica della somma di variabili aleatorie.\\
Siano $X=(X_1,  \dots,  X_n) \sim \varphi_X $ e $ Y=\displaystyle\sum_{k=1}^{n}X_k$. Allora vedendo $Y$ come trasformazione affine
\[
Y=\begin{pmatrix}
1 &\cdots  &1  \\
\end{pmatrix}
\begin{pmatrix}
X_1 \\ \vdots
 \\ X_n
\end{pmatrix} +0
\]
possiamo calcolare
\[
\varphi_Y(u) = \varphi_X \left(
      \begin{pmatrix} 1 \\ \vdots \\ 1 \end{pmatrix} u
    \right) = \varphi_X(u,  \dots,  u)
\]
\begin{oss}$\\$
Quest'ultimo esempio semplifica tantissimo il calcolo delle funzioni caratteristiche marginali di un vettore. Infatti dato $X=(X_1,  \dots,  X_n) \sim \varphi_X $ per calcolare la funzione caratteristica $\varphi_{X_k}$ vediamo $X_k=(0,\dots,0,1,0,\dots,0)X=vX$ e dunque
\[
\varphi_{X_k}(u) = \varphi_X\left(v u\right)=\varphi_X(v)=\varphi_X(0,\dots,0,u,0,  \dots,  0)
\]
\end{oss}
\begin{theorem}[Fattorizzazione della funzione caratteristica per famiglie di VA]
\label{Fattorizzazione della funzione caratteristica per famiglie di VA}
$\\$
Sia $X = (X_1,  \dots, X_n):\SDP\to(\RR^n,\Bc^n)$ vettore aleatorio. Allora
      $$\big\{X_k\big\}_{k=1}^{n} \text{ famiglia di VAR} \indep\ \Longleftrightarrow\ \varphi_X(u_1, \dots, u_n) = \prod_{k=1}^{n} \varphi_{X_k} (u_k) \quad \forall u \in \RR^n$$
\end{theorem}
\begin{corollario}[Legge della somma di variabili indipendenti]$\\$
\label{Legge della somma di variabili indipendenti}
Siano $X_1, \dots , X_n$ VAR indipendenti. Allora
  $$Y = \sum_{k=1}^{n} X_k\ \Longrightarrow\ \varphi_Y(u) = \varphi_{\sum X_k} (u) = \prod_{k=1}^{n} \varphi_{X_k}(u) \quad \forall u \in \RR$$
\end{corollario}

%

Prima procedere con le soluzioni degli esercizi, ci teniamo a fare ulteriori precisazioni in modo da rendere più scorrevoli gli esercizi a venire. 

Abbiamo già visto che: 
\begin{itemize}
\item $X\sim\delta_t\implies\varphi_X(u)=e^{iut}$
\item $X\sim\Pc(\lambda)\implies\varphi_X(u)=e^{ \lambda (e^{i u}-1)}$
\item $Z\sim\Nc(0,1)\implies\varphi_X(u)=e^{-u^2/2}$
\item $X\sim\Nc(\mu,\sigma^2)\implies\varphi_X(u)=e^{iu\mu-\sigma^2u^2/2}$
\end{itemize}
tuttavia utilizzando il teorema (\ref{La funzione caratteristica di una probabilità caratterizza la probabilità}) possiamo dire che l'implicazione vale anche in senso opposto, ovvero (parafrasando tale teorema)
\begin{theorem}
\label{caratterizzazione della f caratteristica}
Date $X,Y:\Omega\to\RR^n$ si ha $P^X=P^Y\iff\varphi_X=\varphi_Y$.
\end{theorem}
Ecco quindi i primi quattro risultati notevoli di questo capitolo
\[
X\sim\delta_t,\ t\in\RR\iff\varphi_X(u)=e^{iut} \qquad X\sim\Pc(\lambda),\ \lambda>0\iff\varphi_X(u)=e^{ \lambda (e^{i u}-1)}
\]
\[
Z\sim\Nc(0,1)\iff\varphi_X(u)=e^{-u^2/2} \qquad X\sim\Nc(\mu,\sigma^2),\ \mu\in\RR,\ \sigma^2>0\iff\varphi_X(u)=e^{iu\mu-\sigma^2u^2/2}
\]
I richiami al teorema (\ref{caratterizzazione della f caratteristica}) verranno omessi, e verrà dimostrato quindi solo il $\implies$.

Verrà anche omesso il riferimento alla definizione (\ref{carmom}) quando scriveremo $\varphi_X(u)=\EE\left[e^{i \langle u|X\rangle}\right]$.

Inoltre verrà fatto uso delle seguenti formule
\begin{enumerate}
\item [(EU)] Formula di Eulero
\[
e^{i \vartheta}=\cos(\vartheta) + i \sin (\vartheta)
\]
\item [(NT)] Binomio di Newton
\[
(a+b)^n=\sum_{k=0}^n\binom{n}{k}a^{n-k}\ b^k
\]
\item [(TY)] Sviluppi in serie di Taylor - Mc Laurin
\begin{gather*}
\begin{aligned}
\cos(x)&=\sum_{n=0}^\infty\frac{(-1)^n}{(2n)!}\ x^{2n}\qquad\forall x\in\RR \\
&=1-\frac{x^2}{2!}+\frac{x^4}{4!}+\cdots +o\left(x^{2n}\right)\\
\sin(x)&=\sum_{n=0}^\infty\frac{(-1)^n}{(2n+1)!}\ x^{2n+1}\qquad\forall x\in\RR\\
&=x-\frac{x^3}{3!}+\frac{x^5}{5!}+\cdots+o\left(x^{2n+1}\right)
\end{aligned}
\end{gather*}
\end{enumerate}

\newpage

%

\ParteEsercizi

\Esercizio{} %1
Sia $\varphi_X$ la funzione caratteristica di una variabile aleatoria reale $X$, quindi $X\sim\varphi_X$.
\begin{enumerate}
\item [(a)] Si mostri che $Y=-X\sim\overline{\varphi}_X$ e che $\overline{\varphi}_X(u)=\varphi_X(-u)$ ($\overline{\ \cdot\ }$ è il complesso coniugato).
\item [(b)] Si mostri che, se le variabili aleatorie $X$ e $W$ sono i.i.d. con funzione caratteristica $\varphi$, allora $X-W\sim|\varphi|^2$.
\end{enumerate}

\Esercizio{} %2
Siano $ n,m \in \NN$, $p\in [0, 1]$ e $\varphi_X$ la funzione caratteristica di una variabile aleatoria $X$.
\begin{enumerate}
\item [(a)] Si mostri che $X\sim B(p)\iff\varphi_X(u)=pe^{iu}+1-p$.
\item [(b)] Si mostri che $X\sim B(n,p)\iff\varphi_X(u)=\left(pe^{iu}+1-p  \right)^n$.
\item [(c)] Si mostri che $X_1,\dots,X_n\iid B(p)\implies X_1+\cdots+X_n\sim B(n,p)$.
\item [(d)] Si mostri che $X\sim B(n,p)\indep Y\sim B(m,p)\implies X+Y\sim B(n+m,p)$.
\end{enumerate}

\Esercizio{} %3
Data una variabile aleatoria reale $X$, si mostri che
\[
X\sim \Gc(p),\ p\in(0,1)\iff \varphi_X(u)=\frac{pe^{iu}}{1-e^{iu}(1-p)}
\]

\Esercizio{} %4
Sia $X$ una variabile aleatoria reale.
\begin{enumerate}
\item [(a)] Si mostri che $X\sim \Uc([-a,a])\iff\varphi_X(u)=\dfrac{\sin(au)}{au}$.
\item [(b)] Si mostri che $X\sim  \Uc([a,b])\iff\varphi_X(u)=e^{iu\frac{a+b}{2}}\ \dfrac{\sin\left(\frac{b-a}{2}u\right)}{\frac{b-a}{2}u}$.
\item [(c)] Si mostri che $X\sim\Ec(\lambda)\iff\varphi_X(u)=\dfrac{\lambda}{\lambda-iu}$.
\item [(d)] Si mostri che $X$ tale che $-X\sim\Ec(\lambda)\iff\varphi_X(u)=\dfrac{\lambda}{\lambda+iu}$.
\end{enumerate}
Si calcolino valore atteso e varianza per queste variabili aleatorie a partire dalle loro funzioni caratteristiche.

\Esercizio{} %5
Sia $(X,Y)$ un vettore aleatorio uniformemente distribuito sul rettangolo $R=\{0\leq x\leq 2,\ 0\leq y\leq 1\}$.
\begin{itemize}
\item [(a)] Si scrivano le densità continue di $(X,Y)$, di $X$ e di $Y$. Le variabili aleatorie $X$ e $Y$ sono indipendenti?
\item [(b)] Si calcolino valore atteso e matrice varianza di $(X,Y)$.
\item [(c)] Si ricavi la funzione caratteristica di $(X,Y)$ in termini delle funzioni caratteristiche di $X$ e $Y$.
\item [(d)] Siano $Z=2X-Y,\ W=Y-2,\ J=-X$. Il vettore aleatorio $(Z,W,J)$ ammette densità continua?
\item [(e)] Si calcolino valore atteso e matrice varianza di $(Z,W,J)$.
\item [(f)] Si ricavi la funzione caratteristica di $(Z,W,J)$ in termini delle funzioni caratteristiche di $X$ e $Y$.
\item [(g)] Si ricavino le funzioni caratteristiche di $(Z,W)$ e $(W,J)$ in termini della funzione caratteristica di $(Z,W,J)$.
\end{itemize}

\Esercizio{} %6
Siano $X_1,\dots,X_n$ variabili aleatorie indipendenti, con $X_k\sim\Pc(\lambda_k),\ k=1,\dots,n$.
\begin{itemize}
\item [(a)] Mostrare che $X_1+\cdots+X_n\sim\Pc(\lambda_1+\cdots+\lambda_n)$.
\item [(b)] Si supponga ora che $X_1,X_2,X_3\iid\Pc(\lambda)$. Calcolare $\PP(X_1+X_2+X_3\geq 3\,|\,X_1\geq 1)$. 
\end{itemize}

\Esercizio{} %7
Date $X_1,\dots,X_n$ variabili aleatorie reali i.i.d. si mostri che
\[
\overline{X}_n=\displaystyle\frac{1}{n}\sum_{k=1}^n X_k,\ n\in\NN\iff \varphi_{\overline{X}_n}(u)=\left(\varphi_{X_1}\left(\frac{u}{n}\right)\right)^n
\]

\Esercizio{} %8
Per ogni $\alpha>0$ e $\lambda>0$, la funzione caratteristica di una variabile aleatoria $X\sim\Gamma(\alpha,\lambda)$ è data da
\[
\varphi_X(u)=\left( \frac{\lambda}{\lambda-iu} \right)^\alpha
\]
\begin{enumerate}
\item [(a)] Si mostri che $X\sim\Gamma(\alpha,\lambda)\indep Y\sim\Gamma(\beta,\lambda)\implies X+Y\sim\Gamma(\alpha+\beta,\lambda)$.
\item [(b)] Si mostri che $X_1,\dots,X_n\iid\Ec(\lambda)\implies X_1+\cdots+X_n\sim\Gamma(n,\lambda)$.
\item [(c)] Si mostri che $Z\sim\Nc(0,1)\implies Z^2\sim\Gamma\left(\frac{1}{2},\frac{1}{2}\right)=\chi^2(1)$.
\item [(d)] Si mostri che $Q=Z_1^2+\cdots+Z_n^2$, con $Z_1,\dots,Z_n\iid\Nc(0,1)\implies Q\sim\Gamma\left(\frac{n}{2},\frac{1}{2}\right)=\chi^2(n)$.
\end{enumerate}

\Esercizio{} %9
Siano $X\sim\Gamma(\alpha,\lambda)$ e $Y\sim\Gamma(\beta,\lambda)$ indipendenti, con $\alpha>0,\ \beta>0$ e $\lambda>0$.
\begin{enumerate}
\item [(a)] Si considerino le variabili aleatorie
\[
T=X+Y\qquad U=\frac{X}{X+Y}
\]
Si mostri che $(T,U)$ ammette densità continua e la si calcoli.
\item [(b)] Si provi che $T\indep U$.
\item [(c)] Si provi che $T\sim\Gamma(\alpha+\beta,\lambda)$ e che $U\sim\text{Beta}(\alpha,\beta)$, ossia $U$ ha distribuzione Beta di parametri $\alpha$ e $\beta$, la cui densità continua è data da
\[
f_U(u)=\frac{\Gamma(\alpha+\beta)}{\Gamma(\alpha)\ \Gamma(\beta)}\ u^{\alpha-1}\ (1-u)^{\beta-1}\ \Ind_{(0,1)}(u).
\]
\item [(d$^\ast$)] Sia $(X_n)_{n\in\NN}$ una successione di variabili aleatorie, con $X_1,X_2,\dots\iid\Ec(\lambda)$. Posto
\[
S_n=X_1+\dots+X_n
\]
si provi che per ogni coppia di interi $(k,n)$ con $1\leq k\leq n$, la variabile aleatoria $S_{k/n}=\dfrac{S_k}{S_n}$ è indipendente da $S_n$. Sfruttando quest'ultimo risultato si calcoli il valore atteso di $S_{k/n}$.
\end{enumerate}

\Esercizio{} %10
Siano $X$ e $Y$ variabili aleatorie congiuntamente gaussiane
\begin{enumerate}
\item [(a$^*$)] Si descriva l'immagine $S$ del vettore aleatorio $(X,Y)$ al variare di $\mu_X,\mu_Y,\sigma_X,\sigma_Y, Cov(X,Y)$.
\item [(b)] Si mostri che $(X,Y)$ ammette densità continua $\fXY\iff\sigma_X>0,\sigma_Y>0,|\rho_{X,Y}|<1$.
\item [(c)] Si mostri che, quando $(X,Y)$ ammette densità continua, si ha
\begin{gather*}
\begin{aligned}
\fXYxy=&\frac{1}{2\pi\sigma_X\sigma_Y\sqrt{1-\rho_{X,Y}^2}}\\ &\times\exp\left\{-\frac{1}{2(1-\rho_{X,Y}^2)}\left[\frac{(x-\mu_X)^2}{\sigma_X^2}-2\rho_{X,Y}\frac{(x-\mu_X)(y-\mu_Y)}{\sigma_X\sigma_Y}+\frac{(y-\mu_Y)^2}{\sigma_Y^2}  \right]\right\}.
\end{aligned}
\end{gather*}
\item [(d)] Si discuta la dipendenza del grafico di $\fXY$ dai parametri $\sigma_X,\sigma_Y,\rho_{X,Y}$.
\end{enumerate}

\Esercizio{} %11
Sia $(X,Y)$ un vettore gaussiano. Le densità di $X$ e di $Y$ sono
\[
f_X(x)=\frac{\sqrt{3}}{\sqrt{\pi}}e^{-3x^2}\qquad\qquad f_Y(y)=\frac{3\sqrt{3}}{2\sqrt{\pi}}e^{-\frac{27}{4}y^2}\qquad\qquad x,y\in\RR
\]
Il coefficiente di correlazione è $\rho_{X,Y}=1/2$.
\begin{enumerate}
\item [(a)] Riconoscere le leggi di $X$ e di $Y$.
\item [(b)] Scrivere media, matrice varianza e densità continua di $(X,Y)$.
\item [(c)] Calcolare la legge di $Z=X+3Y$.
\item [(d)] Calcolare $\PP(X>Y)$.
\end{enumerate}

\Esercizio{} %12
Date $X$ e $Y$ indipendenti di legge $\Nc(0,1)$, si trovi la legge congiunta di $U=X+Y$ e $V=X-Y$. $U\indep V$?

\Esercizio{} %13
Siano $X$ e $Y$ due variabili aleatorie indipendenti, tali che $X\sim\Nc(0,1)$ ed $Y\sim\Nc(0,4)$.
\begin{enumerate}
\item Determinare la legge di $Z=X+Y$.
\item Determinare la matrice varianza del vettore $(X,Z)$.
\item Determinare la legge congiunta di $X$ e $Z$. Ammette densità continua? Se sì, quale?
\item Determinare la funzione caratteristica di $(X,Z)$.
\item Sia $W\sim\Nc(1,2)$, indipendente dal vettore $(X,Z)$. Qual è la legge di $(X,Z,W)$?
\end{enumerate}

\Esercizio{} %14
Dati $X\sim\Nc(0,1)$ e $a>0$, si consideri $\begin{cases}X,&|X|<a,\\-X,&|X|\geq a.  \end{cases}$
\begin{enumerate}
\item [(a)] Si trovi la legge di $Y$.
\item [(b)] Si stabilisca se i vettori $(X,X)$ e $(X,Y)$ sono gaussiani.
\item [(c)] Si calcoli $\PP(X>Y)$.
\item [(d)] Si trovi il coefficiente di correlazione$\rho_{X,Y}$.
\end{enumerate}

\Esercizio{} %15
Sia $X=(X_1,X_2,X_3$ un vettore gaussiano $\Nc(\mu,C)$, dove
\[
\mu=\begin{pmatrix}
0 \\0
 \\1
\end{pmatrix},\qquad\qquad C=\begin{pmatrix}
1 & 1 & 0 \\
1 & 2 & 0 \\
0 &  0& 1 \\
\end{pmatrix}.
\]
\begin{enumerate}
\item [(a)] Il vettore $(X_1,X_2)$ e la variabile $X_3$ sono indipendenti?
\item [(b)] La variabile $X_1$ e il vettore $(X_2,X_3)$ sono indipendenti?
\item [(c)] Determinare, al variare di $\lambda\in\RR$, la legge del vettore $(U,V)$ definito come segue:
\[
\begin{pmatrix}
U \\V

\end{pmatrix}=\begin{pmatrix}
X_1 \\X_2+\lambda X_3
\end{pmatrix}.
\]
\item [(d)] Quanto deve valere $\lambda$ affinché $\PP(U+V>3)>1/2$?
\end{enumerate}

\Esercizio{} %16
Siano $X$ e $Y$ due variabili aleatorie congiuntamente gaussiane con medie $\mu_X$ e $\mu_Y$, varianze $\sigma_X^2$ e $\sigma_Y^2$, coefficiente di correlazione $\rho_{X,Y}$. Detta $W=X-Y$, mostrare che $U=|W-\EE[W]|$ è una variabile aleatoria continua, e calcolarne la densità.

\Esercizio{} %17
Sia $(X,Y)$ un vettore aleatorio con funzione caratteristica
\[
\varphi_{(X,Y)}(u,v)=\exp\left\{i(u+2v)-\frac{1}{2}\left(3u^2+10uv+9v^2\right)  \right\}
\]
\begin{enumerate}
\item [(a)] Riconoscere la legge di $(X,Y)$ e scrivere il vettore delle medie e la matrice varianza.
\item [(b)] Siano $U=\lambda X$ e $V=X-\lambda Y$, con $\lambda\in\RR$. Determinare la legge del vettore aleatorio $(U,V)$.
\item [(c)] Trovare l'unico valore positivo di $\lambda$ tale per cui le variabili aleatorie $U$ e $V$ siano indipendenti.
\end{enumerate}

\Esercizio{} %18
Si consideri il vettore aleatorio $(X_\alpha,Y_\alpha)\sim\Nc\left(\begin{pmatrix}
0 \\1
\end{pmatrix},\begin{pmatrix}
2\alpha &1 \\1&1
\end{pmatrix}\right)$.
\begin{enumerate}
\item [(a)] Determinare i valori ammissibili del parametro $\alpha\in\RR$.
\item [(b)] Riconoscere le distribuzioni di $X_\alpha$ e $Y_\alpha$.
\item [(c)] Determinare il valore del parametro $\overline{\alpha}\in\RR$ tale che $\EE[(X_{\overline{\alpha}}-Y_{\overline{\alpha}})^2]=2$.
\item [(d)] Si consideri ora il vettore $(X_{\overline{\alpha}},Y_{\overline{\alpha}})$. Calcolare $\PP(\max\{X_{\overline{\alpha}},  Y_{\overline{\alpha}}\}=Y_{\overline{\alpha}})$. 
\end{enumerate}

\Esercizio{} %19
Siano $X$ e $Z$ due variabili aleatorie indipendenti tali che $X\sim\Nc(0,1)$ e $Z$ abbia legge: $\PP(Z=1)=1/2$ e $\PP(Z=-1)=1/2$. Si ponga $Y=ZX$.
\begin{enumerate}
\item [(a)] Che legge ha $Y$?
\item [(b)] Calcolare $Cov(X,Y)$.
\item [(c)] Mostrare che $X+Y$ non è una variabile aleatoria gaussiana.
\item [(d)] $(X,Y)$ è un vettore gaussiano?
\item [(e)] $X$ e $Y$ sono indipendenti?
\end{enumerate}

\Esercizio{} %20
Si provi che se $X_1,\dots,X_n$ sono variabili aleatorie i.i.d. $\Nc(\mu,\sigma^2)$, con $\sigma^2>0$, allora 
\[
\sum_{k=1}^{n}\frac{(X_k-\mu)^2}{\sigma^2}\sim\chi^2(n)=\Gamma\left(\frac{n}{2},\frac{1}{2}  \right).
\]

\Esercizio{} %21
Sia $X=(X_1,\dots,X_n)\sim\Nc(\mu,C)$ un vettore aleatorio gaussiano con $C$ invertibile. Si mostri che la variabile aleatoria $(X-\mu)^TC^{-1}(X-\mu)$ ha legge $\chi^2(n)$.

\Esercizio{} %22
Siano $X_1,\dots,X_n$ variabili aleatorie reali i.i.d. $\Nc(\mu,\sigma^2)$.  Si considerino
\[
\overline{X}_n=\frac{1}{n}\sum_{k=1}^nX_k,\qquad Y_k=X_k-\overline{X}_n,\qquad S_n^2=\frac{1}{n-1}\sum_{k=1}^n(X_k-\overline{X}_n)^2.
\]
\begin{enumerate}
\item [(a)] Si calcoli $\varphi_{(\overline{X}_n,Y_1,\dots,Y_n)}$.
\item [(b)] Se ne deduca l'indipendenza di $\overline{X}_n$ e $S_n^2$.
\item [(c)] Si mostri che $Z=\dfrac{\overline{X}_n-\mu}{\sigma/\sqrt{n}}\sim\Nc(0,1)$.
\item [(d)] Sapendo che $Q=\dfrac{n-1}{\sigma^2}\ S_n^2\sim\chi^2(n-1)$, si calcoli la funzione caratteristica di $\varphi_{(\overline{X}_n,S_n^2)}$.
\item [(e)] Si calcolino le funzioni caratteristiche $\varphi_{(\overline{X}_n^2,S_n^2)}$ e $\varphi_{\overline{X}_n^2+S_n^2}$ nel caso $\mu=0$.
\item [(f$^*$)] Si trovi la densità continua della variabile aleatoria $T=\dfrac{\overline{X}_n-\mu}{\sqrt{S_n^2/n}}$.
\end{enumerate}

\ParteSoluzioni

\Soluzione{} %1
\begin{enumerate}
\item [(a)] Si mostri che $Y=-X\sim\overline{\varphi}_X$ e che $\overline{\varphi}_X(u)=\varphi_X(-u)$.

Calcoliamo la funzione di ripartizione di $Y$:
\begin{gather*}
\begin{aligned}
\varphi_Y(u)&=\EE\left[e^{i\langle u|Y \rangle}  \right]=\\
&=\EE\left[e^{-i\langle u|X \rangle}  \right]=\\
&=\{z=re^{i\phi}\implies \overline{z}=re^{-i\phi}  \}=\\
&=\EE\left[\overline{e^{i\langle u|X \rangle}  }\right]=\\
&=\overline{\EE\left[e^{i\langle u|X \rangle}  \right]}=\\
&=\overline{\varphi}_X(u)
\end{aligned}
\end{gather*}
Inoltre
\begin{gather*}
\begin{aligned}
\overline{\varphi}_X(u)&=\EE\left[e^{-i\langle u|X \rangle}  \right]=\\
&=\EE\left[e^{i\langle -u|X \rangle}  \right]=\\
&=\varphi_X(-u)
\end{aligned}
\end{gather*}

\item [(b)] Si mostri che, se le variabili aleatorie $X$ e $W$ sono i.i.d. con funzione caratteristica $\varphi$, allora $X-W\sim|\varphi|^2$.

Calcoliamo la funzione di ripartizione di $X-W$:
\begin{gather*}
\begin{aligned}
\varphi_{X-W}(u)&=\varphi_{X+(-W)}(u)=\\
&\overset{\underset{(\ref{Legge della somma di variabili indipendenti})}{}}{=}\varphi_X(u)\cdot \varphi_{-W}(u)=\\
&=\varphi(u)\cdot\overline{\varphi}(u)=\\
&=|\varphi|^2
\end{aligned}
\end{gather*}
che è quanto volevasi dimostrare.

\end{enumerate}

\Soluzione{} %2
\begin{enumerate}
\item [(a)] Si mostri che $X\sim B(p)\iff\varphi_X(u)=pe^{iu}+1-p$.

Ricordiamo che la densità di una bernoulliana è
\[
p_X(k)=\begin{cases} p&\text{se }k=1 \\ 1-p &\text{se }k=0 \end{cases}
\]
Allora la funzione caratteristica di $X$ è
\begin{gather*}
\begin{aligned}
\varphi_X(u)&=\EE\left[e^{iuX}  \right]=\\
&=\sum_{k=0}^1e^{iuk}\ p_X(k)=\\
&=(1-p)e^{iu0}+pe^{iu1}=\\
&=pe^{iu}+1-p
\end{aligned}
\end{gather*}
Quindi
\[
X\sim B(p),\ p\in[0,1]\iff\varphi_X(u)=pe^{iu}+1-p
\]

\item [(b)] Si mostri che $X\sim B(n,p)\iff\varphi_X(u)=\left(pe^{iu}+1-p  \right)^n$.

Ricordiamo che la densità di una binomiale è
\[
p_X(k)=\binom{n}{k}p^k(1-p)^{n-k}\qquad k=0,\dots,n
\]
Allora la funzione caratteristica di $X$ è
\begin{gather*}
\begin{aligned}
\varphi_X(u)&=\EE\left[e^{iuX}  \right]=\\
&=\sum_{k=0}^ne^{iuk}\ p_X(k)=\\
&=\sum_{k=0}^n\binom{n}{k}\left(pe^{iu}\right)^k(1-p)^{n-k}=\\
&\overset{\underset{\text{NT}}{}}{=}\left(pe^{iu}+1-p\right)^n
\end{aligned}
\end{gather*}
Quindi
\[
X\sim B(n,p),\ n\in\NN,\ p\in[0,1]\iff\varphi_X(u)=\left(pe^{iu}+1-p\right)^n
\]

\item [(c)] Si mostri che $X_1,\dots,X_n\iid B(p)\implies X_1+\cdots+X_n\sim B(n,p)$.

Dato che le bernoulliane sono i.i.d. possiamo applicare il corollario (\ref{Legge della somma di variabili indipendenti}) per calcolare la funzione caratteristica della somma $X_1+\cdot+X_n$:
\[
\varphi_{X_1,\dots,X_n}(u)=\prod_{k=1}^n\varphi_{X_k}(u)=\left(pe^{iu}+1-p\right)^n
\]
Allora grazie al punto (b) possiamo concludere che 
\[
X_1,\dots,X_n\iid B(p),\ n\in\NN,\ p\in[0,1] \implies X_1+\cdots+X_n\sim B(n,p)
\]

\item [(d)] Si mostri che $X\sim B(n,p)\indep Y\sim B(m,p)\implies X+Y\sim B(n+m,p)$.

Analogamente al punto precedente, per il corollario (\ref{Legge della somma di variabili indipendenti}) si ha
\[
\varphi_{X+Y}(u)=\varphi_X(u)\cdot \varphi_Y(u)=\left(pe^{iu}+1-p\right)^{n+m}
\]
Dunque in generale
\[
\begin{cases}X_1,\dots,X_n\text{ indipendenti}\\X_k\sim B(n_k,p),\ n_k\in\NN\ \forall k,\ p\in[0,1]\end{cases} \implies X_1+\cdots+X_n\sim B(n_1+\cdots+n_n,p)
\]
\end{enumerate}

\Soluzione{} %3
Ricordando che la densità di una geometrica è
\[
p_X(k)=p(1-p)^{k-1}\qquad k=1,2,\dots
\]
possiamo calcolare la funzione caratteristica di $X$ nel seguente modo:
\begin{gather*}
\begin{aligned}
\varphi_X(u)&=\EE\left[e^{iuX}  \right]=\\
&=\sum_{k=1}^\infty e^{iuk}\ p_X(k)=\\
&=\sum_{k=1}^\infty e^{iuk}\ p(1-p)^{k-1}=\\
&=p\sum_{t=0}^\infty e^{iu(t+1)}\ (1-p)^{t}=\\
&=pe^{iu}\underbrace{\sum_{t=0}^\infty\left(e^{iu}(1-p)   \right)^t}_{\text{serie geometrica}}=\\
&=\frac{pe^{iu}}{1-e^{iu}(1-p)}
\end{aligned}
\end{gather*}

\Soluzione{} %4
\begin{enumerate}
\item [(a)] Si mostri che $X\sim \Uc([-a,a])\iff\varphi_X(u)=\dfrac{\sin(au)}{au}$.

Calcoliamo la funzione caratteristica di $X$:
\begin{gather*}
\begin{aligned}
\varphi_X(u)&=\EE\left[e^{iuX}  \right]=\\
&=\int_\RR e^{iux}\ \frac{1}{2a}\ \Ind_{[-a,a]}(x)\dx=\\
&=\frac{1}{2a}\int_{-a}^ae^{iux}\dx=\\
&\overset{\underset{\text{EU}}{}}{=}\frac{1}{2a}\left(\int_{-a}^a\underbrace{\cos(ux)}_{\text{pari}}\dx+i\int_{-a}^a\underbrace{\sin(ux)}_{\text{dispari}}\dx  \right)=\\
&=\frac{1}{2a}\ 2\int_{0}^a\cos(ux)\dx=\\
&=\frac{1}{a}\left[\frac{1}{u} \sin(ux)  \right]_0^a=\\
&=\frac{\sin(au)}{au}
\end{aligned}
\end{gather*}
Quindi
\[
X\sim \Uc([-a,a]),\ a\in[0,+\infty)\iff\varphi_X(u)=\frac{\sin(au)}{au}=\text{sinc}(au)
\]
\begin{oss}
La funzione $\dfrac{\sin(au)}{au}=:\text{sinc}(au)$ è detta \emph{seno cardinale}.
\fg{0.5}{8_1}
\end{oss}
Per calcolare media e varianza di $X$ sfruttiamo il teorema (\ref{teo dei momenti}), cioè
\[
X\in L^1\implies\varphi_X\in\Cu,\ \EE[X]=\frac{1}{i}\varphi_X'(0)=-i\varphi_X'(0)
\]
\[
X\in L^2\implies\varphi_X\in\mathcal{C}^2,\ \EE[X^2]=\frac{1}{i^2}\varphi_X''(0)=-\varphi_X''(0)
\]
Allora
\begin{gather*}
\begin{aligned}
&\varphi_X(u)=\frac{\sin(au)}{au}\overset{\underset{\text{TY}}{}}{\simeq}\frac{au-\frac{(au)^3}{3!}+o(u^3)}{au}=1-\frac{(au)^2}{6}+o(u^2) \\
&\varphi_X'(0)=\left[-\frac{a^2}{3}u+o(u) \right]_{u=0}=0\implies\EE[X]=0\\
&\varphi_X''(0)=\left[-\frac{a^2}{3} \right]_{u=0}=-\frac{a^2}{3}\implies\EE[X^2]=\frac{a^2}{3}\implies Var(X)=\frac{a^2}{3}\\
\end{aligned}
\end{gather*}

\item [(b)] Si mostri che $X\sim  \Uc([a,b])\iff\varphi_X(u)=e^{iu\frac{a+b}{2}}\ \dfrac{\sin\left(\frac{b-a}{2}u\right)}{\frac{b-a}{2}u}$.

Notiamo che possiamo ricondurci ad un'uniforme simmetrica semplicemente traslando di $\dfrac{a+b}{2}$:
\fg{0.5}{8_2}
Grazie a questa traslazione possiamo scrivere $X$ in termini dell'uniforme simmetrica
\[
Y\coloneqq X-\frac{a+b}{2}\sim\Uc\left(\left[-\frac{b-a}{2},\frac{b-a}{2}\right]\right)\implies X=\frac{a+b}{2}+Y
\]
Allora
\begin{gather*}
\begin{aligned}
\varphi_X(u)&=\EE\left[e^{iuX}  \right]=\\
&=e^{iu\frac{a+b}{2}}\ \varphi_Y(u)=\\
&=e^{iu\frac{a+b}{2}}\ \frac{\sin\left(\frac{b-a}{2}u\right)}{\frac{b-a}{2}u}
\end{aligned}
\end{gather*}
Quindi
\[
X\sim  \Uc([a,b]),\ a,b\in\RR\iff\varphi_X(u)=e^{iu\frac{a+b}{2}}\ \frac{\sin\left(\frac{b-a}{2}u\right)}{\frac{b-a}{2}u}=e^{iu\frac{a+b}{2}}\ \text{sinc}\left( \frac{b-a}{2}u \right)
\]
Per quanto riguarda media e varianza
\[
\EE[X]\overset{\underset{\dots}{}}{=}\frac{a+b}{2} \qquad \EE[X^2]\overset{\underset{\dots}{}}{=}\frac{(a+b)^2-ab}{3}\qquad Var(X)\overset{\underset{\dots}{}}{=}\frac{(b-a)^2}{12}
\]

\item [(c)] Si mostri che $X\sim\Ec(\lambda)\iff\varphi_X(u)=\dfrac{\lambda}{\lambda-iu}$.

Calcoliamo la funzione caratteristica di $X$:
\begin{gather*}
\begin{aligned}
\varphi_X(u)&=\EE\left[e^{iuX}  \right]=\\
&=\int_\RR e^{iux}\ \lambda e^{-\lambda x}\ \Ind_{[0,+\infty)}(x)\dx=\\
&=\int_0^{+\infty}e^{iux}\ \lambda e^{-\lambda x}\dx=\\
&=\lambda\int_0^{+\infty}e^{(iu-\lambda)x}\dx=\\
&=\lambda\left[\frac{1}{iu-\lambda}\ e^{(iu-\lambda)x}   \right]_0^{+\infty}=\\
&=\displaystyle\frac{\lambda}{iu-\lambda}\left( \underbrace{\lim_{x\to+\infty}e^{(iu-\lambda)x}}_{0}-\underbrace{\lim_{x\to 0}e^{(iu-\lambda)x}}_{1}   \right)=\\
&=\frac{\lambda}{\lambda-iu}
\end{aligned}
\end{gather*}
\begin{nb}
Ma perché $\displaystyle\lim_{t\to+\infty}e^{(iu-\lambda)t}=0$? \\
Considerando il numero complesso $z\coloneqq e^{(iu-\lambda)t}$ possiamo scrivere
\[
z=e^{-\lambda t}\cdot e^{iut}=z_1\cdot z_2
\]
La componente $z_1$ tende a zero per $t\to+\infty$. Invece $z_2$ è un numero complesso appartenente alla circonferenza unitaria in $\CC$, quindi ha modulo unitario. \\
Sapendo che nel confronto tra un esponenziale tendente a zero e un numero complesso limitato a risultare vincitore è proprio il primo, si ha
\[
\displaystyle\lim_{t\to+\infty}z=0
\]
\end{nb}
Quindi
\[
X\sim\Ec(\lambda),\ \lambda>0\iff\varphi_X(u)=\frac{\lambda}{\lambda-iu}
\]
Per quanto riguarda media e varianza
\[
\EE[X]\overset{\underset{\dots}{}}{=}\frac{1}{\lambda} \qquad \EE[X^2]\overset{\underset{\dots}{}}{=}\frac{2}{\lambda^2}\qquad Var(X)\overset{\underset{\dots}{}}{=}\frac{1}{\lambda^2}
\]

\item [(d)] Si mostri che $X$ tale che $-X\sim\Ec(\lambda)\iff\varphi_X(u)=\dfrac{\lambda}{\lambda+iu}$.

Osserviamo che $f_X(x)=\lambda e^{\lambda x}\ \Ind_{(-\infty,0]}(x)$. Allora
\[
\varphi_X(u)=\int_{-\infty}^0e^{iux}\ \lambda e^{\lambda x}\dx=\lambda\left[\frac{1}{\lambda+iu}\ e^{(\lambda+iu)x}   \right]_{-\infty}^{0}=\frac{\lambda}{\lambda+iu}
\]
Ovviamente nell'ultima equaglianza si è usato un discorso analogo al \textbf{NB} precedente.

Si ha
\[
X\text{ tale che }-X\sim\Ec(\lambda)\iff\varphi_X(u)=\dfrac{\lambda}{\lambda+iu}
\]
e
\[
\EE[X]\overset{\underset{\dots}{}}{=}-\frac{1}{\lambda} \qquad \EE[X^2]\overset{\underset{\dots}{}}{=}\frac{2}{\lambda^2}\qquad Var(X)\overset{\underset{\dots}{}}{=}\frac{1}{\lambda^2}
\]
\end{enumerate}

\Soluzione{} %5
\begin{itemize}
\item [(a)] Si scrivano le densità continue di $(X,Y)$, di $X$ e di $Y$. Le variabili aleatorie $X$ e $Y$ sono indipendenti?

La densità congiunta del vettore $(X,Y)$ vale
\[
\fXYxy=\frac{1}{m(R)}\ \Ind_R=\frac{1}{2}\ \Ind_{[0,2]\times [0,1]}(x,y)
\]
Inoltre $X\indep Y$ perché la congiunta fattorizza nelle due marginali
\begin{gather*}
\begin{aligned}
&f_X(x)=\frac{1}{2}\ \Ind_{[0,2]}(x)\implies X\sim\Uc([0,2])\\
&f_Y(y)=\Ind_{[0,1]}(y)\implies Y\sim\Uc([0,1])
\end{aligned}
\end{gather*}

\item [(b)] Si calcolino valore atteso e matrice varianza di $(X,Y)$.

Dato che $X\sim\Uc([0,2]),\ Y\sim\Uc([0,1]),\ X\indep Y$ possiamo subito calcolare
\[
\EE[(X,Y)]=\left(\EE[X],\EE[Y]\right)=\left(1,\frac{1}{2}\right)
\]
\[
Var(X,Y)=\begin{pmatrix}
Var(X) & Cov(X,Y) \\
Cov(X,Y) & Var(Y) \\
\end{pmatrix}=\begin{pmatrix}
1/3 & 0 \\
0 & 1/12 \\
\end{pmatrix}
\]

\item [(c)] Si ricavi la funzione caratteristica di $(X,Y)$ in termini delle funzioni caratteristiche di $X$ e $Y$.

Grazie al teorema (\ref{Fattorizzazione della funzione caratteristica per famiglie di VA}) possiamo calcolare $\varphi_{(X,Y)}$ come
\begin{gather*}
\begin{aligned}
\varphi_{(X,Y)}(u,v)&=\varphi_X(u)\cdot \varphi_Y(v)=\\
&=e^{iu}\ \text{sinc}(u)\cdot e^{\frac{1}{2}iv}\ \text{sinc}\left(v/2\right)=\\
&=e^{i\left(u+\frac{v}{2}  \right)}\ \text{sinc}(u)\ \text{sinc}\left(v/2\right)
\end{aligned}
\end{gather*}

\begin{oss}
Senza usare il teorema (\ref{Fattorizzazione della funzione caratteristica per famiglie di VA}):
\begin{gather*}
\begin{aligned}
\varphi_{(X,Y)}(u,v)&=\EE\left[e^{i\langle(u,v)|(X,Y)  \rangle}   \right]=\EE\left[e^{i(uX+vY)}   \right]=\EE\left[e^{iuX}\ e^{ivY}   \right]\overset{\underset{\indep}{}}{=}\EE\left[e^{iuX}\right]\ \EE\left[e^{ivY}\right]=\\
&=\varphi_X(u)\cdot \varphi_Y(v)=\dots
\end{aligned}
\end{gather*}
\end{oss}

\item [(d)] Siano $Z=2X-Y,\ W=Y-2,\ J=-X$. Il vettore aleatorio $(Z,W,J)$ ammette densità continua?

Ricordiamo che il vettore aleatorio $(Z,W,J)$ è continuo se esiste $f_{(Z,W,J)}:\RR^3\to[0,+\infty)$ misurabile. \\
Tuttavia per come sono definite le tre variabili notiamo che $Z$ è combinazione lineare di $W$ e $J$:
\[
Z=-W-2J-2
\]
Quindi il vettore "vive" in $\RR^2$, che è un sottospazio di $\RR^3$ e pertanto ha misura $m_3=0$. Ciò è assurdo, poiché
\[
\PP((Z,W,J)\in\RR^2)=1\qquad\text{ma}\qquad\PP((Z,W,J)\in\RR^3)=0
\]
Concludiamo quindi che non esiste una funzione densità congiunta e quindi il vettore aleatorio $(Z,W,J)$ non è continuo.

\item [(e)] Si calcolino valore atteso e matrice varianza di $(Z,W,J)$.

Scriviamo $(Z,W,J)$ come trasformazione affine di $(X,Y)$:
\[
\begin{pmatrix}
Z \\
W \\ J
\end{pmatrix}=
\begin{pmatrix}
2X-Y \\Y-2
 \\-X
\end{pmatrix} =\underbrace{\begin{pmatrix}
2 & -1 \\
 0&1  \\
 -1&0  \\
\end{pmatrix}}_{A}\begin{pmatrix}
X \\
Y
\end{pmatrix}+\underbrace{\begin{pmatrix}
0 \\
-2 \\ 0
\end{pmatrix}}_{b}
\]
Allora
\begin{gather*}
\begin{aligned}
&\EE[(Z,W,J)]=A\EE[(X,Y)]+b=\begin{pmatrix} 3/2 \\ -3/2  \\-1 \end{pmatrix}   \\
&Var(Z,W,J)=AVar(X,Y)A^T=\begin{pmatrix}
17/12 & -1/12 & -2/3 \\
-1/12 &  1/12& 0 \\
 -2/3& 0 & 1/3 \\
\end{pmatrix}
\end{aligned}
\end{gather*}

\item [(f)] Si ricavi la funzione caratteristica di $(Z,W,J)$ in termini delle funzioni caratteristiche di $X$ e $Y$.

Grazie al teorema (\ref{t_affine_t}) con $n=2$ e $m=3$ abbiamo
\[
\varphi_{(Z,W,J)}(u)=e^{i\langle u|b\rangle} \ \varphi_{(X,Y)}\left(A^T  u\right)\qquad\forall u\in\RR^3
\]
con
\begin{gather*}
\begin{aligned}
\langle u|b\rangle&=u^Tb=(u_1,u_2,u_3)\begin{pmatrix}
0 \\-2
 \\0
\end{pmatrix}=-2u_2 \\
A^T  u&=\begin{pmatrix}
2 &  0&  -1\\
-1 & 1 & 0 \\
\end{pmatrix}\begin{pmatrix}
u_1 \\
u_2 \\u_3
\end{pmatrix}=\begin{pmatrix}
2u_1-u_3 \\-u_1+u_2
\end{pmatrix}
\end{aligned}
\end{gather*}
Quindi, riprendendo $\varphi_{(X,Y)}$ dal punto (c), abbiamo
\begin{gather*}
\begin{aligned}
\varphi_{(Z,W,J)}(u_1,u_2,u_3)&=e^{-2iu_2}\ \varphi_{(X,Y)}(2u_1-u_3,-u_1+u_2)=\\
&=e^{-2iu_2}\ e^{i\left(2u_1-u_3+\frac{-u_1+u_2}{2}  \right)}\ \text{sinc}(2u_1-u_3)\ \text{sinc}\left((-u_1+u_2)/2\right)=\\
&=e^{i(3u_1-3u_2-2u_3)/2}\ \text{sinc}(2u_1-u_3)\ \text{sinc}\left((-u_1+u_2)/2\right)
\end{aligned}
\end{gather*}

\begin{oss}
Senza usare il teorema (\ref{t_affine_t}):
\begin{gather*}
\begin{aligned}
\varphi_{(Z,W,J)}(u_1,u_2,u_3)&=\EE\left[e^{i\langle(u_1,u_2,u_3)|(Z,W,J)  \rangle}   \right]=\\
&=\EE\left[e^{i(u_1X+u_2W+u_3J)}   \right]=\\
&=\EE\left[e^{i(u_1(2X-Y)+u_2(Y-2)+u_3(-X))}   \right]=\\
&=\EE\left[e^{i(2u_1X-u_1Y+u_2Y-2u_2-u_3X)}   \right]=\\
&=\EE\left[e^{i(2u_1X-u_3X-u_1Y+u_2Y-2u_2)}   \right]=\\
&=e^{-2iu_2}\ \EE\left[e^{i((2u_1-u_3)X+(-u_1+u_2)Y)}   \right]=\\
&=e^{-2iu_2}\ \varphi_{(X,Y)}(2u_1-u_3,-u_1+u_2)=\\
&=\dots
\end{aligned}
\end{gather*}
\end{oss}

\begin{oss}$\\$
Anche se $(Z,W,J)$ non ammette densità congiunta continua la funzione caratteristica è ben definita, anzi, caratterizza proprio $P^{(Z,W,J)}$.
\end{oss}

\item [(g)] Si ricavino le funzioni caratteristiche di $(Z,W)$ e $(W,J)$ in termini della funzione caratteristica di $(Z,W,J)$.

Il trucco sta nel vedere i vettori $(Z,W)$ e $(W,J)$ come trasformazioni affini di $(Z,W,J)$:
\[
\begin{pmatrix}
Z \\ W
\end{pmatrix}=\underbrace{\begin{pmatrix}
1 &0  &0  \\
 0&1  &0  \\
\end{pmatrix}}_{B}\begin{pmatrix}
Z \\W
 \\J
\end{pmatrix}
\]
\[
\begin{pmatrix}
W \\ J
\end{pmatrix}=\underbrace{\begin{pmatrix}
0 &1  &0  \\
 0& 0 &1  \\
\end{pmatrix}}_{C}\begin{pmatrix}
Z \\W
 \\J
\end{pmatrix}
\]
Usando sempre il teorema (\ref{t_affine_t}):
\begin{gather*}
\begin{aligned}
\varphi_{(Z,W)}(u_1,u_2)&=e^{0}\ \varphi_{(Z,W,J)}(B^T(u_1,u_2))=\\
&=\varphi_{(Z,W,J)}(u_1,u_2,0)=\\
&=e^{3i(u_1-u_2)/2}\ \text{sinc}(2u_1)\ \text{sinc}\left((-u_1+u_2)/2\right)\\
\varphi_{(W,J)}(u_2,u_3)&=e^{0}\ \varphi_{(Z,W,J)}(C^T(u_2,u_3))=\\
&=\varphi_{(Z,W,J)}(0,u_2,u_3)=\\
&=e^{-i(3u_2+2u_3)/2}\ \text{sinc}(u_3)\ \text{sinc}\left(u_2/2\right)
\end{aligned}
\end{gather*}

\end{itemize}

\Soluzione{} %6
\begin{itemize}
\item [(a)] Mostrare che $X_1+\cdots+X_n\sim\Pc(\lambda_1+\cdots+\lambda_n)$.

È già stato dimostrato che se $X\sim\Pc(\lambda)$ allora
\[
\varphi_X(u)=e^{\lambda(e^{iu}-1)}
\]
Detta quindi $Y=X_1+\cdots+X_n$, per il corollario (\ref{Legge della somma di variabili indipendenti}) si ha
\[
\varphi_Y(u)=\prod_{k=1}^n\varphi_{X_k}(u)=\prod_{k=1}^n e^{\lambda_k(e^{iu}-1)}=e^{\left(\sum_{k=1}^n \lambda_k  \right)\left(e^{iu}-1   \right)}
\]
Grazie a questo possiamo concludere che 
\[
\begin{cases}X_1,\dots,X_n\text{ indipendenti}\\X_k\sim\Pc(\lambda_k),\ \lambda_k>0\ \forall k\end{cases} \implies X_1+\cdots+X_n\sim\Pc(\lambda_1+\cdots+\lambda_n)
\]

\item [(b)] Si supponga ora che $X_1,X_2,X_3\iid\Pc(\lambda)$. Calcolare $\PP(X_1+X_2+X_3\geq 3\,|\,X_1\geq 1)$. 

Per calcolare $\PP(X_1+X_2+X_3\geq 3\,|\,X_1\geq 1)$ osserviamo che $X_1,X_2,X_3$ sono VA discrete che q.c. assumono valori in $\{0,1,\dots  \}$. Dunque
\[
\PP(X_1+X_2+X_3\geq 3\,|\,X_1\geq 1)=1-\PP(X_1+X_2+X_3< 3\, | \,X_1\geq 1)=1-\frac{\PP(X_1+X_2+X_3< 3\, , \,X_1\geq 1)}{\PP(X_1\geq 1)}
\]
Ma affinché $\PP(X_1+X_2+X_3< 3\, , \,X_1\geq 1)$ deve capitare uno dei seguenti casi
\begin{gather*}
\begin{aligned}
\{ X_1,X_2,X_3\}&=\{1,0,0  \} \\
&=\{2,0,0  \} \\
&=\{1,1,0  \} \\
&=\{ 1,0,1 \}
\end{aligned}
\end{gather*}
Grazie all'ipotesi di indipendenza
\begin{gather*}
\begin{aligned}
&\PP(\{1,0,0  \})=\lambda e^{-3\lambda} \\
&\PP(\{2,0,0  \})=\frac{1}{2}\lambda^2 e^{-3\lambda} \\
&\PP(\{1,1,0  \})=\PP(\{ 1,0,1 \})=\lambda^2 e^{-3\lambda} \\
\implies & \PP(X_1+X_2+X_3< 3\, , \,X_1\geq 1)=\lambda e^{-3\lambda}\left(1+\frac{1}{2}\lambda +2\lambda   \right)=\lambda e^{-3\lambda}\left(1+\frac{5}{2}\lambda  \right)
\end{aligned}
\end{gather*}
Inoltre $\PP(X_1\geq 1)=1-\PP(X_1=0)=1-e^{-\lambda}$.

Allora
\[
\PP(X_1+X_2+X_3\geq 3\,|\,X_1\geq 1)=1-\frac{\lambda e^{-3\lambda}\left(1+\frac{5}{2}\lambda\right)}{1-e^{-\lambda}}
\]

\end{itemize}

\Soluzione{} %7
Dalla funzione di ripartizione di $\overline{X}_n$ possiamo ricondurci a
\begin{gather*}
\begin{aligned}
\varphi_{\overline{X}_n}(u)&=\varphi_{\frac{1}{n}\sum_{k=1}^n X_k}(u)=\\
&=\varphi_{\sum_{k=1}^n X_k}\left(\frac{u}{n}  \right)=\\
&\overset{\underset{(\ref{Legge della somma di variabili indipendenti})}{}}{=}\prod_{k=1}^n \varphi_{X_k}\left(\frac{u}{n}  \right)
\end{aligned}
\end{gather*}
Ma dato che le $X_k$ sono identicamente distribuite, cioè $\varphi_{X_i}=\varphi_{X_j}\ \forall i,j$ allora
\[
\varphi_{\overline{X}_n}(u)=\prod_{k=1}^n \varphi_{X_k}\left(\frac{u}{n}  \right)=\left(\varphi_{X_1}\left(\frac{u}{n}\right)\right)^n
\] 
che conclude la dimostrazione.
\\
\begin{nb}
Al posto di $X_1$ sarebbe andata bene qualsiasi altra $X_k$.
\end{nb}

\Soluzione{} %8
Prima di procedere dimostriamo il risultato notevole
\[
X\sim\Gamma(\alpha,\lambda),\ \alpha>0,\ \lambda>0\iff\varphi_X(u)=\left( \frac{\lambda}{\lambda-iu} \right)^\alpha
\]
Ricordando che
\begin{gather*}
\begin{aligned}
&f_X(x)=\frac{\lambda^\alpha}{\Gamma(\alpha)}\ x^{\alpha-1}\ e^{-\lambda x}\ \Ind_{(0,+\infty)}(x) \\
&\Gamma(\alpha)=\int_{0}^{+\infty} t^{\alpha-1}\ e^{-t}\dt \\
&\Gamma\left(\frac{1}{2}\right)=\sqrt{\pi}
\end{aligned}
\end{gather*}
abbiamo
\begin{gather*}
\begin{aligned}
\varphi_X(u)&=\EE\left[e^{iuX}  \right]=\\
&=\int_\RR e^{iux}\ \frac{\lambda^\alpha}{\Gamma(\alpha)}\ x^{\alpha-1}\ e^{-\lambda x}\ \Ind_{(0,+\infty)}(x)\dx=\\
&=\frac{\lambda^\alpha}{\Gamma(\alpha)}\int_{0}^{+\infty} x^{\alpha-1}\ e^{-(\lambda-iu) x}\dx=\\
&=\{ y\coloneqq(\lambda-iu)x \}=\\
&=\frac{\lambda^\alpha}{\Gamma(\alpha)}\int_{0}^{+\infty} \frac{y^{\alpha-1}}{(\lambda-iu)^{\alpha-1}}\ e^{-y}\ \frac{\text{d}y}{\lambda-iu}=\\
&=\frac{\lambda^\alpha}{\Gamma(\alpha)}\ \frac{1}{(\lambda-iu)^{\alpha}}\int_{0}^{+\infty} y^{\alpha-1}\ e^{-y}\dy =\\
&=\left(\frac{\lambda}{\lambda-iu}\right)^\alpha\ \frac{1}{\Gamma(\alpha)}\underbrace{\int_{0}^{+\infty} y^{\alpha-1}\ e^{-y}\dy}_{\Gamma(\alpha)}=\\
&=\left(\frac{\lambda}{\lambda-iu}\right)^\alpha
\end{aligned}
\end{gather*}
\begin{oss}
Se $Z\sim\Ec(\lambda)$ allora $Z\sim\Gamma(1,\lambda)$, in quanto
\[
\varphi_{\Gamma(1,\lambda)}(u)=\left(\frac{\lambda}{\lambda-iu}\right)^1=\frac{\lambda}{\lambda-iu}=\varphi_{\Ec(\lambda)}(u)
\]
\end{oss}
\begin{enumerate}

\item [(a)] Si mostri che $X\sim\Gamma(\alpha,\lambda)\indep Y\sim\Gamma(\beta,\lambda)\implies X+Y\sim\Gamma(\alpha+\beta,\lambda)$.

Grazie al corollario (\ref{Legge della somma di variabili indipendenti}):
\[
\varphi_{X+Y}(u)=\varphi_X(u)\cdot\varphi_Y(u)=\left(\frac{\lambda}{\lambda-iu}\right)^{\alpha+\beta}
\]
che è proprio la funzione caratteristica di una $\Gamma(\alpha+\beta,\lambda)$.

Quindi in generale
\[
\begin{cases}X_1,\dots,X_n\text{ indipendenti}\\X_k\sim\Gamma(\alpha_k,\lambda),\ \alpha_k>0\ \forall k,\ \lambda>0\end{cases} \implies X_1+\cdots+X_n\sim\Gamma(\alpha_1+\cdots+\alpha_n,\lambda)
\]

\item [(b)] Si mostri che $X_1,\dots,X_n\iid\Ec(\lambda)\implies X_1+\cdots+X_n\sim\Gamma(n,\lambda)$.

Avendo già osservato che $\Ec(\lambda)\equiv\Gamma(1,\lambda)$, e sfruttando il punto (a), possiamo dire che
\[
X_1,\dots,X_n\sim\Gamma(\underbrace{1+\cdots+1}_{n\text{ volte}},\lambda)=\Gamma(n,\lambda)
\]
Quindi
\[
X_1,\dots,X_n\iid\Ec(\lambda),\ n\in\NN,\ \lambda>0\implies X_1+\cdots+X_n\sim\Gamma(n,\lambda)
\]

\item [(c)] Si mostri che $Z\sim\Nc(0,1)\implies Z^2\sim\Gamma\left(\frac{1}{2},\frac{1}{2}\right)=\chi^2(1)$.

Ricordando che
\[
f_Z(t)=\frac{1}{\sqrt{2\pi}}\ e^{-t^2/2}\ \Ind_\RR(x)
\]
calcoliamo la funzione di ripartizione di $Z^2$:
\[
F_{Z^2}(t)=\PP(Z^2\leq t)=\PP(-\sqrt{t}\leq Z\leq\sqrt{t})=F_Z(\sqrt{t})-F_Z(-\sqrt{t})
\]
Allora
\begin{gather*}
\begin{aligned}
f_{Z^2}(t)&=f_Z(\sqrt{t})\ (\sqrt{t})'-f_Z(-\sqrt{t})\ (-\sqrt{t})ì=\\
&=\frac{1}{\sqrt{2\pi}}\ e^{-t^2/2}\left(\frac{1}{2\sqrt{t}}-\frac{1}{-2\sqrt{t}}  \right)=\\
&=\frac{1}{\sqrt{2\pi t}}\ e^{-t^2/2}=\\
&=\frac{\left( \frac{1}{2}  \right)^{1/2}}{\Gamma\left( \frac{1}{2}  \right)}\ t^{\frac{1}{2}-1}\ e^{-\frac{1}{2}t}
\end{aligned}
\end{gather*}
che è proprio la densità di una $\Gamma\left(\frac{1}{2},\frac{1}{2} \right)$.

Quindi
\[
Z\sim\Nc(0,1)\implies Z^2\sim\Gamma\left(\frac{1}{2},\frac{1}{2}\right)=\chi^2(1)
\]
\begin{oss}
Il calcolo mediante la funzione caratteristica è lasciato al lettore.
\end{oss}

\item [(d)] Si mostri che $Q=Z_1^2+\cdots+Z_n^2$, con $Z_1,\dots,Z_n\iid\Nc(0,1)\implies Q\sim\Gamma\left(\frac{n}{2},\frac{1}{2}\right)=\chi^2(n)$.

Sappiamo già che $\forall k=1:n\ \ Z_k^2\sim\Gamma\left(\frac{1}{2},\frac{1}{2}\right)=\chi^2(1)$ e che $Z_1^2,\dots,Z_n^2$ sono indipendenti perché lo sono per ipotesi $Z_1,\dots,Z_n$. Allora per il punto (a) possiamo concludere che
\[
Q\sim\Gamma\left(\underbrace{\frac{1}{2}+\cdots+\frac{1}{2}}_{n\text{ volte}}, \frac{1}{2}  \right)=\Gamma\left(  \frac{n}{2},\frac{1}{2} \right)=\chi^2(n)
\]
In generale
\[
Z_1,\dots,Z_n\iid\Nc(0,1)\implies Z_1^2+\cdots+Z_n^2\sim\Gamma\left(\frac{n}{2},\frac{1}{2}\right)=\chi^2(n)
\]
\end{enumerate}

\Soluzione{} %9
\begin{enumerate}
\item [(a)] Si considerino le variabili aleatorie $T=X+Y$ e $U=X/(X+Y)$. Si mostri che $(T,U)$ ammette densità continua e la si calcoli.

Osserviamo che, in quanto somma di $\Gamma$, $T\in(0,+\infty)$ q.c.; invece per quanto riguarda $U$ notiamo che il denominatore non si annulla mai perché anch'esso somma di $\Gamma$, quindi $U$ è ben definita; inoltre guardando anche il numeratore possiamo dire che $U\in(0,1)$ q.c..

Detto questo, considerando
\[
\begin{pmatrix}
t \\ u
\end{pmatrix}=g\begin{pmatrix}
x \\ y
\end{pmatrix}=\begin{pmatrix}
x+y \\ \frac{x}{x+y}
\end{pmatrix}
\qquad\text{con}\quad g:(0,+\infty)^2\to(0,+\infty)\times (0,1)
\]
possiamo dire che
\begin{itemize}
\item $g$ iniettiva;
\item $g\in\Cu$;
\item lo jacobiano di $g$ vale
\[
J_g(x,y)=\begin{bmatrix}
 1& 1 \\
 \dfrac{y}{(x+y)^2}& \dfrac{-x}{(x+y)^2} \\
\end{bmatrix}
\]
e quindi il suo determinante è
\[
|J_g(x,y)|=\dfrac{-x}{(x+y)^2}-\dfrac{y}{(x+y)^2}=-\dfrac{1}{x+y}
\]
che è $\neq 0\ \ \forall(x,y)\in(0,+\infty)^2$.
\end{itemize}
Possiamo quindi procedere a esplicitare l'inversa di $g$:
\[
\begin{cases}t=x+y\\u=\frac{x}{x+y}\end{cases}
\iff
\begin{cases}\frac{x}{t}=u\\y=t-x\end{cases}
\iff
\begin{cases}x=tu\\y=t(1-u)\end{cases}
\]
cioè
\[
\begin{pmatrix}
x \\ y
\end{pmatrix}=g^{-1}\begin{pmatrix}
t \\ u
\end{pmatrix}=\begin{pmatrix}
tu \\ t(1-u)
\end{pmatrix}
\qquad\text{con}\quad g^{-1}:(0,+\infty)\times (0,1)\to(0,+\infty)^2
\]

Allora possiamo applicare la formula di Jacobi:
\begin{gather*}
\begin{aligned}
f_{(T,U)}(t,u)&=\fXY(g^{-1}(t,u))\ \frac{1}{|J_g(g^{-1}(t,u))|}\ \Ind_{(0,+\infty)\times(0,1)}(t,u)=\\
&=\fXY(tu,t(1-u))\ \frac{1}{\frac{1}{tu+t(1-u)}}\  \Ind_{(0,+\infty)\times(0,1)}(t,u)=\\
&\overset{\underset{\indep}{}}{=}t\ f_X(tu)\ f_Y(t(1-u))\ \Ind_{(0,+\infty)\times(0,1)}(t,u)=\\
&=\frac{\lambda^{\alpha+\beta}}{\Gamma(\alpha)\,\Gamma(\beta)}\ t^{\alpha+\beta-1}\ e^{-\lambda t}\ u^{\alpha-1}\ (1-u)^{\beta-1}\ \Ind_{(0,+\infty)\times(0,1)}(t,u)
\end{aligned}
\end{gather*}
che è la densità congiunta continua del vettore aleatorio $(T,U)$.

\item [(b)] Si provi che $T\indep U$.

Risulta evidente che la densità congiunta appena calcolata fattorizzi in
\begin{gather*}
\begin{aligned}
&f_T(t)=\frac{\lambda^{\alpha+\beta}}{\Gamma(\alpha+\beta)}\ t^{\alpha+\beta-1}\ e^{-\lambda t}\ \Ind_{(0,+\infty)}(t) \\
&f_U(u)=\frac{\Gamma(\alpha+\beta)}{\Gamma(\alpha)\,\Gamma(\beta)}\ u^{\alpha-1}\ (1-u)^{\beta-1}\ \Ind_{(0,1)}(u)
\end{aligned}
\end{gather*}
e dunque $T\indep U$.

\item [(c)] Si provi che $T\sim\Gamma(\alpha+\beta,\lambda)$ e che $U\sim\text{Beta}(\alpha,\beta)$, ossia $U$ ha distribuzione Beta di parametri $\alpha$ e $\beta$, la cui densità continua è data da
\[
f_U(u)=\frac{\Gamma(\alpha+\beta)}{\Gamma(\alpha)\ \Gamma(\beta)}\ u^{\alpha-1}\ (1-u)^{\beta-1}\ \Ind_{(0,1)}(u).
\]

Abbiamo già provato (punto (a) esercizio 8) che $T\sim\Gamma(\alpha+\beta,\lambda)$; invece come diretta conseguenza del punto (b) soprastante abbiamo che $U\sim\text{Beta}(\alpha,\beta)$. 

\begin{oss}[Distribuzione Beta]$\\$
In teoria delle probabilità e in statistica la distribuzione Beta è una distribuzione di probabilità continua definita da due parametri $\alpha>0$ e $\beta>0$  sull'intervallo unitario $[0,1]$. \\
Questa distribuzione trova particolare utilizzo nella inferenza bayesiana perché governa la probabilità uniforme $p\in[0,1]$ di successo di un processo di Bernoulli dopo aver osservato $\alpha -1$ \emph{successi} e $ \beta -1$ \emph{fallimenti}. in altre parole modellizza l'incertezza del parametro $p$ a posteriori. 

Viene definita in termini della \emph{funzione beta} $(\mathrm{B})$ di Eulero (prende infatti da qui il nome):
\[
X\sim\text{Beta}(\alpha,\beta)\implies f_X(x)=\frac{1}{\mathrm{B}(\alpha,\beta)}\ x^{\alpha-1}\ (1-x)^{\beta-1}\ \Ind_{(0,1)}(x)
\]
dove
\[
\mathrm{B}(\alpha,\beta)=\int_0^1 x^{\alpha-1}\ (1-x)^{\beta-1}\dx
\]
Ma abbiamo appena dimostrato che si può anche scrivere
\[
f_X(x)=\frac{\Gamma(\alpha+\beta)}{\Gamma(\alpha)\ \Gamma(\beta)}\ x^{\alpha-1}\ (1-x)^{\beta-1}\ \Ind_{(0,1)}(x)
\]
dunque deduciamo che possiamo esmprimere la \emph{funzione beta} di Eulero in termini della funzione $\Gamma$:
\[
\mathrm{B}(\alpha,\beta)=\frac{\Gamma(\alpha)\ \Gamma(\beta)}{\Gamma(\alpha+\beta)}
\]
\end{oss}

\item [(d$^\ast$)] Sia $(X_n)_{n\in\NN}$ una successione di variabili aleatorie, con $X_1,X_2,\dots\iid\Ec(\lambda)$. Posto
\[
S_n=X_1+\dots+X_n
\]
si provi che per ogni coppia di interi $(k,n)$ con $1\leq k\leq n$, la variabile aleatoria $S_{k/n}=\dfrac{S_k}{S_n}$ è indipendente da $S_n$. Sfruttando quest'ultimo risultato si calcoli il valore atteso di $S_{k/n}$.

Se banalmente $k=n$ allora $S_{k/n}=1$, che è indiendente da $S_{n}$. Se invece $k<n$ allora 
\[
S_n=\underbrace{X_1+\cdots+X_k}_{S_k}+\underbrace{X_{k+1}+\cdots+X_n}_{Z}=S_k+Z
\]
Ma grazie al punto (a) dell'esercizio 8 sappiamo che $S_n\sim\Gamma(n,\lambda),\ S_k\sim\Gamma(k,\lambda),\ Z\sim\Gamma(n-k,\lambda)$ e che quindi per valere $S_k+Z\sim\Gamma(k+n-k,\lambda=S_n$ deve essere $S_k\indep Z$ (del resto, insiemi disgiunti di una famiglia di VA indipendenti sono tra loro indipendenti). \\
A questo punto sfruttiamo il punto (a) di questo esercizio:
\[
\begin{cases}
X=S_k \\
Y=Z
\end{cases}
\implies
\begin{cases}
T=X+Y=S_n \\
U=X/(X+Y)=S_{k/n}
\end{cases}
\]
Per il punto (b) si ha $T\indep U$ e quindi $S_n\indep S_{k/n}$, che conclude la dimostrazione.

Per quanto riguarda il valore atteso:
\[
\EE[S_k]=\EE\left[\frac{S_k}{S_n}\ S_n   \right]\overset{\underset{\indep}{}}{=}\EE[S_{k/n}]\ \EE[S_n]
\]
\[
\implies \EE[S_{k/n}]=\frac{\EE[S_k]}{\EE[S_n]}=\frac{k/\lambda}{n/\lambda}=\frac{k}{n}
\]
\end{enumerate}

\Soluzione{} %10

\Soluzione{} %11

\Soluzione{} %12

\Soluzione{} %13

\Soluzione{} %14

\Soluzione{} %15

\Soluzione{} %16

\Soluzione{} %17

\Soluzione{} %18

\Soluzione{} %19

\Soluzione{} %20

\Soluzione{} %21

\Soluzione{} %22

\chapter{Leggi condizionali}
%!TEX root = ../main.tex



\chapter{Convergenza di variabili aleatorie}
%!TEX root = ../main.tex



\chapter{Legge dei Grandi Numeri. Teorema Centrale del Limite}
%!TEX root = ../main.tex



\chapter{Catene di Markov}
%!TEX root = ../main.tex

\chapter{Catene di Markov}

\ParteEsercizi

\Esercizio{}

\ParteSoluzioni

\Soluzione{}


\end{document}