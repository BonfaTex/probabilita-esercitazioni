%!TEX root = ../main.tex


\section{}
\subsection{}

Si consideri lo spazio di probabilità $( \Omega ,\mathcal{A} ,\mathbb{P}) =(\mathbb{R} ,\mathcal{B} ,m_{0,1})$, dove $m_{0,1} =m( B\cap ( 0,1]) ,\forall B\in \mathcal{B}$, ed $m$ è la misura di Lebesgue sui boreliani. Si considerino inoltre le funzioni $X,Y,Z:\Omega =\mathbb{R}\rightarrow \mathbb{R}$ definite sull'intervallo $( 0,1)$ nel modo seguente:
\begin{equation*}
X( \omega ) =\frac{1}{\omega } ,\ \ \ \ Y( \omega ) =\frac{1}{\sqrt{\omega }} ,\ \ \ \ Z( \omega ) =\log \omega ,
\end{equation*}
per ogni $\omega \in ( 0,1)$. Per quanto riguarda il valore di $X,Y,Z$ su $\mathbb{R} \smallsetminus ( 0,1)$, si sa solamente che $X,Y,Z$ sono costanti su tale insieme.
\begin{enumerate}
\item Mostrare che $X$, $Y$ e $Z$ sono variabili aleatorie ben definite q.c.
\item Determinare le distribuzioni di $X$, $Y$ e $Z$, mostrando in particolare che si tratta di variabili aleatorie continue.
\item Calcolare le densità di probabilità di $X$, $Y$ e $Z$.
\item Calcolare i valori attesi di $X$, $Y$ e $Z$.
\item Esibire un'altra variabile aleatoria $\tilde{X}$ con la stessa legge di $X$, ma definita su un altro spazio di probabilità $\left(\tilde{\Omega } ,\tilde{\mathcal{A}} ,\tilde{\mathbb{P}}\right)$.
\end{enumerate}
\subsection{}

Data una variabile aleatoria reale continua $X$ con densità $f$ simmetrica rispetto a $\mu \in \mathbb{R}$,

si mostri che
\begin{enumerate}
\item $\mu $ è una mediana,
\item Se $\mathbb{E}[| X| ] < +\infty $, allora $\mu =\mathbb{E}[ X]$.
\end{enumerate}
\subsection{(Distribuzione di Cauchy).}

Data una variabile aleatoria di Cauchy $X$, con densità
\begin{equation*}
f( x) =\frac{1}{\pi \left( 1+( x-\alpha )^{2}\right)} ,\ \ \ \ \alpha \in \mathbb{R} ,
\end{equation*}
si calcolino:
\begin{enumerate}
\item la mediana,
\item $\mathbb{E}[ X]$,
\item $\mathbb{E}\left[ X^{2}\right]$.
\end{enumerate}
\subsection{*}

Data $X$ variabile aleatoria positiva e continua, denotata con $F$ la sua funzione di \ ripartizione, si mostri che
\begin{equation*}
\mathbb{E}[ X] =\int\limits _{0}^{+\infty }( 1-F( x)) dx.
\end{equation*}
\subsection{(Distribuzione esponenziale).}

Si consideri $X\sim \mathcal{E}( \lambda )$, variabile esponenziale di parametro $\lambda  >0$, quindi con densità


\begin{equation*}
f( x) =\lambda e^{-\lambda x} I_{[ 0,+\infty )}( x) ,\ \ \ \ x\in \mathbb{R} .
\end{equation*}
\begin{enumerate}
\item Calcolare la moda di $X$.
\item Calcolare il valore atteso $\mathbb{E}[ X]$ e la varianza $\mathrm{Var}( X)$.
\item Calcolare la funzione di ripartizione $F_{X}$ di $X$.
\item (Proprietà di assenza di memoria) Mostrare che $\mathbb{P}( X >s+t\ |\ X >s) =\mathbb{P}( X >t)$ per ogni $s,t >0$.
\item * Si inverta il risultato appena ottenuto: se $T$ è una variabile aleatoria reale strettamente positiva tale che $\mathbb{P}( T >s+t\ |\ T >s) =\mathbb{P}( T >t)$ per ogni $s,t >0$, allora $T\sim \mathcal{E}( \nu )$ con $\nu =-\log(\mathbb{P}( T >1))$.
\end{enumerate}
\subsection{(Distribuzione Gamma).}

Si consideri $X\sim \Gamma ( \alpha ,\lambda )$, variabile aleatoria di distribuzione Gamma di parametri $\alpha  >0$ e $\lambda  >0$. Si ha quindi
\begin{equation*}
f( x) =\frac{\lambda ^{\alpha }}{\Gamma ( \alpha )} x^{\alpha -1} e^{-\lambda x} I_{( 0,+\infty )}( x) ,
\end{equation*}
dove $\Gamma $ è la funzione Gamma (di Eulero) definita da
\begin{equation*}
\Gamma ( \alpha ) :=\int\limits _{0}^{+\infty } t^{\alpha -1} e^{-t} dt,\ \ \ \ \forall \alpha  >0.
\end{equation*}
Si ricordi che
\begin{equation*}
\Gamma ( \alpha +1) =\alpha \Gamma ( \alpha ) ,\ \ \ \ \forall \alpha  >0,\ \ \ \ \ \ \ \ \ \ \ \ \Gamma ( 1) =1,
\end{equation*}
da cui $\Gamma ( n+1) =n!,\ \forall n\in \mathbb{N}$. Infine $\Gamma ( 1/2) =\sqrt{\pi }$.
\begin{enumerate}
\item Si verifichi che $f$ è una densità di probabilità.
\item Si calcolino le mode di $f$.
\item Si calcolino i momenti $\mathbb{E}\left[ X^{k}\right] ,\ k\in \mathbb{Z} ,\ \alpha +k >0$, e la varianza $\mathrm{Var}( X)$.
\item Dato $c >0$, si mostri che $Y=cX\sim \Gamma ( \alpha ,\lambda /c)$.
\item Per $\alpha =n/2,\ n\in \mathbb{N}$, si determini la relazione fra i punti percentuali di una $\Gamma \left(\frac{n}{2} ,\lambda \right)$ e quelli di una $\chi ^{2}( n) =\Gamma \left(\frac{n}{2} ,\frac{1}{2}\right)$.
\item Per $\alpha =n\in \mathbb{N}$, si mostri che la funzione di ripartizione di $X$ è data da
\end{enumerate}
\begin{equation*}
F( t) =\left( 1-\sum\limits _{k=0}^{n-1} e^{-\lambda t}\frac{( \lambda t)^{k}}{k!}\right) I_{( 0,+\infty )}( t) .
\end{equation*}
\subsection{(Distribuzione normale o gaussiana).}

Sia $X\sim \mathcal{N}\left( \mu ,\sigma ^{2}\right)$.
\begin{enumerate}
\item Si calcolino media e varianza di $X$.
\item Si esprimano i punti percentuali $c_{\alpha }$ di $X$ in funzione di quelli di una normale standard.
\end{enumerate}

Si consideri $Y$ di \textit{legge lognormale} di parametri $\mu $ e $\sigma ^{2}$, ovvero $Y=e^{X}$.
\begin{enumerate}
\item Calcolare la distribuzione della variabile aleatoria $Y$, i momenti (ossia $\mathbb{E}[ Y_{k}] ,k\in \mathbb{Z}$) e la varianza.
\item Si calcoli $\mathbb{E}\left[ Xe^{X}\right]$.
\end{enumerate}

Si consideri $Z$ di \textit{legge normale standard}, ossia $Z\sim \mathcal{N}( 0,1)$.
\begin{enumerate}
\item Calcolare la distribuzione della variabile aleatoria $Z^{2}$.
\item Calcolare la legge della variabile aleatoria $| Z| $, il valore atteso, la varianza e i quantili.
\item Si calcoli $\mathbb{E}\left[ Z^{4}\right]$ e $\mathbb{E}\left[ e^{tZ^{2}}\right] ,t\in \mathbb{R}$.
\end{enumerate}
\subsection{}

Sia $X$ una variabile aleatoria con funzione di ripartizione $F$.
\begin{enumerate}
\item Si calcoli la legge di $Y=| X| $.
\item Nel caso in cui $F$ ammetta derivata continua, si mostri che $Y=| X| $ è continua e se ne calcoli la densità.
\item Nel caso $X\sim \mathcal{N}( 0,1)$, si confronti il risultato ottenuto nei punti $( 1)$ e $( 2)$ con quanto trovato nel punto $( 6)$ dell'esercizio $7$.
\end{enumerate}
\subsection{}
\begin{enumerate}
\item Dati $U\sim U(( 0,1))$ e $\lambda  >0$, si mostri che $X=-\frac{1}{\lambda }\log U\sim \mathcal{E}( \lambda )$.
\item Data $X\sim F$ con $F$ invertibile, si mostri che $Y=F( X) \sim U(( 0,1))$.
\item Date $U\sim U(( 0,1))$ e una funzione di ripartizione $F$ invertibile, si mostri che $X=F^{-1}( U) \sim F$.
\item * Data $X\sim F$, con $F$ continua, si mostri che $Y=F( X) \sim U(( 0,1))$. [Suggerimento: sia $t\in \left( 0,1\right)$ e $G\left( t\right) =\inf\left\{x:F\left( x\right) \geq t\right\}$. Utilizzando la continuità di $F$, si mostri che $F\left( G\left( t\right)\right) =t$. Si noti infine che $F\left( x\right) \geq t\iff x\geq G\left( t\right)$.]
\item * Date $U\sim U\left(\left( 0,1\right)\right)$ e una funzione di ripartizione $F$ continua, definita $G\left( t\right) =\inf\left\{x:F\left( x\right) \geq t\right\}$, si mostri che $X=G\left( U\right)$ è una variabile aleatoria di legge $F$.
\item * Date $U\sim U(( 0,1))$ e una distribuzione discreta $P_{0}$ su $\left(\mathbb{R} ,\mathcal{B}\right)$, si trovi una trasformazione misurabile $G$ tale che $X=G\left( U\right) \sim P_{0}$.
\end{enumerate}
\subsubsection{}
\begin{enumerate}
\item Allora\begin{equation*}
U\sim U\left(\left( 0,1\right)\right) \implies F_{U}\left( x\right) =\begin{cases}
0, & x< 0\\
x, & 0\leq x< 1\\
1, & x\geq 1
\end{cases}
\end{equation*}

a partire da questa ci troviamo\begin{gather*}
F_{X}\left( t\right) :=\mathbb{P}\left( X\leq t\right) =\mathbb{P}\left( -\frac{1}{\lambda }\log U\leq t\right)\\
-\frac{1}{\lambda }\log U\leq t\iff \log U\geq -\lambda t\iff U\geq e^{-\lambda t}\\
=\mathbb{P}\left( U\geq e^{-\lambda t}\right) =1-\underbrace{\mathbb{P}\left( U< e^{-\lambda t}\right)}_{F_{U}\left( e^{-\lambda t}\right)} =\\
=1-\begin{cases}
0, & e^{-\lambda t} < 0\ MAI\\
e^{-\lambda t} , & 0\leq e^{-\lambda t} < 1\iff t >0\\
1, & e^{-\lambda t} \geq 1\iff t\leq 0
\end{cases}\\
=\begin{cases}
1-e^{-\lambda t} , & t >0\\
0, & t\leq 0
\end{cases} =\left( 1-e^{-\lambda t}\right) I_{\left( 0,+\infty \right)}\left( t\right)
\end{gather*}

Quindi se $U\sim U\left(\left( 0,1\right)\right)$ e $X=-\frac{1}{\lambda }\log U$ allora $F_{X}\left( t\right) =\left( 1-e^{-\lambda t}\right) I_{\left( 0,+\infty \right)}\left( t\right)$, che è la funzione di ripartizione di un'esponenziale, ovvero $X\sim \mathcal{E}\left( \lambda \right)$.

A partire dalla distribuzione uniforme si può ottenere qualunque altra distribuzione.
\item 
\end{enumerate}
\subsection{}

Data una variabile aleatoria continua, non negativa $X$, con densità di probabilità $f_{X}$ continua su $\left( 0,+\infty \right)$, definiamo "tasso di fallimento" la funzione
\begin{equation*}
h_{X}( t) =\lim _{\varepsilon \rightarrow 0^{+}}\frac{\mathbb{P}( t< X< t+\varepsilon \ |\ X >t)}{\varepsilon } ,\ \ \ \ t >0.
\end{equation*}
Interpretando $X$ come un tempo di attesa, il tasso di fallimento rappresenta la probabilità istantanea di arrivo, sapendo che l'attesa è durata fino al tempo $t$.
\begin{enumerate}
\item Dimostrare che vale la seguente relazione:\begin{equation*}
h_{X}( t) =\frac{f_{X}( t)}{1-F_{X}( t)} ,\ \ \ \ t >0,
\end{equation*}

dove $F_{X}$ indica la funzione di ripartizione di $X$.
\item Dimostrare che $F_{X}( t) =\left( 1-\exp\left( -\int _{0}^{t} h_{X}( s) ds\right)\right) I_{( 0,+\infty )}( t)$.
\item Si prenda $X$ con legge esponenziale di parametro $\lambda $. Determinare $h_{X}$.
\item Si prenda $Y$ con legge Weibull di parametri $\alpha  >0$ e $\lambda  >0$, vale a dire\begin{equation*}
f_{Y}( t) =\alpha \lambda ^{\alpha } t^{\alpha -1} e^{-( \lambda t)^{\alpha }} I_{( 0,+\infty )}( t) .
\end{equation*}

Determinare $F_{Y}$ e $h_{Y}$.
\item Discutere il significato dei risultati ottenuti nei punti $( 3)$ e $( 4)$ e il loro legame.
\end{enumerate}
\subsection{}

Il raggio $R$ di un certo tipo di particella inquinante, espresso in micron, è una variabile aleatoria con densità di probabilità
\begin{equation*}
f( x) =\begin{cases}
c\ x\ e^{-x^{2}} , & \text{per} \ x >0,\\
0, & \text{per} \ x\leq 0.
\end{cases}
\end{equation*}
\begin{enumerate}
\item Determinare il valore della costante reale $c$.
\item Calcolare la funzione di ripartizione di $R$.
\item Calcolare la mediana e la media di $R$.
\item Calcolare la probabilità che una particella abbia un raggio superiore a $2$ micron.
\item Calcolare la legge della variabile aleatoria $X$ che vale $1$ se $R< 2$ e $0$ altrimenti.
\item Supponendo che le particelle inquinanti siano delle sfere, calcolare la probabilità che una particella abbia un volume superiore ad $1$ micron cubo.
\item Detto $V$ il volume di una particella, mostrare che si tratta di una variabile aleatoria continua e calcolarne la densità.
\end{enumerate}
\subsection{}

Sia $X$ una variabile aleatoria non negativa con densità di probabilità
\begin{equation*}
f_{X}( x) =\begin{cases}
\frac{2x}{\lambda } e^{-x^{2} /\lambda } , & \text{per} \ x\geq 0,\\
0, & \text{per} \ x< 0.
\end{cases}
\end{equation*}
\begin{enumerate}
\item Determinare i possibili valori della costante $\lambda \in \mathbb{R}$.
\item Determinare il tasso di fallimento di $X$.
\item Determinare la legge di $X^{2}$ e riconoscerla.
\item Calcolare la media e la varianza di $e^{-X^{2} /\lambda }$.
\item Calcolare $\mathbb{E}\left[\frac{1}{X}\right]$.
\item Stabilire per quali $\alpha \in \mathbb{R}$ si ha che $Y=X^{\alpha } \in L^{1}$.
\end{enumerate}
\subsection{}

Si consideri $X$ variabile aleatoria di densità:
\begin{equation*}
f_{X}\left( x\right) =\begin{cases}
\lambda \ e^{-\left( x-\lambda \right)} , & x >\lambda ,\\
0, & x\leq \lambda .
\end{cases}
\end{equation*}
\begin{enumerate}
\item Determinare il valore della costante reale $\lambda $ e calcolare la funzione di ripartizione.
\item Posta $Y=e^{X}$, si determini la distribuzione di $Y$, e si calcolino $\mathbb{E}\left[ Y\right]$ e $\mathbb{E}\left[ Y^{2}\right]$.
\item Si calcolino $\mathbb{P}\left( X >3\right)$ e $\mathbb{P}\left( X^{3}  >27\right)$.
\item Si calcoli la media di $3X^{2} -1$.
\item Stimare la probabilità che $X$ assuma valori distanti dalla media più di $2$ unità, utilizzando la disuguaglianza di Chebychev.
\item Stabilire per quali $\alpha ,\beta \in \mathbb{R}$ si ha che $Y=X^{\alpha } \in L^{1}$ e $Z=\frac{1}{\left( X-1\right)^{\beta }} \in L^{1}$.
\end{enumerate}
\subsection{}

Siano $X\sim U\left(\left( 0,1\right)\right)$ e $Y\sim \Gamma \left( \alpha ,\lambda \right) ,\alpha ,\lambda  >0$.
\begin{enumerate}
\item Si stabilisca per quali $\beta \in \mathbb{R}$ la variabile aleatoria $Z=\frac{1}{\ \left( 1-X\right)^{\beta }} \in L^{1}$.
\item Si determini, al variare dei parametri $\alpha $, $\lambda $, per quali $\gamma \in \mathbb{R}$ la variabile aleatoria $W=Y^{\gamma } \in L^{1}$.

Si confronti questo risultato con la condizione imposta su $\alpha $ e $k$ nel punto $\left( 3\right)$ dell'esercizio $6$.
\end{enumerate}
\subsection{}

\textit{Ricavo di un'opzione "call" europea.} Un agente finanziario sottoscrive un contratto che gli dà il diritto (ma non l'obbligo) di acquistare un certo titolo ad una data futura fissata, ad un prezzo $k >0$ anch'esso fissato. Detto $S$ il valore di tale titolo alla data fissata, esso è al momento sconosciuto, ma si ritiene che i suoi possibili valori abbiano distribuzione lognormale, ovvero che siano dati da
\begin{equation*}
S=\exp X,\ \ \ \ X\sim \mathcal{N}\left( \mu ,\sigma ^{2}\right) ,\ \ \ \ \mu \in \mathbb{R} ,\sigma  >0.
\end{equation*}
Se il valore $S$ supererà $k$, l'agente eserciterà il suo diritto ricavando la differenza $S-k$, altrimenti non lo eserciterà e avrà ricavo nullo. Sia quindi $R$ il ricavo dell'agente.
\begin{enumerate}
\item Calcolare la probabilità di un ricavo positivo, $\mathbb{P}\left( R >0\right)$.
\item Calcolare la probabilità di un ricavo nullo, $\mathbb{P}\left( R=0\right)$.
\item Calcolare la legge del ricavo $R$. Si tratta di una variabile aleatoria continua? Discreta? Perché?
\item Calcolare il ricavo atteso $\mathbb{E}[ R]$.
\item Calcolare $\lim\limits _{k\rightarrow 0}\mathbb{E}[ R]$.
\end{enumerate}
\subsection{}

Petyr Baelish vende a Lord Varys il diritto di comprare fra un anno $50$ kg di acciaio di Valyria al prezzo di $100$ monete d'oro. Petyr Baelish non conosce il costo $X$ di $50$ kg di acciaio di Valyria fra un anno, ma ritiene che i suoi possibili valori abbiano una distribuzione (approssimativamente) normale di media $\mu =100$ e varianza $\sigma ^{2} =81$. Ovviamente Lord Varys eserciterà il suo diritto solo se sarà $X >100$. Sia quindi $Y$ la differenza che Petyr Baelish dovrà pagare fra un anno.
\begin{enumerate}
\item Scrivere $Y$ in funzione di $X$.
\item Determinare la funzione di ripartizione di $Y$, in termini della funzione di ripartizione della legge normale standard, e tracciarne un grafico qualitativo.
\item Si tratta di una variabile aleatoria continua? Discreta? Perché?
\item Determinare la probabilità che Petyr Baelish non debba aggiungere soldi fra un anno.
\item Determinare il primo e il terzo quartile di $Y$.
\item Determinare il valore atteso e la varianza di $Y$.
\item Se Petyr Baelish vende tale diritto di acquisto ad un prezzo pari al valore atteso di $Y$, con quale probabilità prevede di avere un guadagno positivo?
\end{enumerate}
\subsection{}

Un'industria produce su commissione delle sbarre d'acciaio cilindriche, il cui diametro dovrebbe essere di $4$ cm, ma che tuttavia sono accettabili se hanno diametro compreso fra $3.95$ cm e $4.05$ cm. Il cliente, nel controllare le sbarre fornitegli, constata che il $5\%$ sono di diametro inferiore al minimo tollerato ed il $12\%$ di diametro superiore al massimo tollerato.
\begin{enumerate}
\item Supponendo che le misure dei diametri seguano una legge normale, determinarne media e deviazione standard.
\item Mantenendo la media precedentemente calcolata, determinare quale dovrebbe essere il valore delle deviazione standard affinché la percentuale di sbarre con diametro superiore al massimo tollerato sia minore del $5\%$.
\end{enumerate}
\subsection{}

Sia $( \Omega ,\mathcal{A} ,\mathbb{P})$ uno spazio di probabilità fissato. Data una variabile aleatoria reale continua $X$, si mostri che
\begin{enumerate}
\item $\{X\in \mathbb{Q}\}$ è un evento, ossia $\{X\in \mathbb{Q}\} \in \mathcal{A}$,
\item $\mathbb{P}( X\in \mathbb{Q}) =0$.
\end{enumerate}
\subsection{}

Si mostri che, se $X$ è una variabile aleatoria continua, la variabile aleatoria $W=\min\{X,1\}$ può essere continua o no, a seconda della legge di $X$.
\subsection{}

Data $X$ variabile aleatoria continua di densità $f_{X}$, si determini la legge di $Y\ =aX+b$ per $a\neq 0$. In particolare si riconosca la legge di $Y$ per $X$ di legge $\mathcal{N}\left( \mu ,\sigma ^{2}\right)$, $U([ c,d])$, $\mathcal{E}( \lambda )$.
\subsection{}

Data una variabile aleatoria $X\sim U([ -2,-1] \cup [ 1,2])$, calcolarne densità di probabilità $f$, mediane $m$ e, se esiste, media $\mu $.
\subsection{*}

Data una variabile aleatoria $X\sim U(( 0,2\pi ))$ e un angolo fissato $\vartheta \in \mathbb{R}$, si calcoli la legge di $Y=\sin( X+\vartheta )$.
\subsection{}

Data $X\sim \mathcal{E}( \lambda )$, $\lambda  >0$, se ne consideri il suo arrotondamento per eccesso $Y$, ovvero
\begin{equation*}
Y=\sum\limits _{k=1}^{\ +\infty } kI_{( k-1,k]}( X) =\lceil X\rceil .
\end{equation*}
Se quindi $X$ rappresenta un tempo d'attesa, $Y$ rappresenta il corrispondente tempo d'attesa per un osservatore stroboscopico. Si mostri che $Y$ è una variabile aleatoria e se ne calcoli la distribuzione.
\subsection{}

Data $X\sim \mathcal{N}( 2,5)$, mostrare che esiste un unico $c$ tale che $\mathbb{P}(| X| < c) =0.4$. Trovare un valore approssimato di $c$ con metodi di analisi numerica.
\subsection{}

Aldo possiede un vecchio cronometro che, una volta avviato, si arresta dopo un tempo casuale $X$, che si può considerare una variabile aleatoria con legge esponenziale di media $10$ minuti.
\begin{enumerate}
\item Dopo aver fatto partire il cronometro Aldo evita di guardarlo per $5$ minuti al termine dei quali lo osserva e annota l'ora indicata $Y$. Trovare la legge di $Y$.
\item $Y$ è una variabile aleatoria discreta? È continua?
\item Aldo gioca contro Bruno nel modo seguente: se al momento dell'arresto del cronometro il numero di minuti interamente trascorsi è un numero pari allora vince Aldo (ad esempio se $X=2.5$ oppure $X=0.15$); se invece è dispari allora vince Bruno (ad esempio se $X=5$ oppure $X=7.4$). Calcolare la probabilità di vittoria per Aldo.
\end{enumerate}
\subsection{}

Data la funzione $f:\mathbb{R}\rightarrow \mathbb{R}$ definita come
\begin{equation*}
f( x) :=\begin{cases}
0 & x< 0\\
\frac{1}{2} & 0\leq x< \frac{1}{2}\\
2x-1 & \frac{1}{2} \leq x< 1\\
\frac{1}{4} & 1\leq x\leq 3\\
0 & x >3,
\end{cases}
\end{equation*}
\begin{enumerate}
\item dimostrare che $f$ è una funzione di densità di probabilità,
\item determinare la funzione di ripartizione $F$ avente $f$ come funzione di densità,
\item calcolare il quantile di ordine $3$ di una variabile casuale $X$ avente $f$ come funzione di densità di probabilità,
\item calcolare media e varianza di una variabile casuale $X$ avente $f$ come funzione di densità di probabilità,
\item costruire uno spazio di probabilità e una variabile casuale (reale continua) $X$ avente $f$ come funzione di densità.
\end{enumerate}


