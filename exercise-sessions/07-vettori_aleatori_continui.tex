%!TEX root = ../main.tex

\newcommand{\fXYxy}{f_{(X,Y)}(x,y)}
\newcommand{\fXY}{f_{(X,Y)}}
\newcommand{\NNN}{\mathcal{N}}
\newcommand{\AAA}{\mathcal{A}}
\newcommand{\EEE}{\mathcal{E}}
\newcommand{\CCC}{\mathcal{C}}
\newcommand{\GGG}{\mathcal{G}}
\newcommand{\PPP}{\mathcal{P}}
\newcommand{\UUU}{\mathcal{U}}
\newcommand{\indep}{\mathrel{\text{\scalebox{1.07}{$\perp\mkern-10mu\perp$}}}}
\newcommand{\BBB}{\mathcal{B}}
\newcommand{\dP}{\text{ d}\mathbb{P}}
\newcommand{\Lu}{\mathcal{L}^1}
\newcommand{\Cz}{\mathcal{C}^0}
\newcommand{\Cu}{\mathcal{C}^1}
\newcommand{\Ct}{\Cz\cap\widetilde{\CCC}^1}
\newcommand{\SDP}{(\Omega,\AAA,\PP)}
\newcommand{\SDE}{(E,\EEE)}
\newcommand{\SDF}{(F,\FFF)}
\newcommand{\inta}{\int_{a}^{b}}
\newcommand{\intaa}{\int_{c}^{d}}
\newcommand{\intb}{\int_{-\infty}^{+\infty}}
\newcommand{\intc}{\int_{0}^{+\infty}}
\newcommand{\intd}{\int_{0}^{1}}
\newcommand{\inte}{\int_{\mathbb{R}}}
\newcommand{\intf}{\int_{\mathbb{R}^2}}

% Introduzione

\section{Introduzione}
Vogliamo descrivere uno ad uno gli strumenti che serviranno a svolgere gli esercizi di questo foglio di calcolo.

\subsection{Integrali multipli}
Rinfreschiamo dal corso di Analisi 2 le seguenti definizioni.

\begin{definition}[Domini normali]$\\$
Siano $f_1(x), f_2(x)$ due funzioni continue in $[a,b]\subseteq\RR$ e tali che che $f_1(x)\leq f_2(x)$ $\ \forall x\in[a,b]$. Allora l'insieme
\[
D:=\{(x,y)\in\RR^2\ :\ x\in[a,b],\ f_1(x)\leq y\leq f_2(x)   \}
\]
è misurabile ed è chiamato dominio normale rispetto all'asse $x$.
\fg{0.3}{9_1}
Siano $f_3(y), f_4(y)$ due funzioni continue in $[c,d]\subseteq\RR$ e tali che che $f_3(y)\leq f_4(y)$ $\ \forall y\in[c,d]$. Allora l'insieme
\[
E:=\{(x,y)\in\RR^2\ :\ f_3(y)\leq x\leq f_4(y),\ y\in[c,d]   \}
\]
è misurabile ed è chiamato dominio normale rispetto all'asse $y$.
\fg{0.3}{9_2}
\end{definition}
Osserviamo che esistono anche domini normali rispetto entrambi gli assi.

\begin{definition}[Domini normali regolari]$\\$
Il dominio normale $D$ è detto normale regolare se $f_1,f_2\in\Cu([a,b])$ con $f_1(x)<f_2(x)$ $\ \forall x\in[a,b]$.

Analogamente, il dominio normale $E$ è detto normale regolare se $f_3,f_4\in\Cu([c,d])$ con $f_3(y)<f_4(y)$ $\ \forall y\in[c,d]$.
\end{definition}

\begin{definition}[Domini regolari]$\\$
Un dominio $\Omega\subseteq\RR^2$ si dice regolare se è l'unione finita di domini normali regolari privi di punti interni comuni.
\end{definition}
Osserviamo che per costruzione un dominio regolare è chiuso e limitato, quindi è anche compatto.

Siamo pronti per enunciare
\begin{theorem}[Di Fubini]$\\$
\label{introth1}
Sia $D$ un dominio normale rispetto all'asse $x$ in $[a,b]$. Allora data $f:D\to\RR$ continua si ha
\[
\iint_{D}f(x,y)\dxy=\inta \left(\int_{f_1(x)}^{f_2(x)}f(x,y)\dy  \right)\dx
\]
Sia $E$ un dominio normale rispetto all'asse $y$ in $[c,d]$. Allora data $f:R\to\RR$ continua si ha
\[
\iint_{E}f(x,y)\dxy=\intaa \left(\int_{f_3(y)}^{f_4(y)}f(x,y)\dx  \right)\dy
\]
\end{theorem}

\begin{theorem}[Di Jacobi]$\\$
\label{introth2}
Siano $\Omega,\Omega'\subseteq\RR^2$ due domini regolari. Sia $g:\Omega'\to\Omega$ il campo vettoriale che associa ad ogni coppia $(u,v)\in\Omega'$ la coppia $(x,y)\in\Omega$. Se
\begin{enumerate}
\item [(i)] $g$ è biunivoca
\item [(ii)] $g\in\Cu(\Omega')$
\item [(iii)] $g$ tale che $\det J_g(u,v)\neq0$ per ogni coppia  $(u,v)\in\Omega'$
\end{enumerate}
allora data una funzione $f:\Omega\to\RR$ continua si ha che
\begin{gather*}
\begin{aligned}
\iint_\Omega f(x,y)\dxy&=\iint_{\Omega'}f\left(g^{-1}(u,v)\right)\cdot |\det J_{g^{-1}}(u,v)| \text{ d}u\text{d}v \\
&=\iint_{\Omega'}f\left(g^{-1}(u,v)\right)\cdot \frac{1}{|\det J_{g}(u,v)|} \text{ d}u\text{d}v
\end{aligned}
\end{gather*}
\end{theorem}
Vediamo subito un esempio: calcoliamo l'integrale doppio di $$f(x,y)=e^{2x-y} \text{ su } \Omega=\{(x,y)\in\RR^2\ :\ 2x\leq y\leq2x+5,\ -3x\leq y\leq 15/2-3x  \}$$
Prima di partire riscriviamo meglio l'insieme di integrazione
\[
\Omega=\{(x,y)\in\RR^2\ :\ 0\leq y-2x\leq5,\ 0\leq y+3x\leq 15/2  \}
\]
Allora prendendo $g$ tale che
\[
(u,v)=(y-2x,y+3x)=g(x,y)
\]
si rispettano tutte le ipotesi del teorema di Jacobi (a voi i conti). Allora è ben definita l'inversa di $g$
\[
g^{-1}(u,v)=((v-u)/5,(3u+2v)/5)=(x,y)
\]
con il suo jacobiano
\[
J_{g^{-1}}(u,v)=\begin{pmatrix}
\frac{\partial ((v-u)/5)}{\partial u} &\frac{\partial ((v-u)/5)}{\partial v}  \\
\frac{\partial ((3u+2v)/5)}{\partial u} &\frac{\partial ((3u+2v)/5)}{\partial v}  \\
\end{pmatrix} =\begin{pmatrix}
-1/5 & 1/5 \\
3/5 & 2/5 \\
\end{pmatrix}
\]
tale che $\det J_{g^{-1}}(u,v)=-1/5$. Allora
\begin{gather*}
\begin{aligned}
\iint_\Omega e^{2x-y}\dxy&=\iint_{[0,0.5]\times[0,15/2]} e^{-u}\cdot\frac{1}{5}\text{ d}u\text{d}v=\\
&=\frac{1}{5}\int_0^{0.5}e^{-u}\text{ d}u\int_0^{15/2}1\text{ d}v=\\
&=\cdots =\\
&=\frac{3}{2}(1-e^{-5})
\end{aligned}
\end{gather*}
\fg{0.7}{9_3}
Ora interpreteremo questo importante teorema in funzione delle nozioni probabilistiche del corso.

\subsection{Formule di Jacobi in Probabilità}
Nel capitolo sulle variabili aleatori continue abbiamo già visto come applicare la formula di Jacobi, anche se abbiamo anche detto che nel caso 1 - dimensionale non si riesce a comprendere la vera portata di tale enunciato, in quanto è più semplice passare per il calcolo della funzione di ripartizione. Per completezza, ecco il teorema
\begin{theorem}[Di Jacobi per le variabili continue]$\\$
\label{introth3}
Sia $X$ variabile aleatoria assolutamente continua con densità $f_X$. Inoltre sia $g:\RR\to\RR$ tale che
\begin{enumerate}
\item [(i)] $g$ è iniettiva
\item [(ii)] $g\in\Cu(\RR)$
\item [(iii)] $g$ tale che $g'(x)\neq0$ per ogni $x\in\RR$
\end{enumerate}
Allora $Y=g(X)$ è una variabile aleatoria assolutamente continua con densità
\[
f_Y(y)=\begin{cases}
f_X(g^{-1}(y))\cdot|(g^{-1}(y))'|   &\text{ se }y\in g(\RR) \\
0&\text{ altrove, cioè dove }y\not\in g(\RR)
\end{cases}
\]
Attenzione, delle volte se $X$ prende valori in un certo $S\subseteq\RR$ aperto, allora è comodo definire $g:S\to\RR$ perché potrebbero esserci $g$ non iniettive o derivabili in $\RR$ ma iniettive o derivabili in $S$.
\end{theorem}
Si veda l'esercizio 7 per un esempio pratico.

Ora vediamo il caso $n$ - dimensionale
\begin{theorem}[Di Jacobi per i vettori continui]$\\$
\label{introth4}
Sia $\SDP$ uno spazio di probabilità e $X:\Omega\to\RR^n$ un vettore aleatorio assolutamente continuo con densità $f_X:\RR^n\to\RR$. Inoltre sia $g:\RR^n\to\RR^n$ una funzione tale che
\begin{enumerate}
\item [(i)] $g$ è iniettiva
\item [(ii)] $g\in\Cu(\RR^n)$
\item [(iii)] $g$ tale che $\det J_g(x)\neq0$ per ogni $x\in\RR^n$
\end{enumerate}
Allora $g:\RR^n\to g(\RR^n)$ è biiettiva, ha inversa $g^{-1}:g(\RR^n)\to\RR^n$ anch'essa di classe $\Cu$, e la variabile aleatoria $Y:=g\circ X:\Omega\to\RR^n$ è assolutamente continua con densità
\[
f_Y(y)=\begin{cases}
f_X(g^{-1}(y))\cdot|\det J_{g^{-1}}(y)|   &\text{ se }y\in g(\RR^n) \\
0&\text{ altrove, cioè dove }y\not\in g(\RR^^n)
\end{cases}
\]
dove
\[
J_{g^{-1}}(y)=\frac{1}{J_g(g^{-1}(y))}
\]
Attenzione: $g$ non cambia le dimensioni del vettore; inoltre delle volte se $X$ prende valori in un certo $S\subseteq\RR^n$ aperto (cioè $\PP(X\in S)=1$, $X\in S$ quasi certamente), allora è comodo definire $g:S\to\RR^n$ perché potrebbero esserci $g$ non iniettive o derivabili in $\RR^n$ ma iniettive o derivabili in $S$.
\end{theorem}
Si veda l'esercizio 9 per un esempio pratico.

\subsection{Altri teoremi utili}
Richiamiamo qui altri teoremi che ci servanno per svolgere gli esercizi.

\begin{theorem}[Calcolo delle leggi marginali]
\label{introth5-}
$\\$Se il vettore aleatorio $(X,Y)$ è continuo allora $X$ e $Y$ sono variabili aleatorie continue (il viceversa non vale).
\end{theorem}

\begin{theorem}[Calcolo delle leggi marginali]
\label{introth5}
$\\$Se $(X,Y)$ è un vettore aleatorio assolutamente continuo con densità $f_{(X,Y)}$ allora $X,Y$ sono variabili aleatorie assolutamente continue, con densità
\begin{gather*}
\begin{aligned}
&f_X(x)=\int_\RR f_{(X,Y)}(x,y)\text{ d}y&\forall x\in\RR \\
&f_Y(y)=\int_\RR f_{(X,Y)}(x,y)\text{ d}x&\forall y\in\RR
\end{aligned}
\end{gather*}
\end{theorem}

Tale enunciato non può essere invertito, a meno che non si aggiunga l'ipotesi di indipendenza, cioè
\begin{theorem}[Fattorizzazione della legge congiunta]
\label{introth6}
$\\$$X\indep Y\ \Longleftrightarrow\ f_{(X,Y)}(x,y)=f_X(x)\cdot f_Y(y)$
\end{theorem}

\begin{theorem}[Condizione necessaria sui supporti per l'indipendenza]
\label{introth7}
$\\$Siano $X$ e $Y$ due variabili aleatorie continue e indipendenti con densità rispettivamente $f_X$ su $S_X$ e $f_Y$ su $S_Y$. Sia $(X,Y)$ il vettore aleatorio continuo con densità $f_{(X,Y)}$ su $S_{(X,Y)}$. Allora deve valere che
\begin{equation*}
S_{(X,Y)}=S_X\cdot S_Y
\end{equation*}
\end{theorem}

\begin{theorem}[Regola del valore atteso per vettori continui]
\label{introth8}
$\\$Sia $(X,Y)$ vettore aleatorio continuo su $(\Omega,\mathcal{A},\PP)$ con legge $P^{(X,Y)}$ e densità $f_{(X,Y)}$. Allora $\forall h:\RR^2\to\RR$ borelliana e $\forall h:\RR^2\to[0,+\infty)$
\begin{enumerate}
\item [i)] $h\in L^1(P^{(X,Y)})\ \Longleftrightarrow\ h\cdot f_{(X,Y)}\in L^1(m_2)$
\item [ii)] $\EE[h(X,Y)]=\int_{\RR^2}h(x,y)f_{(X,Y)}(x,y) \text{ d}x\text{d}y$
\end{enumerate}
\end{theorem}

\begin{theorem}[Risultato notevole (esercizio 4 del foglio 5)]
\label{introth9}
$\\$Data $X$ variabile aleatoria positiva e continua, denotata $F$ la sua funzione di ripartizione, allora
\[
\EE[X]=\intc(1-F(x))\dx
\]
\end{theorem}

\begin{theorem}[Coefficiente di correlazione unitario]
\label{introth10}
$\\$Siano $X,Y\in L^2$ tali che $Var(X),Var(Y)>0$. Allora
\[
|\rho_{X,Y}|=1\ \Longleftrightarrow\ \exists a\neq 0,\ \exists b\in\RR\ :\ Y=aX+b
\]
\end{theorem}

\begin{theorem}[Alcune proprietà di valore atteso, varianza e covarianza]
\label{introth11}
$\\$Siano $X,Y\in L^2$ e $a,b,c,d\in\RR$. Allora valgono:
\begin{enumerate}
\item [(a)] $Cov(X,X)=Var(X)$
\item [(b)] $X\indep Y\ \Longrightarrow\ Cov(X,Y)=0\ \Longrightarrow\ \rho_{X,Y}=0$
\item [(c)] $\EE(aX+b)=a\EE[X]$
\item [(d)] $X\indep Y\ \Longrightarrow\ \EE[XY]=\EE[X] \EE[Y]$
\item [(e)] $Var(aX+b)=a^2Var(X)$
\item [(f)] $Cov(aX+b,cY+d)=Cov(aX,cY)=acCov(X,Y)$
\item [(g)] $\rho_{aX+b,cY+d}=\rho_{X,Y}$
\item [(h)] $Var(X+Y)=Var(X)+Var(Y)+2Cov(X,Y)$
\item [(i)] $X\indep Y\ \Longrightarrow\ Var(X+Y)=Var(X)+Var(Y)$
\end{enumerate}
\end{theorem}

\begin{theorem}[Bilinearità della covarianza]
\label{introth12}
$\\$Dati $X,Y,Z,W\in L^2$ e $a,b,c,d\in\RR$ si ha
\[
Cov(aX+bY,cW+dZ)=acCov(X,W)+adCov(X,Z)+bcCov(Y,W)+bdCov(Y,Z)
\]
\end{theorem}

\begin{theorem}[Valore atteso e varianza per trasformazioni affini]
\label{introth13}
$\\$Sia $X:\Omega\to\RR^n$ un vettore aleatorio. Allora la trasformazione affine $Y:=AX+b$ con $A\in\RR^{m\times n},\ b\in\RR^{m}$ è un vettore aleatorio in $\RR^m$ con
\begin{gather*}
\begin{aligned}
\EE[Y]&=A\cdot\EE[X]+b \\
Var(Y)&=A\cdot Var(X)\cdot A^T
\end{aligned}
\end{gather*}
\end{theorem}

Ora anticipiamo due risultati notevoli che verranno poi dimostrati nelle pagine a seguire.
\begin{theorem}[Risultato notevole (esercizio 13 del foglio 7)]
\label{introth14}
$\\$Date $X$ e $Y$ di legge congiunta continua con densità $f_{(X,Y)}$, si consideri $Z=X+Y$. Allora $Z$ èuna variabile aleatoria continua con densità
\[
f_Z(z)=\intb f_{(X,Y)}(z-y,y)\dy
\]
\end{theorem}
\begin{theorem}[Risultato notevole (esercizio 2 del foglio 7)]
\label{introth15}
$\\$Due variabili aleatorie $X$ e $Y$ sono variabili aleatorie continue e indipendenti se e solo se sono congiuntamente continue con densità fattorizzabile, cioè esistono due funzioni $h_1:\RR\to\RR$ e $h_2:\RR\to\RR$ tali che $f_{(X,Y)}(x,y)=h_1(x)\ h_2(y)$ q.o.
\end{theorem}

\subsection{Qualche integrale notevole}
 Vogliamo infine ricordare i seguenti integrali (risolti integrando \textit{pp}: per parti), in modo tale da rendere più fluidi i futuri passaggi.
\begin{enumerate}
\item [$(\alpha)$]
\begin{gather*}
\begin{aligned}
\int te^{-t}\text{ d}t&\overset{\underset{\textit{pp}}{}}{=}\begin{Bmatrix}
\text{fattore finito }&f=t \\ \text{fattore differenziale }&g=-e^{-t}
\end{Bmatrix}=  \\
&=fg-\int f'g\text{ d}t=\\
&=-te^{-t}-\int-e^{-t}\text{ d}t=\\
&=-te^{-t}+\int e^{-t}\text{ d}t=\\
&=-te^{-t}-e^{-t}=\\
&=-(t+1)e^{-t}
\end{aligned}
\end{gather*}
\item [$(\alpha^1)$]
\begin{equation*}
\int_0^{+\infty} te^{-t}\text{ d}t\overset{\underset{(\alpha)}{}}{=}\left[-(t+1)e^{-t}\right]_0^{+\infty}=1
\end{equation*}
\item [$(\beta)$]
\begin{gather*}
\begin{aligned}
\int t^2e^{-t}\text{ d}t&\overset{\underset{\textit{pp}}{}}{=}-t^2e^{-t}-\int-2te^{-t}\text{ d}t=\\
&=-t^2e^{-t}+2\int te^{-t}\text{ d}t=\\
&\overset{\underset{(\alpha)}{}}{=}-t^2e^{-t}-2(t+1)e^{-t}=\\
&=-(t^2+2t+2)e^{-t}
\end{aligned}
\end{gather*}
\item [$(\beta^1)$]
\begin{equation*}
\int_0^{+\infty} t^2e^{-t}\text{ d}t\overset{\underset{(\beta)}{}}{=}\left[-(t^2+2t+2)e^{-t}\right]_0^{+\infty}=2
\end{equation*}
\item [$(\gamma)$]
\begin{gather*}
\begin{aligned}
\int t^3e^{-t}\text{ d}t&\overset{\underset{\textit{pp}}{}}{=}-t^3e^{-t}-\int-3t^2e^{-t}\text{ d}t=\\
&=-t^3e^{-t}+3\int t^2e^{-t}\text{ d}t=\\
&\overset{\underset{(\beta)}{}}{=}-t^3e^{-t}-3(t^2+2t+2)e^{-t}=\\
&=-(t^3+3t^2+6t+6)e^{-t}
\end{aligned}
\end{gather*}
\item [$(\gamma^1)$]
\begin{equation*}
\int_0^{+\infty} t^3e^{-t}\text{ d}t\overset{\underset{(\gamma)}{}}{=}\left[-(t^3+3t^2+6t+6)e^{-t}\right]_0^{+\infty}=6
\end{equation*}

\end{enumerate}

% Fine introduzione

\newpage

\ParteEsercizi

\Esercizio{} % es 1 manca sol
Si esibisca un esempio di due variabili aleatorie continue $X$ e $Y$ con legge congiunta continua. \\
Si esibisca poi un esempio di due variabili aleatorie continue $X$ e $Y$ con legge congiunta non continua.

\Esercizio{$^\ast$}
Si mostri che due variabili aleatorie $X$ e $Y$ sono variabili aleatorie continue e indipendenti se e solo se sono congiuntamente continue con densità fattorizzabile, cioè esistono due funzioni $h_1:\RR\to\RR$ e $h_2:\RR\to\RR$ tali che $f_{(X,Y)}(x,y)=h_1(x)\ h_2(y)$ q.o.

\Esercizio{}
Date $X,Y$ indipendenti ed entrambe di legge $\mathcal{E}(\lambda)$, $\lambda>0$, si calcoli $\PP(Y>X)$.

\Esercizio{}
Sia $(X,Y)$ un vettore aleatorio con densità
\begin{equation*}
f_{(X,Y)}(x,y)=\begin{cases} x(y-x)e^{-y},&0<x<y, \\ 0,&\text{altrove}.\end{cases}
\end{equation*}
\begin{enumerate}
\item [(a)] Calcolare le leggi di $X$ e di $Y$.
\item [(b)] $X$ e $Y$ sono indipendenti?
\item [(c)] Calcolare $\PP(X\leq 2,Y\leq 3)$.
\item [(d)] Calcolare il coefficiente di correlazione $\rho_{X,Y}$.
\item [(e)] Trovare una diversa densità congiunta avente le stesse densità marginali.
\end{enumerate}

\Esercizio{}
Sia $(X,Y)$ un vettore aleatorio di densità
\[
\fXYxy=\begin{cases}
xe^{-(x+y)} &\text{se }x>0,y>0 \\
0 &\text{altrove}
\end{cases}
\]
\begin{enumerate}
\item [(a)] Quali sono le leggi di $X$ e $Y$? Le variabili aleatorie $X$ e $Y$ sono indipendenti?
\item [(b)] Calcolare $Var(X+Y)$.
\item [(c)] Si calcolino i valori attesi di $U=\min(X,Y)$ e $V=\max(X,Y)$.
\end{enumerate}

\Esercizio{}
Il vettore aleatorio $(X,Y)$ ha legge uniforme sul quadrato $Q$ di vertici $(0,2),(2,0),(4,2),(2,4)$.
\begin{enumerate}
\item [(a)] Qual è la legge di $X$?
\item [(b)] Quanto vale $\EE[X]$?
\item [(c)] Quanto vale $\PP(X<Y)$?
\item [(d)] Quanto vale $\PP(X\leq 2,Y\leq 1)$?
\item [(e)] $X$ e $Y$ sono indipendenti?
\end{enumerate}

\Esercizio{}
Una misura di resistenza in un circuito elettrico eseguita con uno strumento di risoluzione pari a 1 ohm dà lettura di 12 ohm. La resistenza $R$ del circuito risulta pertanto descritta da una variabile aleatoria con distribuzione uniforme tra 11.5 e 12.5 ohm. Sia $M=1/R$ la conduttanza del circuito. 
\begin{enumerate}
\item [(a)] $R$ e $M$ hanno legge congiunta continua?
\item [(b)] Calcolare i valori attesi di $R$ e $M$.
\item [(c)] Calcolare la matrice varianza di $(R,M)$.
\item [(d)] Calcolare il coefficiente di correlazione lineare fra $R$ e $M$.
\item [(e)] Determinare la distribuzione di $M$.
\end{enumerate}

\Esercizio{ (Problema dell'ago di Buffon)}
Un ago di lunghezza $l>0$ viene lanciato su un pavimento decorato con linee parallele distanti $d>0$. Qual è la probabilità $p$ che l'ago intersechi almeno una linea?

\Esercizio{}
Siano $X$ e $Y$ variabili aleatorie indipendenti con legge $\UUU([0,1])$.
\begin{enumerate}
\item [(a)] Si scriva la densità di probabilità del vettore aleatorio $(X,Y)$.
\end{enumerate}
Si consideri il vettore aleatorio $(U,V)=(X+1,X+Y)$.
\begin{enumerate}
\item [(b)] Calcolarne il valore atteso e la matrice varianza.
\item [(c)] Se ne calcoli la legge.
\item [(d)] Il vettore ha componenti indipendenti?
\end{enumerate}
Si consideri ora il vettore aleatorio $(U,V)=(X+1,X+Y-XY+X^2Y)$.
\begin{enumerate}
\item [(e)] Calcolarne il valore atteso.
\item [(f)] Se ne calcoli la legge.
\item [(g)] Il vettore ha componenti indipendenti?
\end{enumerate}

\Esercizio{} % es 10 manca sol
Verificare che se $X_1,\dots,X_n$ sono variabili aleatorie esponenziali indipendenti di parametri $\lambda_1,\dots,\lambda_n$ rispettivamente, allora
\[
X_{(1)}=\min\{X_1,\dots,X_n  \}\sim\EEE(\lambda_1,\dots,\lambda_n)
\]

\Esercizio{} % es 11 manca sol
Siano $X_1,\dots,X_n$ sono variabili aleatorie continue i.i.d. con funzione di ripartizione $\CCC^1$ a tratti. 
\begin{enumerate}
\item [(a)] Mostrare che $X_{(n)}=\max\{ X_1,\dots,X_n \}$ è una variabile aleatoria continua e determinarne la densità.
\item [(b)] Mostrare che $X_{(1)}=\min\{ X_1,\dots,X_n \}$ è una variabile aleatoria continua e determinarne la densità.
\end{enumerate}

\Esercizio{}
Un componente elettronico è formato da tre elementi indipendenti in serie, ciascuno dei quali ha un tempo di vita esponenziale di parametro $\lambda=0.3,\mu=0.1,\gamma=0,2$ rispettivamente.
\begin{enumerate}
\item [(a)] Indichiamo con $T$ il tempo di vita del componente. Qual è la legge di $T$?
\item [(b)] Per aumentare l'affidabilità e ridurre gli interventi di sostituzione, viene proprosto di aggiungere un componente identico in parallelo. Qual è la legge del tempo di vita $S$ del nuovo complesso?
\end{enumerate}

\Esercizio{}
Date $X$ e $Y$ di legge congiunta continua con densità $\fXY$, si consideri $Z=X+Y$. Si mostri che $Z$ è una variabile aleatoria continua con densità
\[
f_Z(z)=\intb f_{(X,Y)}(z-y,y)\dy
\]
Si mostri questo risultato procedendo nei due seguenti modi:
\begin{enumerate}
\item [(a)] Si calcoli la funzione di ripartizione di $Z$ e si determini la densità.
\item [(b)] Si calcoli la legge di $(Z,Y)$ e si trovi la legge marginale di $Z$.
\end{enumerate}

\Esercizio{}
Siano $X$ e $Y$ due variabili aleatorie indipendenti entrambe con distribuzione uniforme su $[0,1]$.
\begin{enumerate}
\item [(a)] Determinare media, varianza e legge di $Z=X+Y$.
\item [(b$^*$)] Si trovi la legge di $T=X+Y-\Ind_{\{ X+Y-1 \}}$.
\end{enumerate}

\Esercizio{}
Siano $X$ e $Y$ due variabili aleatorie indipendenti entrambe con distribuzione uniforme su $[0,1]$.
\begin{enumerate}
\item [(a)] Determinare la distribuzione congiunta del vettore $(U,V)$ ove $U=XY$ e $V=\dfrac{Y}{X}$.
\item [(b)] Determinare le marginali di $U$ e $V$.
\end{enumerate}

\Esercizio{}
Siano $X\sim\Gamma(\alpha,\lambda)$ e $Y\sim\Gamma(\beta,\lambda)$ due variabili aleatorie indipendenti; determinare la densità della variabile aleatoria $U=\dfrac{X}{Y}$.

\Esercizio{} % es 17 manca sol
Calcolo delle densità della $t$ d Student e della $F$ di Fisher.
\begin{enumerate}
\item [(a)] Siano $Z\sim\NNN(0,1)$ e $X\sim\chi^2(n)$ due variabili aleatorie indipendenti; determinare la densità di $T=\dfrac{Z}{\sqrt{\frac{X}{n}}}$.
\item [(b)] Siano $X\sim\chi^2(m)$ e $Y\sim\chi^2(n)$ due variabili aleatorie indipendenti; determinare la densità di $F=\dfrac{X/m}{Y/n}$.
\end{enumerate}

\Esercizio{} % es 18 manca sol
Siano $X\sim\UUU([-5,3])$ e $Y\sim\UUU([3,5])$ due variabili aleatorie indipendenti.
\begin{enumerate}
\item [(a)] Determinare media, varianza e legge di $Z=X+Y$.
\item [(b)] Si calcoli il coefficiente di correlazione $\rho_{X,Z}$.
\end{enumerate}

\Esercizio{}
Sia $(X,Y)$ un vettore aleatorio continuo con densità
\[
\fXYxy=\begin{cases} \dfrac{1}{2}(x+y)e^{-(x+y)} &x,y>0 \\ 0&\text{altrove} \end{cases}
\]
\begin{enumerate}
\item [(a)] Si calcoli la media di $(X+Y)^{-1}$.
\item [(b)] Si determini la legge di $X+Y$.
\item [(c)] Si calcoli la media di $X+Y$.
\item [(d)] Si calcolino le marginali di $X,Y$. $X\indep Y$?
\item [(e)] Si calcoli $Cov(X,Y)$.
\item [(f)] Si calcoli $\PP(X\geq 1,Y\geq 2)$.
\end{enumerate}

\Esercizio{} % es 20 manca sol
Due numeri $X,Y$ vengono scelti a caso indipendentemente con distribuzione uniforme su $[0,1]$.
\begin{enumerate}
\item [(a)] Calcolare $\PP(|X-Y|>1/2)$.
\item [(b)] Sia $Z$ la variabile aleatoria che misura la distanza fra $X$ e $Y$. Qual è la legge di $Z$? Qual è la distanza media fra $X$ e $Y$?
\end{enumerate}

\Esercizio{} % es 21 manca punto a
Sia assegnata una successione $N,X_1,X_2,\dots$ di variabili aleatorie indipendenti, tale che $N\sim\GGG(p), p\in(0,1)$, e ciascuna delle $X_i\sim\EEE(\lambda),\lambda>0$. Si ponga $Y=\min\{X_1,\dots,X_N  \}$.
\begin{enumerate}
\item [(a$^*$)] Si mostri che $Y$ è una variabile aleatoria.
\item [(b)] Qual è la legge di $Y$?
\item [(c)] Qual è il valor atteso di $Y$? 
\end{enumerate}

\Esercizio{} % es 22 manca sol
Si mostri che, date due variabili aleatorie indipendenti $X$ e $Y$ in $L^1$, allora necessariamente si ha anche $XY$ in $L^1$. Si mostri però che, se $X$ e $Y$ non sono indipendendenti, allora $XY$ può non essere integrabile.

\Esercizio{$^\ast$} % es 23 manca sol
Date $n$ variabili aleatorie reali $X_1,\dots,X_n$, si definiscono \emph{statistiche d'ordine}
\begin{gather*}
\begin{aligned}
X_{(1)}&=\min\{X_1,\dots,X_n  \}, \\
X_{(2)}&=\text{secondo più piccolo valore di }X_1,\dots,X_n, \\
&\ \vdots \\
X_{(n)}&=\max\{X_1,\dots,X_n  \}.
\end{aligned}
\end{gather*}
\begin{enumerate}
\item [(a)] Si trovi una formula per $X_{(2)}$ nel caso $n=3$ e una formula per $X_{(3)}$ nel caso $n=4$.
\item [(b)] Si trovi una formula per $X_{(k)}$ nel caso $n\in\NN,1\leq k\leq n$.
\item [(c)] Si mostri chele statistiche d'ordine sono variabili aleatorie e che $X_{(1)}\leq X_{(2)}\leq \cdots \leq X_{(n)}$ q.c.
\item [(d)] Sia $B\in\BBB^n$ un boreliano di $\RR^n$ tale che: se $(y_1,\dots,y_n)\in B$ allora $y_1\leq y_2\leq \cdots\leq y_n$. In altre parole, $B\subset\{(y_1,\dots,y_n)\in\RR^n\ :\ y_1\leq y_2\leq \cdots\leq y_n  \}$. Si esprima l'evento $((X_{(1)},\dots,X_{(n)})\in B)$ in termini di $X_1,\dots,X_n$.
\item [(e)] Supponiamo ora che $X_1,\dots,X_n$ siano i.i.d. con densità continua $f$. Si mostri che $X_{(1)},\dots,X_{(n)}$ hanno legge congiuntamente continua con densità
\[
f_{(X_{(1)},\dots,X_{(n)})}(y_1,\dots,y_n)=\begin{cases}n!\displaystyle\prod_{i=1}^n f(y_i) &y_1\leq y_2\leq \cdots\leq y_n \\ 0&\text{altrimenti}   \end{cases}
\]
\end{enumerate}

\Esercizio{}
Le lunghezze $X$ e $Y$ dei cateti di un triangolo rettangolo sono generate a caso, in modo tale che si tratti di variabili esponenziali indipendenti di stesso parametro $\lambda>0$. Sia $A$ l'area del triangolo, $Z=\frac{Y}{X}$ il rapporto dei cateti e $\alpha$ l'angolo (misurato in radianti) opposto al cateto $Y$.
\begin{enumerate}
\item Calcolare il valore atteso dell'area $A$, in funzione di $\lambda$.
\item Calcolare la funzione di ripartizione di $Z$, notando che non dipende da $\lambda$ e disegnarne il grafico.
\item Determinare se $Z$ è assolutamente continua e in tal caso calcolarne la densità e disegnarne il grafico.
\item Calcolare la media di $Z$.
\item Calcolare la funzione di ripartizione di $\alpha$, notando che non dipende da $\lambda$.
\item Determinare se $\alpha$ è assolutamente continua e in tal caso calcolarne la densità.
\end{enumerate}

\ParteSoluzioni

\Soluzione{}
Manca

\Soluzione{}
La dimostrazione si articola in due parti:
\begin{enumerate}
\item [$(\Rightarrow)$] Questo caso è banale, perché basta prendere $h_1(x)=f_X(x)$ e $h_2(y)=f_Y(y)$.

\item [$(\Leftarrow)$] Sappiamo per ipotesi che $f_{(X,Y)}(x,y)=h_1(x)\ h_2(y)$. Allora grazie al teorema (\ref{introth6}) calcoliamo le due marginali come
\begin{gather*}
\begin{aligned}
&f_X(x)=\int_\RR h_1(x)\ h_2(y) \dy=h_1(x)\int_\RR h_2(y) \dy \\
&f_Y(y)=\int_\RR h_1(x)\ h_2(y) \dx=h_2(y)\int_\RR h_1(x) \dx
\end{aligned}
\end{gather*}
Sfruttando il fatto che per definizione di densità deve essere $\int_\RR\fXYxy\dxy=1$, possiamo calcolare
\begin{gather*}
\begin{aligned}
f_X(x)\ f_Y(y)&=h_1(x)\int_\RR h_2(y) \dy\cdot h_2(y)\int_\RR h_1(x) \dx=\\
&=h_1(x)\ h_2(y) \cdot \int_\RR h_1(x)\ h_2(y) \dxy=\\
&=h_1(x)\ h_2(y) \cdot \underbrace{\int_\RR \fXYxy \dxy}_{=1}=\\
&=h_1(x)\ h_2(y)=\\
&=\fXYxy
\end{aligned}
\end{gather*}
e quindi concludere che $X$ e $Y$ sono indipendenti.
\end{enumerate}
L'enunciato è quindi dimostrato.

\Soluzione{}
Il vettore aleatorio $(X,Y):\Omega\to\mathbb{R}^2$ è assolutamente continuo; la sua densità $f_{(X,Y)}(x,y)$, essendo $X$ e $Y$ indipendenti, è data, grazie al  teorema (\ref{introth6}), dal prodotto tra le due densità marginali
\begin{gather*}
\begin{aligned}
&f_X(x)=\lambda e^{-\lambda x}\ \Ind_{[0,+\infty)}(x) \\
&f_Y(y)=\lambda e^{-\lambda y}\ \Ind_{[0,+\infty)}(y)
\end{aligned}
\end{gather*}
Allora per calcolare $\PP(Y>X)$ basta semplicemente vedere l'evento $Y>X$ come $(X,Y)\in T$, con $T$ zona del primo quadrante del piano $X\times Y$ che sta sopra la bisettrice $Y=X$, e poi integrare di conseguenza:
\begin{gather*}
\begin{aligned}
\PP(Y>X)&=\PP((X,Y)\in T)= \\
&=\int_T f_{(X,Y)}(x,y) \text{ d}x\text{d}y= \\
&=\int_0^{+\infty} \int_{x}^{+\infty}\lambda^2e^{-\lambda x}e^{-\lambda y}\text{ d}x\text{d}y= \\
&=\int_0^{+\infty}\lambda e^{-\lambda x} \left(\int_{x}^{+\infty} \lambda e^{-\lambda y}  \text{ d}y  \right)\text{ d}x= \\
&=\int_0^{+\infty}\lambda e^{-\lambda x} \left[ e^{-\lambda y}  \right]_x^{+\infty}\text{ d}x=\\
&=\underbrace{\int_0^{+\infty}\lambda e^{-2\lambda x}\text{ d}x}_{\text{riconduco a }\mathcal{E}(2\lambda)}=\\
&=\frac{1}{2}\underbrace{\int_0^{+\infty}2\lambda e^{-2\lambda x}\text{ d}x}_{=1}=\\
&=\frac{1}{2}
\end{aligned}
\end{gather*}
Esiste un secondo modo, più teorico ma meno calcoloso, per risolvere l'esercizio. Infatti possiamo osservare che
\begin{equation*}
1=\PP(X<Y)+\PP(X=Y)+\PP(X>Y)
\end{equation*}
Diamo un'occhiata ai tre termini:
\begin{enumerate}
\item $\PP(X<Y)$ è la nostra incognita.
\item $\PP(X=Y)=\PP((X,Y)\in r)$ con $r$ bisettrice del primo quadrante del piano $X\times Y$; ma
\begin{enumerate} 
\item [(i)] il vettore $(X,Y)$ è assolutamente continuo rispetto la misura di Lebesgue su $\mathbb{R}^2$
\item [(ii)] la retta $r$ è un sottospazio di $\mathbb{R}^2$ di dimensione $1$ e di conseguenza ha misura di Lebesgue $m_2=0$
\end{enumerate}
$\Longrightarrow \PP(X=Y)=0$.
\item $\PP(X>Y)=\PP(X<Y)$ perché il ruolo di $X$ e $Y$ è interscambiabile dato che hanno la stessa distribuzione.
\end{enumerate}
Allora $1=2\ \PP(X<Y)\ \Longrightarrow\ \PP(Y>X)=\frac{1}{2}$.

\Soluzione{}
\begin{enumerate}
\item [(a)] Calcolare le leggi di $X$ e di $Y$.

Conoscendo la densità congiunta del vettore aleatorio assolutamente continuo $(X,Y)$, per calcolare le leggi marginali usiamo il teorema (\ref{introth5}). Prima di procedere però è sempre meglio capire come è fatto il supporto della densità congiunta, così da capire quale sarà il supporto delle marginali:
\[
S_{(X,Y)}:=\{(x,y)\in\RR^2\ \big|\ f_{(X,Y)}(x,y)>0   \}=\{(x,y)\in\RR^2\ \big|\ 0<x<y   \}
\]
\fg{0.3}{7_17}
Dal grafico della densità congiunta deduciamo un'importante informazione su $X$ e $Y$: $X,Y\geq 0$ quasi certamente, cioè $f_X(x)=0$ per ogni $x<0$ e $f_Y(y)=0$ per ogni $y<0$. Quindi fissiamo $x\geq 0$ e procediamo con il calcolo della prima legge marginale:
\begin{gather*}
\begin{aligned}
f_X(x)&=\int_\RR f_{(X,Y)}(x,y)\text{ d}y= \\
&=\int_x^{+\infty}x(y-x)e^{-y}\text{ d}y= \\
&=\begin{Bmatrix}
\text{cambio di variabile}\\t:=y-x 
\end{Bmatrix}= \\
&=x\int_0^{+\infty}te^{-t-x}\text{ d}t=\\
&=xe^{-x}\underbrace{\int_0^{+\infty}te^{-t}\text{ d}t}_{=1 \text{ per } (\alpha^1)\text{ o per }(\alpha^2)}=xe^{-x}
\end{aligned}
\end{gather*}
con $(\alpha^2)$ che sintetizza la seguente osservazione: data $T\sim\mathcal{E}(\lambda)$ con $\lambda=1$ si ha per definizione
\begin{equation*}
\mathbb{E}[T]:=\int_0^{+\infty}\lambda te^{-t}\text{ d}t=\cdots=\frac{1}{\lambda}=\frac{1}{1}=1
\end{equation*}

Ora fissiamo $y\geq 0$ e calcoliamo la seconda legge marginale:
\begin{gather*}
\begin{aligned}
f_Y(y)&=\int_\RR f_{(X,Y)}(x,y)\text{ d}x= \\
&=\int_0^y x(y-x)e^{-y}\text{ d}x= \\
&=e^{-y}\left( y\int_0^y x\text{ d}x-\int_0^yx^2\text{ d}x  \right)=\\
&=e^{-y}\left( y\left[\frac{x^2}{2}    \right]_0^y-\left[\frac{x^3}{3}    \right]  \right)=\\
&=e^{-y}\left( \frac{y^3}{2}  -\frac{y^3}{3} \right)=\\
&=\frac{1}{6}\ y^3e^{-y}
\end{aligned}
\end{gather*}
Ricapitolando, abbiamo ottenuto
\begin{gather*}
\begin{aligned}
&f_X(x)=xe^{-x}\ \Ind_{[0,+\infty)}(x) \\
&f_Y(y)=\frac{1}{6}\ y^3e^{-y}\ \Ind_{[0,+\infty)}(y)
\end{aligned}
\end{gather*}
Ricordando che
\begin{oss}
$\\$Sia $X\sim\Gamma(\alpha,\lambda)$ con $\alpha,\lambda>0$. Allora
\begin{gather*}
\begin{aligned}
f_X(x)&=\frac{\lambda^\alpha}{\Gamma(\alpha)}\ x^{\alpha-1}e^{-\lambda x}\ \ \ \ \ x\in\RR \\
\Gamma(\alpha)&=(\alpha-1)! \\
\EE[X]&=\frac{\alpha}{\lambda} \\
Var(X)&=\frac{\alpha}{\lambda^2}
\end{aligned}
\end{gather*}
(Quando si ha \textit{densità}$=$\textit{polinomio}$\cdot$\textit{esponenziale} è quasi sempre una $\Gamma$!)
\end{oss}
riconosciamo che $X\sim\Gamma(2,1)$ e $Y\sim\Gamma(4,1)$.
\item [(b)] $X$ e $Y$ sono indipendenti?

Abbiamo già visto con il teorema (\ref{introth6}) che se $X\indep Y$ allora la densità congiunta del vettore $(X,Y)$ fattorizza nelle due marginali. Questo è sufficiente per concludere che in questo caso $X\not\indep Y$.$\\$In realtà si potrebbe usare anche il teorema (\ref{introth7}).
Infatti abbiamo
\begin{gather*}
\begin{aligned}
S_X&=[0,+\infty) \\
S_Y&=[0,+\infty) \\
S_{(X,Y)}&=\{(x,y)\in\RR^2\ \big|\ 0<x<y   \}
\end{aligned}
\end{gather*}
e quindi anche in questo caso concludiamo che $X\not\indep Y$.

\item [(c)] Calcolare $\PP(X\leq 2,Y\leq 3)$.

Se le variabili $X,Y$ fossero state indipendenti sarebbe stato banale il calcolco:
\begin{equation*}
\PP(X\leq 2, Y\leq 3)=\PP(X\leq 2)\cdot \PP(Y\leq 3)
\end{equation*}
Tuttavia non lo sono, quindi dobbiamo ricondurre l'evento $X\leq 2, Y\leq 3$ alla coppia $(X,Y)$, cioè
\begin{gather*}
\begin{aligned}
\PP(X\leq 2, Y\leq 3)&=\PP((X,Y)\in(-\infty,2]\times(-\infty,3])=\\
&=\int_{-\infty}^2\left(\int_{-\infty}^3f_{(X,Y)}(x,y)\text{ d}y   \right)\text{d}x=
\end{aligned}
\end{gather*}
che per 
\fg{0.4}{7_18}
diventa
\begin{gather*}
\begin{aligned}
&=\int_0^2\left(\int_x^3x(y-x)e^{-y}\text{ d}y   \right)\text{d}x=\int_0^2x\left(\underbrace{\int_x^3ye^{-y}\text{ d}y}_{(\alpha)}-x\int_x^3e^{-y}\text{ d}y   \right)\text{d}x=\\
&=\int_0^2x\left(  \left[-(y+1)e^{-y}     \right]_x^3-x\left[-e^{-y}     \right]_x^3    \right)\text{d}x=\int_0^2 x\left(-4e^{-3}+(x+1)e^{-x}+xe^{-3}-xe^{-x}   \right)\text{d}x=\\
&=\int_0^2 x\left(-4e^{-3}+e^{-x}+xe^{-3}   \right)\text{d}x=-4e^{-3}\int_0^2 x\text{ d}x+\underbrace{\int_0^2xe^{-x}\text{ d}x}_{(\alpha)}+e^{-3}\int_0^2{x^2}\text{ d}x=\\
&=-4e^{-3}\left[\frac{x^2}{2}   \right]_0^2+\left[-(x+1)e^{-x}   \right]_0^2+e^{-3}\left[\frac{x^3}{3}   \right]_0^2=-8e^{-3}-3e^{-2}+1+\frac{8}{3}\ e^{-3}=\\
&=1-3e^{-2}-\frac{16}{3}\ e^{-3}
\end{aligned}
\end{gather*}

\item [(d)] Calcolare il coefficiente di correlazione $\rho_{X,Y}$.

Ricordiamo che
\begin{oss}
$\\$Il coefficiente di correlazione lineare si calcola con
\begin{equation*}
\rho_{X,Y}:=\frac{Cov(X,Y)}{\sqrt{Var(X)\cdot Var(Y)}}=\frac{\EE[XY]-\EE[X]\cdot\EE[Y]}{\sqrt{Var(X)\cdot Var(Y)}}
\end{equation*}
\end{oss}

Al volo possiamo già dire che $\EE[X]=2=Var(X)$ e $\EE[Y]=4=Var(Y)$. Invece per $\EE[XY]$ sfruttiamo il teorema (\ref{introth8})
\begin{gather*}
\begin{aligned}
\EE[XY]&\overset{\underset{\text{ii)}}{}}{=}\int_{\RR^2}xyf_{(X,Y)}(x,y) \text{ d}x\text{d}y=\int_0^{+\infty}x^2\left(\int_x^{+\infty}y(y-x)e^{-y}\text{ d}y   \right)\text{d}x=\\
&=\int_0^{+\infty}x^2\left(\underbrace{\int_x^{+\infty}y^2e^{-y}\text{ d}y}_{(\beta)}   \right)\text{d}x-\int_0^{+\infty}x^3\left(\underbrace{\int_x^{+\infty}ye^{-y}\text{ d}y}_{(\alpha)}   \right)\text{d}x=\\
&=\int_0^{+\infty}x^2\left[-(y^2+2y+2)e^{-y} \right]_x^{+\infty}\text{d}x-\int_0^{+\infty}x^3\left[-(y+1)e^{-y} \right]_x^{+\infty}\text{d}x=\\
&=\int_0^{+\infty}x^2(x^2+2x+2)e^{-x}\text{ d}x-\int_0^{+\infty}x^3(x+1)e^{-x}\text{ d}x=\\
&=\int_0^{+\infty}x^4e^{-x}   \text{ d}x+2\int_0^{+\infty}x^3e^{-x}   \text{ d}x+2\int_0^{+\infty}x^2e^{-x}   \text{ d}x-\int_0^{+\infty}x^4e^{-x}   \text{ d}x-\int_0^{+\infty}  x^3e^{-x} \text{ d}x=\\
&=\underbrace{\int_0^{+\infty}x^3e^{-x}   \text{ d}x}_{=6\text{ per }(\gamma^1)}+2\underbrace{\int_0^{+\infty}x^2e^{-x}   \text{ d}x}_{=2\text{ per }(\beta^1)}=6+4=10
\end{aligned}
\end{gather*}
E finalmente
\begin{equation*}
\rho_{X,Y}=\frac{10-2\cdot 4}{\sqrt{2\cdot 4}}=\frac{2}{2\sqrt{2}}=\frac{1}{\sqrt{2}}=\frac{\sqrt{2}}{2}
\end{equation*}
\item [(e)] Trovare una diversa densità congiunta avente le stesse densità marginali.

Banalmente basta prendere la densità data dal prodotto delle marginali, cioè
\begin{equation*}
\widetilde{f}_{(X,Y)}(x,y)=xe^{-x}\ \Ind_{[0,+\infty)}(x)\cdot \frac{1}{6}\ y^3e^{-y}\ \Ind_{[0,+\infty)}(y)
\end{equation*}
\end{enumerate}

\Soluzione{}
\begin{enumerate}
\item [(a)] Quali sono le leggi di $X$ e $Y$? Le variabili aleatorie $X$ e $Y$ sono indipendenti?

Per calcolare le due marginali usiamo il teorema (\ref{introth5}). Quindi, fissato $x>0$, si ha
\begin{gather*}
\begin{aligned}
f_X(x)&=\int_\RR\fXYxy\dy=\\
&=\intc xe^{-(x+y)}\dy=\\
&=\left[-xe^{-(x+y)}  \right]_0^{+\infty}=\\
&=xe^{-x}
\end{aligned}
\end{gather*}
Quindi $f_X(x)=xe^{-x}\ \Ind_{(0,+\infty)}(x)$ e perciò $X\sim\Gamma(2,1)$.

Allo stesso modo fissiamo $y>0$ e calcoliamo
\begin{gather*}
\begin{aligned}
f_Y(y)&=\int_\RR\fXYxy\dy=\\
&=\intc xe^{-(x+y)}\dx=\\
&=e^{-y}\underbrace{\intc xe^{-x}\dx}_{=1\text{ per }(\alpha^1)}=\\
&=e^{-y}
\end{aligned}
\end{gather*}
Quindi $f_Y(y)=e^{-y}\ \Ind_{(0,+\infty)}(y)$ e perciò $Y\sim\Gamma(1,1)\sim\EEE(1)$.

Dato che la congiunta fattorizza nelle due marginali, allora per il teorema (\ref{introth6}) le variabili $X$ e $Y$ sono indipendenti.
\item [(b)] Calcolare $Var(X+Y)$.

Ricordando che $Var(\Gamma(\alpha,\lambda))=\alpha/\lambda^2$ e $Var(\EEE(\lambda))=1/\lambda^2$ possiamo facilmente calcolare
\[
Var(X+Y)\overset{\underset{\indep}{}}{=}Var(X)+Var(Y)=2+1=3
\]
\item [(c)] Si calcolino i valori attesi di $U=\min(X,Y)$ e $V=\max(X,Y)$.

Iniziamo da $U$, osservando che per com'è definita si ha $Im(U)\in(0,+\infty)$; allora fissatto $u>0$ abbiamo
\begin{gather*}
\begin{aligned}
F_U(u)&=\PP(U\leq u)=\\
&=\PP(\min(X,Y)\leq u)=         \\
&=1- \PP(\min(X,Y)> u)=        \\
&= 1-\PP(X>u,Y>u)=        \\
&\overset{\underset{\indep}{}}{=}1-\PP(X>u)\ \PP(Y>u)=         \\
&= 1-(1-F_X(u))(1-F_Y(u))        \\
&=F_X(u)+F_Y(u)-F_X(u)\ F_Y(u)         
\end{aligned}
\end{gather*}
Analogamente, fissato $v>0$, esprimiamo anche il $\max$ in funzione delle due funzioni di ripartizione
\begin{gather*}
\begin{aligned}
F_V(v)&=\PP(V\leq v)=\\
&=\PP(\max(X,Y)\leq v)=         \\
&=\PP(X\leq v,Y\leq v)=        \\
&\overset{\underset{\indep}{}}{=}\PP(X\leq v)\ \PP(Y\leq v)=         \\
&=F_X(v)\ F_Y(v)      
\end{aligned}
\end{gather*}
Dobbiamo quindi calcolarci $F_X$ e $F_Y$ (o le copiamo, tanto sono note, oppure integriamo le $f$ ricavate al punto (a))
\begin{gather*}
\begin{aligned}
&F_X(x)=\int_0^x te^{-t}\dt=1-(x+1)e^{-x}\\  
&F_Y(y)=\int_0^y e^{-t}=1-e^{-y}
\end{aligned}
\end{gather*}
Riprendiamo il calcolo di $F_U$ e $F_V$
\begin{gather*}
\begin{aligned}
F_U(u)&=F_X(u)+F_Y(u)-F_X(u)\ F_Y(u)=\\
&=1-(u+1)e^{-u} + 1-e^{-u} -(1-(u+1)e^{-u})(1-e^{-u})=\\ 
&\overset{\underset{\cdots}{}}{=}1-(u+1)e^{-2u} \\
F_V(v)&=F_X(v)\ F_Y(v)=\\
&=(1-(v+1)e^{-v})(1-e^{-v})=\\
&\overset{\underset{\cdots}{}}{=}(v+1)e^{-2v}-(v+2)e^{-v}+1
\end{aligned}
\end{gather*}
Siamo pronti per calcolare i valori attesi di $U$ e $V$. In particolare faremo vedere per completezza due modi diversi per procedere: per il calcolo di $\EE[U]$ deriviamo $F_U$ trovando $f_U$ e poi usiamo la definizione di $\EE$; per il calcolo di $\EE[V]$ usiamo il risultato notevole (\ref{introth9}). Allora:
\begin{gather*}
\begin{aligned}
f_U(u)&=(F_U(u))'=(2u+1)e^{-2u}\ \Ind_{(0,+\infty)}(u) \\
\Longrightarrow\ \EE[U]&=\intc u\cdot (2u+1)e^{-2u}\du =\\
&\overset{\underset{pp}{}}{=}\left[(2u^2+u)(-\frac{1}{2}e^{-2u})   \right]_0^{+\infty} -\intc (4u+1)(-\frac{1}{2}e^{-2u})\du \\
&\overset{\underset{pp}{}}{=}0+\frac{1}{2}\left\{\left[(4u+1)(-\frac{1}{2}e^{-2u})  \right]_0^{+\infty}+2\intc e^{-2u}\du   \right\}=\\
&=\frac{1}{4}+\left[ -\frac{1}{2}e^{-2u} \right]_0^{+\infty}=\\
&=\frac{1}{4}+\frac{1}{2}=\\
&=\frac{3}{4} \\
\EE[V]&=\intc (1-F_V(v))\dv=\\
&=\intc (v+2)e^{-v}-(v+1)e^{-2v}\dv=\\
&\overset{\underset{pp}{}}{=}\left[-(v+2)e^{-v}+\frac{1}{2}(v+1)e^{-2v}  \right]_0^{+\infty}+\intc e^{-v}\dv-\intc \frac{1}{2}e^{-2v}\dv=\\
&=2-\frac{1}{2}+\left[-e^{-v}  \right]_0^{+\infty}  + \left[\frac{1}{4}e^{-2v}  \right]_0^{+\infty}=\\
&=2-\frac{1}{2}+1-\frac{1}{4}=\\
&=\frac{9}{4}
\end{aligned}
\end{gather*}
\end{enumerate}

\Soluzione{}
Prima di tutto, il testo dice che $(X,Y)\sim\UUU(Q)$ e dunque
\[
\fXYxy=\frac{1}{m(Q)}\ \Ind_Q(x,y)=\frac{1}{8}\ \Ind_Q(x,y)
\]
\fg{0.5}{7_7}
\begin{enumerate}
\item [(a)] Qual è la legge di $X$?

Per calcolare $f_X$ sfruttiamo il teorema (\ref{introth5}):
\[
f_X(x)=\int_\RR\fXYxy\dy
\]
Pe risolvere tale integrale usiamo il teorema di Fubini (\ref{introth1}): vedendo $Q$ come dominio normale rispetto all'asse $x$, ossia compreso tra le funzioni $f_1(x)=|x-2|$ e $f_2(x)=4-|x-2|$, fissato $0\leq x\leq4$, possiamo scrivere
\begin{gather*}
\begin{aligned}
f_X(x)&=\int_{|x-2|}^{4-|x-2|}\frac{1}{8}\dy=\\
&=\left[\frac{1}{8} y  \right]_{|x-2|}^{4-|x-2|}=\\
&=\frac{1}{2}-\frac{1}{4}|x-2|
\end{aligned}
\end{gather*}
Allora
\[
f_X(x)=\left(\frac{1}{4}(2-|x-2|)  \right)\Ind_{[0,4]}(x)
\]

\item [(b)] Quanto vale $\EE[X]$?

Abbiamo
\begin{gather*}
\begin{aligned}
\EE[X]&=\intc x\ f_X(x)\dx=\\
&=\int_0^4 \frac{1}{4}x(2-|x-2|)\dx=\\
&=\int_0^4 \frac{1}{2}x\dx-\int_0^4 \frac{1}{4}x|x-2|\dx=\\
&=\left[\frac{1}{4}x^2\right]_0^4-\left(\int_0^2 -\frac{1}{4}x(x-2)\dx+\int_2^4 \frac{1}{4}x(x-2)\dx  \right)=\\
&=4+\frac{1}{4}\left[ \frac{x^3}{3}-x^2 \right]_0^2-\frac{1}{4}\left[\frac{x^3}{3}-x^2  \right]_2^4=\\
&=2
\end{aligned}
\end{gather*}

\item [(c)] Quanto vale $\PP(X<Y)$?

L'evento $(X<Y)$ corrisponde all'evento $(Y>X)$, cioè alla zona di piano appartenente a $Q$ e soprastante la bisettrice del primo quadrante:
\fg{0.4}{7_8}
Siccome la probabilità è uniforme, la probabilità di "cadere" in metà quadrato è 
\[
\PP(X<Y)=\frac{1}{2}
\]
oppure
\[
\PP(X<Y)=\frac{m(X<Y)}{m(Q)}=\frac{4}{8}=\frac{1}{2}
\]

\item [(d)] Quanto vale $\PP(X\leq 2,Y\leq 1)$?

Analogamente al punto precedente
\[
\PP(X\leq 2,Y\leq 1)=\frac{m(X\leq 2,Y\leq 1)}{m(Q)}=\frac{1/2}{8}=\frac{1}{16}
\]
\fg{0.4}{7_9}

\item [(e)] $X$ e $Y$ sono indipendenti?

Sulla scia del punto (a), vediamo $Q$ come dominio normale rispetto all'asse $y$, cioè per $|2-y|\leq x\leq 4-|2-y|$. Allora
\begin{gather*}
\begin{aligned}
f_Y(y)&=\int_\RR\fXYxy\dx=\\
&=\int_{|2-y|}^{4-|2-y|}\frac{1}{8}\dx=\\
&=\left[\frac{1}{8} x  \right]_{|2-y|}^{4-|2-y|}
\end{aligned}
\end{gather*}
Quindi
\[
f_Y(y)=\left(\frac{1}{4}(2-|y-2|)  \right)\Ind_{[0,4]}(y)
\]
e come possiamo notare $\fXY\neq f_X\cdot f_Y$ cioè $X$ e $Y$ non sono indipendenti.

\end{enumerate}

\Soluzione{}
Siano
\[
R\sim\UUU([a,b])\quad \text{con }a=11.5,\ b=12.5\qquad\qquad\qquad M=\frac{1}{R}
\]
rispettivamente resistenza e conduttanza di un dato circuito.

\begin{enumerate}
\item [(a)] $R$ e $M$ hanno legge congiunta continua?

Se per assurdo il vettore $(R,M)$ fosse assolutamente continuo allora dovrebbe esistere
\[
f_{(R,M)}:\RR^2\to[0,+\infty)
\]
tale che
\[
\PP((R,M)\in B)=\int_B f_{(R,M)}\dxy\qquad\forall B\in\BBB(\RR^2)
\]
Ma prendendo
\[
\overline{B}=\left\{(x,y)\in\RR^2\ :\ 11.5\leq x\leq 12.5,\ y=\frac{1}{x}  \right\}
\]
incorriamo in una contraddizione:
\begin{gather*} 
\PP((R,M)\in \overline{B})=0\qquad\text{perché l'insieme ha misura di Lebesgue }m_2=0 \\
\PP((R,M)\in \overline{B})=\PP\left(R=\frac{1}{M}\right)=1\qquad\text{proprio per come sono definite }R\text{ e }M
\end{gather*}
Abbiamo quindi sbagliato a presupporre che il vettore $(R,M)$ fosse assolutamente continuo.

\begin{oss} Abbiamo chiarito ancora una volta la non invertibilità del teorema (\ref{introth5-}. \end{oss}

\item [(b-c)] Calcolare i valori attesi di $R$ e $M$. Calcolare la matrice varianza di $(R,M)$.

Dato che $R$ è uniforme:
\[
\EE[R]=\frac{a+b}{2}=12 \qquad Var(R)=\frac{(b-a)^2}{12}=\frac{1}{12}=0.08333
\]
Invece per $M$:
\begin{gather*}
\begin{aligned}
\EE[M]&=\EE\left[ \frac{1}{R} \right]=\int_\RR \frac{1}{x}\ f_R(x)\dx=\int_\RR \frac{1}{x}\ \frac{1}{b-a}\ \Ind_{[a,b](x)}\dx=\\
&=\inta \frac{1}{x}\dx=\log\left( \frac{b}{a} \right)=0.08338\\
\EE[M^2]&=\int_\RR \frac{1}{x^2}\ f_R(x)\dx=\inta \frac{1}{x^2}\dx=6.956\cdot 10^{-3}\\
\implies Var(M)&=\EE[M^2]-\EE[M]^2=4.029\cdot 10^{-6}
\end{aligned}
\end{gather*}

Per concludere
\begin{gather*}
\begin{aligned}
Cov(R,M)&=\EE[RM]-\EE[R]\,\EE[M]=1-12\cdot 0.08338 = -5.793\cdot 10^{-4}\\
Var(R,M)&=
\begin{pmatrix}
0.08338 & -5.793\cdot 10^{-4} \\
-5.793\cdot 10^{-4} & 4.029\cdot 10^{-6} \\
\end{pmatrix}
\end{aligned}
\end{gather*}

\item [(d)] Calcolare il coefficiente di correlazione lineare fra $R$ e $M$.

Dalla definizione di coefficiente di correlazione lineare abbiamo
\[
\rho_{R,M}=\frac{Cov(R,M)}{\sqrt{Var(R)}\,\sqrt{Var(M)}}=\frac{-5.793\cdot 10^{-4}}{\sqrt{0.08338}\,\sqrt{4.029\cdot 10^{-6}}}=-0.9998
\]
Nonostante sia parecchio invintante, NON possiamo approssimare $\rho_{R,M}=-1$, in quanto per il teorema (\ref{introth10}) $R$ e $M$ dovrebbero essere legate da una relazione lineare, cosa che invece non avviene.

\item [(e)] Determinare la distribuzione di $M$.

Come sempre (siamo nel caso uni - dimensionale !!), per rispondere alla domanda si passa per il calcolo della funzione di ripartizione di $M$. E come sempre, prima di procedere è utile capire com'è fatto il supporto di $M$:
\[
S_M=\left\{ t\in\RR\ :\ \frac{1}{b}\leq t\leq\frac{1}{a}  \right\}
\]
Allora
\begin{enumerate}
\item [(i)] se $t<\dfrac{1}{b}$ allora $F_M(t)=0$

\item [(ii)] se $t>\dfrac{1}{a}$ allora $F_M(t)=1$

\item [(iii)] se $\dfrac{1}{b}\leq t\leq\dfrac{1}{a}$ allora
\begin{gather*}
\begin{aligned}
F_M(t)&=\PP(M\leq t)=\PP\left(\frac{1}{R}\leq t\right)=\PP\left(R\geq\frac{1}{t}\right)=1-F_R\left(\frac{1}{t}\right)=\\
&=1-\int_{-\infty}^{1/t}\frac{1}{b-a}\ \Ind_{[a,b](x)}\dx=1-\left(\frac{1}{t}-a\right)=12.5-\frac{1}{t}
\end{aligned}
\end{gather*}
\end{enumerate}
Dunque
\[
F_M(t)=
\begin{cases}
0                               &\text{se }t<\dfrac{1}{b} \\
12.5-\dfrac{1}{t}        &\text{se } \dfrac{1}{b}\leq t\leq \dfrac{1}{a}  \\
1                               &\text{se }t>\dfrac{1}{a}
\end{cases}
\]
Notiamo infine che $F_M\in\Ct$ e quindi $M$ è variabile aleatoria assolutamente continua con densità
\[
f_M(t)=F_M'(t)=\frac{1}{t^2}\ \Ind_{\left(\frac{1}{b},\frac{1}{a}\right)}(t)
\]

\begin{oss} Avremmo potuto calcolare $f_M$ anche con la formula di Jacobi (\ref{introth3}), infatti:
\begin{itemize}
\item $g(x)=\frac{1}{x}$ iniettiva su $S=(a,b)$ e con immagine $g(S)=\left(\frac{1}{b},\frac{1}{a}\right)$;
\item $g\in\Cu(S)$;
\item $g'(x)=-\frac{1}{x^2}\neq 0\quad \forall x\in S$.
\end{itemize}
Allora
\[
f_M(t)=f_R(g(y))\ |g'(y)|\ \Ind_{\left(\frac{1}{b},\frac{1}{a}\right)}(y)=\frac{1}{y^2}\ \Ind_{\left(\frac{1}{b},\frac{1}{a}\right)}(y)
\]
\end{oss}

\end{enumerate}

\Soluzione{}
Mettiamo a fuoco la situazione: fissiamo due linee parallele a distanza $d$ sul pavimento e immaginiamo che un ago venga lanciato in modo casuale in tale zona di piano (per "periodicità dell'esperimento" questo è sufficiente)
\fg{0.5}{7_11}
Procediamo per passi graduali:
\begin{enumerate}

\item Riformulazione dell'esperimento in termini astratti.

Osserviamo che la posizione di caduta dell'ago è univocamente determinata dal punto in cui cade il suo baricentro e dall'inclinazione dell'ago rispetto alle linee del pavimento. Viene quindi naturale introdurre le variabili aleatorie
\begin{gather*}
\begin{aligned}
X&=\begin{matrix}
\text{"distanza del baricentro dell'ago dalla} \\
\text{linea del pavimento più vicina"}
\end{matrix} \\
\Theta &=\begin{matrix}
\text{"angolo (antiorario) formato dall'ago} \\ 
\text{rispetto alle linee del pavimento"}
\end{matrix}
\end{aligned}
\end{gather*}
\fg{0.5}{7_12}
Possiamo allora supporre
\begin{align*}
X&\sim\UUU\left(\left[0,\frac{d}{2}\right]\right) &\implies f_X(x)&=\frac{2}{d}\ \Ind_{\left[0,\frac{d}{2}\right]}(x) \\
\Theta&\sim\UUU\left(\left[0,\pi\right]\right) &\implies f_\Theta(\vartheta)&=\frac{1}{\pi}\ \Ind_{\left[0,\pi\right]}(\vartheta) \\
X&\indep\Theta&\implies f_{(X,\Theta)}(x,\vartheta)&=\frac{2}{\pi d}\ \Ind_{\left[0,\frac{d}{2}\right]}(x)\ \Ind_{\left[0,\pi\right]}(\vartheta)
\end{align*}

\item Caratterizzazione dell'evento di interesse, cioè ci chiediamo quando l'ago intersechi almeno una linea del pavimento.
\fg{0.5}{7_13}
Osservando il grafico capiamo che ciò si verifica quando
\[
\boxed{X\leq\frac{l}{2}\sin\Theta}
\]
Possono verificarsi due casi:
\begin{enumerate}

\item [(a)] $l\leq d$, cioè può esserci al più un'intersezione

\item [(b)] $l>d$, cioè possono esserci più di un'intersezione

\end{enumerate}

\medskip

\begin{enumerate}

\item [(a)] L'obiettivo è quello di calcolare $\PP\left(\frac{l}{2}\sin\Theta\geq X\right)$.

Detto 
\[
A:=\left\{(x,\vartheta)\in \left[0,\frac{d}{2}\right]\times \left[0,\pi\right]\ :\ \frac{l}{2}\sin\vartheta\geq x  \right\}
\]
\fg{0.4}{7_14}
abbiamo
\begin{gather*}
\begin{aligned}
\PP(A)&=\int_A f_{(X,\Theta)}(x,\vartheta)\dx\text{d}\vartheta=\\
&=\frac{2}{\pi d}\int_0^\pi \left(\int_0^{\frac{l}{2}\sin\vartheta}1\dx   \right)\text{d}\vartheta=\\
&=\frac{l}{\pi d}\int_0^\pi \sin\vartheta\text{ d}\vartheta=\\
&=\frac{2l}{\pi d}
\end{aligned}
\end{gather*}

\item [(b)] Siamo sempre interessati alla probabilità dell'evento $A$. Tuttavia in questo caso:
\fg{0.6}{7_15}

Denotiamo con $B$ il nuovo insieme di interesse. Allora possiamo vedere $B=B_1\cup B_2\cup B_3=A-V$:
\fg{0.6}{7_16}

Quindi
\begin{gather}
\begin{aligned}
\label{eqn_7_8}
\PP(A)&=\int_A f_{(X,\Theta)}(x,\vartheta)\dx\text{d}\vartheta=\\
&=\frac{2}{\pi d}\int_B 1\dx\text{d}\vartheta=\\
&=\frac{2}{\pi d}\ m_2(B)=\\
&=\frac{2}{\pi d}\ (m_2(A)-m_2(V))=\\
&=\frac{2}{\pi d}\left[\int_0^\pi \left(\int_0^{\frac{l}{2}\sin\vartheta}1\dx \right)\text{d}\vartheta - \int_{\vartheta_1}^{\vartheta_2} \left(\int_{\frac{d}{2}}^{\frac{l}{2}\sin\vartheta}1\dx \right)\text{d}\vartheta  \right]
\end{aligned}
\end{gather}

Calcoliamo separatamente i due integrali

\begin{gather*}
\begin{aligned}
\int_0^\pi \left(\int_0^{\frac{l}{2}\sin\vartheta}1\dx \right)\text{d}\vartheta &=\int_0^\pi \frac{l}{2}\sin\vartheta\text{ d}\vartheta=l \\
\int_{\vartheta_1}^{\vartheta_2} \left(\int_{\frac{d}{2}}^{\frac{l}{2}\sin\vartheta}1\dx \right)\text{d}\vartheta&=\int_{\vartheta_1}^{\vartheta_2} \left(\frac{l}{2}\sin\vartheta-\frac{d}{2}\right)\text{d}\vartheta= \\
&=\left[-\frac{l}{2}\cos\vartheta  \right]_{\vartheta_1}^{\vartheta_2}-\left[\frac{d}{2}\vartheta  \right]_{\vartheta_1}^{\vartheta_2}=\\
&=\frac{l}{2}\cos\vartheta_1-\frac{l}{2}\cos\vartheta_2-\frac{d}{2}\vartheta_2+\frac{d}{2}\vartheta_1 \\
&\overset{\underset{\text{oss}}{}}{=}l\sqrt{1-\left(\frac{d}{l}\right)^2}-\frac{d}{2}\left(\pi-2\arcsin \left(\frac{d}{l}\right) \right)
\end{aligned}
\end{gather*}

\begin{oss}$\\$
Sono state usate le seguenti proprietà trigonometriche
\[
\cos(\arcsin(x))=\sqrt{1-x^2}\qquad \cos(\alpha-\beta)=\cos\alpha\,\cos\beta+\sin\alpha\,\sin\beta
\]
\end{oss}

Riprendendo quindi l'equazione (\ref{eqn_7_8}) otteniamo
\[
\PP(A)=\frac{2}{\pi d}\left(l-l\sqrt{1-\left(\frac{d}{l}\right)^2}+\frac{\pi d}{2}-d\arcsin\left(\frac{d}{l}\right)  \right)=\frac{2l}{\pi d}\left(1-\sqrt{1-\left(\frac{d}{l}\right)^2}\right)+1-\frac{\pi }{2}\arcsin\left(\frac{d}{l}\right)
\]

\end{enumerate}

\end{enumerate}

In conclusione

\[
\PP(A)=
\begin{cases}
\dfrac{2l}{\pi d}    &\text{se }l\leq d \\
\dfrac{2l}{\pi d}\left(1-\sqrt{1-\left(\dfrac{d}{l}\right)^2}\right)+1-\dfrac{\pi }{2}\arcsin\left(\dfrac{d}{l}\right) &\text{se }l> d
\end{cases}
\]

\begin{oss}[Importanza storica dell'esperimento]$\\$
Consideriamo $l\leq d$. Abbiamo trovato che $p=\PP(A)=\dfrac{2l}{\pi d}$ e quindi
\[
\pi=\dfrac{2l}{p d}=2ap^{-1}
\]
con $a$:="rapporto tra lunghezza dell'ago e distanza tra le linee del pavimento".

Interpretando la probabilità in maniera frequentista come $p=f/t$ dove $f$ rappresenta il numero di esperimenti favorevoli (cioè ago interseca linea) e $t$ il numero di esperimenti totali, abbiamo che
\[
p=\displaystyle\lim_{t\to+\infty}\frac{f}{t}
\]
Se il numero di esperimenti è abbastanza grande si può allora ottenere un'approssimazione probabilistica di $\pi$:
\[
\pi=2ap^{-1}=2a\displaystyle\lim_{t\to+\infty}\frac{t}{f}\implies \pi\approx 2a\frac{t}{f}
\]
Si lasciano di seguito alcuni link con simulazioni: \\
\url{https://datagenetics.com/blog/may42015/index.html} \\
\url{https://www.maplesoft.com/support/help/Maple/view.aspx?path=MathApps%2FBuffonsNeedleProblem}
\end{oss}

\Soluzione{}
\begin{enumerate}
\item [(a)] Si scriva la densità di probabilità del vettore aleatorio $(X,Y)$.

Siccome $X,Y$ sono VAR assolutamente continue e indipendenti allora il vettore aleatorio $(X,Y)$ è assolutamente continuo e la sua legge congiunta è
\[
\fXYxy=f_X(x)\cdot f_Y(y)=\Ind_{[0,1]}(x)\cdot \Ind_{[0,1]}(y)=_{[0,1]^2}(x,y)
\]
cioè $(X,Y)\sim\UUU\left([0,1]^2\right)$.

\item [(b)] Si consideri il vettore aleatorio $(U,V)=(X+1,X+Y)$. Calcolarne il valore atteso e la matrice varianza.

Abbiamo
\[
\mathbb{E}\left[\begin{pmatrix}
U \\ V
\end{pmatrix}\right] =
\begin{pmatrix}
\mathbb{E}[U] \\ \mathbb{E}[V]
\end{pmatrix} = 
\begin{pmatrix}
\mathbb{E}[X+1] \\ \mathbb{E}[X+Y]
\end{pmatrix} = 
\begin{pmatrix}
\mathbb{E}[X]+1 \\ \mathbb{E}[X] +\mathbb{E}[Y]
\end{pmatrix} =
\begin{pmatrix}
1/2+1 \\ 1/2+1/2
\end{pmatrix} = 
\begin{pmatrix}
3/2 \\ 1
\end{pmatrix}
\]
\begin{gather*}
\begin{aligned}
Var(U)&=Var(X+1)=Var(X)=1/12 \\
Var(V)&=Var(X+Y)=Var(X)+Var(Y)+\underbrace{2Cov(X,Y)}_{0\text{ perché }X\indep Y}=1/6 \\
Cov(U,V)&=Cov(X+1,X+Y)=\\
&=\underbrace{Cov(X,X)}_{Var(X)}+\underbrace{Cov(X,Y)}_{0}+\underbrace{Cov(1,X)}_{0}+\underbrace{Cov(1,Y)}_{0}=Var(X)=1/12 \\
Var(U,V)&=\begin{pmatrix}
 \dfrac{1}{12}& \dfrac{1}{12} \\
\, & \, \\
\dfrac{1}{12} &\dfrac{1}{6}  \\
\end{pmatrix}  
\end{aligned}
\end{gather*}
In alternativa si poteva usare il teorema (\ref{introth13}), dopo aver introdotto l'opportuna trasformazione affine $(U,V)=A\,(X,Y)+b$. Introdurre tale trasformazione non è utile solo per calcolare valore atteso e matrice varianza, ma anche per ricavare la legge del vettore trasformato. Vediamo nel prossimo punto come fare.

\item [(c)] Se ne calcoli la legge.

Vediamo il vettore $(U,V)$ come
\[
\begin{pmatrix}
U \\ V
\end{pmatrix}  
=
g \begin{pmatrix}
X \\ Y
\end{pmatrix} 
=
\underbrace{\begin{pmatrix}
1 & 0 \\
1 & 1 \\
\end{pmatrix}}_{A}\begin{pmatrix}
X \\ Y
\end{pmatrix}+\underbrace{\begin{pmatrix}
1 \\0
\end{pmatrix}}_{b} 
\]
In tal caso $g:\RR^2\to\RR^2$ è iniettiva, $\Cu$ e ha jacobiano $J_g=A$ con $\det A=1\neq 0$.

\begin{oss}$\\$
Se nel caso 1D $(mx+q)'=m$ nel caso $n$D $(AX+b)'=J_{Ax+b}=A$.
\end{oss}

Possiamo quindi applicare la formula di Jacobi (\ref{introth4}) con $g^{-1}:g(\RR^2)\to\RR^2$ tale che
\begin{gather*}
\begin{aligned}
f_{(U,V)}(u,v)&=\fXY(g^{-1}(u,v))\cdot |\det J_{g^{-1}}(u,v)|=\\
&=\Ind_{[0,1]^2}(g^{-1}(u,v))\cdot\frac{1}{|\det J_g(g^{-1}(u,v))|}=\\
&=\Ind_{g([0,1]^2)}(u,v)\cdot\frac{1}{|\det A|}=\\
&=\Ind_Q(u,v)
\end{aligned}
\end{gather*}
\fg{0.6}{7_19}
Quindi $(U,V)\sim\UUU(Q)$.

\item [(d)] Il vettore ha componenti indipendenti?

Confrontando $f_{(U,V)}$ con $f_U$ e $f_V$ capiamo che $f_{(U,V)}\neq f_U\cdot f_V$ quindi $U$ e $V$ non sono indipendenti.

\begin{oss}Del resto, quando l'insieme di arrivo ha linee di perimetro oblique le due variabili non possono essere indipendenti, perché tali linee avranno forma $y=mx+q$ e quindi apppunto $y$ dipende da $x$ e viceversa. 
\end{oss}

\item [(e)] Si consideri ora il vettore aleatorio $(U,V)=(X+1,X+Y-XY+X^2Y)$. Calcolarne il valore atteso.

Abbiamo
\begin{gather*}
\begin{aligned}
\EE[U]&=\EE[X+1]=3/2 \\
\EE[V]&=\EE[X]+\EE[Y]-\EE[X]\,\EE[Y]+\EE[X^2]\,\EE[Y]=11/12 \\
\implies \EE[(U,V)]&=(3/2,11/12)
\end{aligned}
\end{gather*}

\item [(f)] Se ne calcoli la legge.

Vediamo
\[
\begin{pmatrix}
U \\ V
\end{pmatrix}  
=
g \begin{pmatrix}
X \\ Y
\end{pmatrix}
=
\begin{pmatrix}
X+1 \\ X+Y-XY+X^2Y
\end{pmatrix}
\]
quindi 
\begin{itemize}
\item $g$ iniettiva;
\item $g\in\Cu$;
\item determinante dello jacobiano
$$
\det J_g(x,y)=\det \begin{pmatrix}
\dfrac{\partial (x+1)}{\partial x} &\dfrac{\partial (x+1)}{\partial y}  \\
\, & \, \\
\dfrac{\partial (x+y-xy+x^2y)}{\partial x}   &  \dfrac{\partial (x+y-xy+x^2y)}{\partial y}  \\
\end{pmatrix}
= \det
\begin{pmatrix}
1 & 0 \\
\, &\, \\
1-y+2xy & 1-x+x^2 \\
\end{pmatrix} 
$$
diverso da zero $\forall (x,y)\in S=[0,1]^2$.
\end{itemize}
La trasformazione $g$ soddisfa quindi le ipotesi del teorema di Jacobi (\ref{introth4}), e possiamo allora cercare $g^{-1}(u,v)$:
\[
\begin{cases} u=x+1 \\ v=x+y-xy+x^2y  \end{cases}\implies \begin{cases} x=u-1 \\ y=\frac{v-x}{1-x+x^2}\end{cases}\implies g^{-1}(u,v)=\begin{pmatrix}
u-1 \\ \, \\\dfrac{v-u+1}{u^2-3u+3}
\end{pmatrix}
\]
Allora
\begin{gather*}
\begin{aligned}
f_{(U,V)}(u,v)&=\fXY(g^{-1}(u,v))\cdot |\det J_{g^{-1}}(u,v)|=\\
&=\Ind_{[0,1]^2}(g^{-1}(u,v))\cdot\frac{1}{|\det J_g(g^{-1}(u,v))|}=\\
&=\Ind_{g([0,1]^2)}(u,v)\cdot\frac{1}{1-(u-1)+(u-1)^2}=\\
&=\Ind_{g(S)}(u,v)\cdot\frac{1}{u^2-3u+3}
\end{aligned}
\end{gather*}
Ora manca da calcolare $g(S)$:
\begin{gather*}
g(S)=g(\{(x,y)\in\RR^2\ :\ 0\leq x,y\leq 1 \}) \\
\implies \begin{cases}0\leq x\leq 1\implies 0\leq u-1\leq 1\implies 1\leq u \leq 2 \\ 0\leq y\leq 1\implies 0\leq \frac{v-u+1}{u^2-3u+3} \leq 1 \implies u-1\leq v\leq u^2-2u+2   \end{cases} \\
\implies g(S)=\{(u,v)\in\RR^2\ :\ 1\leq u \leq 2,\ u-1\leq v\leq u^2-2u+2 \}=Q\\
\, \\
\implies f_{(U,V)}(u,v)=\Ind_Q(u,v)\ \frac{1}{u^2-3u+3}
\end{gather*}
\fg{0.6}{7_20}

\item [(g)] Il vettore ha componenti indipendenti?

Per quanto detto al punto (d), a maggior ragione in questo caso, è impossibile che $U$ e $V$ siano indipendenti.

\end{enumerate}

\Soluzione{}
Manca

\Soluzione{}
Manca

\Soluzione{}
Siano 
\begin{align*}
E_1&\sim\EEE(\lambda)=\EEE(0.3) \\
E_2&\sim\EEE(\mu)=\EEE(0.1) \\
E_3&\sim\EEE(\gamma)=\EEE(0.2)
\end{align*}
Sia $T$ la variabile aleatoria che modellizza il tempo di vita del componente elettronico formato dal collegamento in serie dei tre sottocomponenti indipendenti che hanno un tempo di vita modellizzato rispettivamente da $E_1,E_2,E_3$.

\begin{enumerate}
\item [(a)] Qual è la legge di $T$?

Dato che gli elementi sono collegati in serie, l'apparecchio si guasta non appena se ne guasta uno, quindi
\[
T=\min\{E_1,E_2,E_3 \}
\]
Possiamo anche osservare che, poiché $E_i\geq 0$ q.c. $\forall i = 1:3$, si ha $T\geq 0$ q.c. Allora con l'obiettivo di determinare la legge di $T$, fissiamo un $t\geq 0$ e calcoliamo la sua funzione di ripartizione:
\begin{gather*}
\begin{aligned}
F_T(t)&=\PP(T\leq t)=1-\PP(T>t)= \\
&=1-\PP(\min\{E_1,E_2,E_3 \}>t)=\\
&=1-\PP(E_1>t,E_2>t,E_3>t)=\\
&\overset{\underset{\indep}{}}{=}1-\PP(E_1>t)\,\PP(E_2>t)\,\PP(E_3>t)=\\
&=1-(1-F_{E_1}(t))\,(1-F_{E_2}(t))\,(1-F_{E_3}(t))
\end{aligned}
\end{gather*}
Ricordando che
\begin{oss}$\\$
Se $X\sim\EEE(\lambda)$ allora
\[
F_X(x)=\left(1-e^{-\lambda x}\right)\Ind_{[0,+\infty)}(x)
\]
\end{oss}
otteniamo
\begin{gather*}
F_T(t)=1-\left(e^{-\lambda t}\right)\,\left(e^{-\mu t}\right)\,\left(e^{-\gamma t}\right)=1-e^{-(\lambda+\mu+\gamma)t} \\
\implies F_T(t)=\left(1-e^{-(\lambda+\mu+\gamma)t} \right)\Ind_{[0,+\infty)}(t) \\
\implies T\sim\EEE(\lambda+\mu+\gamma)=\EEE(0.6)
\end{gather*}

\item [(b)] Per aumentare l'affidabilità e ridurre gli interventi di sostituzione, viene proprosto di aggiungere un componente identico in parallelo. Qual è la legge del tempo di vita $S$ del nuovo complesso?

Siano $C_1$ e $C_2$ i due componenti identici collegati in parallelo che formano il componente complessivo $C_{\text{TOT}}$, rispettivamente di vita $T,P,S$. Sappiamo che $T\sim\EEE(0.6)\sim P$ e per costruzione è lecito supporre $T\indep P$.

Dato che $C_1$ e $C_2$ sono collegati in parallelo, $C_{\text{TOT}}$ si guasterà quando $C_1$ e $C_2$ saranno entrambi guasti, perciò
\[
S=\max\{T,P\}
\]
Analogamente al punto precedente, fissato $s\geq 0$ si ha:
\begin{gather*}
\begin{aligned}
F_S(s)&=\PP(S\leq s)=\\
&=\PP(\max\{T,P \}\leq s)=\\
&=\PP(T\leq s,P\leq s)=\\
&\overset{\underset{\indep}{}}{=}\PP(T\leq s)\,\PP(P\leq s)=\\
&=F_T(s)\,F_P(s)=\\
&=\left(1-e^{-0.6s}  \right)^2\\
\implies F_S(s)&=\left(1-e^{-0.6s}  \right)^2\Ind_{[0,+\infty)}(s)
\end{aligned}
\end{gather*}
Essendo $\CCC^\infty$, possiamo calcolare
\[
f_S(s)=F_S'(s)=\left(1.2\left(1-e^{-0.6s}  \right)e^{-0.6s}  \right)\Ind_{[0,+\infty)}(s)
\]
\end{enumerate}

\begin{oss}$\\$
Quanto abbiamo quadagnato rispetto al punto (a)?
Prima avevamo
\[
\EE[T]=\frac{1}{0.6}=1.6667
\]
Ora
\[
\EE[S]=\intc \left(1.2\left(1-e^{-0.6s}  \right)e^{-0.6s}  \right)\ds=2.5
\]
Fare una sostituzione ogni 30 mesi rispetto a farla ogni 20 mesi può essere molto più conveniente!
\end{oss}

\Soluzione{}
\begin{enumerate}
\item [(a)] Si calcoli la funzione di ripartiione di $Z$ e si determini la densità.

Per definizione
\[
F_Z(z)=\PP(Z\leq z)=\PP(X+Y\leq z)
\]
Come sempre, vediamo l'insieme $(X+Y\leq z)$ in funzione del vettore $(X,Y)$, cioè
\[
\PP(X+Y\leq z)=\PP((X,Y)\in R_Z)
\]
con 
\[
R_Z=\{(x,y)\in\RR^2\ :\ -\infty\leq x\leq z-y,\ -\infty\leq y\leq +\infty  \}
\]
Allora
\begin{gather*}
\begin{aligned}
F_Z(z)&=\PP((X,Y)\in R_Z)=\\
&=\int_{R_Z}\fXYxy\dxy =\\
&=\intb \left(\int_{-\infty}^{z-y}\fXYxy\dx   \right)\dy
\end{aligned}
\end{gather*}
Arrivati a questo punto è conveniente applicare un cambio di variabili di integrazioni (per l'integrale interno): poniamo $w:=x+y$ in modo tale che l'integrale che va da $-\infty$ a $z-y$ diventi della sola funzione della $z$, dato che dovremo appunto ricavarci la densità di $Z$. Così facendo, per $x\to z-y$ si ha $w\to z$, mentre $\dx=\dw$, e possiamo quindi scambiare l'integrale più esterno con quello più interno:
\begin{gather*}
\begin{aligned}
F_Z(z)&=\intb \left(\int_{-\infty}^{z-y}\fXYxy\dx   \right)\dy=\\
&=\intb \left(\int_{-\infty}^{z}\fXY(w-y,y)\dw   \right)\dy=\\
&=\int_{-\infty}^{z}\left(\intb \fXY(w-y,y)\dy    \right)\dw
\end{aligned}
\end{gather*}
Questo ci dice che $Z$ è una variabile aleatoria assolutamente continua di densità pari a $F_Z'(x)$, cioè l'integranda dell'integrale sopra valutata in $z$:
\begin{gather*}
\begin{aligned}
f_Z(z)&=F_Z'(x)=\\
&=\intb \fXY(z-y,y)\dy
\end{aligned}
\end{gather*}
E con questo si conclude la dimostrazione.
\fg{0.5}{9_4}

\item [(b)] Si calcoli la legge di $(Z,Y)$ e si trovi la legge marginale di $Z$.

Il vettore $(Z,Y)$ può essere visto come una trasformazione affine del vettore $(X,Y)$, cioè
\[
\begin{pmatrix}
Z \\ Y
\end{pmatrix} =\begin{pmatrix}
X+Y \\ Y
\end{pmatrix} = \underbrace{\begin{pmatrix}
1 & 1 \\
0 &  1\\
\end{pmatrix}}_{A}\begin{pmatrix}
X \\ Y
\end{pmatrix}
\]
Allora la funzione
\[
g\begin{pmatrix}
x \\ y
\end{pmatrix}=A\begin{pmatrix}
x \\ y
\end{pmatrix}
\]
è $\Cu$, è biiettiva perché $\det(A)=1\neq0$ ed ha jacobiano $J_g(x,y)=\det(A)=1\neq0$; in particolare è ben definita l'inversa di $g$
\[
g^{-1}\begin{pmatrix}
u \\ v
\end{pmatrix} =A^{-1}\begin{pmatrix}
u \\ v
\end{pmatrix} = \begin{pmatrix}
1 & -1 \\
0 &  1\\
\end{pmatrix}\begin{pmatrix}
u \\ v
\end{pmatrix}=\begin{pmatrix}
u-v \\ v
\end{pmatrix}
\]
Allora grazie alla formula di Jacobi (\ref{introth4}) possiamo calcolare la congiunta
\[
f_{(Z,Y)}(z,y)=\fXY(g^{-1}(z,y))\ \frac{1}{\underbrace{|\det J_g(z,y)|}_{=1\ \forall z,y}}=\fXY(z-y,y)
\]
e dedurne, grazie al teorema (\ref{introth5}), che
\[
f_Z(z)=\intb \fXY(z-y,y)\dy
\]
E con questo si conclude la dimostrazione.
\end{enumerate}

\Soluzione{}
\begin{enumerate}
\item [(a)] Determinare media, varianza e legge di $Z=X+Y$.

Abbiamo
\[
\EE[Z]=\EE[X]+\EE[Y]=\frac{1}{2}+\frac{1}{2}=1\qquad Var(Z)\overset{\underset{\indep}{}}{=}Var(X)+Var(Z)=\frac{1}{12}+\frac{1}{12}=\frac{1}{6}
\]
Per calcolare la legge di $Z$ usiamo il risultato notevole dell'esercizio precedente: $Z$ è variabile aleatoria assolutamente conitnua con densità
\[
f_Z(z)=\int_\RR \fXY(z-y,y)\dy
\]
Osserviamo che
\begin{itemize}
\item $\fXYxy\overset{\underset{\indep}{}}{=}f_X(x)\cdot f_Y(y)\overset{\underset{\UUU}{}}{=}\Ind_{[0,1]^2}(x,y)$
\item $Z\in[0,2]$ q.c.
\end{itemize}
Allora
\[
f_Z(z)=
\begin{cases}
\int_\RR\Ind_{[0,1]^2}(z-y,y)\dy &z\in[0,2] \\
0 &\text{altrove}
\end{cases}
\]
Vediamo com'è fatta $\Ind_{[0,1]^2}(z-y,y)$: per definizione si ha
\[
\Ind_{[0,1]^2}(z-y,y)=
\begin{cases}
1 &(z-y,y)\in[0,1]^2 \\
0 &\text{altrove}
\end{cases}
\]
E $(z-y,y)\in[0,1]^2$ cosa vuol dire? Vuol dire $z-y\in[0,1]$ e $y\in[0,1]$, quindi
\[
\begin{cases}
0\leq y\leq 1 \\
0\leq z-y \leq 1
\end{cases}
\implies
\begin{cases}
0\leq y\leq 1 \\
-z\leq -y \leq 1-z
\end{cases}
\implies
\begin{cases}
0\leq y\leq 1 \\
z-1\leq y \leq z
\end{cases}
\]
Allora detto 
\[
R=\{(z,y)\in\RR^2\ :\ z\in[0,2],\ 0\leq y\leq 1,\ z-i\leq y\leq z  \}
\]

\fg{0.5}{7_21}

si ha che
\[
\Ind_{[0,1]^2}(z-y,y)=\Ind_R(z,y)
\]
e quindi
\[
f_Z(z)=
\begin{cases}
\int_\RR\Ind_R(z,y)\dy &z\in[0,2] \\
0 &\text{altrove}
\end{cases}
=
\begin{cases}
\int_0^z\dy=z &z\in[0,1] \\
\int_{z-1}^1\dy=2-z  &z\in[1,2] \\
0 &\text{altrove}
\end{cases}
\]
che volendo si può riscrivere in
\[
f_Z(z)=
\begin{cases}
\int_{\max(0,z-1)}^{\min(z,1)}\dy=\min(z,1)-\max(0,z-1)=\min(z,2-z) &z\in[0,2] \\
0 &\text{altrove}
\end{cases}
\]

\fg{0.5}{7_22}

\begin{oss} Tale legge vedremo nel capitolo successivo che è il risultato della convoluzione delle due caratteristiche. \end{oss}

\item [(b$^*$)] Si trovi la legge di $T=X+Y-\Ind_{\{ X+Y-1 \}}$.

Siamo nel caso 1D, perciò per trovare la legge di $T$ passiamo per il calcolo della funzione di ripartizione. E come sempre, prima di farlo, cerchiamo di capire com'è fatto il supporto di $T$: $X+Y\in[0,2]$ q.c. allora
\begin{itemize}
\item se $X+Y\in[0,1]$ allora $\Ind_{\{ X+Y-1 \}}=0$ e quindi $T\in[0,1]$
\item se $X+Y\in[1,2]$ allora $\Ind_{\{ X+Y-1 \}}=1$ e quindi $T\in[1,2]-1=[0,1]$
\end{itemize}
Deduciamo che $T\in[0,1]$ q.c.

Allora, fissato $t\in[0,1]$, la funzione di ripartizione è
\begin{gather*}
\begin{aligned}
F_T(t)&=\PP\left(X+Y-\Ind_{\{ X+Y-1 \}}\leq t  \right)=\\
&=\PP(X+Y-1\leq t,X+Y>1)+\PP(X+Y\leq t,X+Y\leq 1)=\\
&=\PP(1<X+Y\leq t+1)+\PP(X+Y\leq t)=\\
&=\int_1^{t+1}2-z\dz+\int_0^tz\dz=\\
&=t
\end{aligned}
\end{gather*}

Quindi
\[
F_T(t)=
\begin{cases}
0 &t<0 \\
t &t\in[0,1] \\
1 &t>1
\end{cases}
\implies T\sim\UUU([0,1])
\]

\end{enumerate}

\Soluzione{}
\begin{enumerate}
\item [(a)] Determinare la distribuzione congiunta del vettore $(U,V)$ ove $U=XY$ e $V=\dfrac{Y}{X}$.

Vediamo
\[
(U,V)=g(X,Y)=(XY,Y/X)
\]
con $g:S=(0,+\infty)^2\to(0,+\infty)^2$. Possiamo dire che
\begin{itemize}
\item $g$ è iniettiva in $S$
\item $g\in\Cu(S)$
\item $\det J_g(x,y)\neq 0\ \ \forall (x,y)\in S$ perché
\[
J_g(x,y)=
\begin{bmatrix}
y & x \\
-y/x^2 & 1/x \\
\end{bmatrix}
\implies \det J_g(x,y)=2y/x
\]
\end{itemize}
Cerchiamo dunque, se esiste, $g^{-1}$: presi $(u,v)\in(0,+\infty)^2$ si ha
\[
\begin{cases}u=xy \\ v=y/x \end{cases}
\implies
\begin{cases}x=u/y \\ y=vx\end{cases} 
\implies
\begin{cases}x=u/y \\ y^2=uv\end{cases} 
\implies
\begin{cases}x=u/y \\ y=\sqrt{uv} \end{cases} 
\implies
\begin{cases}x=\sqrt{u/v} \\ y=\sqrt{uv} \end{cases}
\]
Quindi $g^{-1}$ è ben definita, cioè $g^{-1}:g(S)=(0,+\infty)^2\to S=(0,+\infty)^2$ con $g^{-1}(u,v)=(\sqrt{u/v}, \sqrt{uv})$.

Allora grazie alla formula di Jacobi (\ref{introth4}) la congiunta vale
\begin{gather*}
\begin{aligned}
f_{(U,V)}(u,v)&=\fXY(g^{-1}(u,v))\cdot |\det J_{g^{-1}}(u,v)|=\\
&=\Ind_{[0,1]^2}(g^{-1}(u,v))\cdot\frac{1}{|\det J_g(g^{-1}(u,v))|}=\\
&=\Ind_{[0,1]^2}(\sqrt{u/v},\sqrt{uv})\cdot\frac{1}{2\sqrt{uv}/\sqrt{u/v}}=\\
&=\Ind_{[0,1]^2}(\sqrt{u/v},\sqrt{uv})\cdot\frac{1}{2v}
\end{aligned}
\end{gather*}
Vediamo com'è fatta $\Ind_{[0,1]^2}(\sqrt{u/v},\sqrt{uv})$:
\[
\begin{cases}0\leq\sqrt{u/v}\leq 1  \\ 0\leq \sqrt{uv}\leq 1 \end{cases}
\implies
\begin{cases}0\leq\sqrt{u}\leq \sqrt{v}  \\ 0\leq \sqrt{uv}\leq 1 \end{cases}
\implies
\begin{cases}0\leq u\leq v  \\ u\leq v\leq 1/u \end{cases}
\]
Quindi detto $I=\{(u,v)\in(0,+\infty)^2\ :\ u\in[0,1],\ v\in[u,1/u]  \}$ si ha
\[
f_{(U,V)}(u,v)=\frac{1}{2v}\ \Ind_I(u,v)=\frac{1}{2v}\ \Ind_{[0,1]}(u)\ \Ind_{[u,1/u]}(v)
\]

\item [(b)] Determinare le marginali di $U$ e $V$.

Usando il teorema (\ref{introth5}) abbiamo
\begin{gather*}
\begin{aligned}
u\in[0,1]&\implies f_{U}(u)=\intb f_{(U,V)}(u,v)\dv=\int_u^{1/u}\frac{1}{2v}\dv=-\log u \\
&\implies  f_{U}(u)=-\log u \ \Ind_{[0,1]}(u) \\
v\in[0,1]&\implies f_V(v)=\intb f_{(U,V)}(u,v)\du=\int_0^v \frac{1}{2v}\du=\frac{1}{2} \\
v\in[1,+\infty)&\implies f_V(v)=\intb f_{(U,V)}(u,v)\du=\int_1^{1/v} \frac{1}{2v}\du=\frac{1}{2v^2} \\
&\implies f_V(v)=\frac{1}{2}\ \Ind_{[0,1]}(v)+\frac{1}{2v^2}\ \Ind_{[1,+\infty)}(v)
\end{aligned}
\end{gather*}

\end{enumerate}

\Soluzione{}
Dato che $X$ e $Y$ sono $>0$ q.c. allora $U>0$ q.c. 

Sia quindi $t>0$. Abbiamo
\begin{gather*}
\begin{aligned}
F_U(t)&=\PP(U\leq t)=\PP(X/Y\leq t)=\PP(Y\geq X/t)=\PP((X,Y)\in R)=\\
&=\left\{ R=\{(x,y)\in\RR^2\ :\ y\geq 0,\ 0\leq x\leq ty \} \right\}=\\
&=\int_R \fXYxy\dxy\overset{\underset{\indep}{}}{=}\int_R f_X(x)\cdot f_Y(y)\dxy=\\
&=\int_0^{+\infty}\left(\int_0^{ty}\frac{\lambda^{\alpha+\beta}}{\Gamma(\alpha)\,\Gamma(\beta)}\ x^{\alpha-1}\ y^{\beta-1}\ e^{-\lambda(x+y)}\dx\right)\dy
\end{aligned}
\end{gather*}
Per risolvere tale integrale procediamo in maniera analoga al punto (a) dell'esercizio 13, cioè "facciamo saltar fuori" la funzione di ripartizione di $t$ tramite un cambio di variabili: ponendo $z=x/y$ abbiamo
\begin{gather*}
\begin{aligned}
F_U(t)&=\int_0^{+\infty}\left(\int_0^{t}\frac{\lambda^{\alpha+\beta}}{\Gamma(\alpha)\,\Gamma(\beta)}\ (yz)^{\alpha-1}\ y^{\beta-1}\ e^{-\lambda(yz+y)}\ y\dz\right)\dy= \\
&=\int_0^t\left (\int_0^{+\infty}\frac{\lambda^{\alpha+\beta}}{\Gamma(\alpha)\,\Gamma(\beta)}\ y^{\alpha+\beta-1}\ z^{\alpha-1}\ e^{-\lambda(z+1)y}\dy\right)\dz
\end{aligned}
\end{gather*}
Moltiplicando e dividendo per un'opportuna quantità, ci riconduciamo alla densità di una $\Gamma(\alpha+\beta,\lambda(z+1))$ per risolvere l'integrale interno:
\begin{gather*}
\begin{aligned}
F_U(t)&=\int_0^t z^{\alpha-1}\ \frac{\Gamma(\alpha+\beta)}{(z+1)^{\alpha+\beta)}\,\Gamma(\alpha)\,\Gamma(\beta)}\ \left (\underbrace{\int_0^{+\infty}\frac{(\lambda(z+1))^{\alpha+\beta}}{\Gamma(\alpha+\beta)}\ y^{(\alpha+\beta)-1}\ e^{-(\lambda(z+1))y}\dy}_{=1}\right)\dz=\\
&=\frac{\Gamma(\alpha+\beta)}{\Gamma(\alpha)\,\Gamma(\beta)}\int_0^t\frac{z^{\alpha-1}}{(z+1)^{\alpha+\beta}}\dz
\end{aligned}
\end{gather*}
E quindi
\[
f_U(t)=\frac{\Gamma(\alpha+\beta)}{\Gamma(\alpha)\,\Gamma(\beta)}\ \frac{t^{\alpha-1}}{(t+1)^{\alpha+\beta}}\ \Ind_{(0,+\infty)}(t)
\]

\Soluzione{}
Manca

\Soluzione{}
Manca

\Soluzione{}
\begin{enumerate}
\item [(a)] Si calcoli la media di $(X+Y)^{-1}$.

Abbiamo
\begin{gather*}
\begin{aligned}
\EE\left[ \frac{1}{X+Y} \right]&=\int_{\RR^2}\frac{1}{x+y}\ \fXYxy \dxy=\\
&=\int_{\RR^2}\frac{1}{2}e^{-(x+y)}\dxy=\\
&=\frac{1}{2}\intc e^{-x}\dx\intc e^{-y}\dy=\frac{1}{2}
\end{aligned}
\end{gather*}

\item [(b)] Si determini la legge di $X+Y$.

Usando il risultato notevole dell'esercizio 13 (\ref{introth14}) abbiamo che $Z=X+Y$ è variabile aleatoria assolutamente continua con densità, per $z>0$, data da
\begin{gather*}
\begin{aligned}
f_Z(z)&=\int_\RR\fXY(z-y,y)\dy=\\
&=\int_0^z\fXY(z-y,y)\dy=\\
&=\int_0^z\frac{1}{2}ze^{-z}\dy=\frac{1}{2}z^2e^{-z}
\end{aligned}
\end{gather*}
Quindi $Z\sim\Gamma(3,1)$.

\item [(c)] Si calcoli la media di $X+Y$.

Il valore atteso di una $\Gamma$ è noto, perciò $\EE[X+Y]=3/1=3$.

\item [(d)] Si calcolino le marginali di $X,Y$. $X\indep Y$?

Grazie al teorema (\ref{introth5}):
\begin{gather*}
\begin{aligned}
f_X(x)&=\int_\RR \fXYxy\dy=\\
&=\intc \frac{1}{2}(x+y)e^{-(x+y)}\dy=\\
&=\frac{1}{2}xe^{-x}\underbrace{\intc e^{-y}\dy}_{1}+\frac{1}{2}e^{-x}\underbrace{\intc ye^{-y}\dy}_{1\text{ per }\left(\alpha^1\right)}=\frac{1}{2}(x+1)e^{-x}
\end{aligned}
\end{gather*}
Per simmetria $f_Y(y)=\dfrac{1}{2}(y+1)e^{-y}$, da cui segue che $X\not\indep Y$.

\item [(e)] Si calcoli $Cov(X,Y)$.

Abbiamo
\begin{gather*}
\begin{aligned}
\EE[X]&=\intc \frac{1}{2}x(x+1)e^{-x}\dx=\frac{3}{2}=\EE[Y] \\
\EE[XY]&=\intc \frac{1}{2}xy(x+y)e^{-(x+y)}\dxy=2
\implies Cov(X,Y)&=\EE[XY]-\EE[X]\,\EE[Y]=2-\frac{9}{4}-\frac{1}{4}
\end{aligned}
\end{gather*}

\item [(f)] Si calcoli $\PP(X\geq 1,Y\geq 2)$.

Abbiamo
\begin{gather*}
\begin{aligned}
\PP(X\geq 1,Y\geq 2)&=\int_1^{+\infty}\int_2^{+\infty} \frac{1}{2}(x+y)e^{-(x+y)}\dxy=\\
&=\int_1^{+\infty}\left[-\frac{1}{2}(x+y)e^{-(x+y)}-\frac{1}{2}e^{-(x+y)}  \right]_2^{+\infty}\dx=\\
&=\frac{5}{2}e^{-3}\approx 0.1245
\end{aligned}
\end{gather*}

\end{enumerate}

\Soluzione{}
Manca

\Soluzione{}
\begin{enumerate}
\item [(a$^*$)] Si mostri che $Y$ è una variabile aleatoria.

Manca % manca

\item [(b)] Qual è la legge di $Y$?

Con l'obiettivo di determinare la legge di $Y$, fissiamo $y\geq 0$ perché per costruzione $Y\geq 0$, e calcoliamo la sua funzione di ripartizione:
\[
F_Y(y)=\PP(Y\leq y)=1-\PP(Y>y)
\]
Concentriamoci su $\PP(Y>y)$:
\begin{gather*}
\begin{aligned}
\PP(Y>y)&=\PP\left(\min\{X_1,\dots,X_N  \}>y  \right)=\\
&=\PP\left(X_1>y,\dots,X_N>y,\bigcup_{n=1}^\infty (N=n)  \right)=\\
&=\PP\left(\bigcup_{n=1}^\infty(X_1>y,\dots,X_n>y, N=n)  \right)=\\
&=\sum_{n=1}^\infty \PP(X_1>y,\dots,X_n>y, N=n)=\\
&\overset{\underset{\indep}{}}{=}\sum_{n=1}^\infty \PP(X_1>y)\cdots\PP(X_n>y)\ \PP(N=n)=\\
&=\sum_{n=1}^\infty (1-F_{X_1}(y))\cdots (1-F_{X_n}(y))\ \PP(N=n)=\\
&=\sum_{n=1}^\infty e^{-n\lambda y}\ p(1-p)^{n-1}=\\
&=\sum_{n=0}^\infty e^{-(n+1)\lambda y}\ p(1-p)^{n}=\\
&=pe^{-\lambda y} \underbrace{\sum_{n=0}^\infty \left((1-p)e^{-\lambda y}  \right)^n}_{\text{serie geom.}}=\\
&=\frac{pe^{-\lambda y}}{1-(1-p)e^{-\lambda y}}
\end{aligned}
\end{gather*}
Quindi
\[
F_Y(y)=\left(1-\frac{pe^{-\lambda y}}{1-(1-p)e^{-\lambda y}}  \right)\Ind_{[0,+\infty)}(y)=\frac{1-e^{-\lambda y}}{1-(1-p)e^{-\lambda y}}\ \Ind_{[0,+\infty)}(y)
\]

\item [(c)] Qual è il valor atteso di $Y$? 

Dato che $Y$ è una variabile aleatoria continua e positiva possiamo usare il risultato notevole (\ref{introth9}):
\begin{gather*}
\begin{aligned}
\EE[Y]&=\int_0^{+\infty}(1-F_Y(y))\dy=\\
&=\intc \frac{pe^{-\lambda y}}{1-(1-p)e^{-\lambda y}}\dy=\\
&=\frac{p}{\lambda(1-p)}\intc \frac{\lambda(1-p)e^{-\lambda y}}{1-(1-p)e^{-\lambda y}}\dy=\\
&=\frac{p}{\lambda(1-p)}\left[\log |1-(1-p)e^{-\lambda y}|  \right]_0^{+\infty}=\\
&=\frac{p}{\lambda(1-p)}(-\log |p|)=\\
&=-\frac{1}{\lambda}\, \frac{p}{1-p}\, \log p
\end{aligned}
\end{gather*}

\end{enumerate}

\Soluzione{}
Manca

\Soluzione{}
Manca

\Soluzione{}
Siano $X\sim Y\sim\EEE(\lambda)$ con $X\indep Y$ tali che

\fg{0.4}{7_1}

\begin{enumerate}
\item Calcolare il valore atteso dell'area $A$, in funzione di $\lambda$.

Osservando che $A=XY/2$ abbiamo
\[
\EE[A]=\EE\left[ \frac{XY}{2} \right]=\frac{1}{2}\ \EE[XY]\overset{\underset{\indep}{}}{=}\frac{1}{2}\ \EE[X]\ \EE[Y]=\frac{1}{2\lambda^2}
\]

\item Calcolare la funzione di ripartizione di $Z$, notando che non dipende da $\lambda$, e disegnarne il grafico.

Dato che $X,Y>0$ q.c. si ha $Z>0$ q.c. quindi, fissato $z>0$, la funzione di ripartizione di $Z$ è
\[
F_Z(z)=\PP(Z\leq z)=\PP\left(\frac{Y}{X}\leq z \right)=\PP(Y\leq zX)
\] 
Come sempre, conviene vedere $(Y\leq zX)$ in funzione del vettore $(X,Y)$, quindi detto
\[
R_Z:=\{(x,y)\in\RR^2\ :\ x>0,\ 0<y<zx\}
\]

\fg{0.4}{7_6}

abbiamo 
\begin{gather*}
\begin{aligned}
F_Z(z)&=\PP((X,Y)\in R_Z)= \\
&=\int_{R_Z}\fXYxy\dxy= \\
&\overset{\underset{\indep}{}}{=}\int_{R_Z}f_X(x)\ f_Y(y)\dxy=\\
&=\int_{R_Z}\lambda^2e^{-\lambda(x+y)}\dxy=\\
&=\lambda^2\intc e^{-\lambda x} \left( \int_0^{zx}e^{-\lambda y}\dy  \right)\dx=\\
&=\lambda^2\intc e^{-\lambda x} \left[-\frac{1}{\lambda}\ e^{-\lambda y}  \right]_0^{zx}\dx=\\
&=\lambda\intc e^{-\lambda x} \left( -e^{-\lambda zx}+1  \right)\dx=\\
&=\lambda\intc \left(e^{-\lambda x} -e^{-\lambda (z+1)x}  \right)\dx=\\
&=\lambda\left[-\frac{1}{\lambda}\ e^{-\lambda x}+\frac{1}{\lambda(z+1)}\ e^{-\lambda (z+1)x}  \right]_0^{+\infty}=\\
&=\lambda\left(\frac{1}{\lambda}-\frac{1}{\lambda(z+1)} \right)=\\
&=1-\frac{1}{z+1}=\\
&=\frac{z}{z+1}
\end{aligned}
\end{gather*}
Quindi
\[
F_Z(z)=
\begin{cases}
0 &\text{se }z\leq 0 \\
\displaystyle\frac{z}{1+z} &\text{se }z>0
\end{cases}
\]
\fg{0.5}{7_2}

\item Determinare se $Z$ è assolutamente continua e in tal caso calcolarne la densità e disegnarne il grafico.

Dato che $F_Z\in\Cz\cap\widetilde{\mathcal{C}}^1$ allora $Z$ è una variabile aleatoria assolutamente continua con densità
\[
f_Z(z)=F_Z'(z)=
\begin{cases}
0 &\text{se }z\leq 0 \\
\displaystyle\frac{1}{(1+z)^2} &\text{se }z>0
\end{cases}
\]
\fg{0.5}{7_3}

\item Calcolare la media di $Z$.

Abbiamo
\begin{gather*}
\begin{aligned}
\EE[Z]&=\int_\RR z\ f_Z(z)\dz=\\
&=\intc\frac{z}{(1+z)^2}\dz=\\
&=\left\{ \frac{z}{(1+z)^2}=\frac{1}{1+z}-\frac{1}{(1+z)^2}  \right\}=\\
&=\intc\frac{1}{1+z}\dz-\intc\frac{1}{(1+z)^2}\dz=\\
&=\big[\log |1+z|  \big]_0^{+\infty}-\left[-\frac{1}{1+z}  \right]_0^{+\infty}=\\
&=+\infty
\end{aligned}
\end{gather*}

\item Calcolare la funzione di ripartizione di $\alpha$, notando che non dipende da $\lambda$.

Osserviamo che $\displaystyle\alpha=\arctan\left(\frac{Y}{X}\right)$ e che $\displaystyle \text{Im}(\alpha)=\left(0,\frac{\pi}{2}\right)$. Allora
\begin{itemize}
\item Per $t\leq 0$ si ha $F_\alpha(t)=0$
\item Per $t\geq\pi/2$ si ha $F_\alpha(t)=1$
\item Per $t\in(0,\pi/2)$ si ha
\begin{gather*}
\begin{aligned}
F_\alpha(t)&=\PP(\alpha\leq t)=\\
&=\PP\left(\arctan\left(\frac{Y}{Z}\right)\leq t\right)=\\
&=\PP\left(\frac{Y}{X}\leq\tan(t)\right)=\\
&=\PP(Z\leq\tan(t))=\\
&=F_Z(\tan(t))=\\
&=\frac{\tan(t)}{1+\tan(t)}
\end{aligned}
\end{gather*}
\end{itemize}

Quindi
\[
\begin{cases}
0 &\text{se }t\leq 0 \\
\displaystyle\frac{\tan(t)}{1+\tan(t)} &\text{se }\displaystyle t\in\left(0,\frac{\pi}{2}\right) \\
1 &\text{se }\displaystyle t\geq \frac{\pi}{2}
\end{cases}
\]
\fg{0.5}{7_4}

\item Determinare se $\alpha$ è assolutamente continua e in tal caso calcolarne la densità.

Analogamente al punto 3, osserviamo che $F_\alpha\in\Cz\cap\widetilde{\mathcal{C}}^1$ quindi $\alpha$ è una variabile aleatoria assolutamente continua con densità
\[
\begin{cases}
0 &\text{se }\displaystyle t\not\in\left(0,\frac{\pi}{2}\right) \\
\displaystyle\frac{1+\tan^2(t)}{(1+\tan(t))^2} &\text{se }\displaystyle t\in\left(0,\frac{\pi}{2}\right)
\end{cases}
\]
\fg{0.5}{7_5}
\end{enumerate}
