%!TEX root = ../main.tex

\ParteEsercizi

\Esercizio{}

Date due variabili aleatorie reali X e Y di varianza finita e non nulla, definite sul medesimo spazio di probabilità, si consideri il coefficiente di correlazione $\rho _{X,Y} =\frac{\mathrm{Cov} (X,Y)}{\sigma _{X} \sigma _{Y}}$. Si mostri che:

(a) se $X$ e $Y$ sono indipendenti, allora $\rho _{X,Y} =0$

(b) $\rho _{aX+b,cY+d} =\rho _{X,Y}$ per ogni $a,c >0,b,d\in \mathbb{R}$

(c) $\rho _{X,aX+b} =a/|a|$, per ogni $a\neq 0$

(d) $\mathrm{Var}\left(\frac{1}{\sigma _{Y}} Y-\frac{\rho _{X,Y}}{\sigma _{X}} X\right) =1-\rho _{X,Y}^{2}$

(e) $| \rho _{X,Y}| =1$ implica $Y=aX+b$ per qualche $a\neq 0,b\in \mathbb{R}$. Suggerimento: utilizzare il punto d
\Esercizio{* }

Dato un vettore aleatorio $X=( X_{1} ,\dotsc ,X_{n}) ,$ in cui ciascuna variabile aleatoria $X_{i}$ ha varianza finita, si mostri che $X\in \mathrm{Im} (\mathrm{C} )+\mathbb{E} [\mathrm{X} ]$ q.c., ovvero $\mathbb{P} (\mathrm{X} \in \mathrm{Im} (\mathrm{C} )+\mathbb{E} [\mathrm{X} ])=1$, dove $\mathrm{C}$ è la

matrice varianza di $X$. Suggerimento: si supponga inizialmente che $\mathbb{E} [X]=0$ e, in tal caso, si mostri che $X\in (\mathrm{Ker} (C))^{\perp } =\mathrm{Im} (C)$ q.c.
\Esercizio{}

Date $X_{1} ,\dotsc ,X_{n}$ variabili aleatorie reali i.i.d. con momento secondo finito, introdotti
\begin{gather*}
\mu =\mathbb{E}[ X_{k}] ,\ \ \sigma ^{2} =\mathrm{Var}( X_{k}) ,\\
\overline{X}_{n} =\frac{1}{n}\sum _{k=1}^{n} X_{k} ,\ \ S_{n}^{2} =\frac{1}{n-1}\sum _{k=1}^{n}( X_{k} -\overline{X}_{n})^{2} ,\ \ W_{n}^{2} =\frac{1}{n}\sum _{k=1}^{n}( X_{k} -\mu )^{2}
\end{gather*}
si mostri che:

(a) $\mathbb{E}[\overline{X}_{n}] =\mu ,\mathrm{Var}(\overline{X}_{n}) =\sigma ^{2} /n$

(b) $\sum _{k=1}^{n} X_{k}^{2} =\sum _{k=1}^{n}( X_{k} -\overline{X}_{n})^{2} +n\overline{X}_{n}^{2}$

(c) $\mathbb{E}\left[ S_{n}^{2}\right] =\mathbb{E}\left[ W_{n}^{2}\right] =\sigma ^{2}$
\Esercizio{}

Siano $X_{1} ,\dotsc ,X_{n}$ variabili aleatorie indipendenti con funzioni di ripartizione $F_{X_{1}} ,\dotsc ,F_{X_{n}}$

(a) Qual è la funzione di ripartizione di $X_{(n)} =\max\{X_{1} ,\dotsc ,X_{n}\} ?$

(b) Qual è la funzione di ripartizione di $X_{(1)} =\min\{X_{1} ,\dotsc ,X_{n}\} ?$

(c) $M=\min\{X_{2} ,X_{3}\}$ e $S=X_{1} +X_{7}$ sono indipendenti?

Le variabili aleatorie $X_{(1)}$ e $X_{(n)}$ sono indipendenti? Si consideri, ad esempio, il caso in cui $n=2$ e

$X_{1} ,X_{2}$ sono variabili aleatorie i.i.d. bernoulliane di parametro $p\in (0,1)$

(d) Come sono distribuite $X_{(1)}$ e $X_{(2)}$?

(e) $X_{(1)}$ e $X_{(2)}$ sono indipendenti?
\Esercizio{}

Siano $X,\ Y,\ Z$ variabili aleatorie reali. Mostrare che:

1. Se $Y=c$ q.c., per qualche costante $c\in \mathbb{R} ,$ allora $X$ ed $Y$ sono indipendenti.

$2^{*} .$ Se $\mathbb{P} (Z\in B)\in \{0,1\}$ per ogni boreliano $B,$ allora $Z$ è q.c. costante.

$3^{*} .$ Nel caso $Y=h(X)$, per qualche funzione boreliana $h:\mathbb{R}\rightarrow \mathbb{R} ,X$ e $Y$ sono indipendenti se e solo se $Y=c$ q.c., per qualche costante $c\in \mathbb{R}$
\Esercizio{Esercizio 6.}

Un perito elettrotecnico deve costruire un sistema costituito da tre componenti in serie. Egli pesca i tre componenti da una scatola in cui vi sono tre componenti nuovi, due usati ma funzionanti e due difettosi. Siano $X$ ed $Y$ rispettivamente il numero di componenti nuovi e di componenti usati ma funzionanti tra quelli pescati dalla scatola.

(a) Determinare la legge congiunta di $X$ ed $Y$ e le leggi marginali.

(b) Le variabili $X$ ed $Y$ sono indipendenti?

(c) Calcolare $\mathbb{E} [\mathrm{X} ],\mathbb{E} [\mathrm{Y} ],\mathbb{E} [\mathrm{XY} ].$ Scrivere la matrice varianza di $(X,Y)$ e determinare $\rho _{X,Y}$

(d) Calcolare la legge, il valore atteso e la varianza del numero di componenti pescati funzionanti.

(e) Calcolare la probabilità che l'apparecchio funzioni.
\Esercizio{Esercizio $7.$ }

Siano $X$ ed $Y$ variabili aleatorie i.i.d. bernoulliane di parametro $p=\frac{1}{2}$ e siano $u=X+Y$

e $V=|X-Y|$

(a) Mostrare che $(U,V)$ è un vettore aleatorio e determinarne la legge.

(b) Calcolare la probabilità che $V$ sia minore di $U$

(c) Calcolare la covarianza di $U$ e $V$ e la matrice varianza di $(U,V)$

(d) $U$ e $V$ sono indipendenti?
\Esercizio{Esercizio 8.}

Siano $X$ ed $Y$ due variabili aleatorie con legge congiunta parzialmente data da:
\begin{equation*}
\begin{array}{ c|c|c|c|c }
X\backslash Y & -1 & 5 & 10 & p_{X}\\
\hline
0 &  & 0.12 &  & 0.4\\
\hline
5 &  &  &  & \\
\hline
p_{Y} & 0.3 &  &  & 1
\end{array}
\end{equation*}
(a) Completare la tabella in modo che $X$ ed $Y$ siano indipendenti.

(b) Calcolare $\mathbb{P} (X< Y)$

(c) Calcolare il valore atteso del vettore $(X,Y)$ e $\mathbb{E} [XY]$

(d) Calcolare $\mathbb{P} (|XY|\geq 5)$ e $\mathbb{P} (X+Y >5)$

(e) Siano $U=|XY|$ e $V=X+Y$. Calcolare la legge congiunta di $U$ e $V$ e le leggi marginali.
\Esercizio{Esercizio 9.}

Sia $X$ una variabile aleatoria discreta con legge uniforme sull'insieme discreto $\{-1,1\}$

(a) Determinare la legge di $X$.

Sia ora $Y$ un'altra variabile aleatoria discreta, indipendente da $X,$ ma con la stessa legge di $X$. Si introduca $Z=XY$

(b) Calcolare la legge congiunta di $X$ e $Z$ e le leggi marginali.

(c) $X,Y$ e $Z$ sono indipendenti a coppie?

(d) $X,Y$ e $Z$ sono mutuamente indipendenti?
\Esercizio{Esercizio 10.}

Date $X$ e $Y$ variabili aleatorie indipendenti di legge geometrica di parametro $1/2$, si definisca $Z=\min \{X,Y\}$ e si calcolino:

(a) $\mathbb{P} (Z\leq k)$ per $k\in \mathbb{N}$

(b) la distribuzione di $Z$

(c) $\mathbb{P} (X=Y)$

(d) $\mathbb{P} (X >Y)$
\Esercizio{Esercizio 11.}

Siano $T_{1}$ e $T_{2}$ variabili aleatorie indipendenti di legge geometrica di parametro, rispettivamente, $p_{1}$ e $p_{2} ,$ con cui a tempo discreto si descrive la durata aleatoria di due apparecchiature.

(a) Scrivere la legge congiunta di $T_{1}$ e $T_{2}$

(b) Calcolare la probabilità degli eventi $\{T_{1} =T_{2}\}$ e $\{T_{1} \geq T_{2}\}$

(c) Trovare la legge della durata del sistema composto dalle due apparecchiature collegate in serie.

(d) Trovare la legge della durata del sistema composto dalle due apparecchiature collegate in parallelo.

(e) Trovare la legge congiunta di $U=\min\{T_{1} ,T_{2}\}$ e $V=\max\{T_{1} ,T_{2}\}$

(f) Trovare la legge congiunta di $U$ e $W=V-U$

(g) Trovare la legge di $W$

(h) $U$ e $W$ sono indipendenti?
\Esercizio{Esercizio 12.}

Siano $X,\ Z,\ W$ variabili aleatorie indipendenti con $Z$ e $W$ entrambe con legge di Poisson di parametro $\lambda  >0$ e $X\sim \mathrm{Be} (p),p\in (0,1)$. Definiamo la variabile aleatoria $Y=XZ+W$

(a) Determinare le leggi di $(X,Y)$ e $Y$

(b) Calcolare $\mathbb{E} [Y]$ e $\operatorname{Var} (Y)$

(c) Calcolare $\mathrm{Cov} (X,Y)$. Le variabili aleatorie $X$ e $Y$ sono indipendenti?
\Esercizio{Esercizio 13.}

Siano $N,X_{1} ,X_{2} ,\dotsc $ variabili aleatorie indipendenti, $N$ abbia legge di Poisson di parametro $\lambda  >0,$ e ciascuna delle $X_{k}$ abbia legge di Bernoulli di parametro $p\in (0,1).$ Si consideri la somma aleatoria (per addendi e per numero di addendi)
\begin{equation*}
S=\begin{cases}
0, & \text{se} \ N=0\\
X_{1} +\cdots +X_{N} , & \text{se} \ N\neq 0
\end{cases}
\end{equation*}
(a*) Si mostri che $S$ e $N-S$ sono variabili aleatorie.

(b) Qual è la legge di $S$?

(c) Qual è la legge di $N-S$?

(d) La somma $S$ è indipendente da $N$?

(e) Determinare la densità discreta congiunta di $S$ e $N-S$. La somma $S$ è indipendente da $N-S$ ?
\Esercizio{Esercizio $14.$}

Sia $X$ una variabile aleatoria discreta, con $\mathrm{Im} (X)=\{1,2,\dotsc \},$ che soddisfa
\begin{equation*}
\mathbb{P} (X\geq k)=\frac{1}{k^{\alpha }} ,\ \ k\geq 1
\end{equation*}
dove $\alpha  >0$ è un parametro reale.

(a) Calcolare la densità discreta di $X$.

(b) Calcolare la funzione di ripartizione di $X,$ disegnandone un grafico qualitativo.

(c) Per $\alpha =1$, calcolare il valore atteso di $X$

Si supponga d'ora in poi $\alpha =1.$ Sia $Y$ un'altra variabile aleatoria, indipendente da $X,$ ma con la stessa legge.

(d) Calcolare la densità discreta di $M=\min \{X,Y\}$

(e) Calcolare la densità discreta congiunta di $X$ e $X+Y$.
\Esercizio{Esercizio $15.$}

Si consideri l'estrazione di $n$ palline da un'urna contenente $B$ palline bianche e $R$ palline rosse ($B\geq 1$ e $R\geq 1$). Sia $X_{k}$ la variabile aleatoria bernoulliana che indica se alla $k$-esima estrazione esce una pallina rossa e sia $Y=\sum _{k=1}^{n} X_{k}$ il numero di palline rosse sulle $n$ estratte. Si risponda ai seguenti quesiti, sia nel caso di estrazione con reimmissione sia nel caso senza reimmissione (in quest'ultimo caso $n\leq B+R$).

(a) Qual è la legge di $Y$?

(b) Qual è il valore atteso delle $X_{k} ?$

(c) Qual è il valore atteso di $Y?$

(d) Le $X_{k}$ sono indipendenti?

(e) $Y$ e $X_{1}$ sono indipendenti?

(f) Calcolare $\mathrm{Cov}( X_{1} ,X_{2})$

$(\mathrm{g} )$ Calcolare $\mathrm{Var} (Y)$ per $n=2$
\Esercizio{Esercizio 16.}

Consideriamo infinite prove di Bernoulli indipendenti con probabilità di successo $p\in (0,1)$. Siano quindi $\Omega =\{0,1\}^{\mathbb{N}}$ e $\mathcal{A} =\sigma ( E_{k} |k=1,2,\dotsc )$,
\begin{equation*}
E_{k} =\ \text{successo alla prova } k
\end{equation*}
e sia $\mathbb{P}$ tale che $\{E_{k}\}_{k\in \mathbb{N}}$ risulti una famiglia di eventi indipendenti con $\mathbb{P}( E_{k}) =p$ per ogni $k$. Indicato con $\omega =( \omega _{k})_{k=1}^{\infty }$ il generico esito dello spazio campionario $\Omega $, definiamo infine le variabili aleatorie
\begin{equation*}
X_{k} :\Omega \rightarrow \mathbb{R} ,\ \ X_{k} (\omega )=\omega _{k} ,\ \ k\in \mathbb{N}
\end{equation*}
(a) Le variabili aleatorie $X_{k}$ sono indipendenti?

(b) Le variabili aleatorie $Y=\sum _{k=1}^{2} X_{k}$ e $N=\sum _{k=5}^{7} X_{k}$ sono indipendenti?

(c) Le variabili aleatorie $\overline{X}_{n} =\frac{1}{n}\sum _{k=1}^{n} X_{k}$ e $S_{n}^{2} =\frac{1}{n-1}\sum _{k=1}^{n}( X_{k} -\overline{X}_{n})^{2}$ sono indipendenti?

[Suggerimento: per il punto (b) dell'esercizio 3, si ha che $S_{n}^{2} =\frac{1}{n-1}\sum _{k=1}^{n} X_{k}^{2} -\frac{n}{n-1}\overline{X}_{n}^{2} ;$ notare poi che $X_{k}^{2} =X_{k}$.

(d) Se nelle prime $10$ prove vengono registrati $2$ successi, con quale probabilità almeno un successo si è verificato nelle prime $2$ prove?

(e) Se nelle prime $10$ prove viene registrato almeno un successo, con quale probabilità le prime $8$ prove danno esattamente un successo?

(f) Introdotte anche $Z=$ "numero di prove necessarie per il primo successo" e $W=$ "numero di insuccessi prima del primo successo", calcolare le leggi congiunte delle coppie $(Z,W),(Y,Z)$ $(Y,W).$ Sono coppie di variabili aleatorie indipendenti?

(g) Determinare il coefficiente di correlazione $\rho _{Z,W}$ e la covarianza $\mathrm{Cov} (Z,W)$

cosa sarebbe cambiato se avessimo realizzato una successione di variabili aleatorie $X_{n}$ i.i.d., $X_{n} \sim \mathrm{Be} (p)$, in un altro spazio di probabilità $(\Omega ,\mathcal{A} ,\mathbb{P} )$?
\Esercizio{Esercizio $17.$}

Esibire due diversi spazi $(\Omega ,\mathcal{A} ,\mathbb{P} )$, uno discreto e uno continuo, su cui è possibile definire un vettore aleatorio $( X_{1} ,X_{2}) ,$ con $X_{1}$ e $X_{2}$ indipendenti, $X_{1} \sim \mathrm{Bi}( n_{1} ,p)$, $X_{2} \sim \mathrm{Bi}( n_{2} ,p)$, dove $n_{1} ,n_{2} \in \mathbb{N}$ e $p\in (0,1)$.
\Esercizio{Esercizio 18.}

Alberto usa le lenti a contatto, ma è parecchio maldestro e, quando al mattino cerca di mettersele, spesso nell'operazione ne perde una, se non entrambe. Detto quindi $X$ il numero di lenti a contatto perse al mattino da Alberto, sappiamo che per un qualche valore del parametro $p$ il numero casuale $X$ ha distribuzione
\begin{equation*}
p_{X} (k)=\mathbb{P} (X=k)=\begin{cases}
p & k=0\\
1-p-p^{2} & k=1\\
p^{2} & k=2
\end{cases}
\end{equation*}
1. Stabilire i possibili valori del parametro $p$.

Occupiamoci ora di quel che può capitare in due giorni, considerando indipendenti, e distribuiti entrambi come $X,$ i totali $X_{1}$ e $X_{2}$ di lenti perse alla prima e alla seconda mattina. Quando Alberto perde (almeno) una lente, esce in ritardo nel vano tentativo di cercarla. Siano $Y=$ "numero di lenti a contatto perse da Alberto in due giorni", $\mathrm{Z} =$ "numero di mattine su due giorni in cui Alberto è in ritardo". Trovare:

2. La distribuzione di $Y$.

3. La distribuzione di $Z$.

\ParteSoluzioni

