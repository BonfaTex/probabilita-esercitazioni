%!TEX root = ../main.tex

\ParteEsercizi

\Esercizio{}

Date due variabili aleatorie reali X e Y di varianza finita e non nulla, definite sul medesimo spazio di probabilità, si consideri il coefficiente di correlazione $\rho_{X,Y} =\frac{\mathrm{Cov} (X,Y)}{\sigma_{X} \sigma_{Y}}$. Si mostri che:
\begin{enumerate}[a)]
	\item se $X$ e $Y$ sono indipendenti, allora $\rho_{X,Y} =0$
	\item $\rho_{aX+b,cY+d} =\rho_{X,Y}$ per ogni $a,c >0,b,d\in \mathbb{R}$
	\item $\rho_{X,aX+b} =a/|a|$, per ogni $a\neq 0$
	\item $\mathrm{Var}\left(\frac{1}{\sigma_{Y}} Y-\frac{\rho_{X,Y}}{\sigma_{X}} X\right)=1-\rho_{X,Y}^{2}$
	\item $| \rho_{X,Y}| =1$ implica $Y=aX+b$ per qualche $a\neq 0,b\in \mathbb{R}$. Suggerimento: utilizzare il punto d
\end{enumerate}

\Esercizio{$\star$}

Dato un vettore aleatorio $X=(X_{1} ,\dots ,X_{n}) ,$ in cui ciascuna variabile aleatoria $X_{i}$ ha varianza finita, si mostri che $X\in \mathrm{Im} (\mathrm{C})+\EE[\mathrm{X}]$ q.c., ovvero $\PP (\mathrm{X} \in \mathrm{Im} (\mathrm{C})+\EE[\mathrm{X}])=1$, dove $\mathrm{C}$ è la

matrice varianza di $X$. Suggerimento: si supponga inizialmente che $\EE[X]=0$ e, in tal caso, si mostri che $X\in (\mathrm{Ker} (C))^{\perp } =\mathrm{Im} (C)$ q.c.

\Esercizio{}\label{sec:ese3}

Date $X_{1} ,\dots ,X_{n}$ variabili aleatorie reali i.i.d. con momento secondo finito, introdotti
\begin{gather*}
\mu =\EE[X_{k}] ,\ \ \sigma^{2} =\mathrm{Var}(X_{k}) ,\\
\overline{X}_{n} =\frac{1}{n}\sum_{k=1}^{n} X_{k} ,\ \ S_{n}^{2} =\frac{1}{n-1}\sum_{k=1}^{n}(X_{k} -\overline{X}_{n})^{2} ,\ \ W_{n}^{2} =\frac{1}{n}\sum_{k=1}^{n}(X_{k} -\mu)^{2}
\end{gather*}
si mostri che:
\begin{enumerate}[a)]
	\item $\EE[\overline{X}_{n}] =\mu ,\mathrm{Var}(\overline{X}_{n}) =\sigma^{2} /n$
	\item\label{esercizio3b} $\sum_{k=1}^{n} X_{k}^{2} =\sum_{k=1}^{n}(X_{k} -\overline{X}_{n})^{2} +n\overline{X}_{n}^{2}$
	\item $\EE\left[S_{n}^{2}\right] =\EE\left[W_{n}^{2}\right] =\sigma^{2}$
\end{enumerate}

\Esercizio{}

Siano $X_{1} ,\dots ,X_{n}$ variabili aleatorie indipendenti con funzioni di ripartizione $F_{X_{1}} ,\dots ,F_{X_{n}}$
\begin{enumerate}[a)]
	\item Qual è la funzione di ripartizione di $X_{(n)} =\max\{X_{1} ,\dots ,X_{n}\} ?$
	\item Qual è la funzione di ripartizione di $X_{(1)} =\min\{X_{1} ,\dots ,X_{n}\} ?$
	\item $M=\min\{X_{2} ,X_{3}\}$ e $S=X_{1} +X_{7}$ sono indipendenti?
\end{enumerate}

Le variabili aleatorie $X_{(1)}$ e $X_{(n)}$ sono indipendenti? Si consideri, ad esempio, il caso in cui $n=2$ e $X_{1} ,X_{2}$ sono variabili aleatorie i.i.d. bernoulliane di parametro $p\in (0,1)$

\begin{enumerate}[a),resume]
	\item Come sono distribuite $X_{(1)}$ e $X_{(2)}$?
	\item $X_{(1)}$ e $X_{(2)}$ sono indipendenti?
\end{enumerate}

\Esercizio{}

Siano $X,\ Y,\ Z$ variabili aleatorie reali. Mostrare che:
\begin{enumerate}[a)]
	\item Se $Y=c$ q.c., per qualche costante $c\in \mathbb{R} ,$ allora $X$ ed $Y$ sono indipendenti.
	\item $\star$ Se $\PP (Z\in B)\in \{0,1\}$ per ogni boreliano $B,$ allora $Z$ è q.c. costante.
	\item $\star$ Nel caso $Y=h(X)$, per qualche funzione boreliana $h:\mathbb{R}\rightarrow \mathbb{R} ,X$ e $Y$ sono indipendenti se e solo se $Y=c$ q.c., per qualche costante $c\in \mathbb{R}$
\end{enumerate}

\Esercizio{}

Un perito elettrotecnico deve costruire un sistema costituito da tre componenti in serie. Egli pesca i tre componenti da una scatola in cui vi sono tre componenti nuovi, due usati ma funzionanti e due difettosi. Siano $X$ ed $Y$ rispettivamente il numero di componenti nuovi e di componenti usati ma funzionanti tra quelli pescati dalla scatola.

\begin{enumerate}[a)]
	\item Determinare la legge congiunta di $X$ ed $Y$ e le leggi marginali.
	\item Le variabili $X$ ed $Y$ sono indipendenti?
	\item Calcolare $\EE[\mathrm{X}],\EE[\mathrm{Y}],\EE[\mathrm{XY}].$ Scrivere la matrice varianza di $(X,Y)$ e determinare $\rho_{X,Y}$
	\item Calcolare la legge, il valore atteso e la varianza del numero di componenti pescati funzionanti.
	\item Calcolare la probabilità che l'apparecchio funzioni.
\end{enumerate}


\Esercizio{}

Siano $X$ ed $Y$ variabili aleatorie i.i.d. bernoulliane di parametro $p=\frac{1}{2}$ e siano $u=X+Y$ e $V=|X-Y|$
\begin{enumerate}[a)]
	\item Mostrare che $(U,V)$ è un vettore aleatorio e determinarne la legge.
	\item Calcolare la probabilità che $V$ sia minore di $U$
	\item Calcolare la covarianza di $U$ e $V$ e la matrice varianza di $(U,V)$
	\item $U$ e $V$ sono indipendenti?
\end{enumerate}

\Esercizio{}

Siano $X$ ed $Y$ due variabili aleatorie con legge congiunta parzialmente data da:
\begin{equation*}
	\begin{array}{ c|c|c|c|c }
		X\backslash Y & -1 & 5 & 10 & p_{X}\\
		\hline
		0 &  & 0.12 &  & 0.4\\
		\hline
		5 &  &  &  & \\
		\hline
		p_{Y} & 0.3 &  &  & 1
	\end{array}
\end{equation*}
\begin{enumerate}[a)]
	\item Completare la tabella in modo che $X$ ed $Y$ siano indipendenti.
	\item Calcolare $\PP (X< Y)$
	\item Calcolare il valore atteso del vettore $(X,Y)$ e $\EE[XY]$
	\item Calcolare $\PP (|XY|\geq 5)$ e $\PP (X+Y >5)$
	\item Siano $U=|XY|$ e $V=X+Y$. Calcolare la legge congiunta di $U$ e $V$ e le leggi marginali.
\end{enumerate}

\Esercizio{}

Sia $X$ una variabile aleatoria discreta con legge uniforme sull'insieme discreto $\{-1,1\}$

\begin{enumerate}[a)]
	\item Determinare la legge di $X$.
\end{enumerate}
Sia ora $Y$ un'altra variabile aleatoria discreta, indipendente da $X,$ ma con la stessa legge di $X$. Si introduca $Z=XY$
\begin{enumerate}[a),resume]
	\item Calcolare la legge congiunta di $X$ e $Z$ e le leggi marginali.
	\item $X,Y$ e $Z$ sono indipendenti a coppie?
	\item $X,Y$ e $Z$ sono mutuamente indipendenti?
\end{enumerate}

\Esercizio{}

Date $X$ e $Y$ variabili aleatorie indipendenti di legge geometrica di parametro $1/2$, si definisca $Z=\min \{X,Y\}$ e si calcolino:

\begin{enumerate}[a)]
	\item $\PP (Z\leq k)$ per $k\in \NN$
	\item la distribuzione di $Z$
	\item $\PP (X=Y)$
	\item $\PP (X >Y)$
\end{enumerate}

\Esercizio{}

Siano $T_{1}$ e $T_{2}$ variabili aleatorie indipendenti di legge geometrica di parametro, rispettivamente, $p_{1}$ e $p_{2} ,$ con cui a tempo discreto si descrive la durata aleatoria di due apparecchiature.
\begin{enumerate}[a)]
	\item Scrivere la legge congiunta di $T_{1}$ e $T_{2}$
	\item Calcolare la probabilità degli eventi $\{T_{1} =T_{2}\}$ e $\{T_{1} \geq T_{2}\}$
	\item Trovare la legge della durata del sistema composto dalle due apparecchiature collegate in serie.
	\item Trovare la legge della durata del sistema composto dalle due apparecchiature collegate in parallelo.
	\item Trovare la legge congiunta di $U=\min\{T_{1} ,T_{2}\}$ e $V=\max\{T_{1} ,T_{2}\}$
	\item Trovare la legge congiunta di $U$ e $W=V-U$
	\item Trovare la legge di $W$
	\item $U$ e $W$ sono indipendenti?
\end{enumerate}

\Esercizio{}

Siano $X,\ Z,\ W$ variabili aleatorie indipendenti con $Z$ e $W$ entrambe con legge di Poisson di parametro $\lambda  >0$ e $X\sim \mathrm{Be} (p),p\in (0,1)$. Definiamo la variabile aleatoria $Y=XZ+W$
\begin{enumerate}[a)]
	\item Determinare le leggi di $(X,Y)$ e $Y$
	\item Calcolare $\EE[Y]$ e $\operatorname{Var} (Y)$
	\item Calcolare $\mathrm{Cov} (X,Y)$. Le variabili aleatorie $X$ e $Y$ sono indipendenti?
\end{enumerate}

\Esercizio{}

Siano $N,X_{1} ,X_{2} ,\dots $ variabili aleatorie indipendenti, $N$ abbia legge di Poisson di parametro $\lambda  >0,$ e ciascuna delle $X_{k}$ abbia legge di Bernoulli di parametro $p\in (0,1).$ Si consideri la somma aleatoria (per addendi e per numero di addendi)
\begin{equation*}
	S=
	\begin{cases}
		0, & N=0,\\
		X_{1} +\cdots +X_{N} , & N\neq 0.
	\end{cases}
\end{equation*}
\begin{enumerate}[a)]
	\item $\star$ Si mostri che $S$ e $N-S$ sono variabili aleatorie.
	\item Qual è la legge di $S$?
	\item Qual è la legge di $N-S$?
	\item La somma $S$ è indipendente da $N$?
	\item Determinare la densità discreta congiunta di $S$ e $N-S$. La somma $S$ è indipendente da $N-S$ ?
\end{enumerate}

\Esercizio{}

Sia $X$ una variabile aleatoria discreta, con $\mathrm{Im} (X)=\{1,2,\dots \},$ che soddisfa
\begin{equation*}
\PP (X\geq k)=\frac{1}{k^{\alpha }} ,\ \ k\geq 1
\end{equation*}
dove $\alpha  >0$ è un parametro reale.
\begin{enumerate}[a)]
	\item Calcolare la densità discreta di $X$.
	\item Calcolare la funzione di ripartizione di $X,$ disegnandone un grafico qualitativo.
	\item Per $\alpha =1$, calcolare il valore atteso di $X$
\end{enumerate}
Si supponga d'ora in poi $\alpha =1.$ Sia $Y$ un'altra variabile aleatoria, indipendente da $X,$ ma con la stessa legge.
\begin{enumerate}[a),resume]
	\item Calcolare la densità discreta di $M=\min \{X,Y\}$
	\item Calcolare la densità discreta congiunta di $X$ e $X+Y$.
\end{enumerate}

\Esercizio{}

Si consideri l'estrazione di $n$ palline da un'urna contenente $B$ palline bianche e $R$ palline rosse ($B\geq 1$ e $R\geq 1$). Sia $X_{k}$ la variabile aleatoria bernoulliana che indica se alla $k$-esima estrazione esce una pallina rossa e sia $Y=\sum_{k=1}^{n} X_{k}$ il numero di palline rosse sulle $n$ estratte. Si risponda ai seguenti quesiti, sia nel caso di estrazione con reimmissione sia nel caso senza reimmissione (in quest'ultimo caso $n\leq B+R$).
\begin{enumerate}[a)]
	\item Qual è la legge di $Y$?
	\item Qual è il valore atteso delle $X_{k} ?$
	\item Qual è il valore atteso di $Y?$
	\item Le $X_{k}$ sono indipendenti?
	\item $Y$ e $X_{1}$ sono indipendenti?
	\item Calcolare $\mathrm{Cov}(X_{1} ,X_{2})$
\end{enumerate}
$(\mathrm{g})$ Calcolare $\mathrm{Var} (Y)$ per $n=2$

\Esercizio{}

Consideriamo infinite prove di Bernoulli indipendenti con probabilità di successo $p\in (0,1)$. Siano quindi $\Omega =\{0,1\}^{\NN}$ e $\mathcal{A} =\sigma (E_{k} \mid k=1,2,\dots)$,
\begin{equation*}
E_{k} =\ \text{successo alla prova } k
\end{equation*}
e sia $\PP$ tale che $\{E_{k}\}_{k\in \NN}$ risulti una famiglia di eventi indipendenti con $\PP(E_{k}) =p$ per ogni $k$. Indicato con $\omega =(\omega_{k})_{k=1}^{\infty }$ il generico esito dello spazio campionario $\Omega $, definiamo infine le variabili aleatorie
\begin{equation*}
X_{k} :\Omega \rightarrow \mathbb{R} ,\ \ X_{k} (\omega)=\omega_{k} ,\ \ k\in \NN
\end{equation*}
\begin{enumerate}[a)]
	\item Le variabili aleatorie $X_{k}$ sono indipendenti?
	\item Le variabili aleatorie $Y=\sum_{k=1}^{2} X_{k}$ e $N=\sum_{k=5}^{7} X_{k}$ sono indipendenti?
	\item Le variabili aleatorie $\overline{X}_{n} =\frac{1}{n}\sum_{k=1}^{n} X_{k}$ e $S_{n}^{2} =\frac{1}{n-1}\sum_{k=1}^{n}(X_{k} -\overline{X}_{n})^{2}$ sono indipendenti?
\end{enumerate}
Suggerimento: per il punto \ref{esercizio3b} dell'Esercizio \ref{sec:ese3}, si ha che $S_{n}^{2} =\frac{1}{n-1}\sum_{k=1}^{n} X_{k}^{2} -\frac{n}{n-1}\overline{X}_{n}^{2} ;$ notare poi che $X_{k}^{2} =X_{k}$.
\begin{enumerate}[a),resume]
	\item Se nelle prime $10$ prove vengono registrati $2$ successi, con quale probabilità almeno un successo si è verificato nelle prime $2$ prove?
	\item Se nelle prime $10$ prove viene registrato almeno un successo, con quale probabilità le prime $8$ prove danno esattamente un successo?
	\item Introdotte anche $Z=$ \event{numero di prove necessarie per il primo successo} e $W=$ \event{numero di insuccessi prima del primo successo}, calcolare le leggi congiunte delle coppie $(Z,W),(Y,Z)$ $(Y,W).$ Sono coppie di variabili aleatorie indipendenti?
	\item Determinare il coefficiente di correlazione $\rho_{Z,W}$ e la covarianza $\mathrm{Cov} (Z,W)$
\end{enumerate}
cosa sarebbe cambiato se avessimo realizzato una successione di variabili aleatorie $X_{n}$ i.i.d., $X_{n} \sim \mathrm{Be} (p)$, in un altro spazio di probabilità $(\Omega ,\mathcal{A} ,\PP)$?

\Esercizio{}

Esibire due diversi spazi $(\Omega ,\mathcal{A} ,\PP)$, uno discreto e uno continuo, su cui è possibile definire un vettore aleatorio $(X_{1} ,X_{2}) ,$ con $X_{1}$ e $X_{2}$ indipendenti, $X_{1} \sim \mathrm{Bi}(n_{1} ,p)$, $X_{2} \sim \mathrm{Bi}(n_{2} ,p)$, dove $n_{1} ,n_{2} \in \NN$ e $p\in (0,1)$.

\Esercizio{}

Alberto usa le lenti a contatto, ma è parecchio maldestro e, quando al mattino cerca di mettersele, spesso nell'operazione ne perde una, se non entrambe. Detto quindi $X$ il numero di lenti a contatto perse al mattino da Alberto, sappiamo che per un qualche valore del parametro $p$ il numero casuale $X$ ha distribuzione
\begin{equation*}
	p_{X} (k)=\PP (X=k)=
	\begin{cases}
		p & k=0\\
		1-p-p^{2} & k=1\\
		p^{2} & k=2
	\end{cases}
\end{equation*}
\begin{enumerate}[a)]
	\item Stabilire i possibili valori del parametro $p$.
\end{enumerate}
Occupiamoci ora di quel che può capitare in due giorni, considerando indipendenti, e distribuiti entrambi come $X,$ i totali $X_{1}$ e $X_{2}$ di lenti perse alla prima e alla seconda mattina. Quando Alberto perde (almeno) una lente, esce in ritardo nel vano tentativo di cercarla. Siano $Y=$ \event{numero di lenti a contatto perse da Alberto in due giorni}, $\mathrm{Z} =$ \event{numero di mattine su due giorni in cui Alberto è in ritardo}. Trovare:
\begin{enumerate}[a),resume]
	\item La distribuzione di $Y$.
	\item La distribuzione di $Z$.
\end{enumerate}

\ParteSoluzioni

